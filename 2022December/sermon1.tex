\section{4th Dec 2022: ``Do not be afraid, Mary''}
\subsection*{Text: Luke 1:26-38}
  \begin{quote}
    [26] In the sixth month the angel Gabriel was sent from God to a city of Galilee named Nazareth, [27] to a virgin betrothed to a man whose name was Joseph, of the house of David. And the virgin’s name was Mary. [28] And he came to her and said, “Greetings, O favored one, the Lord is with you!” [29] But she was greatly troubled at the saying, and tried to discern what sort of greeting this might be. [30] And the angel said to her, “Do not be afraid, Mary, for you have found favor with God. [31] And behold, you will conceive in your womb and bear a son, and you shall call his name Jesus. [32] He will be great and will be called the Son of the Most High. And the Lord God will give to him the throne of his father David, [33] and he will reign over the house of Jacob forever, and of his kingdom there will be no end.”

    [34] And Mary said to the angel, “How will this be, since I am a virgin?”

    [35] And the angel answered her, “The Holy Spirit will come upon you, and the power of the Most High will overshadow you; therefore the child to be born will be called holy—the Son of God. [36] And behold, your relative Elizabeth in her old age has also conceived a son, and this is the sixth month with her who was called barren. [37] For nothing will be impossible with God.” [38] And Mary said, “Behold, I am the servant of the Lord; let it be to me according to your word.” And the angel departed from her.
  \end{quote}
\subsection*{Notes}
\begin{itemize}
  \item{``O favoured one'' (v28)
  \begin{itemize}
    \item{Why should Mary be accorded such kind words?  From the world's
    perspective, Mary's background wasn't the best.  Mary was among the
    lowly, she was also probably young, she was also uneducated, living in a
    small country town called Nazareth (Nazareth is a backwater kind of
    place).  From a merely human perspective, Mary was insignificant.  She
    was a nobody in a nothing town in the middle of nowhere.  Yet she was
    given the greatest honor, to bear God's Son.}
    \item{Hence, from Mary's background, we see that grace is for the lowly. In fact, ``lowly'' is a term used by the world to describe other people. For God, He does not view people as ``lowly'', but actually everyone is significant to God, as everyone is made in God's image and is loved by God.}
    \item{God can give the same kind of grace that He gave to Mary, not so much to bear Jesus, but to \textit{bear the presence of Jesus} wherever we are. This is our privilege as we allow God to use us, regardless of how insignificant the world views us. }
    \item{God can use powerful poeple too, but when God chooses to ue what the world views as insignificant, His own power is magnified. However insignificant we view ourselves, we must remember that God doesn't see us that way. And as we bear the presence of Jesus, we bring peace to the world, as Jesus is the prince of peace.}
  \end{itemize}}
  \item{``I am a virgin'' (v34)
  \begin{itemize}
    \item{Gabriel's announcement was very shocking news to Mary, who was a virgin. No wonder Gabriel had to re-assure Mary that she was highly favoured. }
    \item{When Adam and Eve fell into sin, all who come after them were born sinners.}
    \item{God's rescue plan needed a second Adam, who lived in perfect obedience to God. But since all of us are fallen, the second Adam could not come from mandkind. Hence, the second Adam cannot be born through the sexual union of two humans. The second Adam, in God's plan, is the eternal Son of God who took on sinless human flesh.}
    \item{Hence the virgin birth served two purposes: first to mark out Jesus' special divine identity, and secondly to ensure that the human nature of Jesus was untainted by Adam's original sin.}
    \item{Hence, as Jesus was born of a virgin birth, He was born fully human (since He came through Mary's birth canal), and He was also born fully God (since Jesus is the Son of God who took on human flesh).}
    \item{The virgin birth sounds unscientific, but if God can create the world, He can do all miracles.}
    \item{The virgin birth was prophesied 700 years ago in Isaiah 7.  Isaiah
    7 used his message for a double fulfilment.  The first fulfilment was
    that God would be with Israel (Immanuel) in judgment, through the king of
    Assyria.  The king of Assyria would come before a boy born in that time
    could “refuse the evil and choose the good”.  The second fulfilment is in
    Jesus’ birth through Mary, that God would be with His people to save them
    again.  Jesus is the true Immanuel.  OT prophesy is very beautiful lol,
    this double fulfilment stuff is amazing.}
  \end{itemize}}
  \item{``I am the servant of the Lord'' (v38)
  \begin{itemize}
    \item{Mary did not raise any objection, did not protest at all.  Mary did
    not even ask God to explain what would happen after.  But Mary was not
    naive.  She knew the consequences of saying “yes” to God.  She would
    possibly be divorced by Joseph (and that was indeed what Joseph wanted to
    do), and she would be the subject of much gossip.  But as we know, Joseph
    saved the day after an angel spoke to him.}
    \item{Mary also suffered a lot of unexpected suffering, like fleeing to
    egypt to escape Herod.  Mary also suffered opposition when Jesus was
    born, but most of all, she suffered as if a sword pierced her heart when
    Jesus was crucified.  Mary was a woman of great faith.  She had faith in
    God despite the sacrifices that she expected and she did not expect.}
    \item{ For us, we need to be like Mary and say: “By the grace of God,
    through faith in Jesus, have it your way Lord.  I am your servant”.  And
    we can hear God say to us “do not be afraid”, since God who calls us also
    gives us the grace and the strength to perform His calling.  }
  \end{itemize} }
\end{itemize}