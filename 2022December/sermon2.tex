\section{11th December 2022: The Gospel According To The Magnificat}
\subsection*{Text: Luke 1:46-55}
  \begin{quote}
    [46] And Mary said,

    “My soul magnifies the Lord,
    [47]     and my spirit rejoices in God my Savior,
    [48] for he has looked on the humble estate of his servant.
        For behold, from now on all generations will call me blessed;
    [49] for he who is mighty has done great things for me,
        and holy is his name.
    [50] And his mercy is for those who fear him
        from generation to generation.
    [51] He has shown strength with his arm;
        he has scattered the proud in the thoughts of their hearts;
    [52] he has brought down the mighty from their thrones
        and exalted those of humble estate;
    [53] he has filled the hungry with good things,
        and the rich he has sent away empty.
    [54] He has helped his servant Israel,
        in remembrance of his mercy,
    [55] as he spoke to our fathers,
        to Abraham and to his offspring forever.”
  \end{quote}
\subsection*{Notes}
\begin{itemize}
  \item{We are very used to thinking that Christianity is very concerned with
  personal piety.  I.e, Christianity as getting repenting of our own personal
  sins, our behaviour before God, and getting myself to heaven.  I.e,
  Christianity as a faith that is concerned with the immaterial and the
  otherworldly.  All those are true, but does that mean that our Christianity
  has nothing to do with the socio-political realm of our world?}
  \item{If Christianity has nothing to do with the socio-political realm of
  our world, then our religion is really the opium of the masses, because our
  religion causes us to be apathetic to the suffering and injustices of the
  world when we focus on just the personal salvation aspect of our faith.
  The good news according to the Magnificat is that our King Jesus is on the
  side of the poor and the oppressed, and is against the rich and the
  oppressors.  And thus we Christians, as followers of our King Jesus, would
  also be on the side of the poor and the oppressed.}
  \item{Mary's song starts of with a recognition of how much God has blessed
  her and how much grace God has given her.  What exactly is the grace and
  blessing that God has given her though?  One aspect is that Jesus, her Son,
  is the Son of God and is the King of the world.  Hence, one blessing that
  God has given Mary is to elevate Mary to be the mother of Jesus.  In this
  sense, all generations will call her blessed, because of who Jesus is. }
  \item{The first point of today is that Jesus is the universal King.  Our
  tendency today is to spiritualise this, but we must recognise that the term
  ``King'' is a socio-political term.  What that means for us is that just
  like how earthly kings have absolute authority over their subjects, Jesus
  as the universal King has absolute authority over us.  And Jesus as the
  universal King means that we are to give full allegiance to Jesus, not to
  earthly ideologies like democracy, meritocracy, socialism, capitalism etc.
  All of these earthly ideologies must be appraised against our allegiance to
  Jesus' kingship. }
  \item{Now, what kind of King is Jesus?  Jesus is the universal King that is
  on the side of the poor and lowly.  And just as Jesus has elevated Mary
  (who is poor and lowly), Jesus will also elevate all who are poor and
  lowly.  The socio-political condition of Mary's time is that the common
  folk, the majority, are all poor and lowly and oppressed by the rich
  minority who are rich only at the expense of those who are poor.  The
  helpless Jews in Mary's day could only hope that one day, God will free
  them for their opression.  Is their hope a false one?  In the OT, it has
  been explicitly said that God is on the side of the poor and the lowly, and
  thus their hope is not an empty one (c.f Exodus 22:21-27, Isaiah 61).  One
  day, God will set injustices right through His Messiah.  When Mary says
  that ``His mercy is on those who fear him, from generation to generation'',
  this means that God will save all the lowly who trust in Him.}
  \item{However, as we can see with our eyes, the dramatic reversal described
  in the Magnificat hasn't happened yet.  That'll happen at Jesus' second
  coming.  So what did Jesus' first coming accomplish?  Jesus' first coming
  was to gather together a people for Himself that would manifest the Kingdom
  of God through our faithful lives and through our caring for the poor, just
  like how God cares for the poor.  }
  \item{To be clear, it is not wrong to be rich and powerful.  What is wrong
  however are the sins of omission that rich people often commit; if one has
  the resources to help the poor but instead uses his resources for his own
  pleasure and benefit, then that is wrong.  What is even worse is that if
  one is rich at the expense of the poor; then that is really jialat.}
  \item{Since the Church is a manifestation of the Kingdom of God, then the
  Church has to care for the poor in our midst and around us (James 2:15-16).
  We must also be aware of sinful power structures in the Church.  Do we pay
  more attention to rich members of the Church?  God is against that (James
  2:5).  If we, as the Church can't reflect the ideals of Jesus' kingdom,
  then our gospel sharing is just propaganda.  When we are outside the
  Church, we should stand with those who are poor and oppressed in social
  causes.  Or at a more personal level, we are to stand up for colleagues who
  are wrongly accused, we are to give monetary aid to those who need it,
  etc.}
  \item{From Philippians 2:4-8, we see that Jesus was born lowly and
  identified with the weak and the lowly.  When Jesus came, He came in
  solidarity with the poor and the powerless, He spoke up against the
  religious and political opressors, and hence He was crucified for being an
  irritant to the state.  But God raised Jesus from the dead, as a teaser of
  what He'll do in the future when Jesus comes again.  When Jesus comes
  again, those who hunger for justice will be satisfied, all wrongs will be
  made right, and all the opressors will face the consequences of their
  opression.}
\end{itemize}