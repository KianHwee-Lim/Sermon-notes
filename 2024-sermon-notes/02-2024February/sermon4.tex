\setcounter{figure}{0}

\section{25th February 2024: Do not pervert God's justice}
\subsection*{Text: Job 8-9}
  \begin{quote}
    [1] Then Bildad the Shuhite answered and said:

    [2] “How long will you say these things,
        and the words of your mouth be a great wind?
    [3] Does God pervert justice?
        Or does the Almighty pervert the right?
    [4] If your children have sinned against him,
        he has delivered them into the hand of their transgression.
    [5] If you will seek God
        and plead with the Almighty for mercy,
    [6] if you are pure and upright,
        surely then he will rouse himself for you
        and restore your rightful habitation.
    [7] And though your beginning was small,
        your latter days will be very great.


    [8] “For inquire, please, of bygone ages,
        and consider what the fathers have searched out.
    [9] For we are but of yesterday and know nothing,
        for our days on earth are a shadow.
    [10] Will they not teach you and tell you
        and utter words out of their understanding?


    [11] “Can papyrus grow where there is no marsh?
        Can reeds flourish where there is no water?
    [12] While yet in flower and not cut down,
        they wither before any other plant.
    [13] Such are the paths of all who forget God;
        the hope of the godless shall perish.
    [14] His confidence is severed,
        and his trust is a spider’s web.
    [15] He leans against his house, but it does not stand;
        he lays hold of it, but it does not endure.
    [16] He is a lush plant before the sun,
        and his shoots spread over his garden.
    [17] His roots entwine the stone heap;
        he looks upon a house of stones.
    [18] If he is destroyed from his place,
        then it will deny him, saying, ‘I have never seen you.’
    [19] Behold, this is the joy of his way,
        and out of the soil others will spring.


    [20] “Behold, God will not reject a blameless man,
        nor take the hand of evildoers.
    [21] He will yet fill your mouth with laughter,
        and your lips with shouting.
    [22] Those who hate you will be clothed with shame,
        and the tent of the wicked will be no more.”

    [1] Then Job answered and said:

    [2] “Truly I know that it is so:
        But how can a man be in the right before God?
    [3] If one wished to contend with him,
        one could not answer him once in a thousand times.
    [4] He is wise in heart and mighty in strength
        —who has hardened himself against him, and succeeded?—
    [5] he who removes mountains, and they know it not,
        when he overturns them in his anger,
    [6] who shakes the earth out of its place,
        and its pillars tremble;
    [7] who commands the sun, and it does not rise;
        who seals up the stars;
    [8] who alone stretched out the heavens
        and trampled the waves of the sea;
    [9] who made the Bear and Orion,
        the Pleiades and the chambers of the south;
    [10] who does great things beyond searching out,
        and marvelous things beyond number.
    [11] Behold, he passes by me, and I see him not;
        he moves on, but I do not perceive him.
    [12] Behold, he snatches away; who can turn him back?
        Who will say to him, ‘What are you doing?’


    [13] “God will not turn back his anger;
        beneath him bowed the helpers of Rahab.
    [14] How then can I answer him,
        choosing my words with him?
    [15] Though I am in the right, I cannot answer him;
        I must appeal for mercy to my accuser.
    [16] If I summoned him and he answered me,
        I would not believe that he was listening to my voice.
    [17] For he crushes me with a tempest
        and multiplies my wounds without cause;
    [18] he will not let me get my breath,
        but fills me with bitterness.
    [19] If it is a contest of strength, behold, he is mighty!
        If it is a matter of justice, who can summon him?
    [20] Though I am in the right, my own mouth would condemn me;
        though I am blameless, he would prove me perverse.
    [21] I am blameless; I regard not myself;
        I loathe my life.
    [22] It is all one; therefore I say,
        ‘He destroys both the blameless and the wicked.’
    [23] When disaster brings sudden death,
        he mocks at the calamity of the innocent.
    [24] The earth is given into the hand of the wicked;
        he covers the faces of its judges—
        if it is not he, who then is it?


    [25] “My days are swifter than a runner;
        they flee away; they see no good.
    [26] They go by like skiffs of reed,
        like an eagle swooping on the prey.
    [27] If I say, ‘I will forget my complaint,
        I will put off my sad face, and be of good cheer,’
    [28] I become afraid of all my suffering,
        for I know you will not hold me innocent.
    [29] I shall be condemned;
        why then do I labor in vain?
    [30] If I wash myself with snow
        and cleanse my hands with lye,
    [31] yet you will plunge me into a pit,
        and my own clothes will abhor me.
    [32] For he is not a man, as I am, that I might answer him,
        that we should come to trial together.
    [33] There is no arbiter between us,
        who might lay his hand on us both.
    [34] Let him take his rod away from me,
        and let not dread of him terrify me.
    [35] Then I would speak without fear of him,
        for I am not so in myself.

  \end{quote}
\subsection*{Notes}
\begin{itemize}
  \item{Our response to suffering is based on our view and undersading of God
  and other worldviews. Today we will see two view, Bildad's view and Job's
  view. Bildad's view can be summarised as: ``if,$\dots$ then $\dots$''. I.e,
  Bildad's view is one of ``retributive theology''. See for example, chapter
  8:v4-6. Bildad's logic is this:
  \begin{itemize}
    \item{If someone has sinned, then there will be suffering. Therefore, if
  someone is suffering, they must have sinned. Just accept it and repent.}
    % \item{\KH{Though actually when I read the text for myself, I don't see Bildad using the logic as described above.}}
  \end{itemize}
  % \KH{Assuming Bildad's view is really as mentioned above, then this is a
  % logic error, known as ``affirming the consequent.'' E.g, ``if something is
  % a human, then it is mortal. Therefore since something is mortal, it is a
  % human.'' This has the same structure as Bildad's (supposed) logic above,
  % which is an obvious error lol, since animals are also mortal.}}
  }
  \item{Retributive justice is a view that is quite common in the OT. E.g,
  Psalm 1. God's blessing is a consequence of one's faithfulness to the
  Mosaic covenant, and God's punishment and judgement (which leads to
  suffering) falls on those who transgress the covenant. 
  \begin{itemize}
    \item{\KH{Symbolically, we have $ F \implies B$, $\neg F \implies \neg B$ which gives us $F \iff B$, where $F$ stands for faithfulness, $B$ stand for blessing (and hence $\neg B$ stands for a lack of blessing, which is judgement and punishment).}}
  \end{itemize}
  Thus, a lack of blessing would imply one's lack of faithfulness. But this is wrong, e.g John 9:1-3.}
  \item{Job's response is this: ``Truly I know that it is so: But how can a
  man be in the right before God?''. Job's view is this: ``if someone has
  sinned, then there will be suffering. \textbf{But} not all suffering is
  because of sin. I need someone to comfort me''. Job's view is a correct
  view of retributive justice, especially in our fallen world where Satan is
  active. }
  \item{Btw, retributive justice for us in the NT (\KH{and also those in the OT
  who saw the Messiah in the types and shadows of the Mosaic covenant}) must
  be interpreted in the light of Jesus' death and resurrection on the cross.
  All of us have sinned and hence all of us deserve the full punishment for
  our sin, but that has been borne by Jesus on the cross. Thus, for all of us
  who believe, there is no more condemnation for sin, and hence the suffering
  we experience now is not necessarily punishment for sin. \KH{There might
  not be any reason for our suffering that we can understand, but we know God
  will be glorified through our suffering.}}
  \item{\KH{I got lost already... was abit hard \textbf{for me} to follow
  this particular sermon.} }
\end{itemize}