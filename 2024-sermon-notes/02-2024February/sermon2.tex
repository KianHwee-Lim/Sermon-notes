\setcounter{figure}{0}

\section{11th Feb 2024: You have sinned?}
\subsection*{Text: Job 4:1-9, 5:17, 6:24-30}
  \begin{quote}
    [1] Then Eliphaz the Temanite answered and said:

    [2] “If one ventures a word with you, will you be impatient?
        Yet who can keep from speaking?
    [3] Behold, you have instructed many,
        and you have strengthened the weak hands.
    [4] Your words have upheld him who was stumbling,
        and you have made firm the feeble knees.
    [5] But now it has come to you, and you are impatient;
        it touches you, and you are dismayed.
    [6] Is not your fear of God your confidence,
        and the integrity of your ways your hope?


    [7] “Remember: who that was innocent ever perished?
        Or where were the upright cut off?
    [8] As I have seen, those who plow iniquity
        and sow trouble reap the same.
    [9] By the breath of God they perish,
        and by the blast of his anger they are consumed.

    [17] “Behold, blessed is the one whom God reproves;
        therefore despise not the discipline of the Almighty.

    [24] “Teach me, and I will be silent;
        make me understand how I have gone astray.
    [25] How forceful are upright words!
        But what does reproof from you reprove?
    [26] Do you think that you can reprove words,
        when the speech of a despairing man is wind?
    [27] You would even cast lots over the fatherless,
        and bargain over your friend.


    [28] “But now, be pleased to look at me,
        for I will not lie to your face.
    [29] Please turn; let no injustice be done.
        Turn now; my vindication is at stake.
    [30] Is there any injustice on my tongue?
        Cannot my palate discern the cause of calamity?
  \end{quote}
\subsection*{Notes}
\begin{itemize}
  \item{Today, we see the first exchange between Eliphaz (Job 4:1-9,5:17) and Job (Job 6:24-30).}
  \item{Eliphaz says that because of Job’s integrity and uprightness, Job should have been more patient in his suffering}
  \item{Eliphaz has two points to make: 1. Job has sinned, but Job has made himself out to be innocent. 2. Job was impatient to withstand the discipline of the Lord which came about because of his sin.}
  \item{Was Eliphaz right that Job has sinned? From v7-9, Eliphaz re-iterates that the innocent dont perish, the upright will jot br cut off, but that those who do evil will be punished. The doctrine here is that: “you reap what you sow”. This is not wrong, because the Law operates that this is how God operates in his justice. \KH{But as we fast forward to the end of the book, we see that Eliphaz was rebuked by God for “darkens counsel without knowledge”}}
  \item{If the doctrine above was sound, then what is the problem here? Two problems:
  \begin{enumerate}
    \item{Doctrine is sound, but Eliphaz has not seen enough to know all things and everything. Eliphaz did not see what happened in the heavenly realm.}
    \item{Doctrine is sound, but there also are exceptions to this doctrine. Job is one such exceptional case.} 
  \end{enumerate} 
  }
  \item{Was Eliphaz right that God disciplines people for their sin, and that people will be restored if they repent and submit to God? This doctrine is also correct, its found in places like proverbs. But then Eliphaz didnt realise that the world was very complex, there’s an entire spiritual realm that is unseen. Again, Eliphaz did not know about the wager between God and Satan.}
  \item{\KH{My thoughts: Classical physics gives descriptions that are generally correct, but then there are exceptions. But all of these (the classical physics description and the exceptions) can be described with QM. So similarly, Eliphaz’s doctrine of retributive justice is generally correct, but it is just an approximation, that’s why it doesnt handle Job’s case. The underlying idea is God’s passion for His glory, which includes retributive justice but also sometimes a delaying of retributive justice to get more glory.}}
  \item{Now, how did Job respond? And what can we learn?
  \begin{itemize}
    \item{V24-25: honest words are painful. And different people respond to the painful, honest words in different ways. If we see how Saul and David responded when they are confronted by God’s prophets because of their sin, we can learn the difference (we must emulate David who repented rather than Saul). But Job here has not sinned, but was confronted by his friends. Note: Job's friends are not prophets unlike Samuel/Nathan! }
    \item{A godly person (David) is not one who has not sinned, but will find honest rebukes painful but helpful for the sake of repentance. So for Job who is a godly person, he was frustrated by his friend’s rebuke, because Job was unaware of things he needed to repent for and hence he didnt know what to do, because he knew he was innocent.}
    \item{Job here was saying that Eliphaz’s attitude was very mean (v27) because his words has increased his pain.} 
    \item{Lesson for us, when we are called to rebuke someone, we need to examine all the “evidence” carefully. Like a forensic officer, we need to see if the “evidence” is planted, or if it is genuine. We should be willing to listen and hear and change our minds. This is especially so if we claim to be speaking for God, we don't want to be found to be misrepresenting God!}
  \end{itemize}
  }
\end{itemize}