\setcounter{figure}{0}

\section{4th February 2024: The lament}
\subsection*{Text: Job 3:1-26}
  \begin{quote}
    [1] After this Job opened his mouth and cursed the day of his birth. [2] And Job said:

    [3] “Let the day perish on which I was born,
        and the night that said,
        ‘A man is conceived.’
    [4] Let that day be darkness!
        May God above not seek it,
        nor light shine upon it.
    [5] Let gloom and deep darkness claim it.
        Let clouds dwell upon it;
        let the blackness of the day terrify it.
    [6] That night—let thick darkness seize it!
        Let it not rejoice among the days of the year;
        let it not come into the number of the months.
    [7] Behold, let that night be barren;
        let no joyful cry enter it.
    [8] Let those curse it who curse the day,
        who are ready to rouse up Leviathan.
    [9] Let the stars of its dawn be dark;
        let it hope for light, but have none,
        nor see the eyelids of the morning,
    [10] because it did not shut the doors of my mother’s womb,
        nor hide trouble from my eyes.


    [11] “Why did I not die at birth,
        come out from the womb and expire?
    [12] Why did the knees receive me?
        Or why the breasts, that I should nurse?
    [13] For then I would have lain down and been quiet;
        I would have slept; then I would have been at rest,
    [14] with kings and counselors of the earth
        who rebuilt ruins for themselves,
    [15] or with princes who had gold,
        who filled their houses with silver.
    [16] Or why was I not as a hidden stillborn child,
        as infants who never see the light?
    [17] There the wicked cease from troubling,
        and there the weary are at rest.
    [18] There the prisoners are at ease together;
        they hear not the voice of the taskmaster.
    [19] The small and the great are there,
        and the slave is free from his master.

    [20] “Why is light given to him who is in misery,
        and life to the bitter in soul,
    [21] who long for death, but it comes not,
        and dig for it more than for hidden treasures,
    [22] who rejoice exceedingly
        and are glad when they find the grave?
    [23] Why is light given to a man whose way is hidden,
        whom God has hedged in?
    [24] For my sighing comes instead of my bread,
        and my groanings are poured out like water.
    [25] For the thing that I fear comes upon me,
        and what I dread befalls me.
    [26] I am not at ease, nor am I quiet;
        I have no rest, but trouble comes.”
  \end{quote}
\subsection*{Notes}
\begin{itemize}
  \item{Recap: in Ch.1 of Job, we see that Job was the greatest man in the
  region (in terms of wealth, blessing). In Ch.2, we see that Job has quickly
  become the complete opposite, he lost all his possessions, and even his
  health. }
  \item{Today, we'll be looking at chapter 3.}
  \item{Song lyrics have a way of conveying our deepest emotions. In human
  history, there has been a lot of people who wrote emotional songs of lament
  when they felt like life was overwhelming. And those song lyrics usually
  are not to be taken literally. E.g, ``welcome to my life'' by simple plan
  lol.}
  \item{In the first two chapters, we see Job acknowledging God despite all
  that has happened to him. But in this chapter, we see Job's humanness and
  how his spirit is at a breaking point. As Job speaks, his words are filled
  with unbearable anguish and sorrow. Here, we see him wishing that he has
  never been born, and that the day where he was born be erased. He was
  wishing he would rather not be born, so he would avoid all the trouble that
  has befallen him. He also continued on and lamented why he did not just die
  at birth. To Job, the next best thing to never having been born was to have
  died at birth. Then to Job, the best thing after that was to just die now
  and be at peace from his pain and suffering.}
  \item{In this chapter, we see Job going down the spiral of despair. In
  Job's lament, we encounter a real beleiver who went through anguish, despair
  and utter desperation. It is something for us to grasp, that we ourselves,
  if we walk closely with Christ, may go through very deep darkness, deeper
  even eprhaps than if we had not walked faithfully in his footsteps.}
  \item{When we suffer, we may often demand answers from God. Or we sometimes
  try to blackmail God, saying:``if you don't remove my suffering, I will
  walk away from you''. All of that is foolish. \KH{I didn't manage to catch
  what else Ps Stanley was saying, so I extrapolate here: what we should do
  instead is to continue trusting in Jesus despite not hearing any answers.
  But that is very difficult to do, and it is ok to lament like what Job here
  while trusting Jesus. Though I don't know what that looks like in
  practice... }}
  \item{God's ways are really higher than our ways. God is sovereign, and
  though we might not understand the purpose of our suffering, we know that
  He does all things for our good and for His glory. \KH{Thus in our suffering,
  if we can glorify him, then that would fulfill the purpose of our
  suffering.}}
  \item{When we fast forward to the last few chapters of Job, we see that God did not give Job any answers for his suffering. The key to wisdom in life is not to know why things happen to us. The key is to know God who knows why things happen to us.}
  \item{And for us Christians on this side of the cross, we can look to Jesus as our example. He was made like us and He suffered everything on our behalf, so that He can relate with us in our suffering. He knows us, He is not foreign to our suffering, \KH{He cares so much to die on the cross for us so that we can have eternal life and to be free from suffering in the new creation.}}

\end{itemize}