\setcounter{figure}{0}

\section{3th March 2024: Less than what my guilt deserves?}
\subsection*{Text: Excerpts from Job 11,13:15-24,14:13-22}
  \begin{quote}
    [1] Then Zophar the Naamathite answered and said:

    [2] “Should a multitude of words go unanswered,
        and a man full of talk be judged right?
    [3] Should your babble silence men,
        and when you mock, shall no one shame you?
    [4] For you say, ‘My doctrine is pure,
        and I am clean in God’s eyes.’
    [5] But oh, that God would speak
        and open his lips to you,
    [6] and that he would tell you the secrets of wisdom!
        For he is manifold in understanding.
    Know then that God exacts of you less than your guilt deserves.

    [7] “Can you find out the deep things of God?
        Can you find out the limit of the Almighty?
    [8] It is higher than heaven—what can you do?
        Deeper than Sheol—what can you know?
    [9] Its measure is longer than the earth
        and broader than the sea.
    [10] If he passes through and imprisons
        and summons the court, who can turn him back?
    [11] For he knows worthless men;
        when he sees iniquity, will he not consider it?
    [12] But a stupid man will get understanding
        when a wild donkey’s colt is born a man!

    [13] “If you prepare your heart,
        you will stretch out your hands toward him.
    [14] If iniquity is in your hand, put it far away,
        and let not injustice dwell in your tents.
    [15] Surely then you will lift up your face without blemish;
        you will be secure and will not fear.
    [16] You will forget your misery;
        you will remember it as waters that have passed away.
    [17] And your life will be brighter than the noonday;
        its darkness will be like the morning.
    [18] And you will feel secure, because there is hope;
        you will look around and take your rest in security.
    [19] You will lie down, and none will make you afraid;
        many will court your favor.
    [20] But the eyes of the wicked will fail;
        all way of escape will be lost to them,
        and their hope is to breathe their last.”

    [15] Though he slay me, I will hope in him;
        yet I will argue my ways to his face.
    [16] This will be my salvation,
        that the godless shall not come before him.

    [20] Only grant me two things,
        then I will not hide myself from your face:
    [21] withdraw your hand far from me,
        and let not dread of you terrify me.
    [22] Then call, and I will answer;
        or let me speak, and you reply to me.
    [23] How many are my iniquities and my sins?
        Make me know my transgression and my sin.
    [24] Why do you hide your face
        and count me as your enemy?

    [13] Oh that you would hide me in Sheol,
        that you would conceal me until your wrath be past,
        that you would appoint me a set time, and remember me!
    [14] If a man dies, shall he live again?
        All the days of my service I would wait,
        till my renewal should come.
    [15] You would call, and I would answer you;
        you would long for the work of your hands.
    [16] For then you would number my steps;
        you would not keep watch over my sin;
    [17] my transgression would be sealed up in a bag,
        and you would cover over my iniquity.

    [18] “But the mountain falls and crumbles away,
        and the rock is removed from its place;
    [19] the waters wear away the stones;
        the torrents wash away the soil of the earth;
        so you destroy the hope of man.
    [20] You prevail forever against him, and he passes;
        you change his countenance, and send him away.
    [21] His sons come to honor, and he does not know it;
        they are brought low, and he perceives it not.
    [22] He feels only the pain of his own body,
        and he mourns only for himself.”
  \end{quote}
\subsection*{Notes}
\begin{itemize}
  \item{What happens when the person you think can answer your prayers, you feel has also turned against you? This is what Job felt, when his friends imposed their simplistic views of God on Job, which added on to his feeling being cut off from God.}
  \item{How can believers come before God with our honest struggling without feeling that they have sinned against God?}
  \item{Three Ps for today:
  \begin{enumerate}
    \item{Job feeling proscecuted by his friends.}
    \item{Job's plea}
    \item{Job's pitiful plight}
  \end{enumerate}}
  \item{In Job 11, we see Zophar speaking to Job quite harshly. Previously
  Job would have protested his suffering by claiming his innocence. But here,
  Zophar (who is supposed to be Job's friend), is being like a proscecutor
  towards Job. The irony here is that here, Zophar is extolling God, but at
  the same time he is insulting Job. Here, we see here that when we comfort
  someone, we first need to deal with our pride. We should not treat someone
  else like a ``lesser Christian'', and to be like an armchair theologian. We
  should listen and discern first, rather than speaking.}
  \item{Here, God also has not revealed anything to Zophar, yet Zophar is
  claiming to speak on God's behalf when he said that ``God exacts less than
  your guilt deserves''. Here, we must avoid misrepresenting God due to our
  presumption. God being God, might be doing many things through a single
  event. We should not presume on what we think God is doing in someone's life when he suffers (unless God has spoken clearly to us). }
  \item{As Paul said: love is patient. We need to have patience when
  comforting someone in their suffering, we need to be patient to listen
  first before speaking.}
  \item{Here, Zophar's solution to Job's suffering is also overly simplistic.
  From verses 14-16 of chapter 11, we see that essentially Zophar is saying:
  ``if you are more obedient/more faithful, your life will be smooth''. Here,
  the emphasis is on what Job must do to make his life better, rather than on
  God's mercy/grace. Here, Zophar is placing a burden on Job that is too
  heavy to bear, that our suffering or lack thereof is directly correlated to
  our sin. This is a false gospel. We do need to put the focus on God's
  forgiveness and mercy and grace, that His forgiveness is much bigger than
  our sin, when we repent. This frees us from directly correlating our sin
  with the suffering in our lives. }
  \item{In chapter 13, we see Job's response. Job's emotions as portrayed
  here are very complex and multi-faceted. Here, we see Job struggling with
  God but also longing for God. If we focus on verses 15-16, we see that Job
  saying that his hope is only God \KH{oversimplification here, didn't catch fully what Ps Edwin said}. Job is fearful of being too impudent in his approach to God, but he approaches God nonetheless, since he realises that he can't turn to anyone else.}
  \item{Also from verses 20-24 of chapter 13, we see that Job really just wants to be with God. He didn't ask for a restoration of his posessions, he just wants to feel like God is not his enemy.}
  \item{Here, we see how Job is grappling between what he knows of God and how he feels towards God. }
  \item{Unlike Job, we now know about the atoning sacrifice of Jesus. As
  Hebrews says, we can approach the throne of grace with confidence in our
  time of need. So we should not be fearful to go to God in our times of
  suffering, we should not be fearful of sharing our complex emotions with
  God just like what Job does here.}
  \item{In chapter 14, Job is wondering about the fate of all mankind, due to
  sin and death (which is the consequence of sin). In v13-14, Job hopes for
  God to hide him in Sheol for a time until His wrath is past, and then God
  will restore Job. Job here is groaning for a day where he is sure that he
  will not be treated as an enemy, when all will be well (v15-17). This is
  where Job is at, he is longing for a day where there is no more gap between
  God and himself.}
  \item{In Romans 8, Paul says that not only the whole creation, but even us
  Christians, we groan inwardly while we wait for the final day. We are aware
  that things on earth is not fully right. There is still sin and death. We
  long for something, but we have a hope for something without sin and death.
  Compared to Job, for us who live after the cross, we are more privileged.
  Here, our hope is based on the cross, where God has decisively destroyed
  sin and death. We have the hope that God's justice will always prevail,
  though we can't see it yet.}
  \item{In the sermon on the mount, Jesus says: ``blessed are those who mourn, for they will be comforted''. The comforting is not only at the last days, even in our current broken world, when we mourn and groan about the fallenness of creation, we still have a certain communion with God.}
\end{itemize}