\setcounter{figure}{0}

\section{31st March 2024: Christ is risen!}
\subsection*{Text: Romans 6:1-11}
  \begin{quote}
    [1] What shall we say then? Are we to continue in sin that grace may abound? [2] By no means! How can we who died to sin still live in it? [3] Do you not know that all of us who have been baptized into Christ Jesus were baptized into his death? [4] We were buried therefore with him by baptism into death, in order that, just as Christ was raised from the dead by the glory of the Father, we too might walk in newness of life.

    [5] For if we have been united with him in a death like his, we shall certainly be united with him in a resurrection like his. [6] We know that our old self was crucified with him in order that the body of sin might be brought to nothing, so that we would no longer be enslaved to sin. [7] For one who has died has been set free from sin. [8] Now if we have died with Christ, we believe that we will also live with him. [9] We know that Christ, being raised from the dead, will never die again; death no longer has dominion over him. [10] For the death he died he died to sin, once for all, but the life he lives he lives to God. [11] So you also must consider yourselves dead to sin and alive to God in Christ Jesus.
  \end{quote}
\subsection*{Notes}
\begin{itemize}
  \item{Every sunday, we remember Jesus’ death and celebrate His resurrection. The death and resurrection of Jesus is the bedrock, the crux of our Christian faith. Without the death and resurrection of Jesus, there is no hope, no forgiveness, etc.}
  \item{Three points for today, based on viewing the resurrection from the POV of the past, present and future.
  \begin{itemize}
    \item{Did the resurrection happen? (past)}
    \item{How does the resurrection affect me today (present)}
    \item{How does the resurrection affect me in the future (future)}
  \end{itemize}}
  \item{Firstly, w.r.t the past: did the resurrection happen?
  \begin{enumerate}
    \item{First, we see that women were the first witnesses to the empty tomb. In the first century, a woman’s testimony was not legally binding. Thus, in the past, it would be embarrassing for the entire account to be based on a woman’s testimony. Thus, if the empty tomb account was fabricated, then the ancient author would not have chosen women as the witnesses. This means that the author was more concerned about truth than avoiding embarrassment.}
    \item{Secondly, we see that the disciples were willing to die for their faith in the risen Jesus. If the account was fabricated, they wouldn’t suffer for that for 40 years. This means that the disciples believed what they were preaching, i.e they believed that Jesus’ resurrection was true.}
    \item{Thirdly, people who were still alive could refute the resurrection if it wasn’t true (since accounts of the resurrection were around as early as 50AD).}
  \end{enumerate}}
  \item{In the future: Christ died for us (he gave his life willingly for us), and He was raised for us, and He will never die, death having no more dominion over Him. 
  \begin{enumerate}
    \item{Comparing Jesus and Lazarus, we see that Lazarus will die again but Jesus will never die again.}
    \item{Jesus’ resurrection to an eternal life guarantees our own resurrection to an eternal life, though we will eventually die.}
    \item{And Jesus’ eternal life means that He will come again.}
    \item{One day, we will die, but we will not need to be afraid of death. And we will not grieve brothers and sisters in Christ who have died before us, since we will see them again.}
  \end{enumerate}}
  \item{In the present: 
  \begin{enumerate}
    \item{We see that we have been united with Christ in a death like his through baptism (v3,4,6,8)}
    \item{We have also been united with Christ in a resurrection like his (v4,8)}
    \item{This means that our primary identity is in Christ. We are first Christians, then students/etc.}
    \item{This being in Christ is not just something abstract, it is an objective reality. When we have put our faith in Christ, then we are most definitely in Christ. }
    \item{When we are \textbf{in} Christ, then we are \textbf{with} Christ.}
    \item{Practical implications of being united with Christ in a death like his: Christ died to sin (v10), i.e he has taken our sin and died for our sin. Similarly, we too have died to sin. Consider the fourfold state of man (St. Augustine):
    \begin{itemize}
      \item{Adam before the fall: able to sin}
      \item{Adam after the fall: not able not to sin}
      \item{In Christ, on earth: Able not to sin}
      \item{In Christ, in heaven: Not able to sin}
    \end{itemize}
    Thus, for us in Christ now, we are freed from the dominion of sin, we are able not to sin! Compare this with our life before we put our faith in Christ. Though we will still sin, when we are in Christ, the Holy Spirit rebukes us when we sin and teaches us not to sin. And eventually, we will grow in sanctification, if we abide in Christ.}
    \item{The power to not sin also is part of how we have been united with Christ in a baptism like his. We have the freedom to live responsively to God, for his glory.}
    \item{Before we have the resurrected life, hearing the bible is lame. Now, we hear God’s Spirit speaking to us through the Bible and we obey His Word. Additionally, before, we don’t like to serve others, we don’t like to worship, etc. But now, we like all of these.}
    \item{If we don’t experience to some degree the above effects of being in Christ, then maybe we just like the idea of being Christian but we aren’t Christian. We then need to confess our sins and truly believe in Jesus for the forgiveness of sin.}
  \end{enumerate}}
  \item{We must know all of these truths above, and know them not just as theoretical knowledge but as something that affects our behavior. 
  \begin{itemize}
    \item{We must know our primary identity with Christ}
    \item{We must know that we are no longer dead in sin, but we are dead to sin and alive to God. We must remember this especially when we are tempted.}
    \item{Not only are we in Christ, Christ is also in us, through His Holy Spirit.}
  \end{itemize}
  }
  \item{This eternal life that we are living right now prepares us for eternity with Jesus. This is the abundant life that Jesus is talking about. This life is purposeful and meaningful (lived for God), joyful (with God’s presence), and beautiful.}
  \item{And we can live this life because Christ is risen, and we have been raised with him.}
\end{itemize}