\setcounter{figure}{0}

\section{17th March 2024: Responding to suffering}
\subsection*{Text: Job 40-42}
  \begin{quote}
    [1] And the LORD said to Job:

    [2] “Shall a faultfinder contend with the Almighty?
        He who argues with God, let him answer it.”


      [3] Then Job answered the LORD and said:

    [4] “Behold, I am of small account; what shall I answer you?
        I lay my hand on my mouth.
    [5] I have spoken once, and I will not answer;
        twice, but I will proceed no further.”


      [6] Then the LORD answered Job out of the whirlwind and said:

    [7] “Dress for action like a man;
        I will question you, and you make it known to me.
    [8] Will you even put me in the wrong?
        Will you condemn me that you may be in the right?
    [9] Have you an arm like God,
        and can you thunder with a voice like his?


    [10] “Adorn yourself with majesty and dignity;
        clothe yourself with glory and splendor.
    [11] Pour out the overflowings of your anger,
        and look on everyone who is proud and abase him.
    [12] Look on everyone who is proud and bring him low
        and tread down the wicked where they stand.
    [13] Hide them all in the dust together;
        bind their faces in the world below.
    [14] Then will I also acknowledge to you
        that your own right hand can save you.


    [15] “Behold, Behemoth,
        which I made as I made you;
        he eats grass like an ox.
    [16] Behold, his strength in his loins,
        and his power in the muscles of his belly.
    [17] He makes his tail stiff like a cedar;
        the sinews of his thighs are knit together.
    [18] His bones are tubes of bronze,
        his limbs like bars of iron.


    [19] “He is the first of the works of God;
        let him who made him bring near his sword!
    [20] For the mountains yield food for him
        where all the wild beasts play.
    [21] Under the lotus plants he lies,
        in the shelter of the reeds and in the marsh.
    [22] For his shade the lotus trees cover him;
        the willows of the brook surround him.
    [23] Behold, if the river is turbulent he is not frightened;
        he is confident though Jordan rushes against his mouth.
    [24] Can one take him by his eyes,
        or pierce his nose with a snare?

    [1] “Can you draw out Leviathan with a fishhook
        or press down his tongue with a cord?
    [2] Can you put a rope in his nose
        or pierce his jaw with a hook?
    [3] Will he make many pleas to you?
        Will he speak to you soft words?
    [4] Will he make a covenant with you
        to take him for your servant forever?
    [5] Will you play with him as with a bird,
        or will you put him on a leash for your girls?
    [6] Will traders bargain over him?
        Will they divide him up among the merchants?
    [7] Can you fill his skin with harpoons
        or his head with fishing spears?
    [8] Lay your hands on him;
        remember the battle—you will not do it again!
    [9] Behold, the hope of a man is false;
        he is laid low even at the sight of him.
    [10] No one is so fierce that he dares to stir him up.
        Who then is he who can stand before me?
    [11] Who has first given to me, that I should repay him?
        Whatever is under the whole heaven is mine.


    [12] “I will not keep silence concerning his limbs,
        or his mighty strength, or his goodly frame.
    [13] Who can strip off his outer garment?
        Who would come near him with a bridle?
    [14] Who can open the doors of his face?
        Around his teeth is terror.
    [15] His back is made of rows of shields,
        shut up closely as with a seal.
    [16] One is so near to another
        that no air can come between them.
    [17] They are joined one to another;
        they clasp each other and cannot be separated.
    [18] His sneezings flash forth light,
        and his eyes are like the eyelids of the dawn.
    [19] Out of his mouth go flaming torches;
        sparks of fire leap forth.
    [20] Out of his nostrils comes forth smoke,
        as from a boiling pot and burning rushes.
    [21] His breath kindles coals,
        and a flame comes forth from his mouth.
    [22] In his neck abides strength,
        and terror dances before him.
    [23] The folds of his flesh stick together,
        firmly cast on him and immovable.
    [24] His heart is hard as a stone,
        hard as the lower millstone.
    [25] When he raises himself up, the mighty are afraid;
        at the crashing they are beside themselves.
    [26] Though the sword reaches him, it does not avail,
        nor the spear, the dart, or the javelin.
    [27] He counts iron as straw,
        and bronze as rotten wood.
    [28] The arrow cannot make him flee;
        for him, sling stones are turned to stubble.
    [29] Clubs are counted as stubble;
        he laughs at the rattle of javelins.
    [30] His underparts are like sharp potsherds;
        he spreads himself like a threshing sledge on the mire.
    [31] He makes the deep boil like a pot;
        he makes the sea like a pot of ointment.
    [32] Behind him he leaves a shining wake;
        one would think the deep to be white-haired.
    [33] On earth there is not his like,
        a creature without fear.
    [34] He sees everything that is high;
        he is king over all the sons of pride.”

    [1] Then Job answered the LORD and said:

    [2] “I know that you can do all things,
        and that no purpose of yours can be thwarted.
    [3] ‘Who is this that hides counsel without knowledge?’
    Therefore I have uttered what I did not understand,
        things too wonderful for me, which I did not know.
    [4] ‘Hear, and I will speak;
        I will question you, and you make it known to me.’
    [5] I had heard of you by the hearing of the ear,
        but now my eye sees you;
    [6] therefore I despise myself,
        and repent in dust and ashes.”


    [7] After the LORD had spoken these words to Job, the LORD said to Eliphaz the Temanite: “My anger burns against you and against your two friends, for you have not spoken of me what is right, as my servant Job has. [8] Now therefore take seven bulls and seven rams and go to my servant Job and offer up a burnt offering for yourselves. And my servant Job shall pray for you, for I will accept his prayer not to deal with you according to your folly. For you have not spoken of me what is right, as my servant Job has.” [9] So Eliphaz the Temanite and Bildad the Shuhite and Zophar the Naamathite went and did what the LORD had told them, and the LORD accepted Job’s prayer.

    [10] And the LORD restored the fortunes of Job, when he had prayed for his friends. And the LORD gave Job twice as much as he had before. [11] Then came to him all his brothers and sisters and all who had known him before, and ate bread with him in his house. And they showed him sympathy and comforted him for all the evil that the LORD had brought upon him. And each of them gave him a piece of money and a ring of gold.

    [12] And the LORD blessed the latter days of Job more than his beginning. And he had 14,000 sheep, 6,000 camels, 1,000 yoke of oxen, and 1,000 female donkeys. [13] He had also seven sons and three daughters. [14] And he called the name of the first daughter Jemimah, and the name of the second Keziah, and the name of the third Keren-happuch. [15] And in all the land there were no women so beautiful as Job’s daughters. And their father gave them an inheritance among their brothers. [16] And after this Job lived 140 years, and saw his sons, and his sons’ sons, four generations. [17] And Job died, an old man, and full of days.
  \end{quote}
\subsection*{Notes}
\begin{itemize}
  \item{Dust and ashes - human suffering and weakness. In Job's final
  response in chapter 42, here we see Job saying that he ``repents in dust
  and ashes''. Job here is acknowledging his own weakness before God. A
  similar phrase is used by Abraham when he was interceding for Sodom and
  Gomorrah.}
  \item{How do we relate to God as dust and ashes? We do so by speaking
  rightly to God. Here in v7,8 of chapter 42, we see that Eliphaz was rebuked
  for not speaking rightly to God. This is despite him and his friends saying
  things that were quite biblical, the idea that sin leads to suffering. So
  what is wrong about what Eliphaz said? The idea is that they were saying
  right things but they weren't speaking rightly to God. To speak rightly to
  God, we need to be \textbf{honest}. \KH{It is not merely about having right
  theology, since the devil has right theology. It is about having a right
  relationship with God, and if this relationship with God is real, it must
  allow for honesty}. We can speak to God honestly in lamentation (chs 3-39)
  and also in intercession.}
  \item{The Leviathan is under God's control.}
  \item{\KH{Tbh this sermon is abit confusing, I'm lost already. I also think
  the original ESV translation makes more sense to me than Maggie Low's
  translation. Maggie's sermon is based solely on her translation apart from
  the ESV translation. I guess I have to check a commentary on Job.}}
\end{itemize}