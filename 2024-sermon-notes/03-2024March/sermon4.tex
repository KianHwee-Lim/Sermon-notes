\setcounter{figure}{0}

\section{24th March 2024: The triumphal entry}
\subsection*{Text: Mark 11:1-11}
  \begin{quote}
    [1] Now when they drew near to Jerusalem, to Bethphage and Bethany, at the Mount of Olives, Jesus sent two of his disciples [2] and said to them, “Go into the village in front of you, and immediately as you enter it you will find a colt tied, on which no one has ever sat. Untie it and bring it. [3] If anyone says to you, ‘Why are you doing this?’ say, ‘The Lord has need of it and will send it back here immediately.’” [4] And they went away and found a colt tied at a door outside in the street, and they untied it. [5] And some of those standing there said to them, “What are you doing, untying the colt?” [6] And they told them what Jesus had said, and they let them go. [7] And they brought the colt to Jesus and threw their cloaks on it, and he sat on it. [8] And many spread their cloaks on the road, and others spread leafy branches that they had cut from the fields. [9] And those who went before and those who followed were shouting, “Hosanna! Blessed is he who comes in the name of the Lord! [10] Blessed is the coming kingdom of our father David! Hosanna in the highest!”

    [11] And he entered Jerusalem and went into the temple. And when he had looked around at everything, as it was already late, he went out to Bethany with the twelve.
  \end{quote}
\subsection*{Notes}
\begin{itemize}
  \item{Three pints for today:
  \begin{enumerate}
    \item{First point: symbol, what Jesus did was symbolic}
    \item{Second point: scripture, what Jesus did pointed to some parts of scripture}
    \item{Third point: salience, what Jesus did was very notable/significant}
  \end{enumerate}}
  \item{Jesus’ action here was symbolic. He wasn’t just looking for a ride into Jerusalem. When he was looking for a donkey, he was shouting a message to his audience. And in this part of Mark, was the first time Jesus was breaking his usual pattern of being “low key”. This part here marks the transition between Jesus’ public ministry and now His going to His passion.}
  \item{In Jesus’ day, it was a common cultural thing that people were to walk to Jerusalem,  and not ride into Jerusalem. This is because of the culturally perceived holiness of the city. But Jesus rode into Jerusalem, which was very provocative. }
  \item{Also, even though Jesus’ action was very symbolic, he didn’t say anything at all. Usually in the past, the prophets will do something symbolic and then provide a commentary. Here, Jesus provides no commentary at all. Why? Because here, Jesus is counting on scripture to fill in the gaps, that both He and the audience knows.}
  \item{The crowd here says: “blessed is he who comes in the name of the LORD”. This is a quotation from Psalm 118:9. This is part of the liturgy of welcome in those days that the Jerusalem Jews will say to any Jewish pilgrim who comes to Jerusalem. But the crowd also says: “blessed is the coming kingdom of our father david”. So here, the crowd identifies Jesus as the messiah, the Son of David. But why did the crowd do this?}
  \item{Usually, when people ride into a city in triumph, they will ride a horse. E.g, the Roman generals who conquered a city will ride in on their war horses. But here, Jesus rides a donkey. }
  \item{The crowd understood what Jesus did because of Zechariah 9:9-11. The context of that passage is that the messianic king will defeat Israel’s enemies (see chapter 9). But even though the messianic king is victorious here, the focus in the zechariah passage is on the character of the king, that the king is humble and lowly. Though the messianic king in Zechariah 9:9-11 will defeat his enemies, the focus is on the peace that will come (v10). War is made not for its own sake, but the war is for the sake of peace. The messianic king, humble and lowly, defeats his enemies to come to bring peace.}
  \item{When there is such fulfilment of prophecy, we must remember that God remembers His promises and our prayers. He will remember His promises at the right time. }
  \item{What are the salient points that we can learn from Jesus’ action today? First, we see that Jesus is the humble king of Zechariah 9:9-11. Jesus here understands what it means to be afflicted and deprived, though He is king. We must learn from His example. }
  \item{Secondly, we see that Jesus was coming as an inspecting king. In v11, we see Jesus going into the temple, looking around, then going out. Right after, we have the cursing of the fig tree and the cleansing of the temple. Thus, Jesus is not only the humble king, but He is also the inspecting king. Just as Jesus comes to inspect the temple, we must inspect our life. And here, we must remember the crowd. There’s a difference between crowd faith and sincere, personal faith. The crowd faith here is very flimsy, because everybody just “monkey see monkey do” instead of truly believing. They just “herd instinct”. We know this because 5 days later in the story, the crowd goes from “hosanna” to “crucify him”. Sincere personal faith is happy to praise God with the crowd, but is also happy to praise God alone even when the world is against God.}
  \item{Thirdly, if we fast forward abit into the story, we see that the inspecting king is also the sacrificial and loving king. Our king Jesus here inspects Jerusalem, sees where they fall short, and then dies on the cross to make up for they falls short, and rises from the dead to give them the power to live new lives. And this is what happens to us too; jesus sees where we fall short but He also gives himself for us. Jesus is judge, but Jesus is also loving.}
\end{itemize}