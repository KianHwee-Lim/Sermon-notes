\setcounter{figure}{0}

\section{7th January 2024: I have had enough}
\subsection*{Text: Isaiah 1:11-20}
  \begin{quote}
    [11] “What to me is the multitude of your sacrifices?
        says the LORD;
    I have had enough of burnt offerings of rams
        and the fat of well-fed beasts;
    I do not delight in the blood of bulls,
        or of lambs, or of goats.


    [12] “When you come to appear before me,
        who has required of you
        this trampling of my courts?
    [13] Bring no more vain offerings;
        incense is an abomination to me.
    New moon and Sabbath and the calling of convocations—
        I cannot endure iniquity and solemn assembly.
    [14] Your new moons and your appointed feasts
        my soul hates;
    they have become a burden to me;
        I am weary of bearing them.
    [15] When you spread out your hands,
        I will hide my eyes from you;
    even though you make many prayers,
        I will not listen;
        your hands are full of blood.
    [16] Wash yourselves; make yourselves clean;
        remove the evil of your deeds from before my eyes;
    cease to do evil,
    [17]     learn to do good;
    seek justice,
        correct oppression;
    bring justice to the fatherless,
        plead the widow’s cause.


    [18] “Come now, let us reason together, says the LORD:
    though your sins are like scarlet,
        they shall be as white as snow;
    though they are red like crimson,
        they shall become like wool.
    [19] If you are willing and obedient,
        you shall eat the good of the land;
    [20] but if you refuse and rebel,
        you shall be eaten by the sword;
        for the mouth of the LORD has spoken.”
  \end{quote}
\subsection*{Notes}
\begin{itemize}
  \item{In our text, we see that Judah’s offerings in Isaiah’s time are prim and proper. They offer up all the required offerings in the ceremonial law, and they offer them abundantly.}
  \item{But somehow, God is rejecting all of their worship. See v11-15. why?}
  \item{This is because Judah, though they appear prim and proper in their worship, they are actually rebellious and rotten. There is much evil, oppression, and failure to care for the marginalised.}
  \item{This shows us that to God, obedience is more important than sacrifice. And obedience must be whole. It is pointless to obey all the ceremonial laws but ignore all of the laws to fight oppression in the land. What Judah was doing was trying to seek a Holy and Just God through their sacrifices even though their hands are full of sin.}
  \item{This is why God tells them to repent of their sin first (v16-20). If they repent, they’ll be forgiven, but if they don’t repent, they will die by the sword. And this is what happened to Judah.}
  \item{First sunday of the year is always a sermon about worship. This makes sense because there are 51 more sundays in a year where we come to worship. Thus, from our text, we should find out what makes our worship acceptable to God.}
  \item{Firstly, the form of worship is important. A worship service has to be liturgical and participative (liturgy means works of the people). That is, during Sunday worship, we are not just there to have an experience, but we are there to render our works to God. In the past, in time, it would be to bring their burnt offerings to God. And it would be to listen intently to the teaching of the Levites. In our time, we are to participate in the service by singing to each other, to listen intently to the Spirit of God speaking to us through the pastor’s message and the songs. In our form of worship, we must also give of our best. To Judah’s credit, she did give of her best here, by bringing fat animals to sacrifice to God. We can learn this from the ancient Israelites. Do we come for sunday service in a sleep deprived state? For example. Or do we come fully energetic and ready to worship God.}
  \item{But apart from the form of worship, what is more important is the quality of our lives (quality as in from God’s POV, ie our obedience). We must not be just “sunday Christians”, because that would be religious hypocrisy just like Judah here. As James says, it should not be that the same mouth come blessings to God and curses to man. How can we say that love God when we do so much evil to others? Because loving God and loving our neighbour cannot be separated. Our whole lives is to be offered as a worship to God, that is the more important worship.}
  \item{But its not that we should stay away from church unless we’re perfect. \KH{(My analogy: we can think of church as a alcohol rehabilitation centre. A person who is happily addicted to drunkenness and thinks alcohol the best thing ever right, no point going to a alcohol rehabilitation centre. That is hypocrisy. But the alcohol rehabiliation centre is open to people who want to quit alcohol but who struggle with quitting and who perhaps even from time to time take a bit of alcohol out of the weakness of their flesh.) The key is repentance and a desire for obedience. When we have those, and we come to church, God will honour our desire for obedience and cause us to obey Him through His Holy Spirit working in us when we come to church to worship.}}
  \item{Form of worship and obedience is important, but we cannot use it to manipulate God. It is not that when we serve God in church, then we expect God will bless us. \KH{True love of God obeys God not because of the gifts that God gives, but because of gratitude for God’s acts of creating us, redeeming us, and providing for us which reflect His glory and beauty, which makes him worthy of worship just because He is God.}}
  \item{Here, we have a call to repentance. Repentance is key, and repentance is the reason why we gather for public worship. See Hebrews 10: 19-25. Through Jesus’ work, His sacrifice has cleansed us for all time, and His blood perfects all our imperfect (because of sin) service to God. As long as we believe in Jesus and repent, when we gather together, His Holy Spirit works through the service to transform us even as we offer up our voices in worship of God.}
\end{itemize}