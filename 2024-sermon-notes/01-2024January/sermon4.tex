\setcounter{figure}{0}

\section{28th January 2024: Will I still worship God}
\subsection*{Text: Job 1:2-10}
  \begin{quote}
    [1] There was a man in the land of Uz whose name was Job, and that man was blameless and upright, one who feared God and turned away from evil. [2] There were born to him seven sons and three daughters. [3] He possessed 7,000 sheep, 3,000 camels, 500 yoke of oxen, and 500 female donkeys, and very many servants, so that this man was the greatest of all the people of the east. [4] His sons used to go and hold a feast in the house of each one on his day, and they would send and invite their three sisters to eat and drink with them. [5] And when the days of the feast had run their course, Job would send and consecrate them, and he would rise early in the morning and offer burnt offerings according to the number of them all. For Job said, “It may be that my children have sinned, and cursed God in their hearts.” Thus Job did continually.

    [6] Now there was a day when the sons of God came to present themselves before the LORD, and Satan also came among them. [7] The LORD said to Satan, “From where have you come?” Satan answered the LORD and said, “From going to and fro on the earth, and from walking up and down on it.” [8] And the LORD said to Satan, “Have you considered my servant Job, that there is none like him on the earth, a blameless and upright man, who fears God and turns away from evil?” [9] Then Satan answered the LORD and said, “Does Job fear God for no reason? [10] Have you not put a hedge around him and his house and all that he has, on every side? You have blessed the work of his hands, and his possessions have increased in the land. [11] But stretch out your hand and touch all that he has, and he will curse you to your face.” [12] And the LORD said to Satan, “Behold, all that he has is in your hand. Only against him do not stretch out your hand.” So Satan went out from the presence of the LORD.

    [13] Now there was a day when his sons and daughters were eating and drinking wine in their oldest brother’s house, [14] and there came a messenger to Job and said, “The oxen were plowing and the donkeys feeding beside them, [15] and the Sabeans fell upon them and took them and struck down the servants with the edge of the sword, and I alone have escaped to tell you.” [16] While he was yet speaking, there came another and said, “The fire of God fell from heaven and burned up the sheep and the servants and consumed them, and I alone have escaped to tell you.” [17] While he was yet speaking, there came another and said, “The Chaldeans formed three groups and made a raid on the camels and took them and struck down the servants with the edge of the sword, and I alone have escaped to tell you.” [18] While he was yet speaking, there came another and said, “Your sons and daughters were eating and drinking wine in their oldest brother’s house, [19] and behold, a great wind came across the wilderness and struck the four corners of the house, and it fell upon the young people, and they are dead, and I alone have escaped to tell you.”

    [20] Then Job arose and tore his robe and shaved his head and fell on the ground and worshiped. [21] And he said, “Naked I came from my mother’s womb, and naked shall I return. The LORD gave, and the LORD has taken away; blessed be the name of the LORD.”

    [22] In all this Job did not sin or charge God with wrong.

    [1] Again there was a day when the sons of God came to present themselves before the LORD, and Satan also came among them to present himself before the LORD. [2] And the LORD said to Satan, “From where have you come?” Satan answered the LORD and said, “From going to and fro on the earth, and from walking up and down on it.” [3] And the LORD said to Satan, “Have you considered my servant Job, that there is none like him on the earth, a blameless and upright man, who fears God and turns away from evil? He still holds fast his integrity, although you incited me against him to destroy him without reason.” [4] Then Satan answered the LORD and said, “Skin for skin! All that a man has he will give for his life. [5] But stretch out your hand and touch his bone and his flesh, and he will curse you to your face.” [6] And the LORD said to Satan, “Behold, he is in your hand; only spare his life.”

    [7] So Satan went out from the presence of the LORD and struck Job with loathsome sores from the sole of his foot to the crown of his head. [8] And he took a piece of broken pottery with which to scrape himself while he sat in the ashes.

    [9] Then his wife said to him, “Do you still hold fast your integrity? Curse God and die.” [10] But he said to her, “You speak as one of the foolish women would speak. Shall we receive good from God, and shall we not receive evil?” In all this Job did not sin with his lips.
  \end{quote}
\subsection*{Notes}
\begin{itemize}
  \item{None of us can avoid suffering in this current world. Some people might suffer more than others, but everyone will suffer.} 
  \item{Throughout history, there were many views on suffering. E.g: “i am suffering because i must have done something wrong”. This is quite prevalent among folk religion, but even some christians subscribe to this. E.g, “i need to pray the right prayers/do more good works”. }
  \item{One thing we must take away today: suffering is not always to be equated with God’s punishment! \KH{It is true that sometimes suffering is God's discipline in our lives. God sometimes permits us to suffer when we commit sins just to show us the harmful effects of our sin on ourselves and on our r/s with Him. But we need to be careful when interpreting our suffering in this manner, because this is not always the case. Job's suffering is not God's discipline in his life, for example.}}
  \item{Other views on suffering are: “I am suffering because i was at the wrong place at the wrong time, etc”. i.e, these people dont believe in a higher power, everything is random, and suffering comes about just because one is “suay”. These people will try to minimise the randomness as much as possible.}
  \item{The commonality between these two views is that people are trying to take control of their lives. The first is trying to take control of theor lives by manipulating God who they believe in. The sexond is trying to take control of their lives by controlling all the factors that they can. This constant need for control of one’s life leads to restlessness.}
  \item{What does Job tell us about suffering? We have a series of 7 sermons on Job. Today is the first.}
  \item{Firstly, from our text, we see that God knows Job perfectly. And God knows that Job fears Him. Its not that Job was sinless, but he was repentant, i.e v8, he “turned away from evil”. }
  \item{Next, from our text, we see that God blesses the work of Job’s hands. We see that Job is wealthy with a very loving family for example.}
  \item{Skipping forward, we see that Job was to suffer a lot without hearing anything from God. But we are taken behind the scenes to see why Job suffers.}
  \item{We see that Job is suffering because Satan acts as if he knows better than God about Job. Satan lives up to his name here, he’s an adversary, an accuser. So he accused Job before God of worshipping God only out of convenience. Satan here wanted to destroy Job’s relationship with God, he wanted to test Job. He wanted to make it such that Job will blaspheme God and then God, in his justice, will have to judge Job. }
  \item{Satan also sought to accuse God and deny him of his glory. He was telling God that in and of Himself, He is not worthy of worship, and that He is only worthy of worship because of the good things he gives. }
  \item{But in the end, we see that Satan unwittingly ends up proving himself wrong and magnifying God’s worthiness. We see here that the evil and suffering that Job suffers, instead of destroying both Job’s r/s with God and also denying God the glory, in the end, Job’s character is vindicated and God is glorified by Job’s true worship.}
  \item{When we suffer today, we too must guard against satan’s lies and accusations. One way people buy into the lies of satan is that we believe that we are not good enough to be loved and blessed by God. So this is exactly the “works based” thinking that we Christians fall into. Then in the end we think that we deserve blessings just because we are serving in church. }
  \item{The book of Job does leave a lot of questions about suffering unanswered. For example, we dont know why God doesn’t just ignore satan. But what it reveals is sufficient for us to live faithfully. For example, we know that suffering ultimately for our good, and not for our harm. We know that we must not worship God just because of the good things he gives, our r/s with God must not be transactional. And we know that faithful living in times of suffering ultimately leads to glorifying God. }
  \item{What can we learn from the details of Job’s response? First, he actually turned to God to pour out his grief. V20 said he “fell on the ground and worshipped” after shaving his head and tearing his robe. And v21 says that Job did not sin or charge God with wrong. This shows us that it is not sinful to go to God in our grief. In the end, when Job’s wife told Job to curse God and die, Jon didnt.}
  \item{Second, he actually doesn’t have any sense of righteous pride or entitlement. He realised that all that he had came from God. “Naked i came, naked shall I return”; Job came into the world naked, and Job realised that the things he had was pure grace, purely God’s blessing. }
  \item{Job can trust God to act with divine sovereignty and justice. We might not understand why we are suffering, and we cannot understand why we are suffering. God’s ways are really higher than our ways, there could be a million things He is trying to accomplish in our lives and in the lives of those around us. But we know also from the rest of the Bible that God hates it when His righteous servants suffer. \KH{“Precious in the eyes of the Lord is the death of His saints”, Psalm 116:15}. God is not the source of evil, but God in His sovereignty is the one who puts limits on the suffering. Here, God is the one who told satan not to take Job’s life, and Job had to obey.
  \begin{itemize}
    \item{But why did God not just ignore Satan? Again, this is a question that is left unanswered.}
  \end{itemize}
  }
  \item{Here, Job’s innocent suffering, in some sense, foreshadowed Jesus’ innocent suffering. Jesus was a righteous man, sinless, truly fearing the Father. But He went to the cross willingly to demonstrate that God is worthy of worship and that God truly loves His creation. He had no lack, its not like He had to obey to be loved, but He obeyed anyway. }
  \item{We should not think: “Jesus suffered so we dont need to suffer”, but rather, “we suffer just as Jesus suffered so that we can grow to be more like Jesus, and glorify God just as Jesus glorified God”.}
  \item{In the end, there are no easy answers for suffering. It sucks. But the bible shows us that God loves us, even if the suffering seems to indicate otherwise, and also that God’s ways are higher than ours. And we see a glimpse of how God works from the example of Jesus.}
\end{itemize}