\setcounter{figure}{0}

\section{21st January 2024: Can I leave them alone?}
\subsection*{Text: Romans 10:9-17}
  \begin{quote}
    [9] because, if you confess with your mouth that Jesus is Lord and
    believe in your heart that God raised him from the dead, you will be
    saved. [10] For with the heart one believes and is justified, and with
    the mouth one confesses and is saved. [11] For the Scripture says,
    “Everyone who believes in him will not be put to shame.” [12] For there
    is no distinction between Jew and Greek; for the same Lord is Lord of
    all, bestowing his riches on all who call on him. [13] For “everyone who
    calls on the name of the Lord will be saved.”

    [14] How then will they call on him in whom they have not believed? And
    how are they to believe in him of whom they have never heard? And how are
    they to hear without someone preaching? [15] And how are they to preach
    unless they are sent? As it is written, “How beautiful are the feet of
    those who preach the good news!” [16] But they have not all obeyed the
    gospel. For Isaiah says, “Lord, who has believed what he has heard from
    us?” [17] So faith comes from hearing, and hearing through the word of
    Christ.
  \end{quote}
\subsection*{Notes}
\begin{itemize}
  \item{Some people overly idolise certain preachers and evangelists. But
  based on today's text, we see that evangelism is a work of everyone, not
  just for certain ``gifted preachers''.}
  \item{Today is the third of a series of three sermons on the calling of
  Christ's church. The topics in the sermon series are:
  \begin{enumerate}
    \item Worship (not just on Sunday, but with our whole lives). If we
    worship a holy God on Sunday, then shouldn't our lives be holy for the
    rest of the week? Of course we will fail from time to time, but God's
    forgiveness is freely given. And when we respond to God's freely-given
    forgiveness in repentance, we will grow in holiness through the Spirit,
    and our lives will be slowly transformed.
    \item To be a good neighbour to all around us, and especially to those in
    the church. We do this following Jesus' pattern, where He showed
    self-sacrificial love to us even though we were undeserving. And if all
    of us in church can do this, then we will be a loving and embracing
    community.
    \item \textbf{Engage the community, evangelise the lost}. This is our
    topic for today! We should desire to see people being freed from the
    condemnation, power of sin and death, and being gifted the gift of
    eternal life.
  \end{enumerate}}
  \item{In our text today, from v13-15, Paul works backwards from the
  statement: ``For everyone who calls upon the name of the Lord will be
  saved''. Paul says that in v13, then in v14-15, he explains the steps that
  must be taken for this to happen through a series of rhetorical questions.}
  \item{First, for people to call upon the name of the Lord, there must be
  other people who were \textbf{sent} to go to them. And who are the people
  who were sent? It is all of us! See the great commission as well as in John
  20:21-22. We are all sent forth by Jesus to continue Jesus' work of
  spreading the message of the kingdom of God. And we can do this through the
  power of the Holy Spirit. And as Paul says in v15, ``how beautiful are the
  feet of those who preach the good news'', quoting Isaiah 52. In the OT, it
  was the good news of being delivered from Babylonian captivity. But in the
  NT, the good news is better cause the deliverance is better; we are
  delivered from the power and guilt of sin! }
  \item{``Everyone who calls upon the name of the Lord will be saved''. This
  ties in with the idea that everyone is our neighbour. Just as we should not
  make distinctions when we do good works, we should also not make any
  distinctions when we do the ultimate good work of sharing the gospel with
  the people around us. Everyone needs to hear the gospel. Of course good
  works are important (see the example of the good Samaritan), and we should
  do them, but we should also pray for opportunities to speak the gospel when
  we do good works. And these opportunities will come when we are consistent
  with engaging our neighbours and doing good works to our neighbours. We
  just have to pray for these opportunities and pray for courage/obedience to
  take those opportunities!}
  \item{Secondly, for people to call upon the name of the Lord, they must
  hear the word that is preached. Though we are sent by God to preach, and
  when do we preach the gospel, not everyone will hear us with open ears. In
  a sense, this was Jesus' experience as well as the experience of the OT
  prophets. So we can be encouraged that when people don't fully hear the
  word that we speak, we are just following in Jesus' footsteps. Of course,
  we should make it such that our religious hypocrisy is not the reason why
  they are turned away. }
  \item{Thirdly, for people to call upon the name of the Lord, they must
  believe the word that is preached, and they must confess outwardly that
  Jesus is Lord. They must believe that God raised Jesus from the dead, and
  they must thus believe that Jesus (who is Lord) died for us. }
  \item{The above is the whole process through which someone can be lead to
  salvation. But as can be seen, the process begins with people who are sent!
  And continues when those who are sent are faithful to their calling. A
  church without evangelism is a contradiction, much like a fire without
  heat. If the church were to be zealous in their love for God and for
  neighbour, this love should infect those around us. We must be faithful to
  the great commission! We are called to be ``fishers of men'', not ``fishes
  in an aquarium''. The latter situation somewhat describes the case when
  growth in church membership is mostly due to church transfers... and when
  all the pastors in the country are just taking care of the same fish, and
  the church members just look pretty and are inactive.}
  \item{If we as a community can grow in all three points: worship, loving
  our neighbour, and evangelism, then we would have lived up to the call that
  Christ gives to His church.}
\end{itemize}













