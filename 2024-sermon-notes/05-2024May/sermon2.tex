\setcounter{figure}{0}

\section{12 May 2024: A believer's perspective of the world}
\subsection*{Text: James 4:13-5:12}
  \begin{quote}
    [13] Come now, you who say, “Today or tomorrow we will go into such and
    such a town and spend a year there and trade and make a profit”—[14] yet
    you do not know what tomorrow will bring. What is your life? For you are
    a mist that appears for a little time and then vanishes. [15] Instead you
    ought to say, “If the Lord wills, we will live and do this or that.” [16]
    As it is, you boast in your arrogance. All such boasting is evil. [17] So
    whoever knows the right thing to do and fails to do it, for him it is
    sin.
    
    [1] Come now, you rich, weep and howl for the miseries that are coming
    upon you. [2] Your riches have rotted and your garments are moth-eaten.
    [3] Your gold and silver have corroded, and their corrosion will be
    evidence against you and will eat your flesh like fire. You have laid up
    treasure in the last days. [4] Behold, the wages of the laborers who
    mowed your fields, which you kept back by fraud, are crying out against
    you, and the cries of the harvesters have reached the ears of the Lord of
    hosts. [5] You have lived on the earth in luxury and in self-indulgence.
    You have fattened your hearts in a day of slaughter. [6] You have
    condemned and murdered the righteous person. He does not resist you.
    
    [7] Be patient, therefore, brothers, until the coming of the Lord. See
    how the farmer waits for the precious fruit of the earth, being patient
    about it, until it receives the early and the late rains. [8] You also,
    be patient. Establish your hearts, for the coming of the Lord is at hand.
    [9] Do not grumble against one another, brothers, so that you may not be
    judged; behold, the Judge is standing at the door. [10] As an example of
    suffering and patience, brothers, take the prophets who spoke in the name
    of the Lord. [11] Behold, we consider those blessed who remained
    steadfast. You have heard of the steadfastness of Job, and you have seen
    the purpose of the Lord, how the Lord is compassionate and merciful.
    
    [12] But above all, my brothers, do not swear, either by heaven or by
    earth or by any other oath, but let your “yes” be yes and your “no” be
    no, so that you may not fall under condemnation.
  \end{quote}
\subsection*{Notes}
\begin{itemize}
  \item In today’s text, James exhorts his listeners and teaches them how to think about the ways of the world1
  \item Three points for today: God is sovereign over our \textbf{plans} and \textbf{possessions}, and this should help us to be \textbf{patient}.
  \item From v13-14, we see that James is warning against the arrogance of the traders. The traders made many elaborate plans to make money. All they think about is making money, without regard for the Lord. But James reminds us that we are just mist, we are here today and gone tomorrow. James is not discouraging planning, but James is telling us to plan wrt the fact that ultimately God is in control of our plans. We should not have an arrogant attitude and think that what we plan will definitely happen, regardless of God. And that should change what we plan for; instead of “make a profit” as per v14, we should “if the Lord wills, we should do this or that or etc”. We must plan according to the will of the Lord. And we know what the Lord’s will is, it is to love Him and love our neighbour. Just as sinful men can plan to make money, believers can plan to do good. Not doing so is the sin of omission (v17).
  \item In ch5:1-6, we see landowners defrauding their labourers to live in luxury and self-indulgence. This is an example of the pride mentioned in the previous chapter; these landowners think that they can just make plans to oppress people for money. They forget that God, the righteous judge, exists. James says that for these landowners, the wealth that they gain will condemn them. It is not riches that condemn per se, but it is the illegitimate acquiring and use of wealth that condemns.
  \item For us, we might not illegitimately acquiring wealth, but the presence of wealth is still a snare for us, in the sense of how we might be storing wealth up and not using them for good (ch4:17). Rich people should not store up treasure in the last days.
  \begin{itemize}
    \item “God has not appointed gold for rust nor garments for moth, but designed them as aids and helps to human life” (Calvin)
    \item So we can check ourselves thusly: if we have stored up so much that our “gold rusts” and our “garments corrodes”, and if we have “lived in self-indulgence”, then we need to think about how we use our wealth, even if our wealth was obtained legitimately. \KH{Then again, in our globalised capitalised world, is there anyone who truly obtains any wealth legitimately?. All our hands are stained with sin…}
  \end{itemize}
  \item In James 5:7-12, James tells us to be patient in the face of suffering. James gives us three examples here: farmers, prophets, and Job.
  \item Wrt the example of the farmers, we know that we cannot force plants to grow or bear fruit. We can plow the land and sow the seed and water the ground, but God gives the growth. Similarly, we must be patient for the coming of the Lord. 
  \item Wrt the example of the prophets, we see that even as we are patient for the Lord’s coming, we must be dilligent in doing good, like the prophets of old.
  \item And wrt the example of Job, we know that God is compassionate and merciful, and that our suffering will come to an end.
  \item And for the best example of patience, we can look to our Lord Jesus.

\end{itemize}