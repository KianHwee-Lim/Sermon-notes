\setcounter{figure}{0}

\section{19th May 2024: The prayer of faith}
\subsection*{Text: James 5:13-20}
  \begin{quote}
    [13] Is anyone among you suffering? Let him pray. Is anyone cheerful? Let him sing praise. [14] Is anyone among you sick? Let him call for the elders of the church, and let them pray over him, anointing him with oil in the name of the Lord. [15] And the prayer of faith will save the one who is sick, and the Lord will raise him up. And if he has committed sins, he will be forgiven. [16] Therefore, confess your sins to one another and pray for one another, that you may be healed. The prayer of a righteous person has great power as it is working. [17] Elijah was a man with a nature like ours, and he prayed fervently that it might not rain, and for three years and six months it did not rain on the earth. [18] Then he prayed again, and heaven gave rain, and the earth bore its fruit.

    [19] My brothers, if anyone among you wanders from the truth and someone brings him back, [20] let him know that whoever brings back a sinner from his wandering will save his soul from death and will cover a multitude of sins.
  \end{quote}
\subsection*{Notes}
\begin{itemize}
  \item There was a sermon on this text 2 years ago. The points were suffering, sin and sickness. How a church responds to these three things would show her authenticity as a church.1
  \item Today’s sermon on this same text would be more contemplative in nature (can visit the previous sermon for more exposition).
  \item V15: what is the “prayer of faith”? Sickness is a suffering that many wants to be healed from. There are big claims attached to this “prayer of faith”, so we need to understand what it means. 
  \begin{enumerate}
    \item First explanation: it could refer to a prayer by a righteous, prayerful man. So here, with this interpretation, a sick person should visit the elders to pray, because the prayer of the elders are more efficacious. If taken to the extreme, might become superstition.
    \item Second explanation: it could refer to the faith of the person who is praying. 
    \item Third explanation: it could refer to the prayer itself.
  \end{enumerate}
  \item Ps ronnie’s idea is that all explanations are correct. 
  \item Firstly, a person who is suffering may or may not have sinned. Thus, its good to get the elders of the church to come and discern in the person’s life whether he has committed any sin, and if they have discerned that the suffering is a result of sin, then the elders will pray for forgiveness of sin and healing, since the elders have spiritual authority. This is what it means to be praying the “prayer of faith”.
  \item Secondly, proper prayer is praying according to God’s will. And when we pray according to God’s will, our prayer will be answered. This is what it means to be praying the “prayer of faith”. Then, to learn how to pray, one should spend more time with prayerful people. 
  \item Thirdly, wrt the prayer of faith as being the prayer itself, as long as the prayer is theologically sound AND our lives are being transformed as we pray, then the prayer is good.
\end{itemize}