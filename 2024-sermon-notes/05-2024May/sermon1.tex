\setcounter{figure}{0}

\section{5th May 2024: Clothe yourself with humility}
\subsection*{Text: James 4:1-12}
  \begin{quote}
    [1] What causes quarrels and what causes fights among you? Is it not this, that your passions are at war within you? [2] You desire and do not have, so you murder. You covet and cannot obtain, so you fight and quarrel. You do not have, because you do not ask. [3] You ask and do not receive, because you ask wrongly, to spend it on your passions. [4] You adulterous people! Do you not know that friendship with the world is enmity with God? Therefore whoever wishes to be a friend of the world makes himself an enemy of God. [5] Or do you suppose it is to no purpose that the Scripture says, “He yearns jealously over the spirit that he has made to dwell in us”? [6] But he gives more grace. Therefore it says, “God opposes the proud but gives grace to the humble.” [7] Submit yourselves therefore to God. Resist the devil, and he will flee from you. [8] Draw near to God, and he will draw near to you. Cleanse your hands, you sinners, and purify your hearts, you double-minded. [9] Be wretched and mourn and weep. Let your laughter be turned to mourning and your joy to gloom. [10] Humble yourselves before the Lord, and he will exalt you.

    [11] Do not speak evil against one another, brothers. The one who speaks against a brother or judges his brother, speaks evil against the law and judges the law. But if you judge the law, you are not a doer of the law but a judge. [12] There is only one lawgiver and judge, he who is able to save and to destroy. But who are you to judge your neighbor?
  \end{quote}
\subsection*{Notes}
\begin{itemize}
  \item Today’s text has many exhortations, one after another, in rapid fire
  \item One possible overarching theme for all of the exhortations is humility, c.f v6 and v10. All the other exhortations are related to that.
  \item V10 tells us to humble ourselves before the Lord. When we do so, four things should happen:
    \begin{enumerate}
      \item We will relinquish control (7a)
      \item We will resist the devil (7b)
      \item We will return to God (8a)
      \item We will repent from sin (8b)
    \end{enumerate}
  \item Wrt the first point on control, when we are not in control, we feel insecure, and we may even be upset with God. I.e, our life is not going the way that we envision. Or in James’ words, “we desire and do not have, so we murder”. When we submit to God, we are yielding to God, and recognising His sovereignty over our lives, and thus bending our wills to His will. 
  \begin{itemize}
    \item We will not frown at God’s seemingly cross providence, but we will always strive to obey God in all circumstances and try to glorify God in all circumstances
    \item We submit to God as an act of humility, acknowledging that He is God and we are not
  \end{itemize}
  \item When we submit ourselves to God, we will resist the devil, who is a prideful creature (c.f Isaiah 14:12-14).
  \begin{itemize}
    \item Satan wanted to be like God, which is the epitome of pride 
    \item When Satan tempts us, he wants us to follow his footsteps, which is to be proud/egoistic 
    \item Satan deceives us into believing that we can be in control. This is exactly what he did to Adam/Eve 
    \item Thus, when we relinquish control and submit to God, we are resisting the devil. 
    \item The devil is like a roaring lion; when we stick our head into the lion’s mouth, we dampen the sound of the roaring but we don’t increase our chances of being alive.
  \end{itemize}
  \item We should never overly trivialise the devil or overly fear the devil. The second point is because when we resist the devil, he will flee from us.
  \begin{itemize}
    \item Jesus did this! In the temptation in the wilderness.
    \item And Jesus resisted the devil using scripture. 
    \item And Jesus submitted to the will of God in obedience.
    \item And after Jesus resisted the devil, the devil left him.
    \item We see that Jesus here is the epitome of humility; though He is God, he took on our human nature, humbled Himself in obedience to God, all the way to the cross. (Philippians 2)
    \item And as James says, when we submit ourselves to God, God will exalt us, that is what happened with Jesus too.
  \end{itemize}
  \item V8a says that when we submit to God, we are to return to God. Here, James is likely referring to Christians who gave in to temptation and failed. Wrt the example of the prodigal son, we see that even the act of the prodigal son returning took much humility from him. So part of submitting to God is returning to God in humility, realising that you are totally helpless on your own. And as per the story of the prodigal son, when we inch towards God, God runs towards us.
  \item V8b says that when we submit to God, and return to God, we are to repent from sin. We can’t return to God and hold sin in our heart. Here, we are to cleanse our hands (our outward actions), and our hearts (our inward actions). Christianity is a religion of joy (e.g, the joy of the Lord is your strength), and many more verses. While gloom is not a christian characteristic, mourning over our sin is. Thus, James’ exhortation for christians to be wretched and mourn and weep is a calling for us to not trivialise our sin but to instead recognise how costly sin is. The Bible is quite balanced when it comes to how we should treat sin. When we are overwhelmed with guilt of sin, the gospel assures us of forgiveness. But when we trivialise sin, the Law tells us to weep and mourn over how sinful we are, so as to point us to our need for the gospel. (The Law Gospel part is my thoughts). And when we mourn over our sins, the beatitude holds true; blessed are those who mourn, for they shall be comforted.
  \item When we submit ourselves to God and humble ourselves before God, we also are to humble ourselves before others. (V1-3). Conflicts between people arises because of selfish desire. In other words, when we are in a conflict with someone else, there probably is selfish desire in our heart. And this selfish desire comes from pride. This pride comes with a preoccupation with self. Thus, when we humble ourselves before others, we stop caring ourselves as much, and we care about others more. This will help with the pride in our heart, which also helps with the envy, anger, etc which is a fruit of preoccupation with self.
  \item And in verse 11, we see that we are not to speak evil against one another. Speaking evil against someone comes from a place of pride. When we speak evil against someone, its mostly because it brings us some benefit (like makes us feel shiok, gives us tangible benefits, etc). 
  \item V12 tells us not to have a judgmental attitude, not to think that we are better than others (i.e, we might think that we are holier than others). Its not that we can’t judge. We are still to speak the truth in love to others. But love is in itself other-centred. When we are judging others, are we doing so for the good of the other person, or because we want to feel better than others? 
  \item Note that superiority and inferiority are usually both sides of the same coin. Both come with a pre-occupation with self. Thus, both are expressions of pride. Inferiority appears needy and lowly so it might not look like pride, but it often comes with a need for affirmation from others, an envy towards those who are better, and a bitterness towards God for being “inferior”. All of which are bad! The antidote to this is self-denial.
  \item And in conclusion, we see that God opposes the proud but gives grace to the humble. And He gives more grace!s
\end{itemize}
