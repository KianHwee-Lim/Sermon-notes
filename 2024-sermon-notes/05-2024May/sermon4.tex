\setcounter{figure}{0}

\section{26th May 2024: Finding meaning and purpose in our messy world}
\subsection*{Text: Genesis 1:26-27}
  \begin{quote}
    [26] Then God said, “Let us make man in our image, after our likeness. And let them have dominion over the fish of the sea and over the birds of the heavens and over the livestock and over all the earth and over every creeping thing that creeps on the earth.”

    [27] So God created man in his own image, in the image of God he created
    him; male and female he created them.
  \end{quote}
\subsection*{Notes}
\begin{itemize}
  \item Many young people in SG believe in FIRE (financial independence, retire early). 
  \item SG is supposedly the second happiest country in Asia. But everyone in SG feels very stressed over many things, studies, finances, home stress, work, etc.
  \item Many people also struggle with finding meaning/purpose in their lives. And a lot of people are unhappy. This is despite the many pleasures that modern technology and science affords us.
  \item So on the surface, we are very prosperous and we’re supposed to be happy. But yet, a lot of people are unhappy. So what do we make of this?
  \item Christians always say that we have the “good news” of Jesus. In the face of all the problems of modernity canvassed above, what can this good news do?
  \item Non-believers often claim that science can explain the “whys” of life. Why do humans have a longing for justice? Why are we here? Etc. But actually, science can only explain the “hows” of life. My example, science can explain how a murder occured, but it can’t explain why a murder occured.
  \item In fact, in life, the “why” question is more important than the “how” question. And since science can’t explain the “why”, this means that we have to go to philosophy and theology. 
  \item “Why” is something good or meaningful? As humans, we intuitively seek love, goodness, truth, beauty. But what is love/goodness/truth/beauty? We need a absolute standard of truth/goodness/beauty to explain what truth/goodness/beauty is. An analogy: without a well defined “origin”, coordinates of points in geometry have no meaning.
  \item This is not just philosophical musing, because without a well defined, objective standard of good, we can’t say what evil is. 
  \item In our lives, we are caught in the middle between all the goodness we see and enjoy, but also all the evil we also see and experience and grieve. What is the explanation for this? I.e, the questions to ask are: why do we humans have a longing for good? Why do humans desire companionship? Why are humans special among the animals? Why is there even good? Why is there evil?
  \item Christianity has an explanation for all of the above.
  \begin{itemize}
    \item We are made in the image of God. Since God is the ground of all truth, beauty, goodness, we have a natural inclination towards goodness, truth, beauty. As Augustine said: “our hearts will find no rest except in God”.
    \item Also, as images of God, we reflect God’s goodness and glory to one another. And we are to have dominion over the creation; i.e we are to take care of the creation.
    \item But if God made everything good (including us humans), why is there evil? Christians say that the answer to that is Genesis 3. Because humans have chosen to rebel against God, we start to sin because the r/s that humans have with God is broken. For e.g, we don’t have to teach our children how to lie/bully their siblings. They naturally know how to. And because of sin which separates us from God, there is death.
    \item As images of God, we aspire goodness. But as disobedient humans, we are caught up in a vicious cycle of wrongdoing. And because we are caught in this mess, some of us might deny the existence of objective moral values itself.
    \item All the above sounds v mythical, but it does not just exist in the realm of ideas.  It is grounded in an historical event, the birth and death and resurrection of Jesus.
    \item How does Jesus relate to all that was said above about image of God/sin/etc? The birth, death, resurrection of Jesus means that Jesus died to pay for the sins of all humanity, that our r/s with God can be restored. In Romans 1, we see that Jesus was more than just a human, He is not only the son of david, but also the son of God. And when our r/s with God is restored, we can choose good instead of sin. Furthermore, Jesus’ resurrection gives us hope that for those who believe in Him, death is not the final word. There will come a day when all tears will be wiped away.
    \item But the Christian message above is not just a philosophy. God is not just abstract truth, beauty and goodness. He is the answer to the question of the meaning of life.
    \item Our God is a covenantal God, He is a relational God. He wants a r/s with us. It is not sufficient to know that God is good and that God can help us with meaning in life. We also need to know God personally. And God wants to know us personally too.
    \item Thus, we can choose to know God personally.
  \end{itemize}
\end{itemize}