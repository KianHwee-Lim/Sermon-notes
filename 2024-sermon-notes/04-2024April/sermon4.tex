\setcounter{figure}{0}

\section{28th April 2024: True wisdom and worldly wisdom}
\subsection*{Text: James 3:13-18}
  \begin{quote}
    [13] Who is wise and understanding among you? By his good conduct let him show his works in the meekness of wisdom. [14] But if you have bitter jealousy and selfish ambition in your hearts, do not boast and be false to the truth. [15] This is not the wisdom that comes down from above, but is earthly, unspiritual, demonic. [16] For where jealousy and selfish ambition exist, there will be disorder and every vile practice. [17] But the wisdom from above is first pure, then peaceable, gentle, open to reason, full of mercy and good fruits, impartial and sincere. [18] And a harvest of righteousness is sown in peace by those who make peace.
  \end{quote}
\subsection*{Notes}
\begin{itemize}
  \item{First point: what is true wisdom
  \begin{itemize}
    \item Sometimes, we may think we are clever and we might laugh at others,
    but others might think that they are clever and laugh at us. Worldly
    wisdom is often equated with academic success/"street smarts"/"making it
    in life"/cleverness. Or in Chinese culture, those who have more life
    experience are automatically seen as wiser.
    \item But in James, he says that true wisdom is seen by one living a good
    life and doing good works with humility. True wisdom is seen not just in
    talk, but in our actions.
    \item In the news, we see stories of top lawyers getting into trouble
    with the law. They might know the law very well (cleverness/knowledge),
    but because of greed, they go to jail for embezzlement.
    \item James also says that true wisdom is meek. Meekness is not weakness,
    it is more of a deliberate self restraint, a self-sacrifice for the
    greater good of others.
    \item And we can only have this wisdom from above when our relationship
    with God is right.
  \end{itemize} }
  \item{Point 2: Nature of worldly wisdom
  \begin{itemize}
    \item Just like how godly wisdom has good fruits, worldly wisdom also has
    its fruits. These fruits are jealousy and selfish ambition, which leads
    to all kinds of disharmony.
    \item Selfish ambition leads to boasting, quarelling, backstabbing, criticism of rivals, gossip, etc.
    \item Jealousy leads us to rejouce rather than grieve at the sorrows of others. It prevents us from loving them and to be understanding towards them. 
    \item For worldly wisdom that leads to jealousy and selfish ambition, James calls it demonic. We should not follow the ways of this so-called ``wisdom''. If we do, we are denying God's ways and God Himself.
  \end{itemize} }
  \item{Point 3: Characteristics of God's wisdom
  \begin{itemize}
    \item First, we see that God's wisdom is pure, peaceable, gentle, open to
    reason, full of mercy and good fruits, impartial and sincere (taken from
    the text). Pure means that God's wisdom is free from sin. Peaceable means
    that God's wisdom leads to peace. Gentle and open to reason are
    self-explanatory. Full of mercy means that forgiveness is the way of
    God's wisdom.
    \item We see that these characteristics of God's wisdom reflect God's own
    nature. And these characteristics of God's wisdom also reflect God's
    desires and intentions for us. God wants peace in our relationships. And
    thus by following God's righteous ways and living righteously, we can
    invite the world into the ways of righteousness and hence make peace in
    the world.
  \end{itemize}}
  \item{In conclusion, we note that this wisdom comes from Jesus living in us
  through the Holy Spirit. In Him is all wisdom and knowledge. We can also
  obtain this wisdom from the Word of God. And we can also obtain this wisdom
  by prayer. And as per what James says later in the text, we are to submit ourselves therefore to God. We are humbling our hearts and saying to God that we are unable to do what is right and know what is right, and asking God for our help. Thus, we see how humility and godly wisdom are intertwined; without humility; one will never obtain godly wisdom. When we live in the ways of the world instead of God's way, it reflects us lacking trust in God in our heart.
  \begin{quotation}
    Wisdom is knowing how to live God's way in God's world
  \end{quotation}
  We should pray that we would desire this wisdom, so that we may live out God's ways in our lives and in the world.
  }
\end{itemize}