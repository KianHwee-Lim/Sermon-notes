\setcounter{figure}{0}

\section{14th April 2024: Evidence of genuine faith}
\subsection*{Text: James 1:19-27, 2:14-26}
  \begin{quote}
    [19] Know this, my beloved brothers: let every person be quick to hear, slow to speak, slow to anger; [20] for the anger of man does not produce the righteousness of God. [21] Therefore put away all filthiness and rampant wickedness and receive with meekness the implanted word, which is able to save your souls.

    [22] But be doers of the word, and not hearers only, deceiving yourselves. [23] For if anyone is a hearer of the word and not a doer, he is like a man who looks intently at his natural face in a mirror. [24] For he looks at himself and goes away and at once forgets what he was like. [25] But the one who looks into the perfect law, the law of liberty, and perseveres, being no hearer who forgets but a doer who acts, he will be blessed in his doing.

    [26] If anyone thinks he is religious and does not bridle his tongue but deceives his heart, this person’s religion is worthless. [27] Religion that is pure and undefiled before God the Father is this: to visit orphans and widows in their affliction, and to keep oneself unstained from the world.

    [14] What good is it, my brothers, if someone says he has faith but does not have works? Can that faith save him? [15] If a brother or sister is poorly clothed and lacking in daily food, [16] and one of you says to them, “Go in peace, be warmed and filled,” without giving them the things needed for the body, what good is that? [17] So also faith by itself, if it does not have works, is dead.

    [18] But someone will say, “You have faith and I have works.” Show me your faith apart from your works, and I will show you my faith by my works. [19] You believe that God is one; you do well. Even the demons believe—and shudder! [20] Do you want to be shown, you foolish person, that faith apart from works is useless? [21] Was not Abraham our father justified by works when he offered up his son Isaac on the altar? [22] You see that faith was active along with his works, and faith was completed by his works; [23] and the Scripture was fulfilled that says, “Abraham believed God, and it was counted to him as righteousness”—and he was called a friend of God. [24] You see that a person is justified by works and not by faith alone. [25] And in the same way was not also Rahab the prostitute justified by works when she received the messengers and sent them out by another way? [26] For as the body apart from the spirit is dead, so also faith apart from works is dead.
  \end{quote}
\subsection*{Notes}
\begin{itemize}
  \item{Paul says that when we believe in Jesus, we will be saved (Romans 10:19). But what does it mean to believe?
  \begin{itemize}
    \item Analogy: charles blondin could walk across a tightrope blindfolded and carrying a wheelbarrow, but when he asked for an audience member (who claimed to believe in charles’ ability) to sit in a wheelbarrow, nobody wanted to volunteer. Clearly, the audience did not have enough faith in charles
  \end{itemize}}
  \item{Main point: If we listen intently to the word of God and genuinely strive to obey it, then we can have confidence that we will be saved.}
  \item{Listening: “be quick to hear, slow to speak, slow to anger”. This comes after the passage on trials and temptations. We have two ears and one mouth, so a good rule of thumb is to listen twice as much as we speak. We often get angry when we do not get what we want. But maybe listening to others rather than just speaking rashly might help us get what we want. The anger of man does not produce the righteousness of God; when we are speak when we are angry, it is usually counter-productive because our anger causes us to be less teachable, understanding and empathetic.}
  \item{But apart from listening to others, the v19 on James is also sandwiched between v18 (word of truth) and v21 (implanted word). Thus, we are also to intently listen to God’s Word. In fact, as part of fearing God in the midst of trials and temptations, it might be helpful to listen intently to God’s Word rather than to complain to God incessantly about our trials (ofc, there is a place for lamentation and honesty before God. But usually lamentation is also scripture based). As is said in Ecclesiastes, “since God is in heaven and we are on earth, let your words be few”}
  \item{What place do we give God’s Word in our life? Three areas to think about:
  \begin{enumerate}
    \item Sunday worship (e.g, what time do we sleep on sat)
    \item Bible study (e.g, do we prepare before study and do we regularly attend?)
    \item Personal devotion (do we do so? Best to do this first thing in the morn!)
  \end{enumerate}
  When we listen intently to God’s Word, we will have wisdom to deal with trials and temptations…
  }
  \item{Other than listening intently to God’s Word, we are also to obey. Listening intently to God’s word is a good start, but it is not enough. We must also obey God’s Word.
  \begin{enumerate}
    \item God’s Word has the same effect to us as a mirror. Like a mirror, God’s Word points out our sins and shortcomings, and teaches us how we need to improve. When we read the Bible, we are shown who we actually are.
    \item But apart from showing us our sins, the Bible also discloses our need for repentance. It is folly to see our sins and not do anything about it.
    \item God’s Word also assures us our promise of God’s forgiveness.
  \end{enumerate}
  }
  \item{Is obedience to God’s Word a killjoy and restrictive? It is not supposed to be, since God’s law is the “law of liberty”, as per our text. 
  \begin{enumerate}
    \item True freedom is not the absence of rules, but it is the presence of the right rules that set safe boundaries.
    \item \KH{E.g, with the right boundaries in a playground, kids can run as far as they like in the playground because the boundaries are clear (there's actually a psychological study done for this~\footnote{\url{https://www.asla.org/awards/2006/studentawards/282.html}}.) }
    \item Since God made us, He is the one who knows best what boundaries we need.
    \item Every command God gives us is for our good, because they give us the boundaries within which we can flourish!  
  \end{enumerate}}
  \item{How do we distinguish between counterfeit faith and real faith? “Faith by itself, if it does not have works, is dead” (ch. 2:17). “Faith apart from works is useless” (ch. 2:20). ``Faith apart from works is dead'' (ch. 2.26). Two illustrations for fake faith:
  \begin{enumerate}
    \item Fake faith is like seeing someone in need, and just giving talk without any action.
    \item Saying that one believes that God is one; even the demons do so! True believers believe and obey, false believers believe and rebel.
  \end{enumerate}
  Two illustrations for real faith:
  \begin{enumerate}
    \item Abraham’s willingness to give up Isaac. He obeyed God here. 
    \item Rahab’s willingness to shelter the spies 
  \end{enumerate}
  Abraham and Rahab are quite polar opposites when it comes to their backgrounds. But they have similar genuine faith.}
  \item{Five closing remarks:
  \begin{enumerate}
    \item How much do we need to obey? We need to obey all of God’s commandments. When we break one law, we break all the laws. This can be an area of impurity, unforgiveness, pride, etc. God wants us to deal with every sin as the Holy Spirit surfaces it out.
    \item The emphasis of obedience to God is love to others. In a time of persecution, it is easy for people to just care about their own self interests. But what James talks about is loving others, like loving the orphans and widows, and not showing partiality to anyone, and providing for everyone’s needs. 
    \item Mercy triumphs judgment. We are imperfect, but we know that as we try our best, God has mercy on us in virtue of the fact that we are in Christ. We will sin in our mortal bodies, but through continued listening to God and repentance for our sins, through the Holy Spirit, we will sin less and less. \KH{ And in the end, God will keep us blameless, as per Jude 24-25}.
    \item The supposed contradiction between James and Paul is resolved by noting that we are justified by faith alone, but faith is never alone, it produces the fruit of obedience. So we don’t need to look to our good works for justification, but instead we do good works from a state of being justified by God already, with peace with God and thankfulness in our heart.
    \item Finally, obedience to God is not the starting point, but the gospel message of forgiveness of sin through faith is. We are justified by faith, and then that faith produces good works as fruit.  
  \end{enumerate}}
\end{itemize}