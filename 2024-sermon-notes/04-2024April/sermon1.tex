\setcounter{figure}{0}

\section{7th April 2024: Remaining steadfast amidst trials}
\subsection*{Text: James 1:1-18}
  \begin{quote}
    [1] James, a servant of God and of the Lord Jesus Christ,

    To the twelve tribes in the Dispersion:

    Greetings.

    [2] Count it all joy, my brothers, when you meet trials of various kinds, [3] for you know that the testing of your faith produces steadfastness. [4] And let steadfastness have its full effect, that you may be perfect and complete, lacking in nothing.

    [5] If any of you lacks wisdom, let him ask God, who gives generously to all without reproach, and it will be given him. [6] But let him ask in faith, with no doubting, for the one who doubts is like a wave of the sea that is driven and tossed by the wind. [7] For that person must not suppose that he will receive anything from the Lord; [8] he is a double-minded man, unstable in all his ways.

    [9] Let the lowly brother boast in his exaltation, [10] and the rich in his humiliation, because like a flower of the grass he will pass away. [11] For the sun rises with its scorching heat and withers the grass; its flower falls, and its beauty perishes. So also will the rich man fade away in the midst of his pursuits.

    [12] Blessed is the man who remains steadfast under trial, for when he has stood the test he will receive the crown of life, which God has promised to those who love him. [13] Let no one say when he is tempted, “I am being tempted by God,” for God cannot be tempted with evil, and he himself tempts no one. [14] But each person is tempted when he is lured and enticed by his own desire. [15] Then desire when it has conceived gives birth to sin, and sin when it is fully grown brings forth death.

    [16] Do not be deceived, my beloved brothers. [17] Every good gift and every perfect gift is from above, coming down from the Father of lights, with whom there is no variation or shadow due to change. [18] Of his own will he brought us forth by the word of truth, that we should be a kind of firstfruits of his creatures.
  \end{quote}
\subsection*{Notes}
\begin{itemize}
  \item{Writer of James is likely to be James the brother of Jesus, which was the leader of the Jewish christians in jerusalem.}
  \item{But the christians in jerusalem were scattered and dispersed. In this dispersion, many of them were treated unfairly.}
  \item{James wanted to encourage these Christians, because he wanted to encourage these suffering jewish christians and exhort them not to fall away from their faith because of their suffering.}
  \item{For us today, we might also be undergoing suffering, and that might also be affecting our r/s with God. }
  \item{When we think of the word “trials”, we usually associate it with pain, loss. And as far as we could, it is human for us to avoid pain, loss. But here, James is asking people to “count it all joy when they experience trials”.
  \begin{itemize}
    \item{Is James being an insensitive church leader? Probably not.}
    \item{ James is not downplaying the trial and the suffering it causes.
    But James is telling his congregation to think of trials in a christian
    way.}
    \item{It is difficult to define here what “joy” means. But it suffices to
    say that true “joy” is dependent on truth. And the truth that James is
    emphasising here is that God allows us to face trials to refine us, not
    to harm us.}
  \end{itemize}}
  \item{First point: we should change the way we view trials in our life. 
  \begin{enumerate}
    \item Trials are not a sign of God’s inattention or inability to help us
    \item A sovereign and gracious God can use trials to bring about spiritual maturity. There will be tears and loss and grief, but all of these are not in vain. God will bring good into our lives.
    \item The world we live in demands instant gratification. But the growth in spiritual maturity requires time and requires perseverance. This is counter-cultural! Our attitude should be always having a desire to grow into greater christ-likeness.
    \item We also shouldn’t expect other people going through trials to quickly tell you “eh in this season what is God teaching you”? Spiritual maturity and growth in steadfastness takes perseverance and time!
  \end{enumerate}}
  \item{Trials have a way of distorting our perspective of life and of God. Thus, we arrive at our second point: we need to ask God for wisdom to be able to grow in steadfastness in the midst of the trial.
  \begin{enumerate}
    \item The wisdom here is not the ability to immediately solve your problems/remove the trial
    \item The wisdom here is more of the ability to see trials from God’s perspective
    \item James is reminding his congregation that the God we have is always ready to hear us, and thus we always can go to God.
    \item God here is the anchor that helps us to remain “sure and steadfast”, to not go into worldliness in the midst of trials
    \item The doubting here refers to people who want both the ways of the world and the blessings that God gives (e.g maybe peace of mind).
  \end{enumerate}}
  \item{Part of the wisdom that God gives is that God helps us to find delight in what God says is of ultimate value. Thus, the third point is to understand how God values things.
  \begin{enumerate}
    \item The poor are not more holy cause of their poverty
    \item The rich are not more unholy because of their wealth
    \item Both rich and poor Christians are loved by God, and that is the ultimate value. And that is equal to both rich and poor Christians!
    \item Poor christians must guard against low self esteem and envy. They must not view themselves in light of their poverty, but they must view themselves in light of the ultimate value they have in that they are loved by God. (Let the lowly brother boast in his exultation)
    \item Rich christians must guard against pride, and thinking that they are better because they are richer. This is them putting more value in their wealth than in the fact that God loves them (let the rich boast in his humiliation). 
  \end{enumerate}
  }
  \item{Regardless of our circumstances, when we are going through a trial, it is also a potential temptation. It is a temptation because it is an opportunity for our sinful desires to cause us to fall instead of to grow. When we constantly compromise, always giving in to temptation, we will eventually die. And in light of this temptation, we always have to make a choice and we always are responsible for that choice. God is sovereign, but He doesn’t cause us to sin! }
  \item{So here James is reminding them that God is the giver of all good things, and that despite their circumstances, God is still good. And they know this because they have got the most perfect gift already despite their circumstances. This perfect gift is the gospel, that through Jesus they are “brought forth by the word of truth”. They have the power to resist sin and to love God, to enjoy this eternal life. And this gift is to help them to be a firstfruits.}
\end{itemize}