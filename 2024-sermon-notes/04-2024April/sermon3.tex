\setcounter{figure}{0}

\section{21st April 2024: Taming the tongue}
\subsection*{Text: James 3:1-12}
  \begin{quote}
    [1] Not many of you should become teachers, my brothers, for you know
    that we who teach will be judged with greater strictness. [2] For we all
    stumble in many ways. And if anyone does not stumble in what he says, he
    is a perfect man, able also to bridle his whole body. [3] If we put bits
    into the mouths of horses so that they obey us, we guide their whole
    bodies as well. [4] Look at the ships also: though they are so large and
    are driven by strong winds, they are guided by a very small rudder
    wherever the will of the pilot directs. [5] So also the tongue is a small
    member, yet it boasts of great things.

    How great a forest is set ablaze by such a small fire! [6] And the tongue
    is a fire, a world of unrighteousness. The tongue is set among our
    members, staining the whole body, setting on fire the entire course of
    life, and set on fire by hell. [7] For every kind of beast and bird, of
    reptile and sea creature, can be tamed and has been tamed by mankind, [8]
    but no human being can tame the tongue. It is a restless evil, full of
    deadly poison. [9] With it we bless our Lord and Father, and with it we
    curse people who are made in the likeness of God. [10] From the same
    mouth come blessing and cursing. My brothers, these things ought not to
    be so. [11] Does a spring pour forth from the same opening both fresh and
    salt water? [12] Can a fig tree, my brothers, bear olives, or a grapevine
    produce figs? Neither can a salt pond yield fresh water.
  \end{quote}
\subsection*{Notes}
\begin{itemize}
  \item{We all know how destructive/disintegrating bad speech or insulting
  speech can be for a community. All it takes to break up a community is just
  for a few people in the community to talk bad about one another. And this
  is made even worse when this ill-disciplined speech comes from teachers. If
  one wants to be a teacher, he should watch his speech, since if not, his
  judgement will be greater.}
  \item{James' thesis statement here is found in v2 of our text: ``And if
  anyone does not stumble in what he says, he is a perfect man, able also to
  bridle his whole body.''. The word ``perfect'' here is the same word as
  ``mature''. Thus, for a Christian to be mature, he must control his own
  speech. It doesn't matter how much theology one knows if he is
  ill-disciplined with his tongue; he is immature.}
  \item{Three points today: 
  \begin{enumerate}
    \item{We musn't \textbf{underestimate} the power of the tongue. The
    tongue has great power.}
    \item{The tongue can \textbf{undo} a community, other people, and ourselves.}
    \item{The tongue is \textbf{untameable}. }
  \end{enumerate}}
  \item{A lot of us have regrets with respect to things we have previously
  said. Yet we still continue to say regrettable things. Why? Because we
  underestimate the power of the tongue. The analogy that James gives here is
  that just like a small bit in a big horse, or like a small rudder in a big
  ship, the tongue is small but it has a big impact. James also calls the
  tongue a small fire that can set the forest ablaze. If a Christian can control his tongue, he can control his own body and himself.}
  \item{Continuing from the analogy of the tongue as a fire, we see that the
  tongue has very high destructive impact. For example, the body of Christ
  takes a lot of effort to build up, but the tongue can just stain the entire
  body of Christ and makes it impure and destroy it. Just like how stains are
  very hard to remove, the impact of hurtful words are very long lasting. The
  staining, destructive effect of the tongue is on both the hearers and the
  speakers. The hearers can be hurt for decades by hurtful words, and the
  speakers can regret those words for decades.}
  \item{James also calls the tongue the fire from hell. We think that hell is
  an eternal destination for some, and this is correct. But the way James
  uses this analogy here is to say that hell now is no longer simply a
  destination for the future, but its corrupting, harmful and hurtful power
  extends from the future into our present time through the tongue. We see
  this in how our first ancestors fell; they fell cause of the tongue in
  phrases like ``did God say...''. The analogy of the poison here that James
  uses is also helpful; just like how poison usually takes time to kill, the
  effects of bad speech are usually not seen immediately, but it will
  eventually kill. Of course there are more lethal poisons that take effect
  immediately just like how there are words that are immediately hurtful, but
  majority of the hurtful words that we say take time to take effect. The tongue is also restless evil, moving from one place to another. \KH{Perhaps that means that bad words have the effect of spreading, through gossip.}}
  \item{James also says that the tongue is untameable. Though humans have
  tamed many animals, humans can't even tame their own tongue. But this makes
  sense, because Christianity is not about moral achievement, but moral
  redemption. If one thinks that he can tame his tongue by himself, he will
  make sooner or later make statements that he regrets. But if one
  acknowledges that he can't tame his tongue by himself, and which leads him
  to go to God for help to tame his tongue, then God can work. In short, the
  tongue is not tameable by human efforts, but with God's help, the tongue
  can be tamed. This would be an extension of the point that James made
  previously, about asking for wisdom from above. James here emphasises the
  untameability of the tongue to humble us, because of how hurtful it is. And
  only when we are humbled, can God work in our lives to redeem our tongues.}
  \item{Btw, corrupt speech comprises not only hurtful words, but also
  flattery. Flattery is dishonesty, which can undo someone too by causing the
  person to be prideful. }
  \item{One way that God helps us to tame our tongue is through the power of
  community. For example, spouses can remind each other to ``keep it cool,
  don't say words in anger''. Or our Christian friends can remind each other
  of that too. Or we can pray and ask for wisdom to do so.}
  \item{In verse 11, James says that the tongue is like a spring. A spring is
  a source of water. And a salty source brings forth salty water. Similarly,
  when one says hurtful words, it usually comes from a place of hurt that the
  person has previously experienced. The tongue is humanly untameable because
  the source that the words come from is human. But what if the source is
  divine instead of human? When we meditate on the scripture and let our
  vocabulary become the life-giving words of scripture instead of hurtful
  words, when we meditate on what God has done for us to heal the hurts we
  have experienced, then our human source will slowly be transformed by God.
  And we will become like a spring of fresh water, from which fresh water
  will come out. }
  \item{We can also glean some practical wisdom from the analogy of a forest
  fire. Here, the practical wisdom is for people who are on the receiving end
  of hurtful words. Forest fires are usually drastic when the forest has a
  lot of ``dead matter'' like dried leaves/dead trees. On the other hand,
  forest fires are quite manageable and do not spread too much when the
  forest has a lot of ``living matter'' like living trees (which has a lot of
  water content). Thus, when we are the recipients of hurtful words, the
  damage that the words have on us depends on the state of our own souls. If
  we carry a lot of hurt/regrets/insecurites within us, then the damage done
  by these hurtful words will be amplified. On the other hand, if we are very
  assured in God's love for us and our identity as redeemed people of God,
  then the damage done by hurtful words can be contained, though it still
  hurts. Thus, just as how we should clean away the ``dead matter'' in
  forests to minimise the damage of forest fires, this is an exhortation for
  us to clean out the hurts and regrets in our lives by surrending them all
  to God. \KH{We do this by remembering that Christ has borne all our hurts
  and pains, He is the person who was the most wronged, and by His stripes we
  are healed}. When we surrender our past hurts to God, we will have more
  ability to contain the damage done to us by the hurtful words of others. }
  \item{We also note that just as the tongue can do a lot of damage, it can
  also do a lot of good. The book of Proverbs says that the speech of the
  wise is healing. There are many people who carry many hurts, and the
  tongue, if used properly, can help to heal those hurts through words of
  encouragements etc. The ability to communicate is a wonderful gift that God
  has given us which is corrupted by sin. \KH{And we know that the better the
  gift, the worse its effects are when it is corrupted by sin.}. We should
  use this gift of the tongue then for the initial purpose that God has given
  it, which is to love one another. And of course, we can only do this with God's wisdom, so we should ask God for wisdom to do so.}
\end{itemize}