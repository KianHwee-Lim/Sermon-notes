\section{17th July 2022: Where to find wisdom?}
\subsection*{Text: Ecclesiastes 7}
  \begin{quote}
    [1] A good name is better than precious ointment,
        and the day of death than the day of birth.
    [2] It is better to go to the house of mourning
        than to go to the house of feasting,
    for this is the end of all mankind,
        and the living will lay it to heart.
    [3] Sorrow is better than laughter,
        for by sadness of face the heart is made glad.
    [4] The heart of the wise is in the house of mourning,
        but the heart of fools is in the house of mirth.
    [5] It is better for a man to hear the rebuke of the wise
        than to hear the song of fools.
    [6] For as the crackling of thorns under a pot,
        so is the laughter of the fools;
        this also is vanity.
    [7] Surely oppression drives the wise into madness,
        and a bribe corrupts the heart.
    [8] Better is the end of a thing than its beginning,
        and the patient in spirit is better than the proud in spirit.
    [9] Be not quick in your spirit to become angry,
        for anger lodges in the heart of fools.
    [10] Say not, “Why were the former days better than these?”
        For it is not from wisdom that you ask this.
    [11] Wisdom is good with an inheritance,
        an advantage to those who see the sun.
    [12] For the protection of wisdom is like the protection of money,
        and the advantage of knowledge is that wisdom preserves the life of
        him who has it.
    [13] Consider the work of God:
        who can make straight what he has made crooked?


    [14] In the day of prosperity be joyful, and in the day of adversity
    consider: God has made the one as well as the other, so that man may not
    find out anything that will be after him.

    [15] In my vain life I have seen everything.  There is a righteous man
    who perishes in his righteousness, and there is a wicked man who prolongs
    his life in his evildoing.  [16] Be not overly righteous, and do not make
    yourself too wise.  Why should you destroy yourself?  [17] Be not overly
    wicked, neither be a fool.  Why should you die before your time?  [18] It
    is good that you should take hold of this, and from that withhold not
    your hand, for the one who fears God shall come out from both of them.

    [19] Wisdom gives strength to the wise man more than ten rulers who are
    in a city.

    [20] Surely there is not a righteous man on earth who does good and never
    sins.

    [21] Do not take to heart all the things that people say, lest you hear
    your servant cursing you.  [22] Your heart knows that many times you
    yourself have cursed others.

    [23] All this I have tested by wisdom.  I said, “I will be wise,” but it
    was far from me.  [24] That which has been is far off, and deep, very
    deep; who can find it out?

    [25] I turned my heart to know and to search out and to seek wisdom and the
    scheme of things, and to know the wickedness of folly and the foolishness
    that is madness.  [26] And I find something more bitter than death: the
    woman whose heart is snares and nets, and whose hands are fetters.  He who
    pleases God escapes her, but the sinner is taken by her.  [27] Behold, this
    is what I found, says the Preacher, while adding one thing to another to
    find the scheme of things—[28] which my soul has sought repeatedly, but I
    have not found.  One man among a thousand I found, but a woman among all
    these I have not found.  [29] See, this alone I found, that God made man
    upright, but they have sought out many schemes.
  \end{quote}
\subsection*{Notes}
\begin{itemize}
  \item{Recap: The word ``vanity'' in the ESV is translated from the Hebrew
  word for ``breath'' or ``vapor''.  For example, when we say that wealth is
  ``vanity'', we mean that just like a breath or a vapor, wealth is
  ephemeral, and we cannot really grasp it.  When we say toil is ``vanity'',
  we are saying that like vapor, our toil is futile.  When we say that life
  is ``vanity'', we are saying that life is vanity.  So here, the question
  is, how do we resolve the tension between what our faith teaches us, that
  life ought to be meaningful because we have God, and what we observe and
  experience in the struggles of life ``under the sun'' that tell us
  otherwise?  The key is that even when life doesn't make sense, when life is
  monotonous, when God seems absent, we should still fear God, do good, be
  joyful and thankful to God for food and drink that we receive as food from
  His hand.}
  \item{Our text for today, Ecclesiastes 7, is a lot like Proverbs.  There
  are many sayings about wisdom in this chapter.}
  \item{We can find wisdom in the house of mourning.  In contemporary
  language, it is better to attend funerals than go to parties.  The reason
  is that when we attend funerals, we become more aware of our own mortality.
  The physical body that we have is a tent that is easily collapsible.  In
  this sense, life is ``vanity''.  Some of us might think that we are ``too
  young'' to think about our mortality.  As the Psalmist says, ``teach us to
  number our days, that we may have a heart of wisdom''.  In light of our
  mortality, it is wise to ask ``how then shall we live''.  It is good for us
  to reflect on our lives constantly and see if we need to make any changes,
  so that we won't have any regrets when we live.  The goal is to be such
  that at the end of our days, Jesus can say to us: ``well done, my good and
  faithful servant''.  It is not as if we are saved \textit{by} our works, it
  is more of how we are saved \textit{for} good works.  As Paul said, we are
  to ``work out our own salvation with fear and trembling...''.  When we
  reflect about our lives, we must ask ourselves: are we living any different
  from how non-Christians are living?  Does our faith affect the way we
  relate with people?  Does our faith affect the way we relate with our work?
  Etc.  Or is there something that you feel God is asking you to do?  If so,
  then do not tarry thinking that you still have time; our death is just
  around the corner.}
  \item{We can also find wisdom in situations we have no control over
  (v13-18, v23-24).  Straight and crooked, adversity and prosperity, are all
  the work of God.  Things are the way God wants them to be, we can't
  overrule the work of the Almighty.
  % When the Preacher talks about things that are ``crooked'', we are not
  % saying that God is the source of evil (that cannot be true!)
  Sometimes, despite how much we try to keep fit, we still fall sick.  We
  might try our best to live uprightly, yet we are still victims of injustice
  and slander.  These injustices are things that we wish that we could
  straighten our but we can't.  The fact that sometimes God has made things
  crooked can lead us to fatalism, but it shouldn't.  It shouldn't because
  though we have no control over situations and no answers for certain
  situations, these situations help us realise that sometimes our wisdom is
  limited.  Knowing the limits of our wisdom is itself wisdom.  Instead of
  leading us to fatalism, this understanding of the limits of our own wisdom
  should lead us to trust in God, who alone is wise and good and has all the
  answers.  And in fact, just as how sorrow can bring more wisdom than
  laughter, adversity can bring more wisdom that prosperity, because both
  sorrow and adversity lead us to trust in God's goodness.  ``Better is the
  end of a thing than its beginning, therefore patience is better than
  pride''.  Rather than arrogantly thinking that we know best, we should
  patiently trust God, even when we don't have all the answers.  This is part
  of our discipleship and our sanctification.}
  \item{From verse 15 onwards to verse 18, the Preacher meanders abit.  When
  the Preacher says ``do not be overly righteous'', it is more of ``don't be
  self righteous''.  I.e, we should not think that we are more righteous than
  we really are.  When we think that we are more righteous than we are, we
  become bitter when bad things happen to us, because we think that we don't
  deserve bad things.  When we think that we are more righteous than we are,
  we become bitter and prideful at other people.  When the Preacher says ``do
  not be overly wicked'', it is more of ``don't deliberately commit sins''.
  As sinful people, we do sometimes sin unintentionally.  But we should not
  deliberately sin when we can avoid it, because when we deliberately sin, we
  destroy ourselves.}
  \item{From verse 24, we see that wisdom is always distant and difficult to
  find.  The reason for that is that in verse 20 and verse 29, the reason why
  wisdom is hard to find is because of the sin of humanity.  The troubles of
  life can be traced to the problem of sin!  The Preacher did not in the book
  give answers to this problem of sin, apart from reminding us that God will
  finally judge and that we ought to fear God.  To get more definite answers
  to the problem of sin, we must turn to the New Testament.}
  \item{As in our Scripture reading today (1 Corinthians 1:18-24), Jesus
  Christ is the wisdom of God.  Jesus is the solution to the root problem of
  sin that is the cause of all the vanity in this life.  Jesus is our hope in
  life, however troublesome life is.  Whenever we look at the cross, whenever
  we look at Jesus, however troublesome life seems to be, we can rest assured
  that we will not fall outside God's loving hand (see more in Romans
  8:32-39).  It doesn't make sense for God to save us and then let us be
  destroyed by the trials in this world.  Rather, after we are saved, even
  the trials in this world become means through which God works all things
  for our good.  Even our death, which is inevitable, will not destroy us,
  because Jesus has destroyed the sting of death.  Just as Jesus was
  resurrected, we too can look forward to a future resurrection.}
\end{itemize}