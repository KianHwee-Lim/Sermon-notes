\section{29th May 2022: Gospel sunday}
\subsection*{Text: 1 Corinthians 15}
  \begin{quote}
    [1] Now I would remind you, brothers, of the gospel I preached to you, which you received, in which you stand, [2] and by which you are being saved, if you hold fast to the word I preached to you—unless you believed in vain.

    [3] For I delivered to you as of first importance what I also received: that Christ died for our sins in accordance with the Scriptures, [4] that he was buried, that he was raised on the third day in accordance with the Scriptures, [5] and that he appeared to Cephas, then to the twelve. [6] Then he appeared to more than five hundred brothers at one time, most of whom are still alive, though some have fallen asleep. [7] Then he appeared to James, then to all the apostles. [8] Last of all, as to one untimely born, he appeared also to me. [9] For I am the least of the apostles, unworthy to be called an apostle, because I persecuted the church of God. [10] But by the grace of God I am what I am, and his grace toward me was not in vain. On the contrary, I worked harder than any of them, though it was not I, but the grace of God that is with me. [11] Whether then it was I or they, so we preach and so you believed.

    [12] Now if Christ is proclaimed as raised from the dead, how can some of you say that there is no resurrection of the dead? [13] But if there is no resurrection of the dead, then not even Christ has been raised. [14] And if Christ has not been raised, then our preaching is in vain and your faith is in vain. [15] We are even found to be misrepresenting God, because we testified about God that he raised Christ, whom he did not raise if it is true that the dead are not raised. [16] For if the dead are not raised, not even Christ has been raised. [17] And if Christ has not been raised, your faith is futile and you are still in your sins. [18] Then those also who have fallen asleep in Christ have perished. [19] If in Christ we have hope in this life only, we are of all people most to be pitied.

    [20] But in fact Christ has been raised from the dead, the firstfruits of those who have fallen asleep. [21] For as by a man came death, by a man has come also the resurrection of the dead. [22] For as in Adam all die, so also in Christ shall all be made alive. [23] But each in his own order: Christ the firstfruits, then at his coming those who belong to Christ. [24] Then comes the end, when he delivers the kingdom to God the Father after destroying every rule and every authority and power. [25] For he must reign until he has put all his enemies under his feet. [26] The last enemy to be destroyed is death. [27] For “God has put all things in subjection under his feet.” But when it says, “all things are put in subjection,” it is plain that he is excepted who put all things in subjection under him. [28] When all things are subjected to him, then the Son himself will also be subjected to him who put all things in subjection under him, that God may be all in all.

    [29] Otherwise, what do people mean by being baptized on behalf of the dead? If the dead are not raised at all, why are people baptized on their behalf? [30] Why are we in danger every hour? [31] I protest, brothers, by my pride in you, which I have in Christ Jesus our Lord, I die every day! [32] What do I gain if, humanly speaking, I fought with beasts at Ephesus? If the dead are not raised, “Let us eat and drink, for tomorrow we die.” [33] Do not be deceived: “Bad company ruins good morals.” [34] Wake up from your drunken stupor, as is right, and do not go on sinning. For some have no knowledge of God. I say this to your shame.

    [35] But someone will ask, “How are the dead raised? With what kind of body do they come?” [36] You foolish person! What you sow does not come to life unless it dies. [37] And what you sow is not the body that is to be, but a bare kernel, perhaps of wheat or of some other grain. [38] But God gives it a body as he has chosen, and to each kind of seed its own body. [39] For not all flesh is the same, but there is one kind for humans, another for animals, another for birds, and another for fish. [40] There are heavenly bodies and earthly bodies, but the glory of the heavenly is of one kind, and the glory of the earthly is of another. [41] There is one glory of the sun, and another glory of the moon, and another glory of the stars; for star differs from star in glory.

    [42] So is it with the resurrection of the dead. What is sown is perishable; what is raised is imperishable. [43] It is sown in dishonor; it is raised in glory. It is sown in weakness; it is raised in power. [44] It is sown a natural body; it is raised a spiritual body. If there is a natural body, there is also a spiritual body. [45] Thus it is written, “The first man Adam became a living being”; the last Adam became a life-giving spirit. [46] But it is not the spiritual that is first but the natural, and then the spiritual. [47] The first man was from the earth, a man of dust; the second man is from heaven. [48] As was the man of dust, so also are those who are of the dust, and as is the man of heaven, so also are those who are of heaven. [49] Just as we have borne the image of the man of dust, we shall also bear the image of the man of heaven.

    [50] I tell you this, brothers: flesh and blood cannot inherit the kingdom of God, nor does the perishable inherit the imperishable. [51] Behold! I tell you a mystery. We shall not all sleep, but we shall all be changed, [52] in a moment, in the twinkling of an eye, at the last trumpet. For the trumpet will sound, and the dead will be raised imperishable, and we shall be changed. [53] For this perishable body must put on the imperishable, and this mortal body must put on immortality. [54] When the perishable puts on the imperishable, and the mortal puts on immortality, then shall come to pass the saying that is written:

      “Death is swallowed up in victory.”
      [55] “O death, where is your victory?
          O death, where is your sting?”


        [56] The sting of death is sin, and the power of sin is the law. [57] But thanks be to God, who gives us the victory through our Lord Jesus Christ.

    [58] Therefore, my beloved brothers, be steadfast, immovable, always abounding in the work of the Lord, knowing that in the Lord your labor is not in vain.
  \end{quote}
\subsection*{Notes}
\begin{itemize}
  \item{Death is inevitable, whether we choose to think and talk about it, or
  not.  Death sometimes comes in very sudden and surprising ways.  In fact we
  might even know all of these already, yet when death strikes close to home,
  we will still be affected.}
  \item{But from a Christian perspective, we know that death does not have
  the final say.  This is because of Jesus' death on the cross and His
  resurrection.}
  \item{As can be seen in v3 of today's text, ``Christ died for our sins''.
  What does this mean?  When God created everything, everything was good.
  There was no death etc at all.  But when we (Adam and Eve) turned away from
  God, who is the source of life, death came into the world.  So what Jesus
  did on the cross was to take on the penalty of our sin, which is death, so
  that now we no longer need to die.  In fact, Jesus was resurrected from the
  dead, which shows that He was greater than death itself.}
  \item{While Jesus' resurrection might seem hard to believe at first, Paul
  lists a few people who have seen the risen Lord; people like Peter, James,
  etc, and about $500$ more.  So for anyone who wants to verify Paul's
  statement here, they can just go and ask the people that Paul listed here.
  Moreover, anyone who is familiar with the OT would already know that
  whatever that has happened to Jesus has already been prophesied before in
  the past.  One key example is Isaiah 53, who prophesied both Jesus' death
  for our sins and also His resurrection and glorification.}
  \item{Ok but even if Jesus is raised from the dead, what does that have got
  to do with me?  The answer is in v22 of today's text; ``For as in Adam all
  die, so also in Christ shall all be made alive''.  Death gives way to new
  life in Christ.  This is the reason why Christian funeral services have so
  much hope.  When Christians pass on, they are just going on first.  There
  is a certain sadness because we still miss the people who went on ahead,
  but there is also anticipation for the big reunion in the future.}
  \item{But actually we don't have to wait until we die to start this new
  life.  We can have this new life in Christ the moment we believe in Jesus.
  As we walk with Jesus, our mind, character, conduct, etc will all change
  for the better.  I.e, we are a new creation!  Death gives way to new life
  in Christ.  We are also given a new purpose in life, through having a new
  King in our life.  A good life under an evil master won't last very long.
  On the other hand, a good life under Christ, our good king, will last
  forever.  So Christ, in his death, has destroyed ``every rule and every
  authority and power''.  Christ has destroyed all of the evil spiritual
  rulers in the world that are responsible for sin and death, and replaces
  their rule with His own good and perfect rule.  The minute we believe in
  Jesus, we place ourselves under Jesus' good and perfect rule.}
  \item{In the world now there is still evil and suffering, and sin and
  death, but all of these are just machinations of a defeated enemy, who will
  be totally destroyed when Christ comes again.  Right now, God lives in the
  life of Christians through His Holy Spirit that He gives us, and through
  His Holy Spirit, we can overcome all of these darkness that still exists.}
\end{itemize}