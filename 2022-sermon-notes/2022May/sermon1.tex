\section{1st May 2022: Where can we find fulfilment?}
\subsection*{Text: Ecclesiastes 1:12-2:26}
  \begin{quote}
    [12] I the Preacher have been king over Israel in Jerusalem.  [13] And I
    applied my heart to seek and to search out by wisdom all that is done
    under heaven.  It is an unhappy business that God has given to the
    children of man to be busy with.  [14] I have seen everything that is
    done under the sun, and behold, all is vanity and a striving after wind.

    [15] What is crooked cannot be made straight, and what is lacking cannot
    be counted.


    [16] I said in my heart, “I have acquired great wisdom, surpassing all
    who were over Jerusalem before me, and my heart has had great experience
    of wisdom and knowledge.” [17] And I applied my heart to know wisdom and
    to know madness and folly.  I perceived that this also is but a striving
    after wind.

    [18] For in much wisdom is much vexation, and he who increases knowledge
    increases sorrow.


    [1] I said in my heart, “Come now, I will test you with pleasure; enjoy
    yourself.” But behold, this also was vanity.  [2] I said of laughter, “It
    is mad,” and of pleasure, “What use is it?” [3] I searched with my heart
    how to cheer my body with wine—my heart still guiding me with wisdom—and
    how to lay hold on folly, till I might see what was good for the children
    of man to do under heaven during the few days of their life.  [4] I made
    great works.  I built houses and planted vineyards for myself.  [5] I
    made myself gardens and parks, and planted in them all kinds of fruit
    trees.  [6] I made myself pools from which to water the forest of growing
    trees.  [7] I bought male and female slaves, and had slaves who were born
    in my house.  I had also great possessions of herds and flocks, more than
    any who had been before me in Jerusalem.  [8] I also gathered for myself
    silver and gold and the treasure of kings and provinces.  I got singers,
    both men and women, and many concubines, the delight of the sons of man.

    [9] So I became great and surpassed all who were before me in Jerusalem.
    Also my wisdom remained with me.  [10] And whatever my eyes desired I did
    not keep from them.  I kept my heart from no pleasure, for my heart found
    pleasure in all my toil, and this was my reward for all my toil.  [11]
    Then I considered all that my hands had done and the toil I had expended
    in doing it, and behold, all was vanity and a striving after wind, and
    there was nothing to be gained under the sun.

    [12] So I turned to consider wisdom and madness and folly.  For what can
    the man do who comes after the king?  Only what has already been done.
    [13] Then I saw that there is more gain in wisdom than in folly, as there
    is more gain in light than in darkness.  [14] The wise person has his
    eyes in his head, but the fool walks in darkness.  And yet I perceived
    that the same event happens to all of them.  [15] Then I said in my
    heart, “What happens to the fool will happen to me also.  Why then have I
    been so very wise?” And I said in my heart that this also is vanity.
    [16] For of the wise as of the fool there is no enduring remembrance,
    seeing that in the days to come all will have been long forgotten.  How
    the wise dies just like the fool!  [17] So I hated life, because what is
    done under the sun was grievous to me, for all is vanity and a striving
    after wind.

    [18] I hated all my toil in which I toil under the sun, seeing that I
    must leave it to the man who will come after me, [19] and who knows
    whether he will be wise or a fool?  Yet he will be master of all for
    which I toiled and used my wisdom under the sun.  This also is vanity.
    [20] So I turned about and gave my heart up to despair over all the toil
    of my labors under the sun, [21] because sometimes a person who has
    toiled with wisdom and knowledge and skill must leave everything to be
    enjoyed by someone who did not toil for it.  This also is vanity and a
    great evil.  [22] What has a man from all the toil and striving of heart
    with which he toils beneath the sun?  [23] For all his days are full of
    sorrow, and his work is a vexation.  Even in the night his heart does not
    rest.  This also is vanity.

    [24] There is nothing better for a person than that he should eat and
    drink and find enjoyment in his toil.  This also, I saw, is from the hand
    of God, [25] for apart from him who can eat or who can have enjoyment?
    [26] For to the one who pleases him God has given wisdom and knowledge
    and joy, but to the sinner he has given the business of gathering and
    collecting, only to give to one who pleases God.  This also is vanity and
    a striving after wind.
  \end{quote}
\subsection*{Notes}
\begin{itemize}
  \item{Recap: Today's passage seems to point to Solomon as the `preacher'.
  Until the 16th century, both Jewish and Christian interpreters thought that
  Solomon was the author of Ecclesiastes.  The rejection of Solomonic
  authorship is due to linguistic factors etc.  Those who reject Solomonic
  authorship put Ecclesiastes in the time of the return from exile.  In
  Ecclesiastes, there are two voices, that of the narrator and that of the
  Preacher.  The narrator speaks in the prologue and the epilogue, and then
  in the middle we have the voice of the Preacher.  Here, the Preacher is
  speaking from the POV of Solomon.  The goal of the book of Ecclesiastes is:
  what is the meaning of life?}
  \item{Today's sermon point: what will it take for us to find fulfilment?
  What if we were the richest in Singapore, would that make us happy?  Etc.
  If even Solomon, the person with the most resources in the history of the
  world, couldn't find fulfilment through worldly means, what would that mean
  for us?}
  \item{In Eccl 1:12-18, we see the Preacher, in the voice of Solomon,
  telling us about the research he has done to try to find fulfilment.  In
  the Preacher's academic quest to find out what gives fulfilment, he left no
  stone unturned; he even checked if folly can give fulfilment.  In the end,
  he concluded that `it is an unhappy business that God has given to the
  children of men to be busy with', where `unhappy business' refers back to
  the curse of sin that God placed on creation (Gen 3:14-19).  In
  Eccl 1:16-18, we see that no matter how wise we are, because of sin in the
  world, there will always be things that we can't know and things we can't
  fix.  `What is crooked cannot be made straight, and what is lacking cannot
  be counted'.  In the end, much knowledge increases sorrow, since with more
  knowledge comes the growing realisation that knowledge cannot fix
  everything in this world.  Application: we can't find our meaning in life
  through academic means, through signing up for whatever enrichment courses,
  learning from all sorts of gurus, etc.}
  \item{In Eccl 2:1-17, we see Solomon (the Preacher's persona) continuing
  his quest to try to find fulfilment testing if pleasure is the source of
  fulfilment.  Here, the Preacher tries amusment, stimulation with wine,
  finishing many building projects for himself, possessions, collections,
  entertainment, sex, etc.  His conclusion was that all of those means to try
  to find fulfilment were futility.  If we aren't chasing God, we are merely
  chasing the wind.  In fact there is some sort of irony; the harder we try
  to find fulfilment, the emptier we end up.  If we are living in circles, we
  need to turn our lives over to Jesus, the wisdom and power of God.}
  \item{In Eccl 2:18-26, we see that no matter how hard we work to accumulate
  wealth and posessions, we cannot bring them with us to the grave.  The
  lesson here is paralleled with the parable of the rich fool in Luke 12.
  Some people amass great fortune not for their benefit, but for their
  children, but more often than not such great fortunes are very easily
  squandered.  Moreover, such great fortunes for the kids also destorys
  relationships.  Ultimately, hard work is not an end to itself, posessions
  aren't an end to itself, but they are just things that point to God. }
  \item{In the end, in view of eternity, our life on earth is super short in
  comparison.  I.e, if eternity is an infinite line, then our time here is a
  dot (a set of measure zero).  Hence, we should live our lifes with eternity
  in view.  Things on this earth are just meant to point us to eternity.
  True happiness, true joy, is found only in God.  If we treat the things in
  life as an ends in themselves, we will never find fulfilment.  To dethrone
  God is to lose the key to life.  We may pursue many human endeavors, but
  all we will find is vanity, the lack of ultimate fulfilment.  But if we
  treat the things in life as things that point to God, we will find true
  fulfilment in God.  If we enthrone God in our lives, we will enter into the
  fullness of life (John 17:3, Psalm 16:11).  }
  \item{My reflections: in life, we have enough good to point us to God in
  thanksgiving, but also some brokenness due to sin that reminds us that
  these good things depend wholly on God and that these good things aren't an
  ends to themselves.}
\end{itemize}