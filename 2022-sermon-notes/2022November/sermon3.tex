\section{20th November 2022: Christ the King}
\subsection*{Text: Colossians 1:11-20}
  \begin{quote}
    [11] being strengthened with all power, according to his glorious might,
    for all endurance and patience with joy; [12] giving thanks to the
    Father, who has qualified you to share in the inheritance of the saints
    in light.  [13] He has delivered us from the domain of darkness and
    transferred us to the kingdom of his beloved Son, [14] in whom we have
    redemption, the forgiveness of sins.

    [15] He is the image of the invisible God, the firstborn of all creation.
    [16] For by him all things were created, in heaven and on earth, visible
    and invisible, whether thrones or dominions or rulers or authorities—all
    things were created through him and for him.  [17] And he is before all
    things, and in him all things hold together.  [18] And he is the head of
    the body, the church.  He is the beginning, the firstborn from the dead,
    that in everything he might be preeminent.  [19] For in him all the
    fullness of God was pleased to dwell, [20] and through him to reconcile
    to himself all things, whether on earth or in heaven, making peace by the
    blood of his cross.
  \end{quote}
\subsection*{Notes}
\begin{itemize}
  \item{In the time of Paul, Colossae wasn't important or prominent anymore.
  The church at Colossae was planted by Paul's co-worker, Epaphras.  Epaphras
  brought good news to Paul, that the Colossians were growing, but Epaphras
  also brought bad news, that there was some sort of syncretism creeping in
  (Jewish/pagan ascetism and philosophy).  So Paul had to write to them, to
  tell them that these syncretism teachings are bad, and that true wisdom is
  only found in Christ.}
  \item{Paul prays that the Colossians will have patience and endurance, to
  deal with both the vissicitudes of life and to deal with other people (the
  two words patience and endurance have these nuance in Greek).  Endurance
  and patience with joy displays God's glory.  The anchor that allows us to
  have such patience and endurance is the supremacy and pre-eminence of
  Christ.}
  \item{Three concepts of Christ as king: 
  \begin{enumerate}
    \item{Christ is supreme over all things. All are subject to Him.}
    \item{Christ is saviour.  Christ rescued us from the dominion of darkness
    into His Kingdom.  Jesus, who took on human nature, is the start of New
    Creation, God's true people, the new Israel.}
    \item{Christ is sufficient.  The fullness of God, all of God's
    attributes, dwell in Him.  We have all we need in Christ, and seeking or
    succuming to anything else will only lead to impowerishment and bondage.}
  \end{enumerate}}
  \item{Christ is the highest of all Kings in the earth.  Christ is the image
  of the invisible God, which means that Jesus reveals all of the attributes
  of God.  All things were created in Christ, through Christ and for Christ.
  Christ is before all things, even before all creation, and hence He is
  pre-eminent.  Even though creation is tarnished by sin, it does not fall
  apart because Jesus holds it together.  For the young believers in Colossae
  who were looking for something more in their faith, this supremacy of
  Christ would address their concerns.  For us, knowing that Christ rules
  over all gives us comfort and helps us not to be anxious to anything or
  anyone, be it other people or even forces of darkness.  Furthermore, since
  we were created in Christ, through Christ and for Christ, Christ will know
  us the best.  Christ knows us not only because he took on our human nature,
  but also because He created us.}
  \item{And because Christ knows us well when we were made by Him and through
  Him and for Him, He knows how exactly to save us from our imprisonment to
  sin and darkness.  And that was the cross.  So God saved us from the domain
  of darkness through Christ's death and resurrection, and after that,
  transferred us to the kingdom of His Son.  But sadly, some of us forget
  these truths and hence we trap ourselves in dark areas of life, and live in
  a defeated manner.  Paul teaches us that the saviour has conquered all of
  the powers of darkness, and that we are subject only to the loving King.
  We might still be tempted by Satan and still have to do battle with him,
  but we are no longer under Satan's control.}
  \item{And when Jesus took on human nature, He was the start of the new
  creation.  As the Son of God, Jesus had a relationship with the Father.  A
  similar close relationship was intended for us and God when God created us
  too.  But sin came into the world.  So, to fulfil His original plan, God
  sent Jesus to take on human nature to be the start of the new creation.
  And when we put our faith in Jesus, we too are part of the new creation.}
  \item{Since the fulness of God dwells in Christ, we have everything we need
  to deal with whatever life throws at us when we look to Christ.  The
  Colossians wanted to add random stuff to their faith to as to make it more
  mature.  We might fall into similar temptations too.  For example, when God
  doesn't hear our prayer, we might be tempted to pray to someone else.  Or
  when we seek our security in earthly things because we don't have enough
  faith in Jesus.  All of that is wrong, because in Christ is everything that
  is sufficient. In Christ we have all we need for abundant life, now and forever. Seeking something else will only lead to poverty and bondage. }
\end{itemize}