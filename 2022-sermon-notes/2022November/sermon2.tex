\section{13th November 2022: A missional life}
\subsection*{Text: 1 Thessalonians 1:4-10}
  \begin{quote}
    [4] For we know, brothers loved by God, that he has chosen you, [5]
    because our gospel came to you not only in word, but also in power and in
    the Holy Spirit and with full conviction.  You know what kind of men we
    proved to be among you for your sake.  [6] And you became imitators of us
    and of the Lord, for you received the word in much affliction, with the
    joy of the Holy Spirit, [7] so that you became an example to all the
    believers in Macedonia and in Achaia.  [8] For not only has the word of
    the Lord sounded forth from you in Macedonia and Achaia, but your faith
    in God has gone forth everywhere, so that we need not say anything.  [9]
    For they themselves report concerning us the kind of reception we had
    among you, and how you turned to God from idols to serve the living and
    true God, [10] and to wait for his Son from heaven, whom he raised from
    the dead, Jesus who delivers us from the wrath to come.
  \end{quote}
\subsection*{Notes}
\begin{itemize}
  \item{Today's main point: being a missional people wherever God has placed
  us.  I.e, to be effective salt and light.  Being missional means actually
  doing missions right where we are, which means adopting the posture of a
  missionary, learning and adapting to the culture around us while remaining
  biblically sound.}
  \item{There are lots of non-Christians around us, which means that wherever
  we are, we have opportunities for missions.  But first things first - a
  misisonal life flows out of a life transformed by God.  Until and unless
  God transforms us, we lack the core ingredient to be a truly missional
  people.  A missional life is all about loving, serving, and sharing the
  gospel of Christ to people where God has planted us.}
  \item{A missional people is a people who are called to be \textbf{loved}
  and \textbf{chosen} by God.  Paul was trying to remind the Thessalonians
  the truth of their acceptance from God.  We see that Paul started his
  letter with reminding the Thessalonians of their identity in God.  That is
  important, because we need to know who we are and to whom we belong before
  we can be missional.  Just like the Thessalonians, we too are loved and
  chosen by God.  We know God's love because while we were yet sinners,
  Christ died for us.  Knowing our identity as God's beloved children and
  God's chosen agents to be His ambassadors, we are enabled to love and serve
  those around us.  Christ's love compels us to preach the gospel, as was
  mentioned in 2 Corinthians 5:14-21.}
  \item{Sometimes, we forget our identity in Christ, and that causes us to be
  ineffective missionaries.  We all need to regain a fresh appreciation of
  God's love and calling for us.  That can be done through constant
  thanksgiving for God's love through His provision of our daily bread and
  our spiritual blessings.  We can also regain a fresh appreciation by
  remembering our salvation story.}
  \item{We are also a people empowered by the Holy Spirit.  We see how Paul
  and the Thessalonians were thus empowered by the Holy Spirit in verses 5
  and 6 of our text today.  Regardless of whatever means we use to share the
  gospel, the Holy Spirit is the one who empowers the messenger.  For
  example, the Holy Spirit gives us the power to live a holy life, the Holy
  Spirit gives us the words to say at a particular moment, and the Holy
  Spirit might sometimes give us power to work miracles.  The Holy Spirit
  also gives us supernatural joy in the midst of affliction.  All of these
  help testify to the truth of the gospel message.  By nature, it is
  difficult for us to love others, because we are all sinful.  By nature, it
  is difficult for us to by joyful in the midst of trials.  We need the Holy
  Spirit to help us to do things that are not natural to our sinful nature.
  We need to remember this truth and to always pray for the Holy Spirit to
  work in us.}
  \item{Being empowered by the Holy Spirit involves daily surrendering our
  will to God and asking the Holy Spirit to fill and control us as his
  missional people.  God desires that we live our lives not depending on
  ourselves, but by depending on His Holy Spirit.  We must be constantly
  aware of this fact, and we must be constantly submitting our will to God.}
  \item{Being missional also means that we are called to be a people who
  shares and lives out the gospel of Christ.  We share the gospel with words
  and conviction.  That is what Paul did to the Thessalonians (v5), and that
  is what the Thessalonians did (v8).  We do not wait for people to stumble
  on the gospel message themselves, but we use words to share the gospel.}
  \item{Success in witnessing is simply taking the initiative to share Christ
  in the power of the Holy Spirit and leaving the results to God.}
  \item{Practical methods to share the gospel: learn one method (e.g four
  spiritual laws), and learn how to share your testimony of how God has
  worked in your life.}
  \item{Not only do we use words to share the gospel, we live out the gospel
  through actions in life too.  We see that is what the Thessalonians did
  (v6-8).  Note that sharing the gospel in words must go hand-in-hand with
  living out the gospel through our lives.  We need both!  We can't just live
  a holy life and hope people infer the Christian message without us saying a
  word...  }
\end{itemize}