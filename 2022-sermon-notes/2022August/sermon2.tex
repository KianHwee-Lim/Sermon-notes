\section{14th August 2022: Counsel for living when times are uncertain}
\subsection*{Text: Ecclesiastes 11:1-6}
  \begin{quote}
    [1] Cast your bread upon the waters,
        for you will find it after many days.
    [2] Give a portion to seven, or even to eight,
        for you know not what disaster may happen on earth.
    [3] If the clouds are full of rain,
        they empty themselves on the earth,
    and if a tree falls to the south or to the north,
        in the place where the tree falls, there it will lie.
    [4] He who observes the wind will not sow,
        and he who regards the clouds will not reap.

    [5] As you do not know the way the spirit comes to the bones in the womb
    of a woman with child, so you do not know the work of God who makes
    everything.

    [6] In the morning sow your seed, and at evening withhold not your hand,
    for you do not know which will prosper, this or that, or whether both
    alike will be good.
  \end{quote}
\subsection*{Notes}
\begin{itemize}
  \item{Wisdom is important in the life of Christians.  Hence, wisdom
  literature is important for Christians, not just in our thought life, but
  also in our practical life.  Proverbs contains ``conventional wisdom'',
  such as ``a soft answer turns away wrath, but a harsh word stirs up anger''
  (Proverbs 15:1).  Conventional wisdom works like this; when you observe
  swans, if $99$ swans are white, the $100$th would most likely be white too.
  The question is though, what if the $100$th swan is black?  In most cases,
  a gentle answer can de-escalate situations, but what if it doesn't?  It is
  part of wisdom itself to know the limits of conventional wisdom itself, and
  hence we see the limits of conventional wisdom in Job/Ecclesiastes.  This
  is important, because conventional wisdom might not always be applicable in
  our sinful, broken and complicated world.}
  \item{The theme for today is ``wise living in times of uncertainty''.  In
  verses 5-6, the phrase ``you do not know'' appears three times.  In fact,
  chapter 11 is not the first time the theme of ``not knowing'' has appeared
  in Ecclesiastes.  Previously, the theme of how Man ``cannot find out what
  God has done'' (Ecclesiastes 3:11), how Man ``does not know what is to be''
  (Ecclesiastes 8:17) and how Man ``cannot find out the work that is done
  under the sun'' (Ecclesiastes ?:??), and how Man ``does not know his time''
  (Ecclesiastes 9:12).  The uncertainty that the author of Ecclesiastes
  refers to is not uncertainty because of carelessness.  The uncertainty is
  more of how sometimes we can do everything we can, but we still can't find
  out.}
  \item{The key to Ecclesiastes 11:1-6 is found in the latter half of verse
  6; one simply does not know what God is going to do in the future.  One
  cannot be sure of what God is going to do.  So how then should one live?
  We can take all the safeguards humanly possibly, but we still can't assure
  ourselves of the future.}
  \item{Though there is this inherent uncertainty in life, there is still
  some advice that we can heed.  From verse 4, we see a caution against
  idleness/frozenness/inaction.  I.e, it cautions against
  ``analysis-paralysis''.  Doesn't make sense for us to not plant seeds
  because we keep trying to overanalyse the weather.  Though we can't predict
  the weather or whether or crops will receive rain, we should still plant
  seeds.  Though an over-analyser is not a strictly speaking a sluggard (more
  of a paralysed worrywart), the end result is the same (c.f Proverbs 20:4).
  Instead of over-analysing, the answer is in verse 6; in the face of
  uncertainty, we must work, and in fact, we do even more than usual so as to
  extend our safety net.  Not only do we sow our seed in the morning, but we
  also sow our seed in the evening.  Sowing our seed in the morning works
  fine in times of certainty, but in times of uncertainty, we should do more
  than usual.}
  \item{Of course we can go to the extreme and work until we drop dead so as
  to establish the security that we need in our heart.  Yet as was previously
  said, it is impossible to have complete security, because we will never be
  able to fully predict what God does in the future.  We know that clouds
  full of rain will empty themselves (v3), but yet this emptying is not under
  human control.  And as for the latter half of verse three, it is saying
  that we do not know where a tree is going to fall; it is not under human
  control.  ``North and south'' here is abit like ``high and low''.}
  \item{So the balance is this; we do what is responsible in uncertain times,
  which might include working extra, but we also should not try to control
  everything because that is impossible.  Instead of trying to control
  everything, we are to \textbf{trust God}.}
  \item{So a nice term for today is ``the grace of not knowing''.  The grace
  of not knowing what lies ahead, for good or for will, because it frees us
  from the compulsion to control our situation, to secure our own advantage
  in everything and kick ourselves -or curse God- when we guess wrong.  Hence
  while for some of us we might need to do more in times of uncertainty
  (which is responsible), for some of us we actually need to do less and
  trust God more.}
  \item{In v1-2, the idea of ``casting our bread upon the waters'' has two
  main interpretations.  One interpretation is about how we should make wise
  investments.  And when we make these investments, we should diversify our
  portfolio, make ``seven'' or ``eight'' investments, because we don't know
  where disaster will strike.  So here, ``casting your bread'' would be how
  we have ships filled with bread for foreign investment.  Hence, this view
  is more self-focused.}
  \item{In v1-2, the other interpretation is about acts of charity.  Times
  are uncertain not only for you, but also for others.  Hence, this view is
  more other-centred.  When we ``live by the grace of not knowing'', we are
  pushed to also give of ourselves to others, because its like ``since we
  can't predict what will happen to us in the future anyway, we might as well
  go and help others, since they are also facing uncertain times''.  This is
  the favored interpretation of the preacher, and is my favored
  interpretation too.}
\end{itemize}