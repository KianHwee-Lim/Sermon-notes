\section{28th August 2022: Hear it and take it to heart}
\subsection*{Text: Revelation 1:1-8}
  \begin{quote}
    [1] The revelation of Jesus Christ, which God gave him to show to his
    servants the things that must soon take place.  He made it known by
    sending his angel to his servant John, [2] who bore witness to the word
    of God and to the testimony of Jesus Christ, even to all that he saw.
    [3] Blessed is the one who reads aloud the words of this prophecy, and
    blessed are those who hear, and who keep what is written in it, for the
    time is near.

    [4] John to the seven churches that are in Asia:

    Grace to you and peace from him who is and who was and who is to come,
    and from the seven spirits who are before his throne, [5] and from Jesus
    Christ the faithful witness, the firstborn of the dead, and the ruler of
    kings on earth.

    To him who loves us and has freed us from our sins by his blood [6] and
    made us a kingdom, priests to his God and Father, to him be glory and
    dominion forever and ever.  Amen.  [7] Behold, he is coming with the
    clouds, and every eye will see him, even those who pierced him, and all
    tribes of the earth will wail on account of him.  Even so.  Amen.

    [8] “I am the Alpha and the Omega,” says the Lord God, “who is and who
    was and who is to come, the Almighty.”
  \end{quote}
\subsection*{Notes}
\begin{itemize}
  \item{The book of Revelation is a fascinating book, and is often abused by
  unscrupulous teachers who take advantage of people's curiosity about the
  future.  We should not let current world events affect the way we interpret
  the text, we should let the text speak for itself.}
  \item{These eight verses set the tone for how we should understand the rest
  of the book.  Three points: firstly, the book of Revelation is a gift to
  us.  Secondly, these eight verses give us the foundational theology for
  understanding the book.  Lastly, these eight verses give us the testimony
  of the people of God, about what we should do after hearing the
  Revelation.}
  \item{From verse 1, we see that Revelation is a gift of God to Jesus, and
  then from Jesus to John, and then from John to us.  Since Revelation is
  God's gift to us, we should be able to understand it and take to heart all
  that the Lord is saying in the book of Revelation.  God doesn't give us
  gifts we don't understand.  We might not understand it fully, but we will
  know enough.}
  \item{This revelation is about Jesus Christ, from verse 1.  This means that
  a lot of the book is about Jesus.  We should not concern ourselves too much
  with things like ``rapture'', ``tribulation'', etc at the expense of
  forgetting about Jesus.  }
  \item{This book has a very strong sense of the second coming of Jesus
  Christ.  This book is given us for us to prepare ourselves in light of the
  coming of the last days.  Revelation is given to us a guidance for the last
  days, when Jesus' coming is imminent.}
  \item{From ESV, the book of Revelation also comes with a blessing even for
  the one who read it and those who hear, and most important for those who
  keeps what is written in the book.  We are to hear the words of the book,
  and take the words of the book to heart.  This is similar to Psalm 1;
  blessed is the man who does not walk in the way of sinners, who does not
  sit in the seat of scorners, but who meditates on the Law of the Lord day
  and night.  Similarly, the one who reads, hears and takes to heart the book
  of Revelation will be blessed.}
  \item{So, how are we to face the last days?  Thought experiment time;
  imagine for yourselves a future that you really want to come to pass.  Yet
  our imaginations about the future might be wrong, hence it is helpful for
  Jesus to guide our futures.  We think/imagine things that are good for us
  that aren't actually good for us.  What is good for us is God, since God is
  the ultimate good, yet most of the time our imaginations about the future
  doesn't have God in them.  The book of Revelation stresses this by spending
  five verses on giving titles for God, to expand our idea of God.}
  \item{ From verse 4, we see that God is the One who is, who was, and who is
  to come.  This is a throwback to the story of Moses at the burning bush,
  where God told Moses that ``I am who I am''.  God doesn't need any
  definition, He is the one who is the ultimate reality from which all
  reality is defined, He is who He is.  This also means that God is eternal,
  since God who was.  And lastly, God is the one who is to come; present
  tense with a future looking aspect, which means that God's future is
  defined by how He will come for us.}
  \item{God is also called the ``Alpha'' and the ``Omega'', which means that God is the start and the end of reality (since $\alpha$ and $\Omega$ are the first and last greek alphabets).}
  \item{Jesus Christ here is called the faithful witness. }
  \item{The seven spirits here refers to the Holy Spirit.  Seven is the
  number of perfection in OT, which means that the seven spirits talk about
  how the Spirit is perfect.  We see a clue about how the seven spirits refer
  to the Holy Spirit by referring the Revelation 4 which talk about the seven
  spirits together with the seven lights and the seven eyes, which is a
  throwback to the same analogy in Zechariah 4.  In Zechariah, we see that
  the people of God were very discouraged because they were facing lots of
  troubles, and God encouraged them that they will succeed not by might or by
  numbers, but by His Spirit.  And similarly, for us today, our Church might
  be facing lots of pressures, and we might seem to be struggling, yet we
  know that in the end, we will succeed by the Holy Spirit.}
  \item{With all of the theological foundations in place for us to understand
  the future, what is the purpose for us to understand this future?  As in,
  what should we do after we are able to understand the future?  The idea is
  that after we understand the book, we will be prepared to give testimony of
  God in the last days.  The last days will be challenging, it will be
  urgent, and it means that we also have to be decisive in these last days.
  Tracts and social media are not enough; we need people who are actually
  able to give testimony of God through their words and more importantly
  their lives.} 
  \item{In the modern era, when there is so much competition of
  ideas, how do people know whom to trust?  Sometimes people compare
  different ideas by comparing the logic behind the ideas to see which idea
  is correct, but most of the time people compare different ideas by
  comparing whether the different speakers are trustworthy.  And we can prove
  ourselves to be trustworthy speakers if we live consistently with what we
  say about Jesus.  And eventually, though we might not be able to win all
  the verbal arguments, we might be able to convince people through our
  lives.  And this testimony we give through our lives are eternal, though
  our lives on earth are not, because our lives in heaven are eternal, and at the last days, we will be vindicated, though we might have passed on.  }
\end{itemize}