\section{30th January 2022: Living wisely before God}
\subsection*{Text: Proverbs 3:1-12}
\begin{quote}
  [1] My son, do not forget my teaching,
        but let your heart keep my commandments,
  [2] for length of days and years of life
      and peace they will add to you.


  [3] Let not steadfast love and faithfulness forsake you;
      bind them around your neck;
      write them on the tablet of your heart.
  [4] So you will find favor and good success
      in the sight of God and man.


  [5] Trust in the LORD with all your heart,
      and do not lean on your own understanding.
  [6] In all your ways acknowledge him,
      and he will make straight your paths.
  [7] Be not wise in your own eyes;
      fear the LORD, and turn away from evil.
  [8] It will be healing to your flesh
      and refreshment to your bones.


  [9] Honor the LORD with your wealth
      and with the firstfruits of all your produce;
  [10] then your barns will be filled with plenty,
      and your vats will be bursting with wine.


  [11] My son, do not despise the LORD’s discipline
      or be weary of his reproof,
  [12] for the LORD reproves him whom he loves,
      as a father the son in whom he delights.
\end{quote}
\subsection*{Notes}
\begin{itemize}
  \item{Story of the lazy grasshopper and the hardworking ant: idleness leads
  to hunger.  This was a very famous fable in ancient Greece to teach
  dilligence.  In Proverbs, there is a similar story in Proverbs 6:6-8.}
  \item{Every culture has a wisdom tradition.  This makes sense, because
  wisdom helps us to live a good life.}
  \item{Peace in verse 1 is "shalom" in Hebrew, i.e there is a connotation of
  wholeness and abundant welfare.}
  \item{God makes wisdom available to all, God created the earth by wisdom
  (Proverbs 3:19-20).  I.e, God's wisdom is baked into His creation.  The
  created world reflects God's wisdom, and hence it makes sense that we can
  learn wisdom by looking at God's created order (e.g by looking at the ant).
  This is regardless of whether one believes in God or not.  This is nothing
  but the generosity of God, who freely gives his creatures common grace.}
  \item{However, the wisdom as described above is incomplete.  I would add
  that because of sin, the wisdom as described above is fallen; now we have
  wicked people prospering for example.  Sin has affected God's creation such
  that the wisdom baked into God's creation is abit disrupted, and also sin
  has affected us such that we are partially blind to all the wisdom in God's
  creation too.  Hence, all human wisdom must be completed and reframed by
  God's revelation.  To truly live the good life, we are to be like God, and
  we are to submit to God.
  \begin{itemize}
    \item{To be like God, we are to acquire the character of God.  From v3-4,
    we must possess steadfast love and faithfulness, in order to find success
    before the eyes of God and Man.  But steadfast love and faithfulness are
    qualities of God (Exodus 34:6).  This makes sense, because we are made in
    the image of God, and hence reflecting God's character is what we are
    supposed to do anyway.  Hence just as how God shows steadfast love and
    faithfulness to us, we should show steadfast love and faithfulness to
    others.  Of course, showing steadfast love and faithfulness to others
    might lead to us sacrificing something on our end.  However, living the
    good life is not about maximising our self pleasure, but it is about the
    collective shalom of all, which is God's original plan for humanity
    anyway.}
    \item{To submit our lives to God, we need to submit our thinking, our
    possessions, and our negative life experiences.
    \begin{itemize}
      \item{To submit our thinking, we must first note that we are finite
      creatures with a finite understanding.  The worst thing for us is to be
      wise in our own eyes, because if that is the case, we can't grow
      intellectualy.  E.g Proverbs 26:12.  Now, if we note that God is
      infitely wise than us, then we ought not to be wise in our own sight,
      we must trust that God's commandments are far wiser than our own
      understanding of the world.  As Paul said, God's ``foolishness'' is
      infitely wiser than human wisdom.  An example is the flawed human
      ``wisdom'' of ``can do whatever, just don't get caught''.  But God's
      wisdom is Romans 13, submitting to lawful authority.  We are to ``trust
      in the LORD with all your heart, and do not lean on your own
      understanding...''}
      \item{To submit our posessions, we see that in verses 9 and 10 of our
      text.  This is not prosperity gospel, it is not about how when we give
      more, we will receive more.  This wisdom has the context of Israel's
      old covenant, where giving is part of the Law, and obeying the Law is
      about love for God and faithfulness for God.  And as per the OT
      promises, when Israel is faithful to God, God will bless Israel with
      material possessions in this world.  For us in the NT, even though the old
      sacrificial system is not for us, this principle is for us too, just
      that it is transformed; see Luke 12:33.  We should not expect God to
      bless us with material wealth when we submit our possessions to God, because God gives us things far better than material wealth; God gives us treasure in heaven.}
      \item{To submit our negative life experiences to God, we first look at
      Proverbs 13:24 too.  Just like how parents discipline their children,
      God disciplines us too.  However, not all suffering we experience is
      God's discipline/punishment of us.  See Ecclesiastes 8:14 and Job.
      Some of the suffering we experience is a result/consequence of sin.  It
      is wiser to regard suffering not necessarily as God's punishment for us
      (unless God has clearly revealed this to us), but as something that
      will draw us to God.  Suffering itself is not good, but God can bring
      good out of our suffering when we bring it to God.  This is what it
      means to submit our negative life experiences (our suffering) to God.
      In fact, suffering sometimes can lead to greater holiness and deeper
      communion with God, which is part of living the good life, and hence
      paradoxically, we might find greater shalom with God in our suffering.}
    \end{itemize}}
  \end{itemize}}
  \item{The ultimate portrait of wisdom for us is Jesus Christ.  Jesus Christ
  was completely like God, and fully submitted Himself to God.  What Jesus
  did seemed foolish by worldly standards, but it is infitely wise by God's
  standard, and hence God has highly exalted Jesus Christ.  And now we can
  have the mind of Christ through the Holy Spirit.  Let us seek this, and
  hence get the ultimate shalom.}
\end{itemize}