\section{23rd October 2022: Are you a lukewarm Christian?}
\subsection*{Text: Revelation 3:14-22}
  \begin{quote}
    [14] “And to the angel of the church in Laodicea write: ‘The words of the Amen, the faithful and true witness, the beginning of God’s creation.

    [15] “‘I know your works: you are neither cold nor hot. Would that you were either cold or hot! [16] So, because you are lukewarm, and neither hot nor cold, I will spit you out of my mouth. [17] For you say, I am rich, I have prospered, and I need nothing, not realizing that you are wretched, pitiable, poor, blind, and naked. [18] I counsel you to buy from me gold refined by fire, so that you may be rich, and white garments so that you may clothe yourself and the shame of your nakedness may not be seen, and salve to anoint your eyes, so that you may see. [19] Those whom I love, I reprove and discipline, so be zealous and repent. [20] Behold, I stand at the door and knock. If anyone hears my voice and opens the door, I will come in to him and eat with him, and he with me. [21] The one who conquers, I will grant him to sit with me on my throne, as I also conquered and sat down with my Father on his throne. [22] He who has an ear, let him hear what the Spirit says to the churches.’”
  \end{quote}
\subsection*{Notes}
\begin{itemize}
  \item{What does lukewarm even mean? What is Jesus trying to say with the word ``lukewarm''? Three points:
  \begin{enumerate}
    \item{The assessor: who is the one doing the assessment?}
    \item{The assessment: what is the assessment given to the church in Laodicea?}
    \item{The advice: after the assessment, what is the advice that Jesus gives?}
  \end{enumerate}
  This is the last letter, and hence it is quite significant. In mose letters to the churches there is both commendation and rebuke. Except for Philadelphia and Smyrna, where there is only commendation. Here, to Laodicea, there is only rebuke.}
  \item{Point 1: The assessor.
  \begin{itemize}
    \item{The letter starts with Jesus telling the church his titles.  Here
    there are three titles:
    \begin{enumerate}
      \item{The Amen}
      \item{Faithful and true}
      \item{The beginning of God's creation}
    \end{enumerate}}
    \item{Amen means something sure, something firm, something valid.  So
    when we end our prayers with amen, we are saying to God “let our prayers
    be sure, be firm, be valid”.  So when Jesus calls himself the Amen, he is
    saying that he is sure, he is firm.  Steadfast.}
    \item{Next, Jesus is the faithful and true witness.  The word witness and
    amen links to truth.  So what Jesus is saying is true and hence worth
    listening to.}
    \item{Lastly, “beginning of Gods creation” refers to Jesus being the
    creator.  Jesus is the Word of the Father, eternally begotten, through
    whom all things are made.  Hence, since Jesus is the creator, what he
    says about the Laodiceans is significant because he created them.}
    \item{Jesus reminds us of these titles to prompt us to realise.
    Sometimes we know these titles but we forget their significance.  We need
    to hear them anew to learn that we must submit to Jesus’ voice.}
  \end{itemize} }
  \item{Point 2: The assessment.
  \begin{itemize}
    \item{Similar to the other letters, we also have Jesus saying “i know your works”. The church in Laodicea is lukewarm. But what does lukewarm mean? The key to understanding “lukewarm” in this text is to look at the “for” in verse 17.}
    \item{In verse 17, we see that the church in Laodicea had a particular
    assessment of themselves.  They say that they are rich, they say that
    they have prospered (so they got the wealth themselves), and lastly they
    say that they need nothing.  The last one is particularly fatal; when one
    thinks that he is self-sufficient, he thinks that he doesn’t need Jesus.
    Then he is not open to receiving from Jesus.  This is human nature 101;
    when we are full of ourselves, we won’t listen to what other people say
    and we won’t learn anything.  That is pride.  Compare with the
    beautitudes: “blessed are the poor in spirit”.}
    \item{In contrast to their self assessment, Jesus, the faithful and true, told them about their actual state. They are poor and pitiable, wretched, blind and naked.}
    \item{In those days, Laodicea was a very rich city because of their wool trade. They had some sheep that produced glossy black wool lol. Also, they were a city famous for eye doctors (ophthalmologists). The city was so rich that when they had an earthquake, they told their roman overlords that they didnt need any help. }
    \item{Here we see that Jesus' rebuke was especially relevant to the city of Laodicea; naked <-> wool trade, blind <-> eye doctor.}
    \item{So what exactly is lukewarm?  From our archeological evidence, we
    see that Laodicea was on a plateau.  They had no natural sources of
    water.  Their water was piped, and hence they had neither hot water or
    cold water (the water would equilibrate to a lukewarm steady state in the
    pipes).  And lukewarm water back then had no use (can’t make cold food
    nor hot food).  So Jesus is saying that the Laodiceans in their current
    state is of no use to him.  We see that despite the city’s riches, they
    could not solve this water problem themselves.  Similarly, despite the
    city’s supposed self-sufficiency, they cant solve their own spiritual problems.}
  \end{itemize}}
  \item{Point 3: The advice.
  \begin{itemize}
    \item{Since the Laodiceans can't solve their own spiritual problems
    themselves, they need to buy spiritual things from Jesus, like gold
    refined by fire (not like the gold they have they will decay), white
    garments (not like their earthly garments), and eye balm (not like what
    their eye doctors give).}
    \item{Last piece of advice given is that Jesus says that he is at the
    door and knocking.  He wants us to hear his voice and open the door for
    him to come in.  Why is Jesus knocking?  Because my heart's door is
    closed.  But why is my door closed?  Either because of hardness of heart,
    or because of shame.  If it is the latter, then that is silly because we
    must know that we can't deal with shame ourselves.  We need Jesus to help
    us deal with sin and shame, it is wrong to think that we need to deal
    with sin and shame first before we can open our heart to Jesus.  We are
    not self-sufficient.  We need Jesus.}
    \item{And when we open our heart to Jesus, Jesus will fellowship with us, and we will sit with Jesus in glory in the last days.}
  \end{itemize}
  }
\end{itemize}