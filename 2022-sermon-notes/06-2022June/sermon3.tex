\section{19th June 2022: Being an authentic community}
\subsection*{Text: James 5:13-16}
  \begin{quote}
    [13] Is anyone among you suffering?  Let him pray.  Is anyone cheerful?
    Let him sing praise.  [14] Is anyone among you sick?  Let him call for
    the elders of the church, and let them pray over him, anointing him with
    oil in the name of the Lord.  [15] And the prayer of faith will save the
    one who is sick, and the Lord will raise him up.  And if he has committed
    sins, he will be forgiven.  [16] Therefore, confess your sins to one
    another and pray for one another, that you may be healed.  The prayer of
    a righteous person has great power as it is working.
  \end{quote}
\subsection*{Notes}
\begin{itemize}
  \item{This is a continuation of the church's sermon series on being a
  compelling community (first three sermons were preached earlier in the
  year).  Today's text is on being an authentic community.}
  \item{Three `S' for today: suffering, sickness and sin.  How does a church
  deal with these three things in a way that shows faith at work?}
  \item{Without the context of the letter, our text for today sounds like
  standalone instructions that can easily be misinterpreted.  For example,
  verse 13 does not apply to all forms of suffering, such as if we commit a
  crime and go to jail but we pray that the jail will burn down or something
  so we don't have to go.}
  \item{The content of James' letter suggests that the church was more or
  less established, since there were elders and deacons.  Moreover, it is
  said in Galatians that James was ministering to the Jews.  Hence, it is
  quite probable that the letter of James was written to the Jews who were
  dispersed from Jerusalem after the stoning of Stephen.  The letter of James
  starts with how we should stand fast in the face of tribulation.  Towards
  the end of James' letter, James picks up two groups of people for
  condemnation.  The first group are people who are proud (end of chapter 4),
  and the next group are the rich who oppress the poor (start of chapter 5).
  Right after, James exhorts his hearers to be patient in the face of
  tribulation and not grumble about each other.  These are what James was
  talking about right before our text for today.}
  \item{Hence, our text for today on ``suffering'' (v13) is more about how we
  should have faith in the face of suffering and use the time to pray,
  instead of boasting in arrogance or grumbling.  And we should definitely
  not be the source of suffering like those who oppress the poor.  And
  lastly, we should not rashly make oaths in our suffering; we should pray in
  faith, but we should not say things like ``God if you do $X$ for me, then I
  will do $Y$ for you''.  We should pray for strength to not give up, pray
  for patience in the face of suffering, pray for comfort and help, pray that
  our trials will deepen our relationship with God.}
  \item{That being said, it is not wrong to pray for God's provision for
  ``non-spiritual things'', such as job opportunities etc.  It is not that we
  should only pray for spiritual things.  But we can do seek these
  ``non-spiritual'' things in a way that is wrong, and we must be aware of
  that.}
  \item{As for annointing the sick with oil, this is an ancient custom in
  those days.  These days, we can do things like sending a fruit basket etc.
  But why call the elders?  The answer to that is verse 15.  We note that
  first, James had talked about the steadfastness of Job.  Secondly, as per
  John 9, it is normal back then to automatically tie severe sickness with
  sin.  Hence, James called for the elders (who are spiritually mature)
  because he was afraid of how non-spiritually mature people would be like
  Job's three friends and talk rubbish such as tieing the person's sickness
  with his sin.  The elders were called to assess the situation and to
  discern if the sickness is due to sin, or not.  And whether the ``prayer
  that saves'' refers to saving the sick person from his sin, or his
  sickness, this is disputed\footnote{Ok I'm not sure if I got this part
  correctly...}.  And as for ``the Lord will raise him up'', that might be
  referring to how the sick person's spirit will be raised up, just like how
  Job's spirit was raised up.  What this means practically is that if we are
  sick, the church should share our burdens with us.  If we troubled by
  things that we hear with respect to our sickness, we should call the
  elders.  And as for healing, it is up to God's will and according to God's
  providence.  A note on miraculous healing: when Paul was shipwrecked, he
  healed all who were brought to him.  But it seemed like Paul's healing
  ability was limited to his time on the island; when Paul was in Rome, God
  preferred that Paul preached the gospel with boldness and was eventually
  martyred there.  We note also that healing of diseases do not always lead
  to a changed life, as we have seen in Jesus' ministry, but it is faithfully
  living and proclaiming the Word that does.}
  \item{And as for what James talked about sin, we look at Job again.  In Job
  42, we see also that God asked Job to pray for the three friends.  Job is a
  righteous man here, and therefore his prayer worked greatly.  So what is
  James saying us today?  Sin committed against one another must be dealt
  with.  We must confess our sins to God, but also when we sin against each
  other, we must confess our sins to each other and seek restoration of the
  relationship.  We must forgive each other and be humble with each other
  about our faults.  It is actions like these that are actions of
  righteousness, and these actions show the sincerity of our faith which
  leads to the effectiveness of our prayers.  On the other hand, if we
  confess our sins to God but don't do anything about how our sin against
  someone has hurt him/her, we are being hypocrites and our prayers will be
  ineffectual.  }
\end{itemize}