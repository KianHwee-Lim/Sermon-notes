\section{6th March 2022: Find us faithful}
\subsection*{Text: }
  \begin{quote}
    [12] On the following day, when they came from Bethany, he was hungry.
    [13] And seeing in the distance a fig tree in leaf, he went to see if he
    could find anything on it.  When he came to it, he found nothing but
    leaves, for it was not the season for figs.  [14] And he said to it, “May
    no one ever eat fruit from you again.” And his disciples heard it.

    [15] And they came to Jerusalem.  And he entered the temple and began to
    drive out those who sold and those who bought in the temple, and he
    overturned the tables of the money-changers and the seats of those who
    sold pigeons.  [16] And he would not allow anyone to carry anything
    through the temple.  [17] And he was teaching them and saying to them,
    “Is it not written, ``My house shall be called a house of prayer for all
    the nations"?  But you have made it a den of robbers.” [18] And the chief
    priests and the scribes heard it and were seeking a way to destroy him,
    for they feared him, because all the crowd was astonished at his
    teaching.  [19] And when evening came they went out of the city.

    [20] As they passed by in the morning, they saw the fig tree withered
    away to its roots.  [21] And Peter remembered and said to him, “Rabbi,
    look!  The fig tree that you cursed has withered.” [22] And Jesus
    answered them, “Have faith in God.  [23] Truly, I say to you, whoever
    says to this mountain, ``Be taken up and thrown into the sea," and does
    not doubt in his heart, but believes that what he says will come to pass,
    it will be done for him.  [24] Therefore I tell you, whatever you ask in
    prayer, believe that you have received it, and it will be yours.  [25]
    And whenever you stand praying, forgive, if you have anything against
    anyone, so that your Father also who is in heaven may forgive you your
    trespasses.”
  \end{quote}
\subsection*{Notes}
\begin{itemize}
  \item{This cursing of the fig tree was in the week before Passover.  During
  this time, it wasn't the fig season yet; thus obviously there wouldn't have
  been any figs on the fig tree.  Then why did Jesus curse the fig tree for
  not producing food?  The reason was to create a teaching opportunity with
  the disciples.}
  \item{People might think that it wasn't fair for Jesus to curse the tree.
  Yet it was also unfair for the sinless Son of Man to die on the cross.  So
  yea lol.  What was the lesson that Jesus wanted to teach?  Note that the
  cleansing of the temple is in between the cursing and the withering of the
  fig tree.  So the fig tree was being compared to the temple; though it
  looks busy, there is no fruit.  And hence we find Jesus foretelling the
  destruction of the temple in chapter 13; the withering of the fig tree was
  an analogy of the temple.}
  \item{By the way, interesting question; if buying and selling things made
  the temple unclean, then do similar things that happen in church make the
  church unclean? Answer is, it depends; see the below:}
  \item{Jesus overturned the market to create a teaching opportunity, see
  verse 17.  Two verses were quoted here, Isaiah 56:7 for the house of
  prayer, Jeremiah 7:11 for the den of robbers.  Some context: only one type
  of currency was accepted for temple tax, hence a genuine need for money
  changing.  Also, only unblemished animals were allowed as sacrifices, hence
  people would prefer to buy animals in the city themselves.  In King Herod's
  time, he also expanded the temple, to create an outer court for the
  Gentiles.  This wasn't part of God's original temple plan, but perhaps this
  was God's way of allowing the Gentiles to pray in the temple.  The
  religious leaders in those days allowed the outer court to be used for
  trading, buying and selling.  During the time of passover, there would be a
  lot of activity in the outer court, and hence the Gentiles wouldn't have
  been able to pray because of the lack of space.  Hence, the Isaiah verse
  tells us what the temple should have been, but the Jeremiah verse tells us
  what the temple is really used for because of sin.  The religious leaders
  neglected the Isaiah verse w.r.t Gentiles, and focused on the business side
  of things to line their own pockets.}
  \item{So according to Isaiah 56:7, the church should be a house to make
  disciples of all nations.  Like the story of the fig tree, when we focus on
  making the church attractive instead of having actual discipleship, then
  we'll end up like the fig tree.  Though we must stress; it is ok to want to
  beautify the church, and to have other activities in church etc.  But all
  of those things must serve to build up the church in a real spiritual
  manner; if they don't and instead distract us from doing God's work, then
  we should scrap them instead of focusing on them at the expense of actual
  discipleship, lest we end up like a den of robbers.  Real example: coffee
  in cosy corner is not a bad thing, but if considerable time is spent on
  deciding things like the type of coffee instead of the actual discipleship,
  then we have turned the house of God into a den of robbers.}
  \item{Another way that the church can be turned into a den of robbers is
  the prosperity gospel.  Just because scripture is quoted, doesn't mean that
  it is used correctly; even the devil knows how to use the scripture (c.f
  temptation of Jesus in the wilderness).  A correct use of scripture by the
  church leaders should lead to actual transformation of the heart, a true
  turning away from sin and to God.  A wrong use of scripture by the church
  leaders leads to people doing things like asking people to give money to a
  hedge fund (with the details known only to a privileged few), asking people
  to buy books written by the senior pastor to make the book a bestseller,
  etc.}
  \item{Btw, this text can also be used to make statements about how faith in
  God can do miracles, c.f verse 22.  This is how some pastors use this text.
  This use of the text would lead them to say stuff like ``if you ask, you
  will receive; but if you don't receive, means you have no faith''.  But
  that's probably not the proper use of this text; we must always read this
  fig tree episode with the temple in mind (especially w.r.t the temple being
  mentioned again in chapter 13).  Here, Jesus is contrasting faith in God
  and faith in the temple.  Jesus was telling their disciples not to put
  their faith in the temple, but continue to have faith in God even when the
  temple is destroyed.  This way, their continued faith in God would help
  them to overcome challenges to spread the gospel (even moving mountains),
  knowing that anything that they ask for the kingdom's sake will be given to
  them.  So for us today, we must have faith in God, not in our church/our
  leaders/our parachurch organisation etc.  Nothing wrong with liking our
  church, but it is problematic when our leaders are dodgy, and then if our
  faith is ultimately in our leaders, we will be wholly disillusioned and
  might leave our faith.}
  % \item{Point 1}
  % \item{Point 2}
\end{itemize}