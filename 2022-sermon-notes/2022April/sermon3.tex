\section{17th April 2022: He has risen!}
\subsection*{Text: Mark 16:1-8}
  \begin{quote}
    [1] When the Sabbath was past, Mary Magdalene, Mary the mother of James, and Salome bought spices, so that they might go and anoint him. [2] And very early on the first day of the week, when the sun had risen, they went to the tomb. [3] And they were saying to one another, “Who will roll away the stone for us from the entrance of the tomb?” [4] And looking up, they saw that the stone had been rolled back—it was very large. [5] And entering the tomb, they saw a young man sitting on the right side, dressed in a white robe, and they were alarmed. [6] And he said to them, “Do not be alarmed. You seek Jesus of Nazareth, who was crucified. He has risen; he is not here. See the place where they laid him. [7] But go, tell his disciples and Peter that he is going before you to Galilee. There you will see him, just as he told you.” [8] And they went out and fled from the tomb, for trembling and astonishment had seized them, and they said nothing to anyone, for they were afraid.
  \end{quote}
\subsection*{Notes}
\begin{itemize}
  \item{Three points: Certainty of Jesus' death, Certainty of Jesus'
  resurrection, and commisioning of Jesus' disciples.}
  \item{Certainty of Jesus' death: There were three women who knew where
  Jesus were buried.  They brought spices to the tomb because they were
  hoping to annoint Jesus' body.  THe purpose of this was to mask the odor of
  decaying flesh with spices and oil.  They waited until after the Sabbath
  was over to do this.  But as they came, they realised that since Jesus'
  tomb was sealed with a giant stone, someone needed to roll it away for
  them.  Jesus' death was no accident; it was Jesus intentionally giving away
  His life to save everyone.  The timing and manner in which Jesus died was
  in accordance with God's sovereign plan, as it has been attested to
  centuries prior in the Prophets.}
  \item{From 1 John 4:9-10, we see that Jesus died because of Man's sins, but
  also because of God's love.  Death is not natural, death is a result of
  sin.  God is the source of life, and if we turn away from life, the only
  consequence is death.  Jesus didn't come just to be a moral example, He
  came to die, and He came because He loves.  One of the characteristic of
  love is to seek the well-being of others, sometimes self-sacrifically.  God
  gave His only Son to take on human flesh and to die on the cross for our
  sins.  Only Jesus could be the perfect substitute as an atoning sacrifice,
  because of the worthiness of Jesus' person, since Jesus is God's Son.}
  \item{Sin is always costly, instead of downplaying our sins etc, let us
  turn to God in repentance and ask God for forgiveness, so that instead of
  sin, our heart will by gripped by love, a love that will help us walk in
  God's ways.}
  \item{Certainty of Jesus' resurrection: When the women reached the tomb,
  they saw that the rock was removed.  This would have been surprising and
  alarming for them, because if the stone is removed, what about the body
  inside?  Instead of a body, the women saw a white (angelic) figure who told
  them that Jesus is risen, and that they will see Jesus in Galilee.
  Thereafter, the women told the disciples of Jesus, and then the disciples
  also saw the risen Christ, and then the disciples proclaimed the message of
  Jesus' death and resurrection to the whole world.  Now, if this claim was a
  hoax, it would be very easy for the enemies of Jesus to just produce a
  body.  It is also highly unlikely for the disciples to die for a lie.
  Lastly, society in those days was very dismissive of the testimony of
  women; for the writer of Mark's gospel to say that it was the women who
  first saw the risen Christ, this is a mark of authenticity.  If this
  account was made up, Mark would have chosen to use men instead to give
  himself more credibility.}
  \item{Now, if Jesus had not been risen, at most His death could be
  considered heroic.  But His death would still be in vain.  Paul puts it
  nicely in 1 Corinthians 15:17-22.  But Jesus is really risen.  And Jesus'
  resurrection assures us that God has forgiven us of our sins.  Jesus'
  resurrection also assures us that God will free us from sin's curse.  Just
  like Jesus, one day all of us will be given a new body free of pains and
  suffering and death.  Jesus' resurrection shows that God has the power to
  destroy suffering and death.  Apart from God's power in Jesus'
  resurrection, there is nothing else that can end the suffering and death in
  this world. }
  \item{When Jesus rose from the dead, He ushered in a new Kingdom that is
  slowly growing, and as the Kingdom grows, the world will be slowly
  transformed.  As Christians we can all testify to the freedom from the
  bondage of sin.  Though there is still conflict and suffering in the world,
  we know the final end of the world; Jesus will keep His promise and come
  again.  The future renewal of the world does not depend on things like how
  much power/influence us Christians have, but it depends on Jesus' promise;
  and only an alive person can keep promises.}
  \item{Comissioning of Jesus' disciples: After the women met the angelic
  figure, they ran out, for trembling and astonishment has gripped them.
  Their world has been turned upside down; they came to annoint Jesus' dead
  body, but then now they learn that Jesus is alive.  The fact that God chose
  the women to tell the men of Jesus resurrection shows that God uses both
  men and women in His mission.  Despite our failings as men and women, God
  chooses to use both men and women for Gospel proclamation.  The inclusion
  of women in a male-dominated society was really revolutionary.  Now, Mark
  chapter 16 ends at verse 20, though there are a few early manuscripts that
  do not have verses 9-20.  Yet the account in those verses are credible, for
  they are testified to in other gospels.}
  \item{Now, suppose that Mark really ended at verse 8.  What could be Mark's
  point?  The point could be that Mark is trying to say that despite the
  failing of the men (the disciples deserted Jesus) and the failing of the
  women (they were too afraid and didn't tell anyone), God still uses these
  men and women for His own purposes, and that those failures will eventually
  be redeemed as long as the people continue to trust in God.  And in
  history, we do know that those failures are redeemed, for in the other
  gospels, it is testified that the women did tell the disciples.  For us
  today, the possible takeaway for us is that despite our fears, God enables
  both men and women to be faithful in Gospel proclamation.  We can be
  faithful despite our fears because Christ has risen, and that all the
  forces of evil have been decisively defeated.}
  \item{If Jesus really died and rose again, then nothing can stop Him from
  showing forth His glory through us Christians.  Just as Jesus has risen, we
  too will share in that power, which enables us to be faithful, and gives us
  the hope that at the end, all will be made right.}



\end{itemize}