\section{26th June 2022: One in ministry to all the world: Learning to live
beyond ourselves}
\subsection*{Text: Romans 15:5-13}
  \begin{quote}
    [5] May the God of endurance and encouragement grant you to live in such
    harmony with one another, in accord with Christ Jesus, [6] that together
    you may with one voice glorify the God and Father of our Lord Jesus
    Christ.  [7] Therefore welcome one another as Christ has welcomed you,
    for the glory of God.

    [8] For I tell you that Christ became a servant to the circumcised to
    show God’s truthfulness, in order to confirm the promises given to the
    patriarchs, [9] and in order that the Gentiles might glorify God for his
    mercy.  As it is written,

      “Therefore I will praise you among the Gentiles,
          and sing to your name.”


      [10] And again it is said,

        “Rejoice, O Gentiles, with his people.”


      [11] And again,

        “Praise the Lord, all you Gentiles,
            and let all the peoples extol him.”


      [12] And again Isaiah says,

        “The root of Jesse will come,
            even he who arises to rule the Gentiles;
        in him will the Gentiles hope.”


      [13] May the God of hope fill you with all joy and peace in believing,
      so that by the power of the Holy Spirit you may abound in hope.
  \end{quote}
\subsection*{Notes}
\begin{itemize}
  \item{This sermon is related to the talks given at the church retreat on 25
  June 2022.  The talks are recorded on the church website.  Yesterday, we
  have looked at ``one with Christ'' and ``one in Christ''.  Today, we have
  ``one for Christ''.  As mentioned yesterday, we need to be one with Christ
  first, which would allow us to be one in Christ.  And as we worship the God
  together as a Church, we are reminded of God's goodness, and after the
  benediction, we go out into the world together for God's mission.  The God
  who gathers His people also sends them out.  This is what it means to be
  ``one for Christ''.}
  \item{One of the things that Jesus prayed for in John 17 was ``that they
  may be one, as we are one''.  The Lord's longing was for oneness to be
  experienced among the people of God.  However, we often fall short of this
  unity.  In Romans 8:34, we read that Jesus is interceding for us, possibly
  still praying for us to be one.}
  \item{The vertical relationship between God and us is essential for us to
  be unity.  As Paul says in verse 5 of our text, we need to have ``the
  Spirit of unity''.  Unity is not a man-made thing, it is not created
  through human efforts of managing organisations, team building exercises,
  etc.  Unity is a supernatural thing.  This Spirit of unity comes ``in
  accord with Christ Jesus'', or as translated in 1984 NIV, the Spirit of
  unity comes ``as we follow Christ Jesus''.  Without Jesus, there will be no
  success in Church work and no unity.  There might be some success in
  worldly eyes, but from a spiritual lens, we can see that there is no
  success. }
  \item{ In Christian organisations, sadly oftentimes we have meetings to
  discuss important stuff without the individual members spending devotional
  time with God first.  Similarly, we should also meet the Lord ourselves
  first before we gather for corporate worship.  It is wrong to say ``we are
  going to worship in Church anyway, I can skip bible reading in the
  morning''.  We must follow Christ individually first, then when we gather
  for corporate worship, we will be united in our worship of God.  The
  essential point for unity is the Lord Jesus Christ.  Take away Jesus, and
  there will be no unity.  It is the name of Jesus, not some institution or
  some brand or project that gives us spiritual unity.}
  \item{Jesus also said that ``He who is not with me is against me'', and
  ``He who does not gather with me scatters''.  Jesus said this when the
  disciples complained about how there was someone else who is not part of
  them driving out demons in Jesus' name.  This means that as long as there
  is someone else doing God's work at Jesus' name, then that person is also a
  brother in Christ doing the same work.  There is no such thing as ``this is
  your mission field'' and ``this is my mission field''.  The vertical
  relationship with God determines our horizontal relationship with others.
  This is also why Paul writes in verse 7 of our text, ``accept one another,
  just as Christ has accepted you''.  How has Christ accepted us?  See
  Romans 5:8; Christ died for us while we were yet sinners!  Jesus also said
  in John 6:37, ``whoever comes to me I will never drive away''.  In light of
  Jesus' love for us, we must accept other brothers and sisters to the same
  extent.  Or as it is said in John 13:34, we love one another as Jesus has
  loved us.  When the vertical is missing, i.e when we don't understand God's
  love for us, then it is hard for us to love one another truly.  Slogans
  telling us to love one another without reminding us about our vertical
  relationship with God are eventually ineffective.  }
  \item{Or put another way, there is a difference between the
  \textbf{kingdom} of God versus a \textbf{republic} of heaven; in a kingdom,
  there is a king that we must all submit to, but in a repulic, there is no
  king and it is all up to us.}
  \item{Paul also said before that the word is God is near us in our mouth
  and in our heart in Romans 10, which is a paraphrase of Deuteronomy 30.
  And Paul also said that if we confess with our mouth and believe in our
  heart, we will be saved.  In the church, there hence must be unity in both
  our hearts (what we believe) and in our mouths (what we confess and do).
  The image to have here is of an orchestra; our hearts must all be in sync
  with God's heartbeat\footnote{speaking figuratively}, and then our hearts
  will be in sync with each other.  Then when all of us plays the same song,
  our message will be effectively proclaimed to the world.  The contrast is
  if all of us are disunited and if we have many different messages; then it
  will be confusion to the world.}
  \item{In verses 9-12, we have four quotations of the Old Testament.  These
  four quotations all mention the Gentiles.  The first quotation comes from
  David, in 2 Samuel 22:50.  The picture is that of a Jew worshipping God
  among the Gentiles, such as Lydia in Philippi worshipping God in a small
  group because there was no synagogue.  The next quotation is from Moses, in
  Deuteronomy 32:43.  The picture is that of Gentiles rejoicing together with
  the Jews.  This is one step further than the previous picture; in the
  previous verse we had Jews worshipping God among the Gentiles, and then now
  the Gentiles join the Jews in worship.  The third picture is from Psalm
  117:1, here we have the Gentiles praising God on their own.  The Gentiles
  who observed the Jews previously and subsequently joined the Jews
  previously have learnt to praise God on their own.  The last picture is
  from Isaiah 11:10, which is that both the Jews who were sent to the
  Gentiles, and the Gentiles who observed them and then joined them and then
  on their own worshipping God, both the Jews and the Gentiles will be ruled
  by Jesus Christ.}
  \item{The above four pictures in succession depicts what sending a
  missionary overseas looks like.  First the missionary worships God alone,
  then the curious townfolk join in to worship God, then the townfolk learn
  to worship God alone, and finally both the missionary and the townfolk are
  ruled by Jesus.  These four pictures here are an encouragement of what
  missionary work can look like, and these four pictures have occured
  throughout history; the fact that we as Gentiles are in church worshipping
  God now is testament to that!  Thanks to the missionaries who came to SG in
  the past.}
  \item{God's plan is not merely for the Jews, it is for the entire world.
  Will we join God in His mission to the world?  We must give up our small
  ambitions (e.g buying a nice house, getting a nice job, etc.), because they
  will fade away and have no lasting impact.  On the other hand, being
  recruited by God to do His work has lasting impact.  God's mission to the
  world is the greatest project in the world, and if we join God in His
  mission, we will surely lead a blessed (but perhaps not easy) life.  When
  we hear God calling us, will we say: ``not my will, but yours be done''. Participating in God's mission can look like going as far to somewhere like Japan, or it can also look like going one street across to let's say the rental flats to reach the poor and giving them help and good news, or it can also look like going one isle across the office to tell our colleague about Jesus. But for this to happen, we need to experience a oneness with God, a oneness with each other, then we can be one in our mission to the world.}
  \end{itemize}