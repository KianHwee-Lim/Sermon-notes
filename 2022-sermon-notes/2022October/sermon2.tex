\section{9th October 2022: Letter to the church at Sardis}
\subsection*{Text: Revelation 3:1-6}
  \begin{quote}
    [1] “And to the angel of the church in Sardis write: ‘The words of him
    who has the seven spirits of God and the seven stars.

    “‘I know your works.  You have the reputation of being alive, but you are
    dead.  [2] Wake up, and strengthen what remains and is about to die, for
    I have not found your works complete in the sight of my God.  [3]
    Remember, then, what you received and heard.  Keep it, and repent.  If
    you will not wake up, I will come like a thief, and you will not know at
    what hour I will come against you.  [4] Yet you have still a few names in
    Sardis, people who have not soiled their garments, and they will walk
    with me in white, for they are worthy.  [5] The one who conquers will be
    clothed thus in white garments, and I will never blot his name out of the
    book of life.  I will confess his name before my Father and before his
    angels.  [6] He who has an ear, let him hear what the Spirit says to the
    churches.’
  \end{quote}
\subsection*{Notes}
\begin{itemize}
  \item{Sardis, in the Persian period, was a bustling and powerful city
  because its geographical location offered it a lot of natural defenses.
  But by the Roman period, the city has declined a lot cause of war, and
  hence the city can only look back on its glorious past.}
  \item{The letter to the church in Sardis has no commendation, but only
  condemnation.  Compare this with the letter to the church in Smyrna, who
  only had commendation.}
  \item{The church in Sardis had a particular reputation of being alive.
  They received and heard the gospel (v3).  Perhaps in the past they were
  faithful because of the gospel they heard and obeyed, but today they are
  all unfaithful yet their reputation for being faithful and alive persisted.
  Perhaps some of them rested on their laurels about the good old days of
  their faith and thus neglected to work on their faith currently.  This is
  quite similar to the state of Sardis itself, who had an illustrive history
  but is currently rabak.}
  \item{There is nothing wrong for us to look back on the good old days where
  we have been zealous for the Lord and give thanks.  But the important
  question is; what about now?  Are we still passionate for God?  Do we still
  love God as much?  Or do we have the ``been there done that'' mentality
  when it comes to our r/s with God?  This ``been there done that'' mentality
  is dangerous!}
  \item{A quote: ``there is no such thing as stagnating faith.  If your faith
  is not growing, it is likely dying''.  Sometimes, our faith can be dying
  yet we still have a good reputation among people, i.e people who think we
  are very Christian (even to our closest friends!).  We must not deceive
  ourselves by resting on our reputation if we feel like our faith is dying.}
  \item{For us, we must ask ourselves whether we have an abiding r/s with
  Christ and if we are filled with His Spirit (so that we are in the
  invisible church), because if not, we are like whitewashed tombs, outwardly
  looking legit but inside we are totally dead.  Jesus says to the church in
  Sardis: ``\textbf{I know your works}, you have the reputation of being
  alive, but you are dead''.  The phrase: ``I know your works'' is repeated
  to the letters to many of the seven churches here.  The ``works'' here
  include the totality of our Christian conduct, like our acts of service and
  the state of our heart.  For the church at Sardis, their works are
  ``incomplete'', i.e there is something missing in their Christian life,
  perhaps they are acting hypocritically like the Pharisees (got reputation
  but actually dead).  The exhortation to the church at Sardis then is to
  ``remember then what you have heard'', i.e which is to go back to the
  gospel.  The key is to go back to the gospel and be wowed and astounded by
  the magnitude of God's love for us, believe in Jesus who gave us life, and
  then let our gratitude give us motivation to life a Christian life for
  Jesus.}
  \item{The church in Sardis can be described by 2 Time 3:1-5, which says
  especially: ``having the appearance of godliness, but denying its power''.
  }
  \item{The problem with the church in Sardis is that most of them have
  ``soiled their garments''.  From this letter here, the church in Sardis had
  no real problem with persecution, perhaps because most of them are living
  unclean lives like the world anyway.  For us to reflect on: ``have we
  imbibed worldly values into our lives and into the life of our church so
  that we are no different form the world?'' Because if we have, then we have
  soiled our garments. A few examples:
  \begin{itemize}
    \item{In the world, when people are hurt, they just walk away.  But this
    is not what we should do in church.  We are to forgive one another and to
    seek reconciliation with one another.  Practically, what this looks like
    is to talk to the person who has hurt you, and if that is currently too
    hard, then at least talk to the leaders first before leaving.}
    \item{In the world, people are abit too judgey.  We should not be too
    quick to judge in church, especially when people are sharing their
    struggles/sins and their thanksgiving (not everyone who is sharing
    thanksgiving is being boastful).}
    \item{In the increasingly secularised world, the mantra is ``you can
    believe what you want but don't shove your religion down my throat'.
    Would that make us too ashamed of the gospel to share our gospel?  Or do
    we boldly proclaim the gospel and invite people to accept Jesus and
    receive eternal life?} 
  \end{itemize}}
  \item{We must keep watch over our lives and not rest on our reputation.  We
  are saved by faith, but good works and a holy life necessarily flow from
  our faith.  And the good works and a holy life that we produce by our faith
  aren't by our own effort, but by the Holy Spirit who Jesus holds in His
  hand.  Jesus in His hand holds both the churches and the Spirit (v1), and
  He can bring the two together.}
\end{itemize}