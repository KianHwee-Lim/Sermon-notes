\section{24th July 2022: How shall we live under authority?}
\subsection*{Text: Ecclesiastes 8:2-17}
  \begin{quote}
    [2] I say: Keep the king’s command, because of God’s oath to him.  [3] Be
    not hasty to go from his presence.  Do not take your stand in an evil
    cause, for he does whatever he pleases.  [4] For the word of the king is
    supreme, and who may say to him, “What are you doing?” [5] Whoever keeps
    a command will know no evil thing, and the wise heart will know the
    proper time and the just way.  [6] For there is a time and a way for
    everything, although man’s trouble lies heavy on him.  [7] For he does
    not know what is to be, for who can tell him how it will be?  [8] No man
    has power to retain the spirit, or power over the day of death.  There is
    no discharge from war, nor will wickedness deliver those who are given to
    it.  [9] All this I observed while applying my heart to all that is done
    under the sun, when man had power over man to his hurt.

    [10] Then I saw the wicked buried.  They used to go in and out of the
    holy place and were praised in the city where they had done such things.
    This also is vanity.  [11] Because the sentence against an evil deed is
    not executed speedily, the heart of the children of man is fully set to
    do evil.  [12] Though a sinner does evil a hundred times and prolongs his
    life, yet I know that it will be well with those who fear God, because
    they fear before him.  [13] But it will not be well with the wicked,
    neither will he prolong his days like a shadow, because he does not fear
    before God.

    [14] There is a vanity that takes place on earth, that there are
    righteous people to whom it happens according to the deeds of the wicked,
    and there are wicked people to whom it happens according to the deeds of
    the righteous.  I said that this also is vanity.  [15] And I commend joy,
    for man has nothing better under the sun but to eat and drink and be
    joyful, for this will go with him in his toil through the days of his
    life that God has given him under the sun.

    [16] When I applied my heart to know wisdom, and to see the business that
    is done on earth, how neither day nor night do one’s eyes see sleep, [17]
    then I saw all the work of God, that man cannot find out the work that is
    done under the sun.  However much man may toil in seeking, he will not
    find it out.  Even though a wise man claims to know, he cannot find it
    out.
  \end{quote}
\subsection*{Notes}
\begin{itemize}
  \item{One big question in Ecclesiastes: ``What is the meaning of life''?
  The apparent answer is that life has no meaning.  Life is hard, unfair, and
  then we die. This answer doesn't seem to be what we believe in as Christians, but this answer seems to be a fair summary of the book.}
  \item{For today's text, the larger idea is about how good things happen to
  bad people and how bad things happen to good people.  Actually the darkness
  that seemingly envelops Ecclesiastes has a final resolution (somewhat) in
  the book itself.}
  \item{But for today, the focus is just on the point of how we should live
  under authority (Ecclesiastes 8:2-5).  Some practical questions: How should
  Christians submit to bad governments?  Can Christians overthrow corrupt
  governments?  Etc. There are also some other texts in the Bible that talk about a similar topic (e.g Romans 13).}
  \item{Firstly, the Preacher tells us to keep the king's command, because
  the king is chosen by God.  A similar theme is found in Romans 13:3-4,
  where the king is spoken of as God's servant.  The king is said to be God's
  instrument in executing justice in the land, and therefore we should be
  afraid to do evil, or as Ecclesiastes said, we should ``not take your stand
  in an evil cause''.}
  \item{Now for us, we don't have a king in Singapore, and instead, we have a
  democratic republic in Singapore.  In fact, lesser and lesser countries in
  the world today are monarchies.}
  \item{In Ecclesiastes' context, we had kings.  The first king was Saul, and
  then the next king was David.  God's intent was for the king to represent
  Him, and for the king to do God's will in humility and trust.  But by the
  time of Jesus, the people who were in power were the Roman authorities.
  The Jewish ``kings'' (e.g Herod) were weak and corrupt.  Hence, the Jews
  liked neither the Jewish ``kings'' or the Roman government, and as a
  result, there were many rebel groups who wanted to overthrow this
  authority.} 
  \item{Btw, 1 Peter 2 also talks about civil obedience.  As per Peter's
  instructions, slaves are to submit not only to just masters, but also
  unjust masters; and hence insofar as the king is our master, a similar
  thing applies to us.  Peter's words even more remarkable if we remember
  that Peter was writing during the reign of Nero.  So, why do we submit to
  the government?  Firstly, it is because submission to the government
  honours God.  Secondly, it is because governing authorities work for good
  and reduce evil (ok in the ideal case).  But of course, in Ecclesiastes, we
  see that this is not always the case.  Thirdly, we submit to authorities
  because of the example of Christ.  When the soldiers hurled insults at
  Jesus, He didn't retaliate.  Here we see Jesus submitting to unjust human
  authorities.  Jesus also said something interesting to Pilate (John
  19:10-11).  Pilate's power comes from God.}
  \item{As per what Jesus said too, ``render unto caesar what belongs to
  caesar, and render unto God what belongs to God.''}
  \item{A few examples in sg are: should we submit to compulsory education
  act?  Should we submit to covid laws?  Interesting examples but the speaker
  doesn't really address the controversies here.}
  \item{$\dots$ give up taking notes liao lol, the speaker today seems to be
  going in circles.  Plus I'm tired...}
\end{itemize}