\section{27th March 2022: Extravagant devotion}
\subsection*{Text: Mark 14:1-11}
  \begin{quote}
    [1] It was now two days before the Passover and the Feast of Unleavened
    Bread.  And the chief priests and the scribes were seeking how to arrest
    him by stealth and kill him, [2] for they said, “Not during the feast,
    lest there be an uproar from the people.”

    [3] And while he was at Bethany in the house of Simon the leper, as he
    was reclining at table, a woman came with an alabaster flask of ointment
    of pure nard, very costly, and she broke the flask and poured it over his
    head.  [4] There were some who said to themselves indignantly, “Why was
    the ointment wasted like that?  [5] For this ointment could have been
    sold for more than three hundred denarii and given to the poor.” And they
    scolded her.  [6] But Jesus said, “Leave her alone.  Why do you trouble
    her?  She has done a beautiful thing to me.  [7] For you always have the
    poor with you, and whenever you want, you can do good for them.  But you
    will not always have me.  [8] She has done what she could; she has
    anointed my body beforehand for burial.  [9] And truly, I say to you,
    wherever the gospel is proclaimed in the whole world, what she has done
    will be told in memory of her.”

    [10] Then Judas Iscariot, who was one of the twelve, went to the chief
    priests in order to betray him to them.  [11] And when they heard it,
    they were glad and promised to give him money.  And he sought an
    opportunity to betray him.
  \end{quote}
\subsection*{Notes}
\begin{itemize}
  \item{This event here took place close to Passover (as mentioned in v1).
  As seen in v2, someone wanted Jesus dead.}
  \item{Today's main point: Faithful disciples can show their extravagant
  \textbf{commitment} (devotion) in the face of \textbf{criticism} because
  God will \textbf{commend} such efforts. I.e, three `C's.}
  \item{\textbf{Commitment:} in verse 3, we see an unnamed woman.  In these
  times, due to ancient Jewish customs, the fellowship here was probably a
  men's fellowship.  Women only attend such fellowship to serve food etc.
  Hence what the woman did here was going against the social norms.  And
  since women probably won't be able to afford such expensive ointment, this
  ointment was probably all she had.  Thus the lesson here is that faithful
  disciples show their commitment to Jesus despite the challenges
  faced.  Just like how the woman went against social norms to annoint Jesus,
  and gave all she had for Jesus, are we ready to go against social norms and
  give all we have for Jesus?}
  \item{\textbf{Criticism:} in verse 4, we see some men criticising the
  woman's gift as wasted.  During Passover, the Jews were obligated to give
  alms to the poor, and this was why they were thinking of the poor.  But
  what the men implicitly said was that Jesus was not worthy of this
  expensive gift.  Taken the actions of the men negatively, possibly these
  men could be trying to virtue signal, to let other people know how much
  they know about the poor.  Taking the actions of the men as positively as
  possible, in a sense, they put the commandment to `love your neighbour'
  above the commandment to `love God'.  In our context, we note that no
  matter what we do for God, we will always receive criticism of some kind.
  For example, if we give up our high paying job for full-time ministry,
  people might criticise us saying that our tithes could have done more for
  God.  If we stay in our high paying jobs instead of going into full-time,
  people might criticise us for being lovers of money.  We will always face
  criticism of some kind as long as we are faithful disciples of Jesus.}
  \item{\textbf{Commendation:} in verses 6-9, Jesus commends the woman's
  action as beautiful.  Jesus also re-orients the priority of the men there;
  His death was so imminent, whereas the poor would always exist.  Though as
  Christians we must try to alleviate poverty wherever we can, sometimes we
  must weigh the needs of the hour.  The woman here gave what she could, and
  she prepared his body for burial.  And hence, she will be remembered for
  her commitment, just as what we are doing now by remembering her actions.
  Thus, for us today, we must realise that though we may face criticism for
  our faithful service, Jesus will commend our faithful service at the end.
  }
  \item{Some reflections: here the disciples were considered the inner
  circle, but they didn't do what the woman did.  The disciples knew Jesus
  better than the woman, but yet the woman was the one who got it.  And when
  we compare the woman to the scribes and the Pharisees, in this sense, the
  woman was already in the kingdom of God.  For us, this shows that just
  because someone goes to church regularly, goes to seminary regularly etc,
  doesn't mean that he/she is committed to Christ.  Commitment is shown by
  what we do for Jesus, like the woman, not outward religiosity like the
  Pharisees.}
  \item{More reflections: furthermore, God is sovereign; He can make us of
  our faithful service for His kingdom, though we cannot see right now how
  our faithful service can be used.  It is just like the woman, she didn't
  know the effect of her breaking the alabastar jar for God, but God was
  always in control.}
  \item{In a sense also, Jesus is heaven's alabastar jar, broken for us for
  our healing.  He is the Son of God, infinitely precious, yet He took on
  human nature and died on the cross for our sins.  For us then, are we also
  prepared to be broken for Jesus?  I.e, are we ready to give our life to God
  just like how Jesus gave His life for us? }
\end{itemize}