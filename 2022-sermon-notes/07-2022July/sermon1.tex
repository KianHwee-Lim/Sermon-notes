\section{3rd July 2022: What is spiritual friendship?}
\subsection*{Text: 1 Samuel 20:1-17,35-42}
  \begin{quote}
    [1] Then David fled from Naioth in Ramah and came and said before
    Jonathan, “What have I done?  What is my guilt?  And what is my sin
    before your father, that he seeks my life?” [2] And he said to him, “Far
    from it!  You shall not die.  Behold, my father does nothing either great
    or small without disclosing it to me.  And why should my father hide this
    from me?  It is not so.” [3] But David vowed again, saying, “Your father
    knows well that I have found favor in your eyes, and he thinks, ‘Do not
    let Jonathan know this, lest he be grieved.’ But truly, as the LORD lives
    and as your soul lives, there is but a step between me and death.” [4]
    Then Jonathan said to David, “Whatever you say, I will do for you.” [5]
    David said to Jonathan, “Behold, tomorrow is the new moon, and I should
    not fail to sit at table with the king.  But let me go, that I may hide
    myself in the field till the third day at evening.  [6] If your father
    misses me at all, then say, ‘David earnestly asked leave of me to run to
    Bethlehem his city, for there is a yearly sacrifice there for all the
    clan.’ [7] If he says, ‘Good!’ it will be well with your servant, but if
    he is angry, then know that harm is determined by him.  [8] Therefore
    deal kindly with your servant, for you have brought your servant into a
    covenant of the LORD with you.  But if there is guilt in me, kill me
    yourself, for why should you bring me to your father?” [9] And Jonathan
    said, “Far be it from you!  If I knew that it was determined by my father
    that harm should come to you, would I not tell you?” [10] Then David said
    to Jonathan, “Who will tell me if your father answers you roughly?” [11]
    And Jonathan said to David, “Come, let us go out into the field.” So they
    both went out into the field.

    [12] And Jonathan said to David, “The LORD, the God of Israel, be
    witness!  When I have sounded out my father, about this time tomorrow, or
    the third day, behold, if he is well disposed toward David, shall I not
    then send and disclose it to you?  [13] But should it please my father to
    do you harm, the LORD do so to Jonathan and more also if I do not
    disclose it to you and send you away, that you may go in safety.  May the
    LORD be with you, as he has been with my father.  [14] If I am still
    alive, show me the steadfast love of the LORD, that I may not die; [15]
    and do not cut off your steadfast love from my house forever, when the
    LORD cuts off every one of the enemies of David from the face of the
    earth.” [16] And Jonathan made a covenant with the house of David,
    saying, “May the LORD take vengeance on David’s enemies.” [17] And
    Jonathan made David swear again by his love for him, for he loved him as
    he loved his own soul.

    [35] In the morning Jonathan went out into the field to the appointment
    with David, and with him a little boy.  [36] And he said to his boy, “Run
    and find the arrows that I shoot.” As the boy ran, he shot an arrow
    beyond him.  [37] And when the boy came to the place of the arrow that
    Jonathan had shot, Jonathan called after the boy and said, “Is not the
    arrow beyond you?” [38] And Jonathan called after the boy, “Hurry!  Be
    quick!  Do not stay!” So Jonathan’s boy gathered up the arrows and came
    to his master.  [39] But the boy knew nothing.  Only Jonathan and David
    knew the matter.  [40] And Jonathan gave his weapons to his boy and said
    to him, “Go and carry them to the city.” [41] And as soon as the boy had
    gone, David rose from beside the stone heap and fell on his face to the
    ground and bowed three times.  And they kissed one another and wept with
    one another, David weeping the most.  [42] Then Jonathan said to David,
    “Go in peace, because we have sworn both of us in the name of the LORD,
    saying, ‘The LORD shall be between me and you, and between my offspring
    and your offspring, forever.’” And he rose and departed, and Jonathan
    went into the city.
  \end{quote}
\subsection*{Notes}
\begin{itemize}
  \item{Scripture describes the friendship between Jonathan and David as
  covenantal, and very deep.  In today's age, some of us find this weird.
  This is because our culture has elevated romance to the highest form of
  love, so that the deep love that David and Jonathan share is interpreted in
  that light by some.}
  \item{Our culture's obsession with romance has led us to undervalue other
  human relationships, especially friendships.  Friendships have the
  potential to be deeper, stronger and more enduring that romantic
  relationships.}
  \item{The Christian tradition has never elevated romantic relationships
  above friendship.  CS Lewis has likened friendship to the love between
  angels, for example.}
  \item{A spiritual friendship is a friendship founded on the love of God,
  moved by the pursuit of God, and borrught into and experience of God.
  These are the three points for today.}
  \item{A spiritual friendship is founded on the love of God.  Both Jonathan
  and David knew God's love, and they embodied that love in their
  relationship.  What is happening in our text today is that Saul has been
  trying to assasinate David, and hence he has tried to invite him to a
  banquet.  David knows this, and hence he approached Saul's son, Jonathan,
  for help.  Jonathan expects David to ascend Israel's throne, and this means
  that Jonathan is actually David's direct's competitor to the throne.  But
  Jonathan knew the steadfast love of the Lord, and hence he showed David
  that same steadfast love.  Both Jonathan and David knew the steadfast love
  of the Lord themselves first, so that is how they can show it to each
  other.  This necessitates that we can only have spiritual friendship with
  other Christians.  Spiritual friends have been described as friends that
  agree in both human and divine things.}
  \item{Spiritual friendships are covenantal in nature, spritual friends
  aren't just here today and gone tomorrow.  While it is true that friends
  can come and go in different seasons of life, if we remember that spiritual
  friendship is covenantal, we ought to put in effort to preserve such
  friendships, even if the circumstances of life dictate otherwise (e.g if
  people migrate to another country).  And while spiritual friendships
  between sinners will never be perfect, we should be committed to working
  things like disagreements out.}
  \item{Spiritual friendships are sacrifical in nature, spiritual friends
  care for each other just like how God cares for us.  In the passage today,
  Jonathan risked his life to save David.  Our Lord Jesus Christ affirms this
  in John 15, when He said that ``greater love has no one than this, to lay
  down his life for his friends''.  Most of us today won't be in a situation
  when we would need to die for our friends, but since spiritual friendships
  require commitment and hence sacrifice, we must be prepared to do so.  An
  application of this is to stay close to your friends.}
  \item{Spiritual friendships are confidential in nature.  This doesn't mean
  that we need to hide our friendships from other people, but this means that
  spiritual friends confide in one another.  Here we see David seeking out
  Jonathan to share about his fears about Jonathan's father.  Spiritual
  friends share what is on their hearts with each other, even secret things.
  Our Lord Jesus affirms this too, as in John 15.  If this is the case, then
  this doesn't mean that we can be spiritual friends with everyone in church.
  We might have $100$ friends in church, but we might only confide in a few.
  Most of the time, it boils down to whether we can resonate with the other
  person.  The good thing though is that unlike marriages, spiritual
  friendships are not exclusive.  While we can't be spiritual friends with
  everyone in church, we should be open to be spiritual friends with
  Christians we talk to.}
  \item{A spiritual friendship is also moved by the pursuit of God.
  Spiritual friends have a common love for God, so a spiritual friendship
  entails pursuing God and supporting one another along the way.  Here, we
  see that Jonathan knows that the will of God is for David to become king,
  which means that his family will be destroyed because of his dad's enmity
  with David, yet Jonathan still aids David in his ``destiny'' to become
  king.  Spiritual friends support and help one another to do God's will,
  which for us is our sanctification.}
  \item{Spiritual friends encourage each other, a spiritual friend can be a
  channel of divine grace, when the going gets tough.  We see Jonathan doing
  this for David.  Encouragement can be as simple as sharing how God has
  spoken to us with our Christian friends, etc.  Especially when we live in a
  secular age, we are always bombarded with the philosophies of secular
  humanism.  Unless we encourage each other in our pursuit of God, we will
  regress.}
  \item{Spiritual friends counsel and correct each other.  Especially in we
  get stuck in sin, we need spiritual friends to persuade us of the truth and
  wake us up from our spiritual slumber.  Sometimes this might actually
  require rebuke!  As Proverbs say, ``faithful are the wounds of a friend,
  profuse are the kisses of an enemy''.  It is the responsibility of
  spiritual friends to correct one another and to counsel one another, as per
  what is said in James 5:19-20.  It is better to correct your friend's error
  and to feel abit unpleasant about it, rather than to allow your friend to
  stumble.  In any case, a true spiritual friend will take your correction,
  though it might seem painful for a time.}
  \item{A spiritual friendship is brought into an experience of God.  When
  spiritual friends love each other with the love of God, their friendship
  becomes a place when a deeper experience of God's love can take place.
  When we taste the sweetness of spiritual friendships, it is used as a
  channel for us to experience Jesus' friendship for us.  Spiritual
  friendships that are founded on God and moved by a pursuit of God leads to
  us experiencing God.  Our Lord Jesus has called us friends, and our Lord
  Jesus is the best spiritual friend.}
\end{itemize}