\section{15th May 2022: Right relationships with others and with God}
\subsection*{Text: Ecclesiastes 4:4-5:7}
  \begin{quote}
    [4] Then I saw that all toil and all skill in work come from a man’s envy
    of his neighbor.  This also is vanity and a striving after wind.

    [5] The fool folds his hands and eats his own flesh.

    [6] Better is a handful of quietness than two hands full of toil and a striving after wind.

    [7] Again, I saw vanity under the sun: [8] one person who has no other,
    either son or brother, yet there is no end to all his toil, and his eyes
    are never satisfied with riches, so that he never asks, “For whom am I
    toiling and depriving myself of pleasure?” This also is vanity and an
    unhappy business.

    [9] Two are better than one, because they have a good reward for their
    toil.  [10] For if they fall, one will lift up his fellow.  But woe to
    him who is alone when he falls and has not another to lift him up!  [11]
    Again, if two lie together, they keep warm, but how can one keep warm
    alone?  [12] And though a man might prevail against one who is alone, two
    will withstand him—a threefold cord is not quickly broken.

    [13] Better was a poor and wise youth than an old and foolish king who no
    longer knew how to take advice.  [14] For he went from prison to the
    throne, though in his own kingdom he had been born poor.  [15] I saw all
    the living who move about under the sun, along with that youth who was to
    stand in the king’s place.  [16] There was no end of all the people, all
    of whom he led.  Yet those who come later will not rejoice in him.
    Surely this also is vanity and a striving after wind.

    [1] Guard your steps when you go to the house of God.  To draw near to
    listen is better than to offer the sacrifice of fools, for they do not
    know that they are doing evil.  [2] Be not rash with your mouth, nor let
    your heart be hasty to utter a word before God, for God is in heaven and
    you are on earth.  Therefore let your words be few.  [3] For a dream
    comes with much business, and a fool’s voice with many words.

    [4] When you vow a vow to God, do not delay paying it, for he has no
    pleasure in fools.  Pay what you vow.  [5] It is better that you should
    not vow than that you should vow and not pay.  [6] Let not your mouth
    lead you into sin, and do not say before the messenger that it was a
    mistake.  Why should God be angry at your voice and destroy the work of
    your hands?  [7] For when dreams increase and words grow many, there is
    vanity; but God is the one you must fear.
  \end{quote}
\subsection*{Notes}
\begin{itemize}
  \item{Two points today: relationship with others, and relationship with God.}
  \item{Relationships with others: 
  \begin{enumerate}
    \item{Economic relationships: Some people are caught in the trap of
    working harder and harder, and they cannot rest.  Some of these people
    who are caught in this trap do so because of envy.  Another group of
    people are hardly working at all.  In the end, the Preacher concludes
    that it is better to be content with a little (one handful), rather than
    two handfuls obtained after much toil.  We must find balance between work
    and rest.  As for envy, rather than looking at those who are better than
    us, we should look at those who are worse off than us and help them.  The
    Lord is our giver, and from what He has given us, we should help others.}
    \item{Social relationships: It is better to have companions to share your
    joys and your sorrows.  We are not designed to live for ourselves.  On
    the other hand, the world tells us: "it is all about you, live for your
    own dreams".  We are created by God to be social creatures, when we try
    to live for ourselves, we become empty.  Only in a community is there a
    connection between work and reward, rest and support.}
    \item{Political relationships: leaders and positions will come and go.
    There are people who are old, but they aren't wise anymore, and they will
    eventually give way to other people.  There are those who are young but
    wise, and they rise the ranks to become leaders of the country.  But even
    for these young and wise people, even their wisdom doesn't prevent them
    from one day losing their leadership position, not least due to old age.
    }
  \end{enumerate}}
  \item{Relationship with God: When there is silence, that is when we are
  most ready for God to speak to us.  Moreover, dead formalism is worse than
  a lively faith.  On the other hand, if we are too casual with God, that
  also shows a lack of fear (respect/reverence) of God.  There are two types
  of fear; first is the fear of being judged, fear of facing the
  consequences, second is a great respect that results in awe and reverence,
  which comes from a recognition of the majesty, holiness, and glory of God.
  Also, what is our attitude when we come to church?  Do we get angry with
  people when people cut our lane when we drive to church?  Do we get angry
  with our kids when they aren't cooperative?  Etc.
  
  Everything comes from God, we cannot offer to God anything that doesn't
  already belong to him.  Hence, it makes no sense to try to ``bribe'' God
  with sacrifices etc (Psalm 50).  What God wants is not dead formalism, but
  what God wants is a true and lively faith.  And the first mark of a true
  and lively faith is good works (to listen to God and to obey God; e.g true
  religion is helping the widows and orphans in their affliction).  The
  second mark of a true and lively faith is a reverent awe of God, we must be
  very intentional when we approach God.
  
  In fact, our relationship with God shapes our relationships with others.
  The answer to ``for whom am I toiling'' should be ``for God''.  If we have
  a right view of God, if we know that there is a God in heaven who judges,
  we wouldn't wrong our neighbour.  If we know God's love for us, especially
  in the giving of His Son for us, we will love our neighbour.}
\end{itemize}