\section{8th May 2022: Who really is in control?}
\subsection*{Text: Ecclesiastes 3:1-4:3}
  \begin{quote}
    [1] For everything there is a season, and a time for every matter under
    heaven:

    [2] a time to be born, and a time to die;
    a time to plant, and a time to pluck up what is planted;
    [3] a time to kill, and a time to heal;
    a time to break down, and a time to build up;
    [4] a time to weep, and a time to laugh;
    a time to mourn, and a time to dance;
    [5] a time to cast away stones, and a time to gather stones together;
    a time to embrace, and a time to refrain from embracing;
    [6] a time to seek, and a time to lose;
    a time to keep, and a time to cast away;
    [7] a time to tear, and a time to sew;
    a time to keep silence, and a time to speak;
    [8] a time to love, and a time to hate;
    a time for war, and a time for peace.


    [9] What gain has the worker from his toil?  [10] I have seen the
    business that God has given to the children of man to be busy with.  [11]
    He has made everything beautiful in its time.  Also, he has put eternity
    into man’s heart, yet so that he cannot find out what God has done from
    the beginning to the end.  [12] I perceived that there is nothing better
    for them than to be joyful and to do good as long as they live; [13] also
    that everyone should eat and drink and take pleasure in all his toil—this
    is God’s gift to man.

    [14] I perceived that whatever God does endures forever; nothing can be
    added to it, nor anything taken from it.  God has done it, so that people
    fear before him.  [15] That which is, already has been; that which is to
    be, already has been; and God seeks what has been driven away.

    [16] Moreover, I saw under the sun that in the place of justice, even
    there was wickedness, and in the place of righteousness, even there was
    wickedness.  [17] I said in my heart, God will judge the righteous and
    the wicked, for there is a time for every matter and for every work.
    [18] I said in my heart with regard to the children of man that God is
    testing them that they may see that they themselves are but beasts.  [19]
    For what happens to the children of man and what happens to the beasts is
    the same; as one dies, so dies the other.  They all have the same breath,
    and man has no advantage over the beasts, for all is vanity.  [20] All go
    to one place.  All are from the dust, and to dust all return.  [21] Who
    knows whether the spirit of man goes upward and the spirit of the beast
    goes down into the earth?  [22] So I saw that there is nothing better
    than that a man should rejoice in his work, for that is his lot.  Who can
    bring him to see what will be after him?

    [1] Again I saw all the oppressions that are done under the sun.  And
    behold, the tears of the oppressed, and they had no one to comfort them!
    On the side of their oppressors there was power, and there was no one to
    comfort them.  [2] And I thought the dead who are already dead more
    fortunate than the living who are still alive.  [3] But better than both
    is he who has not yet been and has not seen the evil deeds that are done
    under the sun.
  \end{quote}
\subsection*{Notes}
\begin{itemize}
  \item{Three points that relate to the question: ``Who really is in
  control?'' Modern man likes to think that he is in control, but we need to
  be disabused of that notion.  Three points that show us why we aren't
  really in control:
  \begin{itemize}
    \item{Time: we don't control time, time controls us.}
    \item{Transcience: we may think highly of ourselves, but we are temporary.}
    \item{Theology: God and the teaching of God's word.  What is there in
    theology that we can use to fight the ravages of time and also our
    transcience?}
  \end{itemize}}
  \item{Verse 1-8 is a nice poem, usually it is taken in a positive light to
  to say: ``bad things only last for a season, there will be good that come
  after the bad''.  While the poem is certainly beautiful, the author of
  Ecclesiastes means it here in a negative light; see verse 9 and verse 15.
  Especially verse 15, it seems to imply that God is the one that seeks to
  cycle through what has been done before.  Verse 1-8 can also be taken
  negatively; when we want to dance, sometime happens beyond our control that
  causes us to mourn.  When we want to laugh, something happens beyond our
  control that causes us to cry.  The author seems to say that we are in some
  sort of God ordained cycle of good-and-bad that we have no control over.
  The cycles of life also might cause people to be sick of the tedium.  In
  other religions, this cycle of life is also understood from a negative
  perspective.  In Buddhism and Hinduism, we are stuck in the karmic cycle of
  life-death-rebirth, and the goal is to escape this cycle altogether.}
  \item{We can't control time, and that is made worse by the fact that our
  life is short, or transient.  We might have thought that: ``ok I can't
  control time, but if I can live for longer, I might have some limited
  control''.  God says `nope'.  ``Dust we are, and dust we shall return''.
  This is something that occurs to all, rich/poor, popular/unpopular,
  beautiful/ugly.  And ironically, even though we are all transient, God has
  put eternity in our hearts, so that we all (whether we are religious or
  not) aspire to escape the transcience.  Richard Dawkins, well-known
  materialist/atheist, once told his granddaughter: ``when I'm gone, look up
  at the stars, and remember grandpa''.  For even an atheist like Dawkins,
  even he wants to be remembered past his death; he can't handle the
  ``truth'' that he propounds that we are all cosmic accidents that don't
  matter.  In spite of all our human attempts to overcome transcience, we
  humans can't do that, yet we all naturally want to, since God has put
  eternity in our hearts.  How frustrating is that!  Eternity being put in
  our hearts also means that we have this yearning for the big picture, we
  want to find out how our life fits into this big picture.  Sadly, we can't
  see the big picture as limited human beings, the author of Ecclesiastes
  says: ``yet so that he cannot find out what God has done from the beginning
  to the end''.  How frustrating is that!  We are not in control of how
  people remember us after we are gone, we are not in control of the impact
  that we have on the big picture, etc.  Is God trolling us?  We are all
  proud people, and perhaps this is a way for God to humble us.}
  \item{How do we now face the problems of time and transcience that the
  author of Ecclesiastes has raised?  Note that Ecclesiastes has not resolved
  the answer fully, but has only dropped hints.  The final answer/resolution
  is found only in the gospel.  What are some hints that the author of
  Ecclesiastes has dropped?
  \begin{itemize}
    \item{Time is cyclic, but from God's perspective, time actually has a
    ``telos'', a destination that he wants to direct the flow of time to.
    Time is tedious, but God has made everything beautiful in its time.  The
    solution to the tedium of time is to put God in the equation; once we
    realise that God is at work mysteriously, and that God is with us, then
    every mundane/conventional moment can be a moment of beauty.  E.g for a
    housewife who does nothing but housework daily, it might seem really
    tedious, but with God in her heart, because she knows God is working in
    every moment, she can enjoy every moment.  E.g one person might want to
    celebrate, but God has stricken him with a disease outside of his
    control.  What can he do?  With God with him, he can still celebrate, but
    in a different way; once he knows that all this is part of God's plan for
    His purposes, he can work with God's plan joyfully to proclaim God's
    glory even in his sickness.  These are some of the hints that the author
    of Ecclesiastes has dropped for us to answer the tyranny of time.}
    \item{The question to the transcience of time has an answer.  Richard
    Dawkins and all human beings generally want to be remembered.  The answer
    to this is not in this particular text, but in the later parts of
    Ecclesiastes, we see that there is a final judgement and hence a final
    resurrection.  We are not really transient; while people might forget us,
    God will not, and in fact the people who forget us will get to know us
    when we are all resurrected.  There is an eternity for us after our
    death.  The resurrection and the final judgement is also a hint to the
    problem of us not being able to see the big picture; one day we will be
    able to see the big picture, when we are resurrected.  }
    \item{Now, if we aren't really transient, if we know that the big picture
    exists (though we can't see it), if we know that God ultimately directs
    time to a beautiful end (though we can't see it), we know that our work
    is not meaningless but is ultimately meaningful.  Hence, we can go about
    our work joyfully today, even though we can't see the meaning today,
    because we know that there is an ultimate meaning that we are
    participating in.  We can be joyful in all circumstances, in times of
    rejoicing and also in times of mourning, because we know that things are
    in the hands of God.  Things are in the hands of God, who works all
    things for our good (Romans 8:28). }
  \end{itemize}}
\end{itemize}