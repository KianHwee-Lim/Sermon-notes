\section{22nd May 2022: TITLE}
\subsection*{Text: Ecclesiastes 5:8-6:}
  \begin{quote}
    [8] If you see in a province the oppression of the poor and the violation
    of justice and righteousness, do not be amazed at the matter, for the
    high official is watched by a higher, and there are yet higher ones over
    them.  [9] But this is gain for a land in every way: a king committed to
    cultivated fields.

    [10] He who loves money will not be satisfied with money, nor he who
    loves wealth with his income; this also is vanity.  [11] When goods
    increase, they increase who eat them, and what advantage has their owner
    but to see them with his eyes?  [12] Sweet is the sleep of a laborer,
    whether he eats little or much, but the full stomach of the rich will not
    let him sleep.

    [13] There is a grievous evil that I have seen under the sun: riches were
    kept by their owner to his hurt, [14] and those riches were lost in a bad
    venture.  And he is father of a son, but he has nothing in his hand.
    [15] As he came from his mother’s womb he shall go again, naked as he
    came, and shall take nothing for his toil that he may carry away in his
    hand.  [16] This also is a grievous evil: just as he came, so shall he
    go, and what gain is there to him who toils for the wind?  [17] Moreover,
    all his days he eats in darkness in much vexation and sickness and anger.

    [18] Behold, what I have seen to be good and fitting is to eat and drink
    and find enjoyment in all the toil with which one toils under the sun the
    few days of his life that God has given him, for this is his lot.  [19]
    Everyone also to whom God has given wealth and possessions and power to
    enjoy them, and to accept his lot and rejoice in his toil—this is the
    gift of God.  [20] For he will not much remember the days of his life
    because God keeps him occupied with joy in his heart.

    [1] There is an evil that I have seen under the sun, and it lies heavy on
    mankind: [2] a man to whom God gives wealth, possessions, and honor, so
    that he lacks nothing of all that he desires, yet God does not give him
    power to enjoy them, but a stranger enjoys them.  This is vanity; it is a
    grievous evil.  [3] If a man fathers a hundred children and lives many
    years, so that the days of his years are many, but his soul is not
    satisfied with life’s good things, and he also has no burial, I say that
    a stillborn child is better off than he.  [4] For it comes in vanity and
    goes in darkness, and in darkness its name is covered.  [5] Moreover, it
    has not seen the sun or known anything, yet it finds rest rather than he.
    [6] Even though he should live a thousand years twice over, yet enjoy no
    good—do not all go to the one place?

    [7] All the toil of man is for his mouth, yet his appetite is not
    satisfied.  [8] For what advantage has the wise man over the fool?  And
    what does the poor man have who knows how to conduct himself before the
    living?  [9] Better is the sight of the eyes than the wandering of the
    appetite: this also is vanity and a striving after wind.

    [10] Whatever has come to be has already been named, and it is known what
    man is, and that he is not able to dispute with one stronger than he.
    [11] The more words, the more vanity, and what is the advantage to man?
    [12] For who knows what is good for man while he lives the few days of
    his vain life, which he passes like a shadow?  For who can tell man what
    will be after him under the sun?
  \end{quote}
\subsection*{Notes}
\begin{itemize}
  \item{Greed-the desire for more money than we need-is not good.  Money is
  not itself good too.  Yet it is not hard to see that today, money makes the
  world go round.  Money has no eternal significance, and any desire to
  accumulate more money will lead to sadness.  The only way to satisfaction
  is in Christ.}
  \item{Firstly, we see how greed leads to misery for others.  Even though
  the officials of the land were supposed to serve the common folk, through
  their corruption and greed they exploited the poor peasants.  Chapter 5
  verse 9 is hard to understand in the Hebrew, a better translation would be
  the NLT over the ESV, who says ``even the king milks the land for his own
  profit''.  The greed of the ruling class has brought about a lot of misery
  for the common people.  Greed causes one to care only for his own material
  gain, at the expense of others.  Greed runs contrary to how we should love
  others, which is central to the Christian faith.  As Paul says in
  Philippians 2, we are to look not only to our own needs, but to the needs
  of others.  Love is the antithesis of greed; love shares with others, greed
  hoards for yourself.  We can see this in our world today; just look at some
  billionaires who become rich by exploiting their employees.}
  \item{The reason why people chase and accumulate wealth is because we think
  it makes us happy.  Yet greed, a chase for more money, only make us more
  miserable.  The preacher observes in this text here that greedy and rich
  people are often miserable.  In verse 11, we see that as we have more
  money, more people will come and help us spend it.  Moreover in verse 12,
  we see that the richer we become, the more anxiety we will have over losing
  that money; for example a poor laborer doesn't need to worry about the
  crash of the US stock market lol.  It is a tragedy to chase money so hard
  for happiness.  Beware of greed that leads only to misery by yourself.
  This misery is compounded by the fact that money has no significance at
  all.  We can't take money with us when we die.  As Ecclesiastes 5:15 say,
  we go out of the world as naked and as empty as we came.  Moreover, money
  doesn't help us find favour in God's sight; in fact, to whom more money is
  given, more is expected.  In Luke 16, we see the parable of the rich man
  and Lazarus; the rich man ended up in further misery in the afterlife
  because he didn't use his money properly.  Money ensnares us and draws us
  further from God.  No one can serve two masters, for either he will hate
  the one and love the other, or he will be devoted to the one and despise
  the other.  You cannot serve God and money.  Money is not a neutral entity,
  it is a master that demands to be served.  It has the power to stoke our
  desire, and to entice us to serve it.  The desire for material good is
  vanity, because there is always new stuff to desire.  In our first world
  country, if we want to seek happiness by chasing material goods, we realise
  that there is always more and more material goods to chase.  E.g if we earn
  money to get the iPhone 13, next year there'll be the iPhone 14.}
  \item{But greed, although sinful, is a search for happiness.  People are
  greedy only because they want happiness, but sadly they don't realise that
  money cannot give happiness.  The desire for happiness is actually a good
  thing, its just that using money (which cannot fulfill that desire) as a
  means to happiness is vanity.  So what can give us happiness in this world?
  The answer is Jesus.  Haha.  Through Jesus we can have communion with God,
  and find eternal rest in that communion.  Jesus Christ is our true riches,
  our pearl of great price, our hidden treasure in the field.  When we have
  Jesus, we have everything.  For us that have Jesus as our treasure, we can
  learn to resist the temptations of money.}
  \item{One way we can do that is to enjoy the here and now, in thanksgiving
  to God.  We need to eat and drink everyday to survive, and we need to work
  everyday to survive.  There is nothing especially attractive in these
  mundane activities.  But rather than thinking that we can be happy only in
  the future, e.g only when we have our first million etc, we should realise
  that we can be happy in the here and now, no matter what we are going
  through.  E.g we don't need to go on an expensive vacation with our family
  to enjoy family time; we can enjoy family time at home.  All these mundane
  moments are given to us by God, and we should enjoy them in thanksgiving.
  Who knows when the last mundane moment we will have is...}
  \item{Next, we can resist the temptation to accumulate more money by giving
  money away to the poor.  This is also an obedience to Jesus' command for us
  to love our neighbour.  The more we learn to give money away, the less
  money is able to tempt us into greed.  If we have more than enough for
  ourselves and our family, then we should consider letting the extra money
  go.}
  \item{Ultimately, we are called to seek first the kingdom of God and his
  righteousness.  Whatever we need in our life will be given to us by God.
  Of course we must still work and earn an honest living, but we do so in a
  utilitarian manner (using money to love God and love neighbour and for our
  basic upkeep) rather than seeking money itself as a means to an end.}
\end{itemize}