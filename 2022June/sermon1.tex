\section{5th June 2022: Sent out to witness}
\subsection*{Text: Acts 1:1-11}
  \begin{quote}
    [1] In the first book, O Theophilus, I have dealt with all that Jesus
    began to do and teach, [2] until the day when he was taken up, after he
    had given commands through the Holy Spirit to the apostles whom he had
    chosen.  [3] He presented himself alive to them after his suffering by
    many proofs, appearing to them during forty days and speaking about the
    kingdom of God.

    [4] And while staying with them he ordered them not to depart from
    Jerusalem, but to wait for the promise of the Father, which, he said,
    “you heard from me; [5] for John baptized with water, but you will be
    baptized with the Holy Spirit not many days from now.”

    [6] So when they had come together, they asked him, “Lord, will you at
    this time restore the kingdom to Israel?” [7] He said to them, “It is not
    for you to know times or seasons that the Father has fixed by his own
    authority.  [8] But you will receive power when the Holy Spirit has come
    upon you, and you will be my witnesses in Jerusalem and in all Judea and
    Samaria, and to the end of the earth.” [9] And when he had said these
    things, as they were looking on, he was lifted up, and a cloud took him
    out of their sight.  [10] And while they were gazing into heaven as he
    went, behold, two men stood by them in white robes, [11] and said, “Men
    of Galilee, why do you stand looking into heaven?  This Jesus, who was
    taken up from you into heaven, will come in the same way as you saw him
    go into heaven.”
  \end{quote}
\subsection*{Notes}
\begin{itemize}
  \item{Christians all share the same mission, regardless of their individual
  backgrounds. They are all supposed to be on the same team.}
  \item{Luke wanted us to understand that Jesus' mission didn't end with
  Jesus going to the Father into heaven.  In fact, Jesus' mission continues
  through His disciples, as God works through them through His Spirit.}
  \item{We are on a mission because Jesus is the risen Lord.  In this text,
  from verse 6, we see that the disciples anticipated Jesus to restore the
  kingdom to Israel.  After all, if Jesus could conquer death, surely He
  could restore the kingdom to Israel.  But Jesus didn't answer the
  disciples' question about the kingdom directly.  Jesus told them that God's
  kingdom will indeed be established (God is sovereign and His purposes
  cannot be thwarted).  But more importantly, Jesus told them not to worry
  about the details about how the kingdom would come, but instead Jesus
  commissioned them (v8) to continue carrying on Kingdom work, and Jesus
  promised them that they would have the power to do so through the power of
  the Holy Spirit.}
  \item{Now, since ``God so loved the world'', and that ``God wills nobody to
  perish, but all to repent'', then God's kingdom work cannot end with the
  disciples' lifetime; it carrys on through the disciples of the disciples,
  and through their disciples, etc, all the way now to us.  }
  \item{We cannot be indifferent to how we conduct our life here, we cannot
  live as if Jesus' death, resurrection and ascension has no effect on us.
  When we confess that ``Jesus is seated at the right hand of the Father'',
  our confession should have an effect on our lives, for example in how we
  relate to other people.}
  \item{Now, also, using the metaphor of the Church being Jesus' body, since
  the Head is on a mission the body must be too!  I.e, just as Jesus who is
  God's Son carried out God's mission, we are to participate in His mission.}
  \item{The purpose of discipleship is to train us and to shape us to perform
  the mission that God called us to.  The process of discipleship trains us
  to understand how things in the world relate to God's mission as revealed
  in His Word, and also produces spiritual fruit in us that is essential for
  God's mission.  It is more than just coming to church, making friends,
  reading the Bible together as a hobby, etc.  We have to be mission-minded.}
  \item{Btw, our mission is to be a witness of the Risen Lord.  A witness is
  someone who testifies to what he has seen or experienced.  So for us, we
  have experienced God's amazing love in our life, and we are to witness to
  it.  In the OT, Israel was said to be God's witness (Isaiah 43), and
  because of their failure, God sent His Son, Jesus to be His perfect
  witness.  And for us today, though we have not seen Jesus' resurrection,
  the content of our witness is also still to be centred on truths about the
  person and work of Christ.  Who is this person Jesus, is Jesus truly Lord
  and truly savior?}
  \item{In our witness, there are the objective truths (that Jesus is a
  historical figure who was crucified under Pontus Pilate, and who was
  resurrected and ascended to God, also that Jesus is the Son of God, etc
  etc), and there also are subjective truths (our conversion experience, how
  Jesus has changed our lives, etc).  Both are important.  Btw our subjective
  truths have to be constantly updated, it is not sufficient to talk about
  how Jesus saved you $10$ years ago, we also need to talk about how Jesus
  saves us now.}
  \item{The scope of our witness also extends to everyone and every aspect of
  our lives.  ``...in Jerusalem and in all Judea and Samaria, and to the end
  of the earth...'' it is said.  Jerusalem is a hostile place to the
  disciples (since Jesus was crucified there), Samaria historically has
  baggage with Jewish people, and ``the ends of the earth'' wasn't even well
  defined back then.  The difficulties are super difficult, but the disciples
  still did it.}
  \item{In our lives today, we might need to witness in ``Jerusalem'', a
  familiar but hostile place, perhaps to our family.  We might also need to
  witness in ``Judea'', still kinda familiar but might be awkward, perhaps to
  our friends.  We might also need to witness in ``Samaria'', to people who
  we don't really know and might hate us.}
  \item{Our witness to the world becomes more convincing if it can also be
  seen in our lives.  Hence, we need to be pure in heart, etc, we need the
  fruits of the Spirit.}
  \item{God empowers us with His Spirit for our mission.  God's Spirit is our
  Helper, and God's Spirit mediates to us the presence of God and His Son.
  See John 14:16-18.  In contemporary Judaism, there is the ``wailing wall'',
  where people go there and cry because under their view, God's presence has
  left them, and they want God to return to them.  But under the Christian
  view, we have God's presence with us through His Spirit!}
  \item{The Spirit is God who convicts hearts and transform lives.  Nothing
  else can make spiritually dead people come back to spiritual life, and that
  power is God's divine power through His Spirit.  As it is said in Ezekiel,
  God's Spirit gives us a heart of flesh and takes away our heart of stone.
  If we are born again, we would also have experienced this power in our
  lives.  God's Spirit will continue to convict the world of sin, and that as
  we try our best through God's Spirit to preach the Word faithfully and as
  to live faithful lives, ultimately it is God's power who changes the hearts
  of those who hear and see us.}
\end{itemize}