\section{20th March 2022: False Piety or True Devotion}
\subsection*{Text: }
  \begin{quote}
    [38] And in his teaching he said, “Beware of the scribes, who like to
    walk around in long robes and like greetings in the marketplaces [39] and
    have the best seats in the synagogues and the places of honor at feasts,
    [40] who devour widows’ houses and for a pretense make long prayers.
    They will receive the greater condemnation.”

    [41] And he sat down opposite the treasury and watched the people putting
    money into the offering box.  Many rich people put in large sums.  [42]
    And a poor widow came and put in two small copper coins, which make a
    penny.  [43] And he called his disciples to him and said to them, “Truly,
    I say to you, this poor widow has put in more than all those who are
    contributing to the offering box.  [44] For they all contributed out of
    their abundance, but she out of her poverty has put in everything she
    had, all she had to live on.”
  \end{quote}
\subsection*{Notes}
\begin{itemize}
  \item{The two stories in today's passage are deliberately arranged as so to
  contrast the scribes with the widow.  The scribes neither loved God nor
  people, but the widow loved God with total devotion.}
  \item{The scribes appeared to be lovers of God, but they really are lovers
  of self (v38-39).  They like to be viewed as pious by people, such as
  making long prayers in public, as compared to what Jesus said in the Sermon
  on the mount about private prayer.  But we know that their piety is false,
  because their actions are unjust; they are said to "devour" widows' houses.
  The scribes were hypocrites, exploiting the poor while still looking pious.
  This is a clear violation of the teaching of the Law and the Prophets, e.g
  in Isaiah 10:1-4.  They did not love their neighbours, especially those who
  were vulnerable.  And this is especially egregious because the scribes were
  supposed to know the Law.  The widow in the story \textit{could} have been
  one who's house was devoured by the scribes.}
  \item{In the 20th century, Hitler was a master of outward religiosity with
  no inward piety. Reflections for us:
  \begin{itemize}
    \item{For leaders in the church, do y'all use your authority and power to
    exploit your sheep for the people under you?}
    \item{For all: are we lovers of self when we should be lovers of God?  Do
    you do things just to get the approval of others, to be seen as a 'good
    Christian'?  Are we pious only for people to see?  We might be able to
    fool the people around us, but we can't fool God.  Do we live a double
    life?  Are we well-behaved in church but for the rest of the week we're
    mean to our spouse/children?  The solution is to repent from this
    hypocrisy, to turn away from sin and to turn to God.}
  \end{itemize}}
  \item{The widow offered a small offering, but Jesus did not despise it.
  The widow offered two copper coins, where a copper coin was a hundredth of
  a denarius, where a denarius is a day's wage for a labourer.  In fact,
  Jesus commended the widow for giving more than the rich.  In a sense, what
  Jesus looked at was not the absolute amount of money $x$ given by a person,
  but the percentage of money given $x/x_{\text{Total}}$.  There was not as
  much sacrifice on the part of the rich, when they gave, but for the widow,
  her offering was costly for her.}
  \item{The widow serves thus as a vivid model for sacrificial discipleship,
  complete surrender, and total trust.  Without the trust that God will care
  for her, she probably wouldn't have given all her money.}
  \item{ ``The story of the poor widow reminds us that in God's economy, the
  size of the gift is of no consequence, what is of consequence is the size
  of the giver's heart''.
  Even if our gift is small, God can multiply it
  anyway (c.f the story of the 5 loaf and 2 fishes).  Reflections for us:
  \begin{itemize}
    \item{Is God speaking to you and nudging you to give more sacrificially?}
    \item{Besides giving, in what other ways can we practice sacrificial
    discipleship that Jesus requires of us?}
    \item{Is our discipleship merely based on convenience?}
  \end{itemize}
  One example: do we plan our day/life such that we give God the remainder of
  our money/time only after we have spent the money/time on ourselves?  Or do
  we give to God the firstfruits of our time/money/energy?}
  \item{Application to worship: online service is definitely more convenient
  than coming to church, but it is a poor substitute.  Do we then do things
  based on our convenience, or do we sacrifice a bit of our convenience so
  that we can obey God's revealed will for us to come together to meet to
  exhort each other.}
  \item{Let us ... (conclusion to be added later)}
\end{itemize}