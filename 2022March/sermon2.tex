\section{13th March 2022: At the Heart of God's Law}
\subsection*{Text: Mark 12:23-34}
  \begin{quote}
    [28] And one of the scribes came up and heard them disputing with one
    another, and seeing that he answered them well, asked him, “Which
    commandment is the most important of all?” [29] Jesus answered, “The most
    important is, ‘Hear, O Israel: The Lord our God, the Lord is one.  [30]
    And you shall love the Lord your God with all your heart and with all
    your soul and with all your mind and with all your strength.’ [31] The
    second is this: ‘You shall love your neighbor as yourself.’ There is no
    other commandment greater than these.” [32] And the scribe said to him,
    “You are right, Teacher.  You have truly said that he is one, and there
    is no other besides him.  [33] And to love him with all the heart and
    with all the understanding and with all the strength, and to love one’s
    neighbor as oneself, is much more than all whole burnt offerings and
    sacrifices.” [34] And when Jesus saw that he answered wisely, he said to
    him, “You are not far from the kingdom of God.” And after that no one
    dared to ask him any more questions.
  \end{quote}
\subsection*{Notes}
\begin{itemize}
  \item{In Jesus' time, many religious leaders confronted Jesus with
  questions.  But they weren't interested in finding out the truth; they were
  hoping to use these questions to trap Jesus so that Jesus would lose his
  credibility.  But not the scribe in this passage; this scribe was said to
  be not far from the kingdom of God.  The scribe had a genuine question,
  which was to find out what the heart of godly living was.  And in this
  passage, Jesus answered the scribe directly.}
  \item{Instead of quoting a particular law, Jesus quoted the Shema
  (Deuteronomy 6).  There is only one God, and we are to worship God
  single-mindedly.  God graciously revealed Himself to Abraham and His
  descendants, and as a response to God's loving self revelation, Israel was
  supposed to pledge their allegiance to God.  The LORD God allows Israel
  (and us) to know about Himself; His gracious self revelation provides us
  with true knowledge about Him and His nature.  God is one, this means that
  God is unique, and that there is only one God.  And hence we are to worship
  God alone.}
  \item{God's law's are meant to reveal His nature.  When God gave Israel His
  law, He has already rescued them out of Egypt.  It is not that Israel has
  to earn God's love by following His law; God already loved Israel when He
  gave them the Law, and the purpose of the Law giving was to teach Israel
  how to maintain a relationship with Him, especially since Israel was
  amongst the pagan nations.}
  \item{To love God is more than a fuzzy feeling towards God; love for God is
  doing actions demonstrated in faithfulness.  If we read on in Deuteronomy
  6, there are some actions that God suggests would be an expression of
  loving Him with all our heart, soul, mind and strength.  These actions
  include passing down God's truth to the next generation (Deuteronomy
  6:6-9).  These actions also include giving thanks for our food, because
  ``when we eat and are full, then take care lest you forget the LORD...''}
  \item{Running towards other idols will destroy us; God is the only source
  of life, and when we turn towards other idols and away from God, we are
  turning away from life and turning unto death.}
  \item{Apart from loving God, the next greatest commandment was loving our
  neighbour.  What Jesus did that unique amongst the rabbis of His time; in
  the past, other rabbis also summarised the Law as loving our neighbour, but
  what Jesus did was to combine loving God and loving our neighbour.  We
  cannot truly love God unless we love our neighbour, and we cannot truly
  love our neighbour unless we love God.  For example, when we look at
  portions like Leviticus 19:10,18 we realise that we love our neighbour
  because ``I am the LORD your God''.}
  \item{The scribe understood what Jesus said (see Mark 12:34).  Instead of
  merely a confession of faith, God calls also for a demonstration of love
  towards Him and others.  E.g in Leviticus, it is said that when we gather
  grapes for the vineyard, we should leave some for the sojourner.  To do so
  ungrudgingly, one must first realise that God is the one who blesses his
  vineyard, and one must first not be anxious about his harvest, e.g one must
  first trust God.  And when one trusts God, he can love his neighbor easily
  because there is no longer any anxiety etc preventing him from loving his
  neighbour by leaving grapes.}
  \item{The scribe understood that it is not about outward religiousity, but
  it is about the attitude of the heart.  See Isaiah 1:10-20, our scripture
  reading for today.  Or see many other of the OT prophets haha...  Christian
  living is more than just mechanicallly memorising scripture and turning up
  on Sunday because it is an obligation, true Christian living includes
  loving our neighbour too (which would demonstrate our love for God).}
  \item{However, when we try to love God with all our heart, soul, mind and
  strength, and to love our neighbour as ourselves, we'll soon realise that
  we can't do it with our own strength.  We need a savior.  And this savior
  is Jesus.  Even in the OT, we know that even as the sincere, faithful OT
  people tried their best to love God, they still needed to trust in God for
  the forgiveness of their sins, because they realise that they fall short
  (e.g all the Psalms, like Psalm 51).  Now that God has revealed Himself to
  us through the person of His Son, we put our trust in Jesus, who is the
  image of the invisible God and the exact imprint of God's glory.  Just as
  the OT saints trusted in God to forgive their failures to love God
  perfectly, today we trust in Jesus to forgive us our failures to love God
  perfectly.  And we also know that as we put our trust in Jesus, He will
  slowly transform us to help us to better love Him and our neighbours.}
  \item{Loving our neighbours means more than doing good deeds (it is not
  less than that), loving our neighbours also means inviting them to follow
  Jesus, and helping them to be disciples of Jesus, because Jesus is the
  source of life.}
\end{itemize}