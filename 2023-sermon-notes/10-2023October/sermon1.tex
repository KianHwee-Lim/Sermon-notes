\setcounter{figure}{0}

\section{1st October 2023: In the beginning was the Word}
\subsection*{Text: John 1:1-18}
  \begin{quote}
    [1] In the beginning was the Word, and the Word was with God, and the Word was God. [2] He was in the beginning with God. [3] All things were made through him, and without him was not any thing made that was made. [4] In him was life, and the life was the light of men. [5] The light shines in the darkness, and the darkness has not overcome it.

    [6] There was a man sent from God, whose name was John. [7] He came as a witness, to bear witness about the light, that all might believe through him. [8] He was not the light, but came to bear witness about the light.

    [9] The true light, which gives light to everyone, was coming into the world. [10] He was in the world, and the world was made through him, yet the world did not know him. [11] He came to his own, and his own people did not receive him. [12] But to all who did receive him, who believed in his name, he gave the right to become children of God, [13] who were born, not of blood nor of the will of the flesh nor of the will of man, but of God.

    [14] And the Word became flesh and dwelt among us, and we have seen his glory, glory as of the only Son from the Father, full of grace and truth. [15] (John bore witness about him, and cried out, “This was he of whom I said, ‘He who comes after me ranks before me, because he was before me.’”) [16] For from his fullness we have all received, grace upon grace. [17] For the law was given through Moses; grace and truth came through Jesus Christ. [18] No one has ever seen God; the only God, who is at the Father’s side, he has made him known.
  \end{quote}
\subsection*{Notes}
\begin{itemize}
  \item{This sermon series will be on the seven “I AM” discourses in John.}
  \item{Today’s sermon is the prologue to John. Three parts for today: mystery of the Word, the coming of the light, and the Word make flesh.}
  \item{The word “Word” is translated from the greek Logos. This word had many different meanings to the different groups of people back then. The jews would have one understanding of Logos (e.g Philo of Alexandria), the stoics would have another.}
  \item{For the jews specifically, the Logos was God’s creative power in creation. So if John stopped at “the Word was God”, it would probably be ok to the Jews, c.f Philo. I.e, since the Word is God’s creative power that creates and sustains everything, the Word is God. So, proverbs 8.}
  \item{But John went on to say “the Word was with God”, and that would have riled up the Jews, because that would contradict what they thought was monotheism. So here we have John introducing the idea of the Trinity. So here, John is preparing for later where be will introduce Jesus as the Son of God, as God’s Word. }
  \item{The next statement about Jesus being the true light that is coming into the world. This one is also quite cool, it plays on the common understanding that light is good and darkness is bad. Such as in Genesis, where God said “let there be light”.}
  \item{So far, John had not said anything too mind blowing yet. So far everything he said was metaphysical in nature, and we know how philosophers like to discuss these metaphysical things. Even if the Logos of God is another divine person apart from God, this is still metaphysical.}
  \item{The most mind-blowing part of this prologue is “the Word became flesh”. This is abit mind-blowing because the Word is a metaphysical concept, but flesh is a concrete concept. Especially in the dualistic worldview of spiritual vs physical, this statement would be offensive. And especially for the Jews, since the Logos is transcendent above space and time, so how could the Word take on flesh and be embodied, stuck in a certain space-time point? \KH{(By space-time point is meant a \textit{world line})}.}
  \item{So here, John is saying that the Word is a divine person apart from God, and the Word took on human flesh to reveal the Father to the world, to be the light that gives life to all who believes, so that to those who believe, they have the right to be the children of God. }
\end{itemize}