\section{1st January 2023: The True Way Misison statement}
\subsection*{Text: Ephesians 4:7-16}
  \begin{quote}
    [7] But grace was given to each one of us according to the measure of Christ’s gift. [8] Therefore it says,

    “When he ascended on high he led a host of captives,
        and he gave gifts to men.”

    [9] (In saying, “He ascended,” what does it mean but that he had also descended into the lower regions, the earth? [10] He who descended is the one who also ascended far above all the heavens, that he might fill all things.) [11] And he gave the apostles, the prophets, the evangelists, the shepherds and teachers, [12] to equip the saints for the work of ministry, for building up the body of Christ, [13] until we all attain to the unity of the faith and of the knowledge of the Son of God, to mature manhood, to the measure of the stature of the fullness of Christ, [14] so that we may no longer be children, tossed to and fro by the waves and carried about by every wind of doctrine, by human cunning, by craftiness in deceitful schemes. [15] Rather, speaking the truth in love, we are to grow up in every way into him who is the head, into Christ, [16] from whom the whole body, joined and held together by every joint with which it is equipped, when each part is working properly, makes the body grow so that it builds itself up in love.
  \end{quote}
\subsection*{Notes}
\begin{itemize}
  \item{In Ephesians 2, there is the imagery of Christ as the cornerstone,
  the apostles and prophets as the foundation, and then we are built up on
  the prophets and the apostles.  In our text, a similar thing happens, but
  now the imagery is that of a human body, made up of many parts but working
  together, and all these parts are built up in love and grow due to Christ
  giving the church apostles and teachers and prophets and evangelists.}
  \item{True way has a mission statement and a vision statement. The vision statement is what our church aspires to be, and the mission statement is what our church wants to do everyday to achieve that vision statement. The mission statement is that: compelled by the love of God, we engage the community, evangelise the lost, establish the faith of the saints, and equip the saints to do mission work.}
  \item{Firstly, let’s talk about the motivation aspect in our mission
  statement, the “love of God that compels us”.  Here “love of God” can mean
  two things, either God’s love for us or our love of God.  For the former,
  we have 1 John 4:9 as an example.  And for the latter, we have 1 John 2:4-5
  as an example.  The latter is also an echo to the great commandment.  How
  are the former and latter connected?  The former leads to the latter; we
  love because He first loved us (1 John 4:19).  God loving us is the
  starting point to us loving God.  This motivation is important; if the task
  is too difficult, the sacrifice too great, it is this motivation that keeps
  us going.}
  \item{Next, lets talk about the “mission” aspect of our mission statement.
  The rest of our mission statement is just an elaboration of the great
  commission (Matt 28:18-20).  In the great commission text, we have the
  “going”, the “make disciples”, the “baptising” and the “teaching”.  The
  first two words above are related to the “engage and evangelise” part of
  our mission, and the next two words are related to “establish and equip”
  part of our mission.}
  \item{Lastly, lets talk about the means that are given to us to carry out
  the mission.  The key here is Eph 4:7.  As the Father sends the Spirit
  through the Son to us (in the economy of salvation), the Spirit gives us
  all individual gifts that we use to build up the body of Christ, to build
  each other up in love.  }
  \item{The method by which the body of Christ is built up is that God gives the Church prophets and the apostles, which are the foundation of the Church. Then God also gives the Church evangelists and pastors and teachers which build upon this foundation of the church, who preach and teach the rest of the saints to \textbf{equip} the saints. }
  \item{But what are the rest of the saints equipped to do? They are equipped to do the work of ministry. So it is not that all the work that God wants us to do is done by the evangelists and pastors, the majority of God’s work is actually done by the rest of the saints! The pastors and evangelists are just the ones that equip the rest of the saints. It is as if the pastors and evangelists are just the blacksmiths and the war academy, but the actual army is the rest of the saints. But of course, we don’t discount the work of the pastors and the teachers and the evangelists. These teaching ministries help us to be mature in Christ, which help us to discern truth from heresy for example, and ultimately help us to \textbf{speak the truth in love to one another}, which helps us to build up the body of Christ (the establishing and equipping part of our mission). } 
  \item{Examples: after the sermon by the pastors, we (the congregation) can share with each other a lesson from the sermon. This is the best example of how the pastor equips us to build up the church. We can also listen to each other’s conversion story, we can also pray for each other, we can also share how God has been working in our lives. If we are in a DG, we can prepare for the BS and participate. We can also remind each other, to rebuke each other in love. We can also serve one another using the gifts God has given us, we can welcome each other and love each other. }
  \item{We should not come to church expecting to be served, with a consumer mindset. That leads to much complaining. But we should come to church as builders, as establishers, so we will want to solve problems in Church when they arise (which they will)!}
  \item{Conclusion: motivation $\rightarrow$ mission $\rightarrow$ means $\rightarrow$ method. And what is the goal? The goal is in v13-15 of our text, and basically it is to make more people members of the Church and make each member of the Church more like Jesus. }
\end{itemize}