\section{8th January 2023: The True Way Misison statement - Engaging and Evangelising}
\subsection*{Text: 2 Corinthians 5:10-21}
  \begin{quote}
    [10] For we must all appear before the judgment seat of Christ, so that
    each one may receive what is due for what he has done in the body,
    whether good or evil.

    [11] Therefore, knowing the fear of the Lord, we persuade others.  But
    what we are is known to God, and I hope it is known also to your
    conscience.  [12] We are not commending ourselves to you again but giving
    you cause to boast about us, so that you may be able to answer those who
    boast about outward appearance and not about what is in the heart.  [13]
    For if we are beside ourselves, it is for God; if we are in our right
    mind, it is for you.  [14] For the love of Christ controls us, because we
    have concluded this: that one has died for all, therefore all have died;
    [15] and he died for all, that those who live might no longer live for
    themselves but for him who for their sake died and was raised.

    [16] From now on, therefore, we regard no one according to the flesh.
    Even though we once regarded Christ according to the flesh, we regard him
    thus no longer.  [17] Therefore, if anyone is in Christ, he is a new
    creation.  The old has passed away; behold, the new has come.  [18] All
    this is from God, who through Christ reconciled us to himself and gave us
    the ministry of reconciliation; [19] that is, in Christ God was
    reconciling the world to himself, not counting their trespasses against
    them, and entrusting to us the message of reconciliation.  [20]
    Therefore, we are ambassadors for Christ, God making his appeal through
    us.  We implore you on behalf of Christ, be reconciled to God.  [21] For
    our sake he made him to be sin who knew no sin, so that in him we might
    become the righteousness of God.
  \end{quote}
\subsection*{Notes}
\begin{itemize}
  \item{As per last week, we are equipped by the pastors and the teachers to
  do the work of ministry.  And part of the work of ministry is the
  \textit{ministry of reconciliation}, which is the key theme in today's
  text (the word reconciliation appears like $5$ times).}
  \item{The purpose of engaging and evangelising those who are not Christians
  yet is so that we can help them be reconciled with God.}
  \item{In this ministry of reconciliation, the \textbf{message} of
  reconciliation is key.  And the key message is found in v21 of our text,
  which can be paraphrased as: ``For God made CHrist, who never sinned, to be
  the offering for oursin, so that we could be made right with God
  \underline{through} Christ''.  The ministry of reconciliation implies that
  that is something that needs reconciliation, and that something is sin;
  because of the Fall, all of us are born with sinful human natures and are
  by nature enemies of God, unless God reconciles us to himself. And God has extended this offer of reconciliation through the death of His Son for our sins.}
  \item{After we have been reconciled, we have a new \textbf{master}.  In the
  past while we were sinners, we lived for ourselves.  We were our own
  master.  Now, God is our master, and we are to live for Christ rather than
  for ourselves.  If our Master then has given us the ministry of
  reconciliation, shouldn't we take this task seriously?  This ministry of
  reconciliation is difficult.  We will be rejected, and we will get hurt.
  But we must still do it.  When Jesus was on earth, He was similarly
  rejected by those He was sent to; we should similarly expect rejection.  }
  \item{What compels us to live for Christ reather than for ourselves?  What
  is our \textbf{motivation} for doing this ministry of reconciliation
  despite how difficult it is?  The motivation is the \underline{love of
  Christ}!  In fact, when we love Jesus, we will be motivated to love other
  people too.  And the best way to love another person is to tell the person
  about the message of Jesus.  This is the primary motivation.}
  \item{Another secondary motivation though is that at the end of our lives,
  we all have to appear before the judgment seat of Christ.  This judgment is
  not for our salvation, but to determine our reward.  In short, the
  secondary motivation here is the \underline{fear of the Lord}.  Of course,
  this secondary motivation is related to the primary motivation; the fear of
  the Lord is related to the love of the Lord!  }
  \item{As ambassadors of Christ, we are like high ranking diplomats sent to
  a foreign country.  We are in the world but not of the world; as diplomats,
  our citizenship is in the Kingdom of heaven.  Hence, as ambassadors of the
  Kingdom of heaven, we are to live our Kingdom values, to make the Kingdom
  of God an attractive place to others.}
  \item{Now, being an ambassador means that the model of ministry is
  ``incarnational''.  Just like how diplomats are more useful if they go out
  of the embassy and talk to others, we are more useful if we go out into the
  world to live among the world and serve them.  Our service is part of how
  we \textit{engage} the lost, and this engagement is the stepping stone to
  \textit{evangelism}.}
  \item{In True Way, we have organised ministries like the Tuition ministry,
  the Ukelele ministry, etc.  These are opportunites for us to engage and
  evangelise.  But apart from these organised ministries, we can also be
  ambassadors wherever God places us in our lives.  We should always be
  praying for and on the lookout for opportunites for spiritual conversations
  with whoever God brings into our lives.  We can have these spiritual
  conversations in an impromptu manner with whoever crosses our paths, but we
  should also have these spiritual conversations in a more intentional
  manner.  Being intentional would mean things like having a list of people
  that we want to engage, and then constantly reaching out to them in an
  intentional manner.  }
  \item{Closing remarks: 
  \begin{itemize}
    \item{What if people are not interested? Ans: continue to be their friend! Don't make them feel like you have an ulterior motive, that you only want to share the gospel with them. Let us be genuinely interested in our lives and let us love them truly.}
    \item{What if we don't feel adequate? For example, what if we feel that we are not good at explaining the gospel, or what if we feel that we don't know how to direct the conversations to spiritual conversations? Ans: this is where the ``equipping and establishing'' comes in. There are programmes in church that equip us to do stuff. Or in a less formal setting, we can ask other more experienced Christians to help us!}
    \item{What if we don't see any results: Ans: don't measure our results by
    how many people we bring to Christ, but measure our results by how much
    we can \textbf{show} and \textbf{share}.  As long as we are effective in
    \textbf{showing} the gospel through how we live, and as long as we are
    effective in \textbf{sharing} the gospel through the conversations that
    we have, we can consider ourselves to have had good results.}
  \end{itemize}}
\end{itemize}