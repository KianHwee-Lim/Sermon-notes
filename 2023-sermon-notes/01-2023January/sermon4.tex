\section{22nd Jan 2023: The worship of the creator}
\subsection*{Text: }
  \begin{quote}
    [1] After this I looked, and behold, a door standing open in heaven!  And
    the first voice, which I had heard speaking to me like a trumpet, said,
    “Come up here, and I will show you what must take place after this.” [2]
    At once I was in the Spirit, and behold, a throne stood in heaven, with
    one seated on the throne.  [3] And he who sat there had the appearance of
    jasper and carnelian, and around the throne was a rainbow that had the
    appearance of an emerald.  [4] Around the throne were twenty-four
    thrones, and seated on the thrones were twenty-four elders, clothed in
    white garments, with golden crowns on their heads.  [5] From the throne
    came flashes of lightning, and rumblings and peals of thunder, and before
    the throne were burning seven torches of fire, which are the seven
    spirits of God, [6] and before the throne there was as it were a sea of
    glass, like crystal.

    And around the throne, on each side of the throne, are four living
    creatures, full of eyes in front and behind: [7] the first living
    creature like a lion, the second living creature like an ox, the third
    living creature with the face of a man, and the fourth living creature
    like an eagle in flight.  [8] And the four living creatures, each of them
    with six wings, are full of eyes all around and within, and day and night
    they never cease to say,

    “Holy, holy, holy, is the Lord God Almighty,
        who was and is and is to come!”

    [9] And whenever the living creatures give glory and honor and thanks to
    him who is seated on the throne, who lives forever and ever, [10] the
    twenty-four elders fall down before him who is seated on the throne and
    worship him who lives forever and ever.  They cast their crowns before
    the throne, saying,

    [11] “Worthy are you, our Lord and God,
        to receive glory and honor and power,
    for you created all things,
        and by your will they existed and were created.”
  \end{quote}
\subsection*{Notes}
\begin{itemize}
  \item{Chapter 4 opens with ``after this'', hence we know chapter 4 marks a
  new point in the book of Revelation.  Also, there is a change in setting;
  chapters 1-3 have the setting of the island of Patmos.  Now in chapter 4,
  John is ``spirited away'' into the heavenly throne room. John is seeing a vision of God's heavenly throne room generated by the Holy Spirit. }
  \item{Chapters 4-5 are meant to be one unit, and they are the basis of what
  happens in the rest of the book (much like how the vision of the Son of Man
  is the basis of chapters 1-3).}
  \item{Chapter 4 is purely descriptive.  So, what does this text tell its
  readers?  Instead of trying to tell us something, chapter 4 is trying to
  show us something.  And what chapter 4 is showing us evokes a certain
  response from us.  Chapter 4 shows us the glory of God in display, so that
  we can respond in true worship.  The Christians of John's day need to see
  God's glory, because they were either under persecution or being tempted by
  the lures of the world.  They need to be reminded that God is worthy of
  worship and hence they should persevere in their faith despite persecution
  of temptations.}
  \item{For us, our world today is highly volatile.  Many things can change
  very quickly.  An example was the covid-19 pandemic.  This leads to us
  having fear and anxiety, because despite our best attempts to control our
  future, we can't.  For us, we need a vision of God that is majestic so as
  to remind us that God is in control and that we can continue trusting in
  God.  The glorious vision of God in chapter 4 compels us to worship God and
  hence trust God with our future, since true worship leads to trust.}
  \item{John's description in chapter 4 is less like a photograph of heaven,
  and more like a surrealist painting of heaven.  I.e, it is not literal.  We
  know this is not literal because John describes God the Father as a human
  figure sitting on the throne, but theologically, we know God is Spirit. If we try to take John's description literally, instead of trying to see what the description points to, we will be missing the point.}
  \item{The images that John describes in chapter 4 are very familiar to his
  first century Christian audience (intertextuality!).
  \begin{itemize}
    \item{First of all, John sees only one mega throne.  What this is trying
    to say is that there is only one God, something that harks back to the
    shema.}
    \item{Next of all, John sees the throne made of jasper and carnelian, both precious stones. These stones emphasise that God is glorious.}
    \item{Next of all, John sees a rainbow that has the appearance of
    emerald.  The rainbow could be a reference to the Noahic covenant, or it
    could be just something that elaborates that God is glorious.  The
    reference to the Noahic covenant could be a reference to how in the later
    chapters, God will judge the earth yet preserve it through judgment.}
    \item{Next of all, John sees the sea of glass, and lightning and thunder
    that issued from the throne.  The lighting and thunder that issues forth
    from the throne reminds the readers of the Exodus, and it brings across
    the awesomeness and fearsomeness of God.  While we might find lightning
    and thunder frightful, they are under God's control.}
    \item{The seven torches around the throne refer to the seven spirits of
    God, and this is the same imagery that is found in Zechariah 4.  Hence,
    the seven spirit of God refer to the sevenfold Spirit of God, the Holy
    Spirit (the number seven is just used to symbolise completeness).  }
    \item{The four living creatures are almost the same as those in Ezekiel.
    Possibly, the lion represents the greatest wild animal, the ox represents
    the greatest domesticated animal, the human represents man, and the eagle
    represents the greatest flying animal.  Hence, this shows us that since
    even the greatest living creatures are made to continually praise God,
    hence all living creatures are to continually praise God. Nature is suffused with God's glory, and reflect God's glory. The marvellous aspect of nature in John's description are fitting to describe the glory of God. }
    \item{Forming an outer ring around the mega throne are 24 elders around
    the throne.  This is a re-enactment of an ancient court.  In the gospel,
    Jesus said that Christians are the ones who would be given white robes,
    crowns, and will be seated on thrones.  Hence, the 24 elders could be a
    reference to the people of God.  And the number 24 is because the number
    12 has always been used to refer to God's people.  We have $24 = 12 +
    12$, and the first 12 is a reference to the OT saints (12 tribes), and
    the other 12 is a reference to the NT saints (12 apostles).  The 24
    elders continually affirm that God is the one who creates everything and
    sustains everything, and hence God has control over the all creation.}
  \end{itemize}}
  \item{Now, to focus on the last point above, we note that we do not worship
  sheer power.  Sheer power is not worthy of worship.  Hitler was a powerful
  man, but people wouldn't spontaneously worship him.  Hitler could only
  coerce people to worship him.  In our text, we see the 24 elders
  spontaneously worshipping God.  This means that there must be something
  apart from sheer power in God that is worthy to worship, and that something
  else would be His goodness (see also chapter 5).  From this chapter itself,
  we also see that God's goodness is displayed through His act of creation.
  God does not need to create; He exists as a triune community of love with
  His Son and His Spirit.  Creation is not necessary to God, yet God created
  out of love so that we can enjoy the good creation that He created and so
  we can enjoy His goodness.}
  \item{Hence, our posture as Christians today must be the same as the 24
  elders, we must worship God with true worship.  And trust is necessary to
  true worship of God; hence despite all the anxiety and chaos of daily life, we must trust God instead of ourselves. }
  \item{Now, with all that has been said about the goodness of creation, we
  see that right now, the world is not as perfect as it should have been.
  That is the source of anxiety and chaos that we see in our lives.  This is
  not because God is absent, but this is because of human sin (the fall).
  John's vision shows us that what we see today, raging seas and crashing
  waves, is not what things should be, or what things will be (compare raging
  seas with crystalline, still sea).  In the new creation, all the chaos will
  come to an end, and we can truly see creation as the reflection of God's
  glory, as it should be.}
\end{itemize}