\section{29th January 2023: The Redeeming Lamb}
\subsection*{Text: Revelation 5}
  \begin{quote}
    [1] Then I saw in the right hand of him who was seated on the throne a scroll written within and on the back, sealed with seven seals. [2] And I saw a mighty angel proclaiming with a loud voice, “Who is worthy to open the scroll and break its seals?” [3] And no one in heaven or on earth or under the earth was able to open the scroll or to look into it, [4] and I began to weep loudly because no one was found worthy to open the scroll or to look into it. [5] And one of the elders said to me, “Weep no more; behold, the Lion of the tribe of Judah, the Root of David, has conquered, so that he can open the scroll and its seven seals.”

    [6] And between the throne and the four living creatures and among the elders I saw a Lamb standing, as though it had been slain, with seven horns and with seven eyes, which are the seven spirits of God sent out into all the earth. [7] And he went and took the scroll from the right hand of him who was seated on the throne. [8] And when he had taken the scroll, the four living creatures and the twenty-four elders fell down before the Lamb, each holding a harp, and golden bowls full of incense, which are the prayers of the saints. [9] And they sang a new song, saying,

    “Worthy are you to take the scroll
        and to open its seals,
    for you were slain, and by your blood you ransomed people for God
        from every tribe and language and people and nation,
    [10] and you have made them a kingdom and priests to our God,
        and they shall reign on the earth.”

    [11] Then I looked, and I heard around the throne and the living creatures and the elders the voice of many angels, numbering myriads of myriads and thousands of thousands, [12] saying with a loud voice,

    “Worthy is the Lamb who was slain,
    to receive power and wealth and wisdom and might
    and honor and glory and blessing!”

    [13] And I heard every creature in heaven and on earth and under the earth and in the sea, and all that is in them, saying,

    “To him who sits on the throne and to the Lamb
    be blessing and honor and glory and might forever and ever!”

    [14] And the four living creatures said, “Amen!” and the elders fell down and worshiped.
  \end{quote}
\subsection*{Notes}
\begin{itemize}
  \item{Three points for today: disappointment, solution, and response.}
  \item{Verse 4 tells us at the start, there was a scroll that was sealed
  completely (7 seals, 7 is the number of perfection).  Unfortunately,
  nothing in creation (in heaven, on earth or under the earth) could open the
  scroll.  This scroll was important because it contained God's plan of
  salvation and judgment.  If nobody could open the scroll, then the plan of
  salvation and judgment could not be carried out.  Hence, the weeping.}
  \item{We live in a world that is VUCA (volatile, uncertain, complex and
  ambiguous).  We face many disappointments in life, from people, from
  unanswered prayers, etc.  And that is because of sin and the fall of Man.
  Verse 4 tells us that creation cannot save itself.  This means that when we
  look to someone or something for hope, we can't look to someone or
  something in creation for hope.  We must look outside.}
  \item{Verses 5-9 give us the solution to the problem in v4.  An elder told
  John that $\exists$ (there exists) someone who can break the seals and open
  the scroll, and that someone was the Lion from the tribe of Judah (c.f Gen
  49:9-10), from the root of Jesse.  Obviously, this someone was the
  long-promised messiah. }
  \item{But shockingly, when John turned around, he saw a lamb instead! The expectation of the Lion of Judah that was to come was fulfilled by the Lamb of God. Lion represents an animal that is strong, and Lamb represents an animal that is meek. A few questions that comes to mind:
  \begin{enumerate}
    \item{Who is the Slain Lamb? (Jesus as the fulfilment of the passover Lamb).}
    \item{Why is He worthy to open the scroll and the seals?  ( He has made
    all of us sinners a people for God through His blood.  He has already
    triumphed over evil by taking on the world's evil on himself, dying and
    resurrecting.)}
    \item{How did He conquer?  (Not with might and strength like the rulers
    of the world, but with the sacrificial meekness of a Lamb.  This is
    mind-blowing!  This Lamb is actually very powerful, with seven horns.
    This Lamb is also all knowing, with seven eyes.  THis Lamb is very
    powerful, but He wins the victory through His sacrificial death and
    through His resurrection.)}
    \item{What does it mean for us today?  (We are redeemed to be witnesses
    of Christ, to live like Christ.  This means that for us Christians today,
    the way we win is the same way Christ wins; we win not by strength,
    military or political, but we win by self-sacrificial meekness that
    touches the hearts of others.)}
  \end{enumerate}}
  \item{Also, Christ has redeemed people from every tribe and nation and
  tongue and tribe.  This means that God, from the start, desires a
  multi-cultural body of Christ.  There is unity in diversity, rather than
  uniformity.  True worship focuses on something more important than
  ourselves, it is focused on Christ.  All races stand in need of a saviour,
  and all races have the same savior.  The savior is what unites us.}
  \item{Our savior Jesus died for our sins but He rose again, and He will
  come again to reign.  When we follow Jesus' example, we don't need to be
  afraid of dying. For like Jesus, it is through death that we overcome, and
  like Jesus, God will raise us up to reign with Jesus through His Holy Spirit.}
  \item{Verses 8-14 is the response of all creation.  They sang a new song,
  just like Psalm 98.  When we know the magnitude of what Christ has done for
  us, and when we know that Christ can do what He was done only because of
  His worthiness, we will naturally worship and celebrate.  The
  worship of the Lamb here in chapter 5 is the same as the worship of the one
  on the throne in chapter 4.  Just like how the figure on the throne in
  chapter 4 is in the middle of all things, the Lamb is also in the middle of
  all things.  Hence, Jesus, like the Father, is central to all creation.
  Hence, all creation's appropriate response to God is worship.  The ultimate
  destiny of Mankind can only either be to join the eternal worship chorus,
  or to rebel against God.  }
  \item{This vision of the throne room in chapters 4 and 5 is meant to give
  the seven churches a sight of the true spiritual realities.
  \begin{enumerate}
    \item{For churches like Smyrna and Philadelphia who are persecuted, this
    vision reminds them that Rome is not the centre of creation, but God is.
    Only God is worthy, and God has the power to destroy evil.  Hence, this
    would encourage the Christians there to continue holding on to their
    faith in the midst of suffering, because one day, their faith in God who
    is worthy will be vindicated.}
    \item{For churches like Laodicea and Sardis, this vision reminds them
    that they must repent and turn back to God, because only God is worthy.
    If not, one day, they will be destroyed when the Lamb opens the scroll.}
  \end{enumerate}}
\end{itemize}