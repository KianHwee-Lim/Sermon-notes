\section{19th February 2023: Who can stand?}
\subsection*{Text: Revelation 7:1-17}
  \begin{quote}
    [1] After this I saw four angels standing at the four corners of the
    earth, holding back the four winds of the earth, that no wind might blow
    on earth or sea or against any tree.  [2] Then I saw another angel
    ascending from the rising of the sun, with the seal of the living God,
    and he called with a loud voice to the four angels who had been given
    power to harm earth and sea, [3] saying, “Do not harm the earth or the
    sea or the trees, until we have sealed the servants of our God on their
    foreheads.” [4] And I heard the number of the sealed, 144,000, sealed
    from every tribe of the sons of Israel:

    [5] 12,000 from the tribe of Judah were sealed,
    12,000 from the tribe of Reuben,
    12,000 from the tribe of Gad,
    [6] 12,000 from the tribe of Asher,
    12,000 from the tribe of Naphtali,
    12,000 from the tribe of Manasseh,
    [7] 12,000 from the tribe of Simeon,
    12,000 from the tribe of Levi,
    12,000 from the tribe of Issachar,
    [8] 12,000 from the tribe of Zebulun,
    12,000 from the tribe of Joseph,
    12,000 from the tribe of Benjamin were sealed.


    [9] After this I looked, and behold, a great multitude that no one could
    number, from every nation, from all tribes and peoples and languages,
    standing before the throne and before the Lamb, clothed in white robes,
    with palm branches in their hands, [10] and crying out with a loud voice,
    “Salvation belongs to our God who sits on the throne, and to the Lamb!”
    [11] And all the angels were standing around the throne and around the
    elders and the four living creatures, and they fell on their faces before
    the throne and worshiped God, [12] saying, “Amen!  Blessing and glory and
    wisdom and thanksgiving and honor and power and might be to our God
    forever and ever!  Amen.”

    [13] Then one of the elders addressed me, saying, “Who are these, clothed
    in white robes, and from where have they come?” [14] I said to him, “Sir,
    you know.” And he said to me, “These are the ones coming out of the great
    tribulation.  They have washed their robes and made them white in the
    blood of the Lamb.

    [15] “Therefore they are before the throne of God,
        and serve him day and night in his temple;
        and he who sits on the throne will shelter them with his presence.
    [16] They shall hunger no more, neither thirst anymore;
        the sun shall not strike them,
        nor any scorching heat.
    [17] For the Lamb in the midst of the throne will be their shepherd,
        and he will guide them to springs of living water,
    and God will wipe away every tear from their eyes.”
  \end{quote}
\subsection*{Notes}
\begin{itemize}
  \item{Last week, we looked at the breaking of the seven seals on the scroll
  by the Lamb, which brings forth the judgment of God on the idolatrous and
  the sinful world.  But right after the seventh seal is broken, there is an
  interlude. The interlude here focuses on God's people. }
  \item{When the sixth seal was broken, there was a great earthquake, a great
  flood, etc.  In short, it was a cosmic disaster.  All the kings of the
  world, all the slave and the free, went to hide in the caves to try to hide
  from the wrath of the Lamb.}
  \item{Who can stand?  When the wrath of God is poured out in judgment on
  this idolatrous and wicked world.  The answer, from our text, is:
  \begin{enumerate}
    \item{The servants of God who are sealed (v3)}
    \item{The 144000 who are sealed (v4-8)}
    \item {The great multitude from every nation (v9-10)}
  \end{enumerate}
  These groups of people all refer to Christians of all ages, the universal
  Church.}
\item{Verse one of this chapter starts with ``after this''.  The ``after
this'' shouldn't be taken to mean chronological order, but it could mean a
different view of the same event.  I.e, it just signals a change of scene,
nothing to do with chronology.  Recall that the four horseman in the previous
chapter is a throwback to Zechariah 6:1-5.  And in Zechariah 6:1-5, the four
horsemen are going out to the four winds of heaven.  Hence, if we identify
the four horsemen from chapter 6 with the four angels in this chapter, the
fact that the angels are still holding back the winds means that the judgment
hasn't started yet.}
\item{Now, when we identify the $144000$ with the Church, then we are saying
that the Church is the true Israel (c.f Romans 9-11).  Also, when we identify
the $144000$ with the Church, then we are interpreting the number
symbollically.  $12$ has always been a number used for the people of God, e.g
$12$ apostles or $12$ tribes of Israel.  Hence, $12 \times 12 = 144$
represent the entirety of the redeemed church, then $144 \times 1000$
represent a great multitude.}
\item{Now, when John \textbf{hears} about the declaration of the sealing of
this $144000$ by God, he \textbf{looked} and he saw a great multitude.  What
John hears is the declaration of this total, what he sees is a great
multitude.  This is a throwback to chapter $5$, where John \textbf{heard}
about the Lion of the tribe of Judah but He saw a Lamb.  Just like how in
that case we identify the lion with the lamb, we should also identify here
the $144000$ with the multitude.}
\item{The great multitude is said to come out of the great tribulation.  This
great tribulation refers not to a specific point in time, but it refers to
the entire church age.  As Jesus said, in this world we will have trouble.
The great tribulation is already upon us, and it will only get more intense
as we get closer toward Christ's second coming.  Yet all of us who are sealed
will make it out of this great tribulation, even if not physically, then
surely spiritually.}
\item{Now, how are we sealed?  Similarly to our scripture reading of Ezekiel
9, we are sealed on our foreheads.  And similar to the reading of Ezekiel 9,
the mark is to be given to God's faithful only.  That is from the OT.  From
the NT, we see that we are sealed by the Holy Spirit (c.f 2 Corinthians
1:21-22, Ephesians 1:13-14).  Later on, we will see that the beast will also
give his followers a mark on their forehead, to parody this sealing that
faithful believers in God will get.}
\item{What is the consequence of this sealing that we have received?  Just
like how a seal identifies an who an object belongs to, us being sealed by
the Holy Spirit tells us that we belong to God.  When we are sealed, we will
not be harmed by the calamities brought about by the horsemen/winds.  It does
not mean we will not suffer (we will still suffer sickness and death and
poverty and etc), but through this suffering, our faith will be protected and
preserved.  As the Heidelberg Catechism puts it, ``all things are subservient
to our salvation''.  Essentially, those who are sealed will persevere to the
end in their faith.  No matter how bad things get, God will always give us
the way out and will give us the faith.  }
\item{Now, what is the outward sign and seal of this reality that we have
been sealed with the Holy Spirit?  It is Holy Baptism!  In Acts, we see that
conversion and baptism and being sealed by the Spirit all happen at the same
time (Acts 2:38).  And as people who are sealed by the Spirit grow in their
faith, what would characterise them?  As Jesus said, ``The
wind blows where it wishes, and you hear its sound, but you do not know where
it comes from or where it goes.  So it is with everyone who is born of the
Spirit.'' Being born of the Spirit is mysterious, but the effects of being
born of the Spirit is obvious and can be felt (like wind).  The marks of
being sealed are the Spirit are, according to Jonathan Edwards:
\begin{enumerate}
  \item{Esteen Jesus as Son of God and Saviour}
  \item{Oppose the reign of Satan and turn from sin}
  \item{Have an increased interest in God's Word}
  \item{Have a good grasp of sound doctrine and a desire to defend it against error}
  \item{Demonstrate love}
\end{enumerate}}
\item{So ya, who can stand?  The sealed can stand before the throne of God
forever.}
\item{The $144000$ can also be more specifically identified with the Church
millitant, because they look like the census of fighting men from the book of
Numbers.  The Church millitant is the Church on earth facing persecution and
opposition, and once those who are in the Church millitant pass from death
into eternal life, they become part of the Church triumphant, which can be
more specifically identified as the multitude in heaven.}
\item{And as we can see in v9-13, the multitude are seen to be wearing white
robes (v9,13).  This is because the blood of Jesus washes us clean from our
sin.  The multitude are also seen to wave palm branches (v9), which is a sign
of triumph.  This waving of palm branches is a throwback to the feast of
tabernacles in the past, where the people praised God for the harvest that
have come in.}
\item{The saints above sing ``salvation belongs to our God who sits on the
throne, and to the Lamb''.  Our salvation is the will of the Father, the work
of the Son, and the conviction of the Spirit, and hence they are truly worthy
of our praise.  But there is no reason to wait till heaven to sing this,
since our salvation now on earth is as secure as the saints in heaven.  The
saints in heaven are ``more happy, but not more secure''.}
\item{Not only is our salvation secure, from the last part of the text, we
can also look forward to God's protection.  God will shelter us from the sun
and from scorching heat, and we will be provided for such that the Lamb our
shepherd will guide us to springs of living water. While this protection will be realised in its fullness in the New Creation, right now we already experience a foretaste of this. For example, David experienced something like this in Psalm 23. }
\item{This text today gives us many things to be excited about; salvation,
sealing and security, provision, protection and presence.  No wonder the
heavenly hosts and the four living creatures and etc all bow down and marvel
at God's wisdom and mercy, and sing ``Amen!  Blessing and glory and wisdom
and thanksgiving and honor and power and might be to our God forever and
ever!  Amen.''}
\item{In conclusion, who can stand in the day of God's wrath?  All who have
placed their trust in Jesus and have been sealed with the Holy Spirit.  So if
you haven't put your faith in Jesus yet, do not tarry any longer.  We, the
Church millitant, belong to the $144000$.  Though we might have to face
tribulation and persecution, we will hold fast to the end through the sealing
of God's Spirit, and we will make it to the end as the Church triumphant.
This should give us many reasons to worship God now, right now, right here.
As the church in Singapore, we kinda face less calamities and less
tribulations.  But the Christians in other places face persecution,
execution, natural disasters, famine, slander, etc.  In Singapore, we should
be grateful for God's mercies as we navigate not so much the outward
persecution that the other Christians around the world face, but as we
navigate the more subtle persecution. And we can have faith that we'll make it out safely, with God's help. }
\end{itemize}