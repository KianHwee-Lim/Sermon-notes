\section{5th February 2023: To be Salt and Light}
\subsection*{Text: Matthew 5:13-20}
% \begin{enumerate}
%   \item{Psalms 112:1-9}
%   \item{Isaiah 58}
%   \item{1 Corinthians 2:1-12}
%   \item{\textbf{Matthew 5:13-20}}
% \end{enumerate}}
  \begin{quote}
    [13] “You are the salt of the earth, but if salt has lost its taste, how shall its saltiness be restored? It is no longer good for anything except to be thrown out and trampled under people’s feet.

    [14] “You are the light of the world. A city set on a hill cannot be hidden. [15] Nor do people light a lamp and put it under a basket, but on a stand, and it gives light to all in the house. [16] In the same way, let your light shine before others, so that they may see your good works and give glory to your Father who is in heaven.

    [17] “Do not think that I have come to abolish the Law or the Prophets; I have not come to abolish them but to fulfill them. [18] For truly, I say to you, until heaven and earth pass away, not an iota, not a dot, will pass from the Law until all is accomplished. [19] Therefore whoever relaxes one of the least of these commandments and teaches others to do the same will be called least in the kingdom of heaven, but whoever does them and teaches them will be called great in the kingdom of heaven. [20] For I tell you, unless your righteousness exceeds that of the scribes and Pharisees, you will never enter the kingdom of heaven.
  \end{quote}
  \subsection*{Auxiliary texts:
  \begin{enumerate}
    \item{Psalms 112:1-9}
    \item{Isaiah 58}
    \item{1 Corinthians 2:1-12}
  \end{enumerate}}
\subsection*{Notes}
\begin{itemize}
  \item{Today happens to be the fifth sunday of Epiphany.  Ephiphany puts the
  focus on Jesus and what He revealed about himself in His identity and His
  mission.}
  \item{There are four passages that we will look at today.  The text from
  Psalms tell us about how the righteous man behaves.  The text from Isaiah
  tells us about how the righteous man seeks God.  The text from 1
  Corinthians 2 tells us about how Paul wants to know nothing except Christ
  crucified and His glory.  The text from Matthew tells us how we are to live
  in light of this knowledge of Christ crucified and His glory.}
  \item{The whole idea is that the more we know about the ministry and
  identity of Jesus, in our lives we will be able to reflect Jesus.}
  \item{In the previous sermons, we saw that Jesus is God's Son, God's
  messiah, and God's annointed Lamb.  We also saw that in order to follow
  Jesus, we must encounter Jesus.  Then we must know Jesus.  Only then can we
  know how to follow Jesus.  We also saw how Jesus called His first
  disciples; the disciples heard that they will become fishers of men, rather
  than fishermen.  But in their idea, the thought of the disciples was that
  they will be the 12 rulers over the kingdom of God.  Their assumption of
  the idea of the messiah was of one that would be a political figure to
  overthrow the romans.  That is why they asked to be on Jesus' left and
  right.  As we all know now, that is wrong lol.}
  \item{Hence, in today's sermon, we see that to know Jesus, we must know
  Jesus through how Jesus fulfils the Law and the Prophets.  We see that to
  know Jesus, we must know Christ crucified.  From v17-18 of the Matthew
  text, we see that Jesus comes not to abolish the Law and the Prophets, but
  to fulfil them.  So we see how God's word in the OT will be fulfilled in
  the NT.  So when the NT writers are referring to the Law and the Prophets,
  they are referring to the OT scriptures, since the NT hasn't been fully
  written yet.  }
  \item{For example, for baptism and the Holy Communion elements, the bread
  and the cup, are the fulfilment of the Passover, and that baptism is the
  fulfulment of circumcision.  It is not that circumcision has been
  abolished, but it is fulfiled in baptism.  Baptism, just like circumcision,
  is an entrance into the covenant community of God.  Likewise, it is not
  that the Passover has been abolished.  Just like how the Passover
  commemorates the deliverance from Egypt through the slain lamb, the holy
  communion commemorates our deliverance from sin through Christ crucified.
  Similarly, the temple in the past prefigures how the Holy Spirit dwells in
  us to make us temples of God so that we can worship in Spirit and in
  Truth.}
  \item{For us today, what distinguishs us as God's people is that we are
  baptised, we partake of the holy communion, and we gather as God's people.
  When we do this, we are fulfilling the OT, and hence we can meet Jesus.}
  \item{With all of the context about how we can know Jesus through our
  fulfilment of the OT and through Christ crucified, we know how we can be
  salt and light in the world.  To be salt and light is to be more than just
  ``good people''.  Other religions also teach us to be ``good people''.  To
  be salt and light, for us, is to be more like Jesus and to continue the
  ministry of Jesus.  We are salt because we are to function as salt.  Salt
  preserves food from preserving it from being spoilt.  Similarly, if we know
  Jesus and are following Jesus, we fight against the corruption caused by
  sin in the world, both individually and corporately.  Hence, as Jesus says,
  if we aren't functioning as salt, we will be thrown away.  Just as how salt
  functions as salt, Christians are to function as Christians.}
  \item{Prior to the text in Matthew, we have the beautitudes.  There are 8
  beautitudes, first four telling us how we are to relate to God, and the
  next four telling us how we are to relate to others.  Christians are to
  function according to these beautitudes, and hence we will be blessed.  If
  as Christians we aren't functioning as how these beautitudes describe, we
  need to return to Jesus to rediscover who Jesus is and to recommit to
  follow Jesus.  }
  \item{We are to be light in the sense of how we walk in the light, and in
  the sense of how we are to illumine the darkness in the world.  The
  darkness is due to people's ignorance of Jesus, and due to hardness of
  heart due to sin.  Through our lives and our words, we are to be the light
  to illumine that darkness.  Hence, a new challenge for us in this new year; if
  there are people around us who aren't Christians, we are to illumine the
  darkness in their lives.}
  \item{To reiterate, to be salt and light is more than just doing good works. It is to live out our function and identity as Christians, to reveal Christ to the world. We are to function as salt and light, and if our functioning as salt and light is correct, then it cannot be hidden. The final goal of this is to glorify God. }
\end{itemize}