\section{12th February 2023: Seven seals}
\subsection*{Text: Revelation 6:1-7, 8:1-6}
  \begin{quote}
  [1] Now I watched when the Lamb opened one of the seven seals, and I heard
  one of the four living creatures say with a voice like thunder, “Come!” [2]
  And I looked, and behold, a white horse!  And its rider had a bow, and a
  crown was given to him, and he came out conquering, and to conquer.

  [3] When he opened the second seal, I heard the second living creature say,
  “Come!” [4] And out came another horse, bright red.  Its rider was
  permitted to take peace from the earth, so that people should slay one
  another, and he was given a great sword.

  [5] When he opened the third seal, I heard the third living creature say,
  “Come!” And I looked, and behold, a black horse!  And its rider had a pair
  of scales in his hand.  [6] And I heard what seemed to be a voice in the
  midst of the four living creatures, saying, “A quart of wheat for a
  denarius, and three quarts of barley for a denarius, and do not harm the
  oil and wine!”

  [7] When he opened the fourth seal, I heard the voice of the fourth living
  creature say, “Come!” [8] And I looked, and behold, a pale horse!  And its
  rider’s name was Death, and Hades followed him.  And they were given
  authority over a fourth of the earth, to kill with sword and with famine
  and with pestilence and by wild beasts of the earth.

  [9] When he opened the fifth seal, I saw under the altar the souls of those
  who had been slain for the word of God and for the witness they had borne.
  [10] They cried out with a loud voice, “O Sovereign Lord, holy and true,
  how long before you will judge and avenge our blood on those who dwell on
  the earth?” [11] Then they were each given a white robe and told to rest a
  little longer, until the number of their fellow servants and their brothers
  should be complete, who were to be killed as they themselves had been.

  [12] When he opened the sixth seal, I looked, and behold, there was a great
  earthquake, and the sun became black as sackcloth, the full moon became
  like blood, [13] and the stars of the sky fell to the earth as the fig tree
  sheds its winter fruit when shaken by a gale.  [14] The sky vanished like a
  scroll that is being rolled up, and every mountain and island was removed
  from its place.  [15] Then the kings of the earth and the great ones and
  the generals and the rich and the powerful, and everyone, slave and free,
  hid themselves in the caves and among the rocks of the mountains, [16]
  calling to the mountains and rocks, “Fall on us and hide us from the face
  of him who is seated on the throne, and from the wrath of the Lamb, [17]
  for the great day of their wrath has come, and who can stand?”

  [1] When the Lamb opened the seventh seal, there was silence in heaven for
  about half an hour.  [2] Then I saw the seven angels who stand before God,
  and seven trumpets were given to them.  [3] And another angel came and
  stood at the altar with a golden censer, and he was given much incense to
  offer with the prayers of all the saints on the golden altar before the
  throne, [4] and the smoke of the incense, with the prayers of the saints,
  rose before God from the hand of the angel.  [5] Then the angel took the
  censer and filled it with fire from the altar and threw it on the earth,
  and there were peals of thunder, rumblings, flashes of lightning, and an
  earthquake.

  [6] Now the seven angels who had the seven trumpets prepared to blow them.
  \end{quote}
\subsection*{Notes}
\begin{itemize}
  \item{A quick question that might arise after this passage: what is going
  on?  Why is there so much destruction?  To answer this, we must first go
  back to chapter 5.  In chapter 5 and 6, we see that nobody was worthy to
  open the scroll, except the Lamb that was slain.  When initially John saw
  that nobody could open the scroll, he wept.  He wept because that scroll
  was very important; if nobody could open the scroll, then God's plan on
  earth cannot be fulfilled.  And God's plan is to bring in His Kingdom on
  the earth.  }
  \item{Three points for today: what are God's priorities in implementing His
  plan?  What is the process by which the plan is to be implemented?  And who
  are the participants in bringing in God's plan?}
  \item{First point: what are God's priorities?  John was taken up to heaven
  to see what is to come.  But before God showed John what is to come, John
  had to see the glorious vision of God in chapter 4 and 5.  From there, we
  see that one of God's priority is to be known as the only one true God, and
  to consequently to eliminate idolatry on earth.  Another of God's priority
  is to be known as the one Lord.  When we worship the one true God, not only
  do we say that God is the creator, we say that God is the one Lord over
  all.  This is depicted by the throne of God in the centre.  What are the
  priorities of the Lamb?  First of all, the Lamb's priority is to save us
  from idolatry.  To summarise, when God brings in His plan, His priorities
  is to be known as one creator God, one Lord, and through the Lamb, one
  saviour.}
  \item{The next point: what is the process?  Clearly, there are still
  idolators on earth and there will still be God's idolator on earth.  Hence,
  part of the process of bringing in His plan includes judgment.  We see that
  in these seven seals, we can split the judgment into two parts; God's
  passive judgement in seals 1-4, and God's active judgement in seals 5-7.
  Passive judgment refers to God giving people up to the consequences of
  their sins and to their subsequent hardness of hearts.  This is best
  explained in Romans 1, ``God gave them up to their passions...''.
  Theologically, it is God withholding His restraining grace on sinners and
  letting sinners do what they like.  }
  \item{For example, the first seal depicts how people would give in to their
  innate desire to conquer and for authority.  The second seal depicts how
  people would die because of this warfare that is a consequence of people's
  desire to conquer.  The third seal depicts how there would be
  hyper-inflation.  A denarius is a labourer's daily wage, and ordinarily,
  that would be able for him to feed his family.  However, after the third
  seal, there would be famine due to the war and now, a labourer can only
  afford a quart of wheat.  From ancient calculations (c.f Heroditus), we
  know that a quart of wheat is enough only to feed one single person.  I.e,
  a labourer can no longer feed his family, but only himself!  Barley is
  cheaper, and hence the labourer can buy three quarts.  The fourth seal
  depicts how there would be much death due to the first three seals.}
  \item{The sixth seal depicts God's active judgement.  This where God
  directly intervenes to actively judge sinners.  And this is depicted as
  something very fierce.}
  \item{Hence, we see that judgment is part of God's process for bringing in
  His kingdom.  In today's world, we don't like to talk about God as judge.
  Yet we as a society know that we need judges (and hence we have a
  judiciary).  And yet we as a society balk when there is injustice, for
  example when a criminal gets away due to a loophole in our human justice
  system etc.  So in a sense, we as humans all crave for some sort of final
  justice so that those who got away from the human justice system will
  eventually be held accountable.  So why do people hate the idea of God as
  judge despite them all wanting just judgement?  This is because they don't
  like what God is judging.  People all want to think that they are righteous
  and beyond judgment, and they don't like the idea that they too are so bad
  as to warrant God's judgment!  This is why we need the gospel of the Lord
  Jesus to convict the world of sin.}
  \item{The last point: the participants.  Here, we look at seals $5$ and
  $7$.  Seal $5$ tells us why God has to judge the world.  Not only do the
  people of the world want to be king and rulers and hence unleash a chain of
  bad consequences on the world, they also want to persecute the remnant who
  actually speak out for justice.  The remnant are here are those who are
  faithful in their witness to God's word.  Here, we see that while God gives
  the world up to their sinful passions through His act of passive judgement,
  He also leaves a remnant on earth to speak His truth.  So in this sense, we
  are the participants of God's judgment in the sense that we participate by
  witnessing to God's Word and His justice, even if we might lose our lives
  in the process.  Also, seal $5$ also shows us that God has His own
  timeline.  While sometimes we want God to act immediately to redress our
  injustice, we must realise that God has His own timeline.}
  \item{Seal $7$ is interesting because it is quite anti-climatic.  Seal $6$
  leads to so much crying and groaning from the peoples of the earth, yet
  seal $7$ leads to silence.  If we look at the sequel, we see that after the
  seventh seal, there are seven trumpets and seven bowls.  So we must think
  of the seven trumpets and the seven bowls as part of the seventh seal.  The
  silence in a sense, is meant for the angels to prepare (see chapter 8 verse
  6) and for prayers.  In chapter 8 verse 3, we see the angel taking the
  incense which represents the prayers of the saints.  And when the angel
  pours the prayers of the saints on the earth, we see ``peals of thunder,
  rumblings, flashes of lightning, and an earthquake''.  These are the same
  things that appear after the end of the trumpet series, and after the end
  of the bowl series. }
  \item{God did not just save us, He co-opted us as workers in His harvest
  field.  When God brings in His kingdom, He could have done it alone, but He
  chose to do it through us.  The seventh seal reminds of this; God's kingdom
  will come, God's righteous judgment will come to right the world, through
  the prayers of the saints.  The prayers of the saints comes as the climax
  of the seal series, which reminds us how important our prayers are.  It is
  God's desire that in our prayer, we resonate with God's heartbeat.  In our
  prayer, our will should be conformed to God's will, so that we want to will
  the same thing God wills.  }
  \item{We participate in God's process of bringing in the Kingdom through
  our prayers for God's righteousness to be revealed and for His justice to
  be done.  God is pleased to work through our prayers.  We participate in
  God's process of bringing in the Kingdom through our witnessing for God's
  Word and perhaps losing our lives in the process.  Witnessing for God's
  Word mean putting God's priorities as our priorities.  The fact that God
  uses our work and prayers in His process of bringing in His kingdom is a
  good reminder for us that our work for Him is not futile, and in fact is
  very important.  So we can take heart and be courageous, joyful and willing
  witnesses for God.}
  \item{After the sixth seal, we see the kings of the earth trying to flee
  from God.  That is futile, as we can see from Psalm 139.  What they should
  have done instead is to flee to God.  It is counter-intuitive, but it is
  true.}
\end{itemize}