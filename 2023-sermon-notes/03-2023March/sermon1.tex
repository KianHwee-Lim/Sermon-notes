\section{5th March 2023: The role of God's people}
\subsection*{Text: Revelation 10:1-11, 11:1-13}
  \begin{quote}
    [1] Then I saw another mighty angel coming down from heaven, wrapped in a
    cloud, with a rainbow over his head, and his face was like the sun, and
    his legs like pillars of fire.  [2] He had a little scroll open in his
    hand.  And he set his right foot on the sea, and his left foot on the
    land, [3] and called out with a loud voice, like a lion roaring.  When he
    called out, the seven thunders sounded.  [4] And when the seven thunders
    had sounded, I was about to write, but I heard a voice from heaven
    saying, “Seal up what the seven thunders have said, and do not write it
    down.” [5] And the angel whom I saw standing on the sea and on the land
    raised his right hand to heaven [6] and swore by him who lives forever
    and ever, who created heaven and what is in it, the earth and what is in
    it, and the sea and what is in it, that there would be no more delay, [7]
    but that in the days of the trumpet call to be sounded by the seventh
    angel, the mystery of God would be fulfilled, just as he announced to his
    servants the prophets.

    [8] Then the voice that I had heard from heaven spoke to me again,
    saying, “Go, take the scroll that is open in the hand of the angel who is
    standing on the sea and on the land.” [9] So I went to the angel and told
    him to give me the little scroll.  And he said to me, “Take and eat it;
    it will make your stomach bitter, but in your mouth it will be sweet as
    honey.” [10] And I took the little scroll from the hand of the angel and
    ate it.  It was sweet as honey in my mouth, but when I had eaten it my
    stomach was made bitter.  [11] And I was told, “You must again prophesy
    about many peoples and nations and languages and kings.”

    [1] Then I was given a measuring rod like a staff, and I was told, “Rise
    and measure the temple of God and the altar and those who worship there,
    [2] but do not measure the court outside the temple; leave that out, for
    it is given over to the nations, and they will trample the holy city for
    forty-two months.  [3] And I will grant authority to my two witnesses,
    and they will prophesy for 1,260 days, clothed in sackcloth.”

    [4] These are the two olive trees and the two lampstands that stand
    before the Lord of the earth.  [5] And if anyone would harm them, fire
    pours from their mouth and consumes their foes.  If anyone would harm
    them, this is how he is doomed to be killed.  [6] They have the power to
    shut the sky, that no rain may fall during the days of their prophesying,
    and they have power over the waters to turn them into blood and to strike
    the earth with every kind of plague, as often as they desire.  [7] And
    when they have finished their testimony, the beast that rises from the
    bottomless pit will make war on them and conquer them and kill them, [8]
    and their dead bodies will lie in the street of the great city that
    symbolically is called Sodom and Egypt, where their Lord was crucified.
    [9] For three and a half days some from the peoples and tribes and
    languages and nations will gaze at their dead bodies and refuse to let
    them be placed in a tomb, [10] and those who dwell on the earth will
    rejoice over them and make merry and exchange presents, because these two
    prophets had been a torment to those who dwell on the earth.  [11] But
    after the three and a half days a breath of life from God entered them,
    and they stood up on their feet, and great fear fell on those who saw
    them.  [12] Then they heard a loud voice from heaven saying to them,
    “Come up here!” And they went up to heaven in a cloud, and their enemies
    watched them.  [13] And at that hour there was a great earthquake, and a
    tenth of the city fell.  Seven thousand people were killed in the
    earthquake, and the rest were terrified and gave glory to the God of
    heaven.
  \end{quote}
\subsection*{Notes}
\begin{itemize}
  \item{In chapter $7$, we see an interlude between the sixth and seventh seal.
  Here, we also see an interlude between the sixth and seventh trumpet.
  First, chapter $10$ starts with a vision of a glorious angel.  The angel
  here reflects God's power and God's glory, and that would help us to
  remember God's acts of salvation and of God's faithfulness.  First, we see
  that the angel here has a rainbow (throwback to Noahic covenant), the angel
  here also has legs like a pillar of fire (throwback to Sinai).}
  \item{The latter half of chapter $10$ has a vision of John eating the
  little scroll.  This is a throwback to the OT prophets like Ezekiel.  This
  scroll eating is God commissioning the prophet as His speaker, because it
  is like the scroll contains the words God wants to say, and hence after
  eating the scroll, everything that the prophet says will be God's word.  So
  here John is being comissioned to speak like the OT prophets of old.  The
  sweet but bitter scroll here could be analogous to how the gospel message
  is sweet yet the experience of a Christian after accepting the gospel is
  often bitter (persecution, rejection by friends and family, etc).  Like
  John and the prophets of old, we too have received God's word through His
  Spirit speaking to us through the Bible.  And similarly, like John and the
  prophets of old, we must be a prophetic voice to the world around us, to
  tell them to repent and turn from their sins, and to turn to God, and that
  their sins will be fully forgiven in Christ if they do so.}
  \item{John's measuring of the temple symbolises God's commitment to
  preserve His church during tribulation (since a temple is just a symbol for
  God's people).  The nations of Revelation 11 here who are trampling the
  outer court refer to non-Christians who despise God's people and profane
  what is holy.  The two witnesses here refer to the Church's witness through
  the power of the Holy Spirit.  The vision of the two witnesses here is a
  throwback to Zechariah 4 (not elaborated upon in the sermon).  Cool quote:
  ``we are all immortals until our work here is done''.  Until we have
  finished our work in witnessing for God, God will keep us safe and preserve
  us and give us the strength to do our work.  Here, we see that the beast
  only kills the two witnesses after the two witnesses finished their
  testimony.  After the church has finished her testimony, the beast
  temporarily overwhelms the church and then the people of the world rejoice
  over the fall of the church.  But that is not the end, in the end, we see
  God raising up the two witnesses and bringing them up to heaven.  In the
  end, we will be victorious, though the world persecute us.  }
  \item{Three points for today: 
  \begin{itemize}
    \item{\textbf{R}emember God's faithfulness.}
    \item{\textbf{E}vangelise to everyone.}
    \item{\textbf{D}epend on the Holy Spirit, God will defend and protect His people.}
  \end{itemize}}
\end{itemize}