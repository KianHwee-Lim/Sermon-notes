\setcounter{figure}{0}

\section{5th November 2023: I AM the resurrection and the life}
\subsection*{Text: John 11:17-44}
  \begin{quote}
    [17] Now when Jesus came, he found that Lazarus had already been in the tomb four days. [18] Bethany was near Jerusalem, about two miles off, [19] and many of the Jews had come to Martha and Mary to console them concerning their brother. [20] So when Martha heard that Jesus was coming, she went and met him, but Mary remained seated in the house. [21] Martha said to Jesus, “Lord, if you had been here, my brother would not have died. [22] But even now I know that whatever you ask from God, God will give you.” [23] Jesus said to her, “Your brother will rise again.” [24] Martha said to him, “I know that he will rise again in the resurrection on the last day.” [25] Jesus said to her, “I am the resurrection and the life. Whoever believes in me, though he die, yet shall he live, [26] and everyone who lives and believes in me shall never die. Do you believe this?” [27] She said to him, “Yes, Lord; I believe that you are the Christ, the Son of God, who is coming into the world.”

    [28] When she had said this, she went and called her sister Mary, saying in private, “The Teacher is here and is calling for you.” [29] And when she heard it, she rose quickly and went to him. [30] Now Jesus had not yet come into the village, but was still in the place where Martha had met him. [31] When the Jews who were with her in the house, consoling her, saw Mary rise quickly and go out, they followed her, supposing that she was going to the tomb to weep there. [32] Now when Mary came to where Jesus was and saw him, she fell at his feet, saying to him, “Lord, if you had been here, my brother would not have died.” [33] When Jesus saw her weeping, and the Jews who had come with her also weeping, he was deeply moved in his spirit and greatly troubled. [34] And he said, “Where have you laid him?” They said to him, “Lord, come and see.” [35] Jesus wept. [36] So the Jews said, “See how he loved him!” [37] But some of them said, “Could not he who opened the eyes of the blind man also have kept this man from dying?”

    [38] Then Jesus, deeply moved again, came to the tomb. It was a cave, and a stone lay against it. [39] Jesus said, “Take away the stone.” Martha, the sister of the dead man, said to him, “Lord, by this time there will be an odor, for he has been dead four days.” [40] Jesus said to her, “Did I not tell you that if you believed you would see the glory of God?” [41] So they took away the stone. And Jesus lifted up his eyes and said, “Father, I thank you that you have heard me. [42] I knew that you always hear me, but I said this on account of the people standing around, that they may believe that you sent me.” [43] When he had said these things, he cried out with a loud voice, “Lazarus, come out.” [44] The man who had died came out, his hands and feet bound with linen strips, and his face wrapped with a cloth. Jesus said to them, “Unbind him, and let him go.”
  \end{quote}
\subsection*{Notes}
\begin{itemize}
  \item{Everyone will die one day :(. Our passage today is a common passage used in funerals to give comfort to those who grieve. }
  \item{Before our text today, we had John 11:3-5. Jesus said: “this illness does not lead to death, it is for the glory of God”. Jesus also said that He loved Martha, Mary and Lazarus. }
  \item{Then Jesus stayed two days longer where He was before going to Judea, then He said: “Lazarus has died”. Does this contradict what Jesus said in the previous few verses? This will be explained later.}
  \item{Now, when Jesus went to meet Mary and Martha, Martha was grieving. She said: “Lord, if you had been here, my brother would not have died”. She was clearly grieving. Yet she still had faith in Jesus, she said: “but even now I know that whatever you ask from God, God will give you”. }
  \item{Jesus told Martha that her brother will rise again, then Martha said: “i know that he will rise again on the last day”. But Jesus promises Martha something better, that He says that “i am the resurrection and the life. Whoever believes in me, though he die, yet shall he live, and everyone who lives and believes in me shall never die”. This statement is very profound, and here the word “live” is used to denote more than just physical life. Our hope is not on an event (the resurrection), but it is on a Person, Jesus Christ. And this means that those who believe in Jesus and have died, are not really dead, they are currently alive in Christ, waiting for the final resurrection.}
  \item{In verse 33, we see that Jesus was deeply moved in his spirit and greatly troubled. This is also very profound, since He waited for two days before going to Judea, and in v14 He even said that He was glad that He wasn’t there when Lazarus died. So in a sense, Lazarus’ death was part of Jesus plan. So why then did Jesus weep? This shows that while God might make use of suffering and death in His providence, but God also cares a lot for all of us who are going through suffering and death, and that we can always go to God in our suffering. And while God makes use of suffering and death (which exists in our world because of our sin), sin is not part of God’s original plan, and it will not be present in God’s final plan. And hence, God is also deeply troubled by the presence of sin and death in the world, and He will not rest until sin and death is finally taken care of.}
  \item{Now, sin and death and suffering comes as a result of the fall of man (see Romans 5). While sin and death and suffering will finally be eradicated when Jesus comes again, even in this life, we can experience eternal life and fullness of life by believing in Jesus. Like Mary, we must believe: “I believe that you are the Christ, the Son of God, who is coming into the world”. In this way, though we may die, we will always be alive in Christ.}
  \item{From v38 onwards, we see Jesus raising Lazarus from the dead. This was all part of Jesus plan to display His identity, so that God will be glorified and that people (primarily His disciples) will believe (v40, v14). And thus He did, He raised Lazarus from the dead, thereby partially fulfilling His promise that Lazarus’ illness will not lead to death. This is just an example of God demonstrating His power over death. But we know that after Lazarus was raised, he died again too. The final fulfilment of Jesus’ words are that Lazarus is now alive in Christ, and that when He comes again, Lazarus will be raised from the dead. }
  \item{And we know that God has power over death not just because of Lazarus, but because of the resurrection of Christ. The resurrection of Christ shows that Christ has defeated sin and death. He is the resurrection and the life. }
  \item{One last thing; if even a dead man can listen to Jesus voice and obey (Lazarus, come out), how much more should we who are living listen to Jesus and obey!}
\end{itemize}