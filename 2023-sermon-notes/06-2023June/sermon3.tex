\setcounter{figure}{0}

\section{18th June 2023: That his spirit may be saved}
\subsection*{Text: 1 Corinthians 5:1-12}
  \begin{quote}
    [1] It is actually reported that there is sexual immorality among you, and of a kind that is not tolerated even among pagans, for a man has his father’s wife. [2] And you are arrogant! Ought you not rather to mourn? Let him who has done this be removed from among you.

    [3] For though absent in body, I am present in spirit; and as if present, I have already pronounced judgment on the one who did such a thing. [4] When you are assembled in the name of the Lord Jesus and my spirit is present, with the power of our Lord Jesus, [5] you are to deliver this man to Satan for the destruction of the flesh, so that his spirit may be saved in the day of the Lord.

    [6] Your boasting is not good. Do you not know that a little leaven leavens the whole lump? [7] Cleanse out the old leaven that you may be a new lump, as you really are unleavened. For Christ, our Passover lamb, has been sacrificed. [8] Let us therefore celebrate the festival, not with the old leaven, the leaven of malice and evil, but with the unleavened bread of sincerity and truth.

    [9] I wrote to you in my letter not to associate with sexually immoral people—[10] not at all meaning the sexually immoral of this world, or the greedy and swindlers, or idolaters, since then you would need to go out of the world. [11] But now I am writing to you not to associate with anyone who bears the name of brother if he is guilty of sexual immorality or greed, or is an idolater, reviler, drunkard, or swindler—not even to eat with such a one. [12] For what have I to do with judging outsiders? Is it not those inside the church whom you are to judge?
  \end{quote}
\subsection*{Notes}
\begin{itemize}
  \item{Part of a three part series on being an embracing community. This sermon is the last part about “the limits of embrace”. }
  \item{1 Corinthians 5 is church discipline. It is about the most extreme form of church discipline, which is excommunication. Excommunication is a topic that is rarely talked about.}
  \item{Church discipline has always been a vexed issue. Some question whether church discipline is the best way to deal with a church member living in sin or a church leader promoting heresy. Perhaps a gentler approach is better (?). Even those who are for church discipline question when to excommunicate a person. Furthermore, church discipline is hard to implement in modern day protestant churches too, since Protestant churches do not have enough institutional ties between denominations.}
  \item{In our text, we see that Paul was appalled by how the Corinthian church dealt with immorality in the Corinthian church. Paul suggests excommunication here in this text. Apart from Matthew 18:1-15, this text is the most important text in church discipline.}
  \item{A few questions for today:
  \begin{enumerate}
    \item{What is church discipline in general and excommunication in particular?}
    \item{Who must the church discipline?}
    \item{What is the logic of church discipline}
  \end{enumerate}}
  \item{What is church discipline? There are two types: formative church discipline (when members spur each other to love and good works) and corrective church discipline. }
  \item{Corrective church discipline occurs any time sin is corrected within the body... It occurs most fully when the church body announces that the covenant between church and member is already broken beacuse the nember has proven to be unsubmissive in his or her discipleship to Christ.}
  \item{In the teaching of the reformers, corrective church discipline is one of the two keys of the church (the other key being the preaching of the gospel and the administration of the sacraments). See Matthew 19; whatever the Church binds on earth will be bound in heaven, and etc etc. see also Heidelberg Catechism qn 83. And WCF qn 30. So we see that in the teaching of the reformers, corrective church discipline is as central as the preaching of the gospel. And in the OT, deuteronomy provides a basis for excommunication.}
  \item{And when excommunication occurs (the highest form of corrective church discipline), the person is forbidden from participating in the life of the church, such as worship and holy communion.}
  \item{Now, the qn is, who shall be excommunicated? In church tradition, there are two classes of sin: serious moral delinquency, and the second class of sin is heresy. \KH{(My thoughts: what is heresy though? haha. Is Armininiansim a heresy?)}.}
  \item{In our text, we have this first case here. And in the later part of our text, the list expands from just sexual immorality to greed, swindlers and idolators.}
  \item{The second case, which is heresy, is also explained in the NT. E.g in Jude, and in 1 Peter, and in Matthew. There is always an injunction against false teachers in the NT. For example, Paul instructs Timothy to rebuke false teachers in the presence of all. If those people repent, then its ok, but if they don’t, they he should be excommunicated.}
  \item{What is the logic/purpose of corrective church discipline, especially excommunication? What do we hope to achieve in our excommunication? (This will be answered later).}
  \item{Futhermore, didnt Jesus command us to even love our enemy? How much more than should we love those who profess to be Christian? (Despite their heresy and their moral failings). Is excommunication compatible with this? Isn’t excommunication very untolerant and brutal?}
  \item{Firstly, as an extreme form of church discipline, excommunication should only be used sparingly and as the last resort for very severe moral failings \KH{(My thoughts: what exactly is a ``severe moral failing?'' If ``greed'' is a criteria as Paul mentions here, then should we excommunicate everyone who desires luxuries in life such as flying business class/owning large mansions/etc?)}}
  \item{Secondly, sometimes these corrective church discipline is necessary for the spiritual life of the community. “A little leaven leavens the whole lump”. What this means is that when we keep outwardly unrepentant people in our congregation and in our church life, they will influence others to sin like them too.}
  \item{Thirdly, this excommunication is also good for the offender. Here Paul says: “that his spirit or soul may be saved in the day of the Lord”. Church discipline is meant to be redemptive. But what does Paul means here by “deliver to satan for the destruction of the flesh”? Maybe this means to put the person back from the kingdom of light (the church) to the kingdom of darkness, i.e by excluding him from fellowship. The logic is that when the person is excluded from church fellowship, he will feel fomo and then understand the severity of his own sin and repent. The “destruction of his flesh” here then is to destroy the person’s sinful nature. The purpose of excommunication is not to condemn, but to correct.}
\end{itemize}