\setcounter{figure}{0}

\section{25th June 2023: Will you go or flee?}
\subsection*{Text: Jonah 1}
  \begin{quote}
    [1] Now the word of the LORD came to Jonah the son of Amittai, saying, [2] “Arise, go to Nineveh, that great city, and call out against it, for their evil has come up before me.” [3] But Jonah rose to flee to Tarshish from the presence of the LORD. He went down to Joppa and found a ship going to Tarshish. So he paid the fare and went down into it, to go with them to Tarshish, away from the presence of the LORD.

    [4] But the LORD hurled a great wind upon the sea, and there was a mighty tempest on the sea, so that the ship threatened to break up. [5] Then the mariners were afraid, and each cried out to his god. And they hurled the cargo that was in the ship into the sea to lighten it for them. But Jonah had gone down into the inner part of the ship and had lain down and was fast asleep. [6] So the captain came and said to him, “What do you mean, you sleeper? Arise, call out to your god! Perhaps the god will give a thought to us, that we may not perish.”

    [7] And they said to one another, “Come, let us cast lots, that we may know on whose account this evil has come upon us.” So they cast lots, and the lot fell on Jonah. [8] Then they said to him, “Tell us on whose account this evil has come upon us. What is your occupation? And where do you come from? What is your country? And of what people are you?” [9] And he said to them, “I am a Hebrew, and I fear the LORD, the God of heaven, who made the sea and the dry land.” [10] Then the men were exceedingly afraid and said to him, “What is this that you have done!” For the men knew that he was fleeing from the presence of the LORD, because he had told them.

    [11] Then they said to him, “What shall we do to you, that the sea may quiet down for us?” For the sea grew more and more tempestuous. [12] He said to them, “Pick me up and hurl me into the sea; then the sea will quiet down for you, for I know it is because of me that this great tempest has come upon you.” [13] Nevertheless, the men rowed hard to get back to dry land, but they could not, for the sea grew more and more tempestuous against them. [14] Therefore they called out to the LORD, “O LORD, let us not perish for this man’s life, and lay not on us innocent blood, for you, O LORD, have done as it pleased you.” [15] So they picked up Jonah and hurled him into the sea, and the sea ceased from its raging. [16] Then the men feared the LORD exceedingly, and they offered a sacrifice to the LORD and made vows.

    [17] And the LORD appointed a great fish to swallow up Jonah. And Jonah was in the belly of the fish three days and three nights.
  \end{quote}
\subsection*{Notes}
\begin{itemize}
  \item{Sermon series on the book of Jonah! This is the first sermon in that series!}
  \item{Church tradition tells us that Jonah is historical. Jonah is also mentioned in 2 Kings under the reign of Jeroboam, which places Jonah under the time 780+BC.}
  \item{“Arise, go to Nineveh, that great city…”. Nineveh was the city of the Assyrians, and the Assyrians would conquer the Northern kingdom in the next 30 years. But Jonah, instead of going to Nineveh, went to Tarschich, which was in the opposite direction. Jonah was disobeying God on purpose. But why? The reason will be explained in Jonah 4.}
  \item{On the ship, there was a mighty storm caused by God. Everyone was
  panicking and crying out to their God and etc. Even throwing things down
  into the sea. But not Jonah. Jonah was just sleeping in the ship. But why?
  My guess: cause Jonah really didnt want to go to nineveh and he would
  rather die lol. So when there was a tempest, Jonah went to the deck to
  sleep, he was content with dying. Then the captain went to find him.}
  \item{Then there was a casting of lots to find out why there was a tempest
  (in ancient thinking, disasters come about because of sin). Then the lot
  fell on Jonah. The lot fell on Jonah to show that Jonah was the cause of
  the evil. But very interestingly, even after the lot fell on Jonah, the
  captain still asked him: “tell us on whose account this evil has come upon
  us”. Why would the captain still ask Jonah that after the lot had fell on
  him? Perhaps the captain wanted a further explanation, like what was
  Jonah’s occupation, where was he from, etc. Then Jonah explained that he
  worships a sovereign God who made heaven and earth but he was running away
  from God lol.}
  \item{Then Jonah suggested that the sailors just throw him off the boat. I think again, since Jonah wanted to die. But the sailors didnt listen to Jonah, they didnt want to kill Jonah. Instead, they sought a solution that doesnt involve killing Jonah, so they tried to row back to dry land.}
  \item{But they can’t row to dry land because the storm got worse. They got even more fearful and then they called out to God, which was the only deity they haven’t called out to yet. And then they finally took Jonah’s suggestion, threw Jonah overboard, and while doing so, they cried for mercy from God to not be guilty of doing so. }
  \item{After doing that, the sea calmed down, and the men realised that God is really sovereign and powerful, and they feared the LORD, offered a sacrifice to the LORD and made vows.}
  \item{Application: Jonah is like an example of a Christian who sins wilfully and who doesn’t repent when given the opportunity to do so. The end result is that Jonah was directly chastised by God. When a Christian sins wilfully, they cause a lot of damage. This is Jonah and the tempest. Then when given an opportunity to repent, we should. If we don’t, the damage we cause will continue to increase. This is Jonah’s refusal to repent when the lot fell on him (he could have called out to God and apologised for not going to Nineveh). Lastly, when there is a unrepentant sinning christian, the only way to stop the damage is to excommunicate him. This is throwing Jonah over the boat. }
\end{itemize}