\documentclass[12pt, a4paper]{article} %possible sizes are 10,11,12
\title{Sermon Notes 2022}
\author{Kian Hwee}
%To input the current time
\usepackage[nodate, 12hr]{datetime}
\date{\small Last updated: \today, \currenttime} %leave blank if dont want date

%Packages for paper formatting, figures, hyperref, etc
\usepackage{microtype}
\usepackage{geometry} %package to adjust the margins n shit of the paper
\geometry{a4paper, margin = 2cm, top = 2cm, bottom = 2cm}
\usepackage{xcolor}
\usepackage{parskip} %package to tweak paragraph skipping
\usepackage[space]{grffile} %For file names with spaces
\usepackage{graphicx}
%\graphicspath{{./Figures/}}
\usepackage{chngcntr}
\counterwithin{figure}{section}
%\counterwithin{footnote}{section}
\usepackage{float}
\usepackage{comment} 
\usepackage{wrapfig}
\setcounter{tocdepth}{2}
%\setlength{\parskip}{0.1cm} %to set the line spacing between paragraphs. I.e,
%how compact shit should be. Also affects the list environment.
\PassOptionsToPackage{hyphens}{url}
\usepackage[hidelinks]{hyperref}
\hypersetup{
    colorlinks,
    linkcolor={blue!50!black},
    citecolor={blue!50!black},
    urlcolor={blue!80!black}
}
%For quotation marks to behave well
\usepackage{csquotes}
\MakeOuterQuote{"}

%Blind text
\usepackage[english]{babel}
\usepackage{blindtext}
\blindmathtrue

%Math packages
\usepackage{amsmath}
\usepackage{amssymb}
\numberwithin{equation}{section} %numbers equations according to sections
%\allowdisplaybreaks %equations in environment can page break
\usepackage{physics}
\usepackage{amsthm}
\theoremstyle{plain}
\newtheorem{theorem}{Theorem}[section]
\theoremstyle{remark}
\newtheorem{corollary}{Corollary}[theorem]
\theoremstyle{plain}
\newtheorem{lemma}[theorem]{Lemma}
\theoremstyle{definition}
\newtheorem{definition}{Definition}[section]
\theoremstyle{remark}
\newtheorem*{remark}{Remark}

%Alternate fonts 
% \usepackage{newtxtext,newtxmath} %this is good
\usepackage{fourier} %this is good
% \usepackage{newpxtext,newpxmath} %this is good
%The rest are...meh
% \usepackage{libertine,libertinust1math} %this is quite good
% \usepackage{fouriernc} %this is quite good
% \usepackage{mathptmx} %looks quite bad
%For new look
% \usepackage{newpxtext} \usepackage[euler-digits]{eulervm} %looks disgusting

\setcounter{secnumdepth}{0}

\begin{document}
  \maketitle
  \section*{Foreword}
    Herein contains my summary of the sermons for 2022.  These sermons notes
    are mostly typed out during the sermon as the pastor preaches.  Sometimes
    I will add in some of my own clarifications/notes/thoughts to the points
    given by the pastors, hence one can safely assume that any theological
    errors found herein are to be attributed to me, not the pastors. 
  \tableofcontents %optional
  \section{23rd January 2022: A Father's invitation to wisdom}
\subsection*{Text: Proverbs 1}
  \begin{quote}
    [1] The proverbs of Solomon, son of David, king of Israel:

    [2] To know wisdom and instruction,
        to understand words of insight,
    [3] to receive instruction in wise dealing,
        in righteousness, justice, and equity;
    [4] to give prudence to the simple,
        knowledge and discretion to the youth—
    [5] Let the wise hear and increase in learning,
        and the one who understands obtain guidance,
    [6] to understand a proverb and a saying,
        the words of the wise and their riddles.


    [7] The fear of the LORD is the beginning of knowledge;
        fools despise wisdom and instruction.


        [8] Hear, my son, your father’s instruction,
        and forsake not your mother’s teaching,
    [9] for they are a graceful garland for your head
        and pendants for your neck.
    [10] My son, if sinners entice you,
        do not consent.
    [11] If they say, “Come with us, let us lie in wait for blood;
        let us ambush the innocent without reason;
    [12] like Sheol let us swallow them alive,
        and whole, like those who go down to the pit;
    [13] we shall find all precious goods,
        we shall fill our houses with plunder;
    [14] throw in your lot among us;
        we will all have one purse”—
    [15] my son, do not walk in the way with them;
        hold back your foot from their paths,
    [16] for their feet run to evil,
        and they make haste to shed blood.
    [17] For in vain is a net spread
        in the sight of any bird,
    [18] but these men lie in wait for their own blood;
        they set an ambush for their own lives.
    [19] Such are the ways of everyone who is greedy for unjust gain;
        it takes away the life of its possessors.


        [20] Wisdom cries aloud in the street,
        in the markets she raises her voice;
    [21] at the head of the noisy streets she cries out;
        at the entrance of the city gates she speaks:
    [22] “How long, O simple ones, will you love being simple?
    How long will scoffers delight in their scoffing
        and fools hate knowledge?
    [23] If you turn at my reproof,
    behold, I will pour out my spirit to you;
        I will make my words known to you.
    [24] Because I have called and you refused to listen,
        have stretched out my hand and no one has heeded,
    [25] because you have ignored all my counsel
        and would have none of my reproof,
    [26] I also will laugh at your calamity;
        I will mock when terror strikes you,
    [27] when terror strikes you like a storm
        and your calamity comes like a whirlwind,
        when distress and anguish come upon you.
    [28] Then they will call upon me, but I will not answer;
        they will seek me diligently but will not find me.
    [29] Because they hated knowledge
        and did not choose the fear of the LORD,
    [30] would have none of my counsel
        and despised all my reproof,
    [31] therefore they shall eat the fruit of their way,
        and have their fill of their own devices.
    [32] For the simple are killed by their turning away,
        and the complacency of fools destroys them;
    [33] but whoever listens to me will dwell secure
        and will be at ease, without dread of disaster.”
  \end{quote}
\subsection*{Notes}
\begin{itemize}
  \item{Authorship of Proverbs: Mainly Solomon (see chapter 1:1, chapter
  10:1, chapter 25:1) but also others like Agur, Lemuel, ``the wise'', etc.
  Though Solomon's wisdom is greater than all of the others' wisdom (1 Kings
  4:29-34).}
  \item{Apparently, Solomon was also a man of science; his study of nature
  helped him to see wisdom in the created order, like when he provides an
  analogy to the ant to explain the value of hard work and diligence.  As
  will be seen in the later sermons, God's wisdom is baked into the whole
  created order (Proverbs 8, Proverbs 3) and hence one can get wisdom by
  studying the created order.}
  \item{That being said, books of Wisdom like Proverbs, Ecclesiastes, Job are
  revealing divine truths about the world, more than just information about
  the world.}
  \item{Proverbs teaches divine truth, teaching God's people what godly
  living in day to day life looks like.  Though on this note, Proverbs
  teaches us general principles instead of going into specifics; for example,
  in Proverbs 13 when the child and the rod are mentioned, the text does not
  tell us how many times we should use the rod, how much force to use with
  the rod, etc.}
  \item{Again, we must emphasize that that wisdom is from God above (2
  Chronicles 1:8-10), though it is available to all by God's common grace.}
  \item{Proverbs invites us to reflect upon God's truths rather than to
  promise us immediate rewards for our choices.}
  \item{Wisdom literature in its entirety (Job and Ecclesiastes also) teach God's people the entire complexity of life.
  \begin{itemize}
    \item{Proverbs gives us general principles for us to ponder and consider as we think about living righteously.}
    \item{Job and Ecclesiastes handle the edge cases (i.e, exceptions to the
    general principles in Proverbs) that now occur because of sin.  E.g,
    Proverbs say that generally the wicked will suffer for their wickedness,
    but Job and Ecclesiastes show the exceptions to the above general rule.}
  \end{itemize}}
  \item{Wisdom is more about knowledge, it is also about skillfulness (Exodus
  31:1-3).  More than being knowledgeable, being wise is to be skillful in
  applying God's truth in our life.}
  \item{True wisdom is inseparable from righteousness.  Wisdom teaches us how
  to do righteousness, and teaches us how to see beneath the surface to make
  sound judgements.
  \begin{itemize}
    \item{The Law teaches us what is right and wrong, but the emphasis of the
    Law is not on the ``how''.}
    \item{Wisdom emphasises that right living leads to life, wrong living
    leads to death (Proverbs 1:29-32), then goes on to give us general
    principles on what right living looks like.}
  \end{itemize}}
  \item{``Wisdom cries out in the street'' God's wisdom is available to all,
  not hidden.  It is just whether people want to hear or not.}
  \item{Wisdom is received on bended knees (Proverbs 1:7)
  \begin{itemize}
    \item{To fear God is to reverence God and to humbly submit to God's power
    and moral authority over all things.}
    \item{Fools disregard God and ignore God's moral authority, e.g Adam and
    Eve and the Fall.}
    \item{God's moral authority is baked into God's creation, hence people
    who go against God's moral authority go against God's created order and
    hence generally hurt themselves.  E.g, addiction to drugs, addiction to
    pornography, sex outside of marriage etc all are scientifically proven to
    cause harm to a person.
    \begin{itemize}
      \item{Though we must emphasize again that the devastating effect of
      going against God's wisdom in the created order is obscured because of
      sin, i.e sin throws the balance of the creation abit off (e.g thorns
      and thistles), and sin darkens our mind such that we don't really see
      the devastating, natural consequences of sin until it is too late.}
    \end{itemize}}
  \item{Wisdom from God is best exemplified in Jesus (1 Corinthians 1:30).
  Fear of God will lead us to be transformed through jesus.  Not our own
  wisdom that transforms us, but Jesus makes us wise.}
  \item{Wisdom is learnt not alone, but in the context of community and
  discipleship}
  \end{itemize}}
\end{itemize}
  \section{30th January 2022: Living wisely before God}
\subsection*{Text: Proverbs 3:1-12}
\begin{quote}
  [1] My son, do not forget my teaching,
        but let your heart keep my commandments,
  [2] for length of days and years of life
      and peace they will add to you.


  [3] Let not steadfast love and faithfulness forsake you;
      bind them around your neck;
      write them on the tablet of your heart.
  [4] So you will find favor and good success
      in the sight of God and man.


  [5] Trust in the LORD with all your heart,
      and do not lean on your own understanding.
  [6] In all your ways acknowledge him,
      and he will make straight your paths.
  [7] Be not wise in your own eyes;
      fear the LORD, and turn away from evil.
  [8] It will be healing to your flesh
      and refreshment to your bones.


  [9] Honor the LORD with your wealth
      and with the firstfruits of all your produce;
  [10] then your barns will be filled with plenty,
      and your vats will be bursting with wine.


  [11] My son, do not despise the LORD’s discipline
      or be weary of his reproof,
  [12] for the LORD reproves him whom he loves,
      as a father the son in whom he delights.
\end{quote}
\subsection*{Notes}
\begin{itemize}
  \item{Story of the lazy grasshopper and the hardworking ant: idleness leads
  to hunger.  This was a very famous fable in ancient Greece to teach
  dilligence.  In Proverbs, there is a similar story in Proverbs 6:6-8.}
  \item{Every culture has a wisdom tradition.  This makes sense, because
  wisdom helps us to live a good life.}
  \item{Peace in verse 1 is "shalom" in Hebrew, i.e there is a connotation of
  wholeness and abundant welfare.}
  \item{God makes wisdom available to all, God created the earth by wisdom
  (Proverbs 3:19-20).  I.e, God's wisdom is baked into His creation.  The
  created world reflects God's wisdom, and hence it makes sense that we can
  learn wisdom by looking at God's created order (e.g by looking at the ant).
  This is regardless of whether one believes in God or not.  This is nothing
  but the generosity of God, who freely gives his creatures common grace.}
  \item{However, the wisdom as described above is incomplete.  I would add
  that because of sin, the wisdom as described above is fallen; now we have
  wicked people prospering for example.  Sin has affected God's creation such
  that the wisdom baked into God's creation is abit disrupted, and also sin
  has affected us such that we are partially blind to all the wisdom in God's
  creation too.  Hence, all human wisdom must be completed and reframed by
  God's revelation.  To truly live the good life, we are to be like God, and
  we are to submit to God.
  \begin{itemize}
    \item{To be like God, we are to acquire the character of God.  From v3-4,
    we must possess steadfast love and faithfulness, in order to find success
    before the eyes of God and Man.  But steadfast love and faithfulness are
    qualities of God (Exodus 34:6).  This makes sense, because we are made in
    the image of God, and hence reflecting God's character is what we are
    supposed to do anyway.  Hence just as how God shows steadfast love and
    faithfulness to us, we should show steadfast love and faithfulness to
    others.  Of course, showing steadfast love and faithfulness to others
    might lead to us sacrificing something on our end.  However, living the
    good life is not about maximising our self pleasure, but it is about the
    collective shalom of all, which is God's original plan for humanity
    anyway.}
    \item{To submit our lives to God, we need to submit our thinking, our
    possessions, and our negative life experiences.
    \begin{enumerate}
      \item{To submit our thinking, we must first note that we are finite
      creatures with a finite understanding.  The worst thing for us is to be
      wise in our own eyes, because if that is the case, we can't grow
      intellectualy.  E.g Proverbs 26:12.  Now, if we note that God is
      infitely wise than us, then we ought not to be wise in our own sight,
      we must trust that God's commandments are far wiser than our own
      understanding of the world.  As Paul said, God's ``foolishness'' is
      infitely wiser than human wisdom.  An example is the flawed human
      ``wisdom'' of ``can do whatever, just don't get caught''.  But God's
      wisdom is Romans 13, submitting to lawful authority.  We are to ``trust
      in the LORD with all your heart, and do not lean on your own
      understanding...''}
      \item{To submit our posessions, we see that in verses 9 and 10 of our
      text.  This is not prosperity gospel, it is not about how when we give
      more, we will receive more.  This wisdom has the context of Israel's
      old covenant, where giving is part of the Law, and obeying the Law is
      about love for God and faithfulness for God.  And as per the OT
      promises, when Israel is faithful to God, God will bless Israel with
      material possessions in this world.  For us in the NT, even though the old
      sacrificial system is not for us, this principle is for us too, just
      that it is transformed; see Luke 12:33.  We should not expect God to
      bless us with material wealth when we submit our possessions to God, because God gives us things far better than material wealth; God gives us treasure in heaven.}
      \item{To submit our negative life experiences to God, we first look at
      Proverbs 13:24 too.  Just like how parents discipline their children,
      God disciplines us too.  However, not all suffering we experience is
      God's discipline/punishment of us.  See Ecclesiastes 8:14 and Job.
      Some of the suffering we experience is a result/consequence of sin.  It
      is wiser to regard suffering not necessarily as God's punishment for us
      (unless God has clearly revealed this to us), but as something that
      will draw us to God.  Suffering itself is not good, but God can bring
      good out of our suffering when we bring it to God.  This is what it
      means to submit our negative life experiences (our suffering) to God.
      In fact, suffering sometimes can lead to greater holiness and deeper
      communion with God, which is part of living the good life, and hence
      paradoxically, we might find greater shalom with God in our suffering.}
    \end{enumerate}}
  \end{itemize}}
  \item{The ultimate portrait of wisdom for us is Jesus Christ.  Jesus Christ
  was completely like God, and fully submitted Himself to God.  What Jesus
  did seemed foolish by worldly standards, but it is infitely wise by God's
  standard, and hence God has highly exalted Jesus Christ.  And now we can
  have the mind of Christ through the Holy Spirit.  Let us seek this, and
  hence get the ultimate shalom.}
\end{itemize}
  \section{6 Feb 2022: A disciple's heart}
\subsection*{Text: Proverbs 4:20-27}
\begin{quote}
    [20] My son, be attentive to my words;
        incline your ear to my sayings.
    [21] Let them not escape from your sight;
        keep them within your heart.
    [22] For they are life to those who find them,
        and healing to all their flesh.
    [23] Keep your heart with all vigilance,
        for from it flow the springs of life.
    [24] Put away from you crooked speech,
        and put devious talk far from you.
    [25] Let your eyes look directly forward,
        and your gaze be straight before you.
    [26] Ponder the path of your feet;
        then all your ways will be sure.
    [27] Do not swerve to the right or to the left;
        turn your foot away from evil.
\end{quote}
\subsection*{Notes}
\begin{itemize}
  \item{The key point of today's text is verse 23.  For the ancient
  Israelites, faith is not separated from knowledge and not separated from
  one's experience of the world.  Two points today: guarding the heart, and
  acting in wisdom.  Btw, the word ``heart'' here refers not to the physical
  heart, but to the entire inner life of the person.}
  \item{Wisdom starts with inclining our ears to our parent's saying.  That's
  how we learnt things like ``don't touch the hot stove'' etc.}
  \item{Guarding our hearts means to watch our inner man with all vigilance.
  Whatever happens to the heart will affect everything a person does.  I.e,
  if the heart is corrupt, then the affections, motives, pursuits of a person
  will be corrupt. The heart is the core of the person.}
  \item{As Augustine said: ``You have made us for yourself, O Lord, and our
  heart is restless until it rests in you''.  The restlessness is because of
  sin; sin causes us to try to fill our heart with worldly things, but as we
  all know, only God can fill our heart.}
  \item{The heart is what God transforms in the New Covenant, e.g that
  Jeremiah new covenant verse.  And once the heart is transformed, we will
  have abundant life in the sense of John 10:10b.  Also, from John 7:37-38,
  we see that believing in Jesus, is the way to have a transformed heart.}
  \item{We guard our hearts practically by Psalm 119:11.  Guarding our
  hearts keeps us from sinning, and helps us to live the abundant life that
  God has for us.  The more we hear God's word, the more we will recognise
  God's voice.}
  \item{In life we watch many worldly things carefully, like our weight, our
  children's grades, sales, etc.  But the text today says, \textbf{above all
  else}, guard your hearts.}
  \item{If we are parents/teachers, we help our kids guard their hearts
  through sound instruction in the fear of God.}
  \item{If we give our hearts to Jesus, Jesus must control everything, not
  just certain portions of our lives.}
  \item{Walking in wisdom: first, we note that our actions follow our
  beliefs.  Our behavior is a good indication of what is in our heart.  By
  the way, actions here include what we say (v24).  As per what Jesus says,
  what comes from the mouth first comes from the heart.  To walk wisely, we
  must be careful with what we say with our mouths; that is a practical
  application of guarding our hearts.  Our actions here also include what we
  see (v25).  So we should keep our eyes on heavenly things, and look at all
  things in the world through the lens of God's word.  Keeping our eyes on
  heavenly things mean to keep heavenly things as our primary motivation, e.g
  Paul and running the race, keeping his eyes on the prize.  Our actions here
  also include what we do with our feet (v26).  There are two paths we can
  choose to walk; the straight and the narrow path, or the wide path that
  leads to destruction.  As another application of guarding our hearts, we
  must daily constantly walk on the straight and narrow, and turn away from
  the wide path that leads to destruction.}
  \item{Our perfect role model for all of the above is our Lord Jesus Christ,
  who perfectly obeyed the voice of His Father and who perfectly did the will
  of His Father.  Because He is the perfect Son, He brings many sons and
  daughters into glory through the cross.  With Christ's help, it is possible
  for us to guard our hearts, through the Holy Spirit that Jesus sends us to
  help us walk in His example.}
  \item{In conclusion, let us guard our hearts and walk in wisdom (which is
  the result/expression of us guarding our hearts), and we can walk in wisdom
  by watching the things we say, the things we see, and the things we choose
  to do (our feet).}

\end{itemize}
  \section{13th February 2022: Do you want to be blessed?}
\subsection*{Text: Jeremiah 17:5-10}
  \begin{quote}
    [5] Thus says the LORD:
    “Cursed is the man who trusts in man
        and makes flesh his strength,
        whose heart turns away from the LORD.
    [6] He is like a shrub in the desert,
        and shall not see any good come.
    He shall dwell in the parched places of the wilderness,
        in an uninhabited salt land.


    [7] “Blessed is the man who trusts in the LORD,
        whose trust is the LORD.
    [8] He is like a tree planted by water,
        that sends out its roots by the stream,
    and does not fear when heat comes,
        for its leaves remain green,
    and is not anxious in the year of drought,
        for it does not cease to bear fruit.”


    [9] The heart is deceitful above all things,
        and desperately sick;
        who can understand it?
    [10] “I the LORD search the heart
        and test the mind,
    to give every man according to his ways,
        according to the fruit of his deeds.”
  \end{quote}
\subsection*{Notes}
\begin{itemize}
  \item{The bible has a different take on what it means to be blessed, as
  compared to the secular idea of blessings being in the form of health and
  wealth.  Of course God can bless us with such health and wealth, but having
  health and wealth is not a necessary condition for one consider himself
  blessed by God (see Luke 6:17-26).  Let's understand why the poor etc can
  consider himself blessed.}
  \item{In the Jeremiah text, there are two groups; those who trust in man vs
  those who trust in God.  Those who trusted in man would be those, in
  Jeremiah's time, would be those who trust in military might, in false gods
  and false prophets.  Those who trusted in God, in Jeremiah's time, would be
  those who rely on God for deliverance.  There are also two analogies
  corresponding to these two groups: shrub in the desert vs the tree planted
  by the stream.
  \begin{itemize}
    \item{Those who trust in man are like the shrub who don't bear any fruit.}
    \item{Those who trust in God are like the tree which always bears fruit,
    whose leaves are always green.}
  \end{itemize} Both the shrub and the tree might both experience drought,
  but because the roots of the shrub are not deep, the shrub will wither.
  But for the tree, since the roots are deep, it can still bear fruit in
  times of drought.}
  \item{I.e, those who trust in man are cursed, and those who trust in God
  are blessed.  What is depicted here in Jeremiah is a throwback to the
  Mosaic Law, in Deuteronomy 28.  There will be blessings for obedience and
  curses for disobedience.  Faith and obedience go hand in hand; those who
  trust and have faith in God will obey God naturally.  When we trust God, we
  will obey.  And when we obey, we will be blessed.  Hence we can say that
  those who trust in God will be blessed.}
  \item{In Jeremiah's time, Judah was a nation that didn't trust in God,
  hence they experienced the curses of the Law, experiencing things like
  famine etc and even eventually exile.  But for those in Judah that were
  going through the same difficult times, they are still blessed as long as
  their trust is in the LORD.}
  \item{Hence in Luke, even in adverse circumstances (e.g poverty), we can
  still be blessed when we obey.  Blessing does not necessarily remove
  suffering; in the Jeremiah text, we see that even the tree experiences
  drought (Jeremiah 17:8).  Hence we can say that the poor who trust in God
  are still blessed.}
  \item{What are some of these blessings? They are:
  \begin{itemize}
    \item{Peace that transcends all understanding.}
    \item{Communion and friendship with God.}
    \item{Treasures in heaven.}
  \end{itemize}}
  \item{So how is all of these relevant for us?  The starting point of a
  blessed life is to put our trust in God for our salvation.  We must trust
  in Jesus' finished work on the cross rather than in our own works; our
  hearts are deceitful above all things, and hence our own works are rubbish.
  Jesus has come to seek and save the lost, only Jesus can save us, because
  only He has died on the cross for our sins.  For His suffering on our
  behalf, our sins are cleansed, but if and only if we put our trust in
  Jesus.  If we reject Jesus and put our trust in ourselves (i.e in man)
  instead, we are cursed.  The ``good works'' that we do will never be enough
  before a holy and just God.  Apart from Jesus, we are under the wrath of
  God for our sins; and this will manifest itself especially on judgement
  day.  Hence, the starting point of a blessed life is to trust in Jesus'
  completed work for our salvation.}
  \item{We can also put our trust in man when we trust in the physical
  blessings that God give us rather than recognising that God is behind all
  of those blessings.  When those physical blessings distract us from God,
  then those physical blessings will lead us to be cursed.  E.g, some
  parents' lives revolve around their children so much so that they have no
  time to spend with God, no time to participate in church life, etc.  We
  should work hard and study hard and take care of our kids etc, this is our
  testimony before the world.  But we should not let these things distract us
  from God, because if not we will be like the shrub in the desert.  Hence
  let us repent and seek God and His kingdom first.}
  \item{When we are going through difficult times, are we turning our hearts
  away from God in anger and disappointment or do we cling on to God with
  confidence and hope?  If we turn away from the source of blessings (i.e
  God), how can we be blessed? There are two examples which Pastor Kien Seng talked about:
  \begin{itemize}
    \item{Example 1: Very sian because of physical sickness, don't even want
    to come to church.}
    \item{Example 2: Physical sickness has led to greater trust and
    dependence in God.}
  \end{itemize}
  Here we see two people in the same circumstance but one is evidently more
  blessed than the other, because he is closer to God. The latter person is what it means to be blessed in God's eyes.}
  \item{So do you want to be blessed?  If you do,
  \begin{itemize}
    \item{Believe in Jesus, cultivate your relationship with Jesus.  Cling on
    to Him, put your trust in Him.  Put our trust in who God is and what He
    has done for us, especially in the finished work of the cross.  Putting
    our trust in God also means putting our trust in His Word, because His
    Word reveals who He is and what He has done.  We also put our trust in
    God's promises.  And when we trust in God's promises and His Word, we are
    like the wise man who builds his house on the rock.}
  \end{itemize}}
\end{itemize}


















  \section{20th February 2022: Turning from anger}
\subsection*{Text: Proverbs 15:1,18;16:32;27:3,4}
  \begin{quote}
    [1] A soft answer turns away wrath,
        but a harsh word stirs up anger.
    [18] A hot-tempered man stirs up strife,
        but he who is slow to anger quiets contention.

    [32] Whoever is slow to anger is better than the mighty,
        and he who rules his spirit than he who takes a city.   

    [3] A stone is heavy, and sand is weighty,
        but a fool's provocation is heavier than both.
    [4] Wrath is cruel, anger is overwhelming,
        but who can stand before jealousy?
  \end{quote}
\subsection*{Notes}
\begin{itemize}
  \item{Anger is a common emotion; even animals can get angry.  It is
    possible to get angry without sinning; see Ephesians (Be angry and do not
    sin).  Anger is not sin it and of itself, but our response to our anger
    could be sin.}
  \item{Anger is part of our body's fight or flight response, more
    specifically our fight response; anger can help us to protect
    ourselves/others, stand up against injustices of the world.}
  \item{To see that anger can be non-sinful, conisder that God can be angry;
    however, God's anger is wholly without sin.  God is also not fundamentally
    angry, he is fundamentally righteous; his anger is directed at the
    sinfulness of Man.}
  \item{But anger can of course be sinful, when it controls us and makes us
    meaningful, full of hatred and bitterness. One simple example is road rage.}
  \item{For Man, anger is usually an emotional response to a threat against
  our \textbf{ego}.  Human experience makes this abundantly clear.}
  \item{To help manage our anger, there are 3 main points:
  \begin{itemize}
    \item{First point: restraint, not react.  Instictively, when we hear
    something we don't like, sometimes we just react based on the emotions we
    are feeling.  These feelings are often based on our first impressions of
    the things we have heard, and contain a lot of our biases. Most of the time, people are triggered when:
    \begin{itemize}
      \item{They feel threatened.}
      \item{They feel frustrated or powerless, hence angry at ourselves.  Or
      angry at the situation.}
      \item{They feel invalitated or unfairly treated.}
      \item{They or their possessions are not respected.}
    \end{itemize}
    For us personally, it is helpful to find out what are our triggers.
    However, all of the above is not an excuse for sin; just because triggers
    exist doesn't mean we have to be triggered.  Self-control is a Christian
    virtue, a fruit of the Spirit (Galatians 5:22-23).  Practically, we can
    restrain when we deliberately interrupt our `reactive' response, by
    walking away, slowly counting to 5, taking a break from the situation,
    etc.  If we must respond, we are to give a `soft answer'.  If we can't
    give a `soft answer', it is better to keep quiet.  If we can't give a
    `soft answer' and shout back because of our anger, then the other person
    will get angry also, and then there'll be a relationship breakdown.  The
    above might seem hard to do, but remember that Jesus faced the biggest
    injustice in the world, yet he was like a lamb led to the slaughter; he
    opened not his mouth.}
    \item{Second point: re-evaluate, not relish.  When we re-evaluate
    something, we do it more objectively as compared to our instinctive
    response, and we widen the possible eplanations or interpreations of the
    actions or what was said.  When we relish, we take delight in going through
    the situation over and over again to stir up our emotions.  We become more
    angry and our burden becomes heavier.  We should also re-evaluate the
    consequences of our possible angry actions, especially in light of James
    1:19-20.  We must remember that our human anger does not produce the
    righteousness of God. One such consequence is that our unrestrained anger is a poor testimony to our Christian faith.}
    \item{Third point: release, not retain.  Keeping our anger within us is
    described as a burden, as in Proverbs 27:3-4.  A stone is a heavy thing,
    and as for sand, a lot of it is a heavy thing.  Sand is a particularly
    good analogy; if everytime we retain our anger we store up a grain of
    sand, then in half a lifetime, all that sand will accumulate to be
    something super heavy.  And also, as per the proverbs verse, one possible
    thing that stirs up our anger is jealousy.  To release our anger to God,
    we should:
    \begin{itemize}
      \item{Be truthful with God about your feelings.  Lament them to God, as
        per in the Psalms.}
      \item{Be ready to forgive those who said those things that made us
        angry.  Just as God forgiven us, we should also forgive.}
      \item{Ask God to turn our problems into solutions.}
      \item{To combat jealousy, we thank God that we are made differently and
        given different gifts.  We affirm the good in our lives and recognise
        the source of goodness which is from God.}
    \end{itemize}}
  \end{itemize}}
  \item{Our angry, sinful response is merely a symptom of a greater problem,
  which is sin.  The Bible is not just an anger management manual; when Jesus
  died on the cross for our sins, He removed the root of our human anger,
  which is sin.  The ultimate remedy for anger is to kill the sin in our
  lives, through the ministry of the Holy Spirit and of the Word.  We are to
  emuluate Jesus' example, as per Philippians 2:3-11, and when we do, anger
  will vanish.  When we are slighted, we must remember that we forgive as God
  forgiven us.}
\end{itemize}
  \section{27th February 2022: Fleeing from sexual immorality}
\subsection*{Text: Proverbs 5:1-23}
  \begin{quote}
    [1] My son, be attentive to my wisdom;
        incline your ear to my understanding,
    [2] that you may keep discretion,
        and your lips may guard knowledge.
    [3] For the lips of a forbidden woman drip honey,
        and her speech is smoother than oil,
    [4] but in the end she is bitter as wormwood,
        sharp as a two-edged sword.
    [5] Her feet go down to death;
        her steps follow the path to Sheol;
    [6] she does not ponder the path of life;
        her ways wander, and she does not know it.


    [7] And now, O sons, listen to me,
        and do not depart from the words of my mouth.
    [8] Keep your way far from her,
        and do not go near the door of her house,
    [9] lest you give your honor to others
        and your years to the merciless,
    [10] lest strangers take their fill of your strength,
        and your labors go to the house of a foreigner,
    [11] and at the end of your life you groan,
        when your flesh and body are consumed,
    [12] and you say, “How I hated discipline,
        and my heart despised reproof!
    [13] I did not listen to the voice of my teachers
        or incline my ear to my instructors.
    [14] I am at the brink of utter ruin
        in the assembled congregation.”


    [15] Drink water from your own cistern,
        flowing water from your own well.
    [16] Should your springs be scattered abroad,
        streams of water in the streets?
    [17] Let them be for yourself alone,
        and not for strangers with you.
    [18] Let your fountain be blessed,
        and rejoice in the wife of your youth,
    [19]     a lovely deer, a graceful doe.
    Let her breasts fill you at all times with delight;
        be intoxicated always in her love.
    [20] Why should you be intoxicated, my son, with a forbidden woman
        and embrace the bosom of an adulteress?
    [21] For a man’s ways are before the eyes of the LORD,
        and he ponders all his paths.
    [22] The iniquities of the wicked ensnare him,
        and he is held fast in the cords of his sin.
    [23] He dies for lack of discipline,
        and because of his great folly he is led astray.
  \end{quote}
\subsection*{Notes}
\begin{itemize}
  \item{Sex is a very precicous and powerful gift given to us by God.  It is
  so powerful and precious that if sex is abused, it will have dire
  consequences.  Scripture doesn't pull any punches in this topic, because
  the stakes are high.  This is an issue that affects everyone,
  married/unmarried, men/women. While proverbs 5 is addressed literally to men, we can easily apply it to women. The reason why it is addressed to men is cultural and will not be explored here.}
  \item{Three main points:
  \begin{itemize}
    \item{Attraction: the power of attraction and how it works.}
    \item{Attrition: the wear and tear, the pain and comes our way when we
    fall into sexual immorality.}
    \item{Attention: how are we to discipline our attention to avoid sexual
    immorality.}
  \end{itemize}}
  \item{Attraction: There is hardly a Christian person that wakes up thinking
  ``today I want to fall into sexual immorality''.  But Christians still fall
  nonetheless because we are led along by sin, i.e we are attracted to sin.
  Two avenues by which we are attracted is via the ear and the eye; see
  proverbs 5:3.  And also, by the scent, or by the touch.  Different people
  are more susceptible to different things.  So we must know what our
  weakness is, so we can take appropriate measures to flee.  Specifically for
  the ear and the eye, we have pornography; there are hardly any silent
  pornographic films for example, they use the two-pronged approach.  This is
  an issue that is especially difficult in the Internet age.  There are many
  strategies that we can take to counter this temptation; though those
  strategies can be inconvenient, we \textbf{must} do it, because if we
  don't, we will be led along.}
  \item{Attrition: What if we are led along?  If we are, we can look at
  Proverbs 5:4-14.  I.e, a large proportion of Proverbs 5 describes the great
  and dire consequences of being led along by sexual sin, so that when we
  enter into temptation, we might consider the consequences and get out of
  temptation.  E.g, Proverbs 5:4-5.  There is bitterness, loss of honor, lost
  of years (e.g when your whole life with your spouse come crashing down
  because of infidelity).  Being attracted to sin is like borrowing from a
  loan shark; if we borrow, we might need to pay it back with a lot of
  interest.  E.g for one moment of pleasure with an adulteress, we might need
  to spend many, many years rebuilding family relationships. The price of sexual immorality is real and hefty.
  
  The long description of attrition here also tells us that God wants our
  life to flourish, and that our life flourishes only when we walk according
  to the framework of God's Law, which is the divine design for a flourishing
  human life. We need to daily consider the stakes, to help us flee from sexual immorality.}
  \item{Attention: So what are we to do with this frightening monster?  There
  are external things we can do, but there are also internal things (see
  Proverbs 5:7) we can do, which we shall classify as disciplining our
  attention.  What we give our attention to can be trained; we must be
  intentional about what we think about, what we look at, to know when to
  stay and dwell, to know when to flee like Joseph in Popithar's house.  One
  thing we can dwell in is to our allegiance to our spouse and our family,
  see Proverbs 5:15-20.  We must daily put our attention there, so that when
  temptation comes, we will know that we must flee.  Another idea we can
  dwell in is to discipline our affection; see verse 18.  We must always
  strive to dwell on and develop our affection for our spouse.  If we think
  that our relationship is getting `stale', we must then take time to put our
  attention to appreciate the depths of beauty in our spouse.  People will
  age, but the inner beauty is permanent (charm is deceitful, beauty is vain,
  but the fear of the Lord...).  In fact, the word ``intoxicated'' here for
  the wife is the same as ``leading astray'' used for the adulteress above.
  Thus, the author of Proverbs is contrasting the two types of ``leading
  astray'' here.}
  \item{In conclusion, remember the three points here; the
  \textbf{attraction} of sexual sin, the \textbf{attrition} caused by sexual
  sin, and to discipline our \textbf{attention}.  Especially in the NT
  context, remember that our bodies are temples of the Holy Spirit (1
  Corinthians 6:19-20).  Sexual sin can be forgiven by your spouse, but the
  real life consequences, the hurt and the pain, are real.  Sexual sin can be
  forgiven but not forgotten.  So don't fall into it, consider the
  consequences, and discipline your attention.  And for the unmarried, we
  must still discipline our attention, so that we don't get led astray into
  destruction.  And when we do so, we learn to discern what is important in a
  life partner.}
\end{itemize}
  \section{6th March 2022: Find us faithful}
\subsection*{Text: }
  \begin{quote}
    [12] On the following day, when they came from Bethany, he was hungry.
    [13] And seeing in the distance a fig tree in leaf, he went to see if he
    could find anything on it.  When he came to it, he found nothing but
    leaves, for it was not the season for figs.  [14] And he said to it, “May
    no one ever eat fruit from you again.” And his disciples heard it.

    [15] And they came to Jerusalem.  And he entered the temple and began to
    drive out those who sold and those who bought in the temple, and he
    overturned the tables of the money-changers and the seats of those who
    sold pigeons.  [16] And he would not allow anyone to carry anything
    through the temple.  [17] And he was teaching them and saying to them,
    “Is it not written, ``My house shall be called a house of prayer for all
    the nations"?  But you have made it a den of robbers.” [18] And the chief
    priests and the scribes heard it and were seeking a way to destroy him,
    for they feared him, because all the crowd was astonished at his
    teaching.  [19] And when evening came they went out of the city.

    [20] As they passed by in the morning, they saw the fig tree withered
    away to its roots.  [21] And Peter remembered and said to him, “Rabbi,
    look!  The fig tree that you cursed has withered.” [22] And Jesus
    answered them, “Have faith in God.  [23] Truly, I say to you, whoever
    says to this mountain, ``Be taken up and thrown into the sea," and does
    not doubt in his heart, but believes that what he says will come to pass,
    it will be done for him.  [24] Therefore I tell you, whatever you ask in
    prayer, believe that you have received it, and it will be yours.  [25]
    And whenever you stand praying, forgive, if you have anything against
    anyone, so that your Father also who is in heaven may forgive you your
    trespasses.”
  \end{quote}
\subsection*{Notes}
\begin{itemize}
  \item{This cursing of the fig tree was in the week before Passover.  During
  this time, it wasn't the fig season yet; thus obviously there wouldn't have
  been any figs on the fig tree.  Then why did Jesus curse the fig tree for
  not producing food?  The reason was to create a teaching opportunity with
  the disciples.}
  \item{People might think that it wasn't fair for Jesus to curse the tree.
  Yet it was also unfair for the sinless Son of Man to die on the cross.  So
  yea lol.  What was the lesson that Jesus wanted to teach?  Note that the
  cleansing of the temple is in between the cursing and the withering of the
  fig tree.  So the fig tree was being compared to the temple; though it
  looks busy, there is no fruit.  And hence we find Jesus foretelling the
  destruction of the temple in chapter 13; the withering of the fig tree was
  an analogy of the temple.}
  \item{By the way, interesting question; if buying and selling things made
  the temple unclean, then do similar things that happen in church make the
  church unclean? Answer is, it depends; see the below:}
  \item{Jesus overturned the market to create a teaching opportunity, see
  verse 17.  Two verses were quoted here, Isaiah 56:7 for the house of
  prayer, Jeremiah 7:11 for the den of robbers.  Some context: only one type
  of currency was accepted for temple tax, hence a genuine need for money
  changing.  Also, only unblemished animals were allowed as sacrifices, hence
  people would prefer to buy animals in the city themselves.  In King Herod's
  time, he also expanded the temple, to create an outer court for the
  Gentiles.  This wasn't part of God's original temple plan, but perhaps this
  was God's way of allowing the Gentiles to pray in the temple.  The
  religious leaders in those days allowed the outer court to be used for
  trading, buying and selling.  During the time of passover, there would be a
  lot of activity in the outer court, and hence the Gentiles wouldn't have
  been able to pray because of the lack of space.  Hence, the Isaiah verse
  tells us what the temple should have been, but the Jeremiah verse tells us
  what the temple is really used for because of sin.  The religious leaders
  neglected the Isaiah verse w.r.t Gentiles, and focused on the business side
  of things to line their own pockets.}
  \item{So according to Isaiah 56:7, the church should be a house to make
  disciples of all nations.  Like the story of the fig tree, when we focus on
  making the church attractive instead of having actual discipleship, then
  we'll end up like the fig tree.  Though we must stress; it is ok to want to
  beautify the church, and to have other activities in church etc.  But all
  of those things must serve to build up the church in a real spiritual
  manner; if they don't and instead distract us from doing God's work, then
  we should scrap them instead of focusing on them at the expense of actual
  discipleship, lest we end up like a den of robbers.  Real example: coffee
  in cosy corner is not a bad thing, but if considerable time is spent on
  deciding things like the type of coffee instead of the actual discipleship,
  then we have turned the house of God into a den of robbers.}
  \item{Another way that the church can be turned into a den of robbers is
  the prosperity gospel.  Just because scripture is quoted, doesn't mean that
  it is used correctly; even the devil knows how to use the scripture (c.f
  temptation of Jesus in the wilderness).  A correct use of scripture by the
  church leaders should lead to actual transformation of the heart, a true
  turning away from sin and to God.  A wrong use of scripture by the church
  leaders leads to people doing things like asking people to give money to a
  hedge fund (with the details known only to a privileged few), asking people
  to buy books written by the senior pastor to make the book a bestseller,
  etc.}
  \item{Btw, this text can also be used to make statements about how faith in
  God can do miracles, c.f verse 22.  This is how some pastors use this text.
  This use of the text would lead them to say stuff like ``if you ask, you
  will receive; but if you don't receive, means you have no faith''.  But
  that's probably not the proper use of this text; we must always read this
  fig tree episode with the temple in mind (especially w.r.t the temple being
  mentioned again in chapter 13).  Here, Jesus is contrasting faith in God
  and faith in the temple.  Jesus was telling their disciples not to put
  their faith in the temple, but continue to have faith in God even when the
  temple is destroyed.  This way, their continued faith in God would help
  them to overcome challenges to spread the gospel (even moving mountains),
  knowing that anything that they ask for the kingdom's sake will be given to
  them.  So for us today, we must have faith in God, not in our church/our
  leaders/our parachurch organisation etc.  Nothing wrong with liking our
  church, but it is problematic when our leaders are dodgy, and then if our
  faith is ultimately in our leaders, we will be wholly disillusioned and
  might leave our faith.}
  % \item{Point 1}
  % \item{Point 2}
\end{itemize}
  \section{13th March 2022: At the Heart of God's Law}
\subsection*{Text: Mark 12:23-34}
  \begin{quote}
    [28] And one of the scribes came up and heard them disputing with one
    another, and seeing that he answered them well, asked him, “Which
    commandment is the most important of all?” [29] Jesus answered, “The most
    important is, ‘Hear, O Israel: The Lord our God, the Lord is one.  [30]
    And you shall love the Lord your God with all your heart and with all
    your soul and with all your mind and with all your strength.’ [31] The
    second is this: ‘You shall love your neighbor as yourself.’ There is no
    other commandment greater than these.” [32] And the scribe said to him,
    “You are right, Teacher.  You have truly said that he is one, and there
    is no other besides him.  [33] And to love him with all the heart and
    with all the understanding and with all the strength, and to love one’s
    neighbor as oneself, is much more than all whole burnt offerings and
    sacrifices.” [34] And when Jesus saw that he answered wisely, he said to
    him, “You are not far from the kingdom of God.” And after that no one
    dared to ask him any more questions.
  \end{quote}
\subsection*{Notes}
\begin{itemize}
  \item{In Jesus' time, many religious leaders confronted Jesus with
  questions.  But they weren't interested in finding out the truth; they were
  hoping to use these questions to trap Jesus so that Jesus would lose his
  credibility.  But not the scribe in this passage; this scribe was said to
  be not far from the kingdom of God.  The scribe had a genuine question,
  which was to find out what the heart of godly living was.  And in this
  passage, Jesus answered the scribe directly.}
  \item{Instead of quoting a particular law, Jesus quoted the Shema
  (Deuteronomy 6).  There is only one God, and we are to worship God
  single-mindedly.  God graciously revealed Himself to Abraham and His
  descendants, and as a response to God's loving self revelation, Israel was
  supposed to pledge their allegiance to God.  The LORD God allows Israel
  (and us) to know about Himself; His gracious self revelation provides us
  with true knowledge about Him and His nature.  God is one, this means that
  God is unique, and that there is only one God.  And hence we are to worship
  God alone.}
  \item{God's law's are meant to reveal His nature.  When God gave Israel His
  law, He has already rescued them out of Egypt.  It is not that Israel has
  to earn God's love by following His law; God already loved Israel when He
  gave them the Law, and the purpose of the Law giving was to teach Israel
  how to maintain a relationship with Him, especially since Israel was
  amongst the pagan nations.}
  \item{To love God is more than a fuzzy feeling towards God; love for God is
  doing actions demonstrated in faithfulness.  If we read on in Deuteronomy
  6, there are some actions that God suggests would be an expression of
  loving Him with all our heart, soul, mind and strength.  These actions
  include passing down God's truth to the next generation (Deuteronomy
  6:6-9).  These actions also include giving thanks for our food, because
  ``when we eat and are full, then take care lest you forget the LORD...''}
  \item{Running towards other idols will destroy us; God is the only source
  of life, and when we turn towards other idols and away from God, we are
  turning away from life and turning unto death.}
  \item{Apart from loving God, the next greatest commandment was loving our
  neighbour.  What Jesus did that unique amongst the rabbis of His time; in
  the past, other rabbis also summarised the Law as loving our neighbour, but
  what Jesus did was to combine loving God and loving our neighbour.  We
  cannot truly love God unless we love our neighbour, and we cannot truly
  love our neighbour unless we love God.  For example, when we look at
  portions like Leviticus 19:10,18 we realise that we love our neighbour
  because ``I am the LORD your God''.}
  \item{The scribe understood what Jesus said (see Mark 12:34).  Instead of
  merely a confession of faith, God calls also for a demonstration of love
  towards Him and others.  E.g in Leviticus, it is said that when we gather
  grapes for the vineyard, we should leave some for the sojourner.  To do so
  ungrudgingly, one must first realise that God is the one who blesses his
  vineyard, and one must first not be anxious about his harvest, e.g one must
  first trust God.  And when one trusts God, he can love his neighbor easily
  because there is no longer any anxiety etc preventing him from loving his
  neighbour by leaving grapes.}
  \item{The scribe understood that it is not about outward religiousity, but
  it is about the attitude of the heart.  See Isaiah 1:10-20, our scripture
  reading for today.  Or see many other of the OT prophets haha...  Christian
  living is more than just mechanicallly memorising scripture and turning up
  on Sunday because it is an obligation, true Christian living includes
  loving our neighbour too (which would demonstrate our love for God).}
  \item{However, when we try to love God with all our heart, soul, mind and
  strength, and to love our neighbour as ourselves, we'll soon realise that
  we can't do it with our own strength.  We need a savior.  And this savior
  is Jesus.  Even in the OT, we know that even as the sincere, faithful OT
  people tried their best to love God, they still needed to trust in God for
  the forgiveness of their sins, because they realise that they fall short
  (e.g all the Psalms, like Psalm 51).  Now that God has revealed Himself to
  us through the person of His Son, we put our trust in Jesus, who is the
  image of the invisible God and the exact imprint of God's glory.  Just as
  the OT saints trusted in God to forgive their failures to love God
  perfectly, today we trust in Jesus to forgive us our failures to love God
  perfectly.  And we also know that as we put our trust in Jesus, He will
  slowly transform us to help us to better love Him and our neighbours.}
  \item{Loving our neighbours means more than doing good deeds (it is not
  less than that), loving our neighbours also means inviting them to follow
  Jesus, and helping them to be disciples of Jesus, because Jesus is the
  source of life.}
\end{itemize}
  \section{20th March 2022: False Piety or True Devotion}
\subsection*{Text: }
  \begin{quote}
    [38] And in his teaching he said, “Beware of the scribes, who like to
    walk around in long robes and like greetings in the marketplaces [39] and
    have the best seats in the synagogues and the places of honor at feasts,
    [40] who devour widows’ houses and for a pretense make long prayers.
    They will receive the greater condemnation.”

    [41] And he sat down opposite the treasury and watched the people putting
    money into the offering box.  Many rich people put in large sums.  [42]
    And a poor widow came and put in two small copper coins, which make a
    penny.  [43] And he called his disciples to him and said to them, “Truly,
    I say to you, this poor widow has put in more than all those who are
    contributing to the offering box.  [44] For they all contributed out of
    their abundance, but she out of her poverty has put in everything she
    had, all she had to live on.”
  \end{quote}
\subsection*{Notes}
\begin{itemize}
  \item{The two stories in today's passage are deliberately arranged as so to
  contrast the scribes with the widow.  The scribes neither loved God nor
  people, but the widow loved God with total devotion.}
  \item{The scribes appeared to be lovers of God, but they really are lovers
  of self (v38-39).  They like to be viewed as pious by people, such as
  making long prayers in public, as compared to what Jesus said in the Sermon
  on the mount about private prayer.  But we know that their piety is false,
  because their actions are unjust; they are said to "devour" widows' houses.
  The scribes were hypocrites, exploiting the poor while still looking pious.
  This is a clear violation of the teaching of the Law and the Prophets, e.g
  in Isaiah 10:1-4.  They did not love their neighbours, especially those who
  were vulnerable.  And this is especially egregious because the scribes were
  supposed to know the Law.  The widow in the story \textit{could} have been
  one who's house was devoured by the scribes.}
  \item{In the 20th century, Hitler was a master of outward religiosity with
  no inward piety. Reflections for us:
  \begin{itemize}
    \item{For leaders in the church, do y'all use your authority and power to
    exploit your sheep for the people under you?}
    \item{For all: are we lovers of self when we should be lovers of God?  Do
    you do things just to get the approval of others, to be seen as a 'good
    Christian'?  Are we pious only for people to see?  We might be able to
    fool the people around us, but we can't fool God.  Do we live a double
    life?  Are we well-behaved in church but for the rest of the week we're
    mean to our spouse/children?  The solution is to repent from this
    hypocrisy, to turn away from sin and to turn to God.}
  \end{itemize}}
  \item{The widow offered a small offering, but Jesus did not despise it.
  The widow offered two copper coins, where a copper coin was a hundredth of
  a denarius, where a denarius is a day's wage for a labourer.  In fact,
  Jesus commended the widow for giving more than the rich.  In a sense, what
  Jesus looked at was not the absolute amount of money $x$ given by a person,
  but the percentage of money given $x/x_{\text{Total}}$.  There was not as
  much sacrifice on the part of the rich, when they gave, but for the widow,
  her offering was costly for her.}
  \item{The widow serves thus as a vivid model for sacrificial discipleship,
  complete surrender, and total trust.  Without the trust that God will care
  for her, she probably wouldn't have given all her money.}
  \item{ ``The story of the poor widow reminds us that in God's economy, the
  size of the gift is of no consequence, what is of consequence is the size
  of the giver's heart''.
  Even if our gift is small, God can multiply it
  anyway (c.f the story of the 5 loaf and 2 fishes).  Reflections for us:
  \begin{itemize}
    \item{Is God speaking to you and nudging you to give more sacrificially?}
    \item{Besides giving, in what other ways can we practice sacrificial
    discipleship that Jesus requires of us?}
    \item{Is our discipleship merely based on convenience?}
  \end{itemize}
  One example: do we plan our day/life such that we give God the remainder of
  our money/time only after we have spent the money/time on ourselves?  Or do
  we give to God the firstfruits of our time/money/energy?}
  \item{Application to worship: online service is definitely more convenient
  than coming to church, but it is a poor substitute.  Do we then do things
  based on our convenience, or do we sacrifice a bit of our convenience so
  that we can obey God's revealed will for us to come together to meet to
  exhort each other.}
  \item{Let us ... (conclusion to be added later)}
\end{itemize}
  \section{27th March 2022: Extravagant devotion}
\subsection*{Text: Mark 14:1-11}
  \begin{quote}
    [1] It was now two days before the Passover and the Feast of Unleavened
    Bread.  And the chief priests and the scribes were seeking how to arrest
    him by stealth and kill him, [2] for they said, “Not during the feast,
    lest there be an uproar from the people.”

    [3] And while he was at Bethany in the house of Simon the leper, as he
    was reclining at table, a woman came with an alabaster flask of ointment
    of pure nard, very costly, and she broke the flask and poured it over his
    head.  [4] There were some who said to themselves indignantly, “Why was
    the ointment wasted like that?  [5] For this ointment could have been
    sold for more than three hundred denarii and given to the poor.” And they
    scolded her.  [6] But Jesus said, “Leave her alone.  Why do you trouble
    her?  She has done a beautiful thing to me.  [7] For you always have the
    poor with you, and whenever you want, you can do good for them.  But you
    will not always have me.  [8] She has done what she could; she has
    anointed my body beforehand for burial.  [9] And truly, I say to you,
    wherever the gospel is proclaimed in the whole world, what she has done
    will be told in memory of her.”

    [10] Then Judas Iscariot, who was one of the twelve, went to the chief
    priests in order to betray him to them.  [11] And when they heard it,
    they were glad and promised to give him money.  And he sought an
    opportunity to betray him.
  \end{quote}
\subsection*{Notes}
\begin{itemize}
  \item{This event here took place close to Passover (as mentioned in v1).
  As seen in v2, someone wanted Jesus dead.}
  \item{Today's main point: Faithful disciples can show their extravagant
  \textbf{commitment} (devotion) in the face of \textbf{criticism} because
  God will \textbf{commend} such efforts. I.e, three `C's.}
  \item{\textbf{Commitment:} in verse 3, we see an unnamed woman.  In these
  times, due to ancient Jewish customs, the fellowship here was probably a
  men's fellowship.  Women only attend such fellowship to serve food etc.
  Hence what the woman did here was going against the social norms.  And
  since women probably won't be able to afford such expensive ointment, this
  ointment was probably all she had.  Thus the lesson here is that faithful
  disciples show their commitment to Jesus despite the challenges
  faced.  Just like how the woman went against social norms to annoint Jesus,
  and gave all she had for Jesus, are we ready to go against social norms and
  give all we have for Jesus?}
  \item{\textbf{Criticism:} in verse 4, we see some men criticising the
  woman's gift as wasted.  During Passover, the Jews were obligated to give
  alms to the poor, and this was why they were thinking of the poor.  But
  what the men implicitly said was that Jesus was not worthy of this
  expensive gift.  Taken the actions of the men negatively, possibly these
  men could be trying to virtue signal, to let other people know how much
  they know about the poor.  Taking the actions of the men as positively as
  possible, in a sense, they put the commandment to `love your neighbour'
  above the commandment to `love God'.  In our context, we note that no
  matter what we do for God, we will always receive criticism of some kind.
  For example, if we give up our high paying job for full-time ministry,
  people might criticise us saying that our tithes could have done more for
  God.  If we stay in our high paying jobs instead of going into full-time,
  people might criticise us for being lovers of money.  We will always face
  criticism of some kind as long as we are faithful disciples of Jesus.}
  \item{\textbf{Commendation:} in verses 6-9, Jesus commends the woman's
  action as beautiful.  Jesus also re-orients the priority of the men there;
  His death was so imminent, whereas the poor would always exist.  Though as
  Christians we must try to alleviate poverty wherever we can, sometimes we
  must weigh the needs of the hour.  The woman here gave what she could, and
  she prepared his body for burial.  And hence, she will be remembered for
  her commitment, just as what we are doing now by remembering her actions.
  Thus, for us today, we must realise that though we may face criticism for
  our faithful service, Jesus will commend our faithful service at the end.
  }
  \item{Some reflections: here the disciples were considered the inner
  circle, but they didn't do what the woman did.  The disciples knew Jesus
  better than the woman, but yet the woman was the one who got it.  And when
  we compare the woman to the scribes and the Pharisees, in this sense, the
  woman was already in the kingdom of God.  For us, this shows that just
  because someone goes to church regularly, goes to seminary regularly etc,
  doesn't mean that he/she is committed to Christ.  Commitment is shown by
  what we do for Jesus, like the woman, not outward religiosity like the
  Pharisees.}
  \item{More reflections: furthermore, God is sovereign; He can make us of
  our faithful service for His kingdom, though we cannot see right now how
  our faithful service can be used.  It is just like the woman, she didn't
  know the effect of her breaking the alabastar jar for God, but God was
  always in control.}
  \item{In a sense also, Jesus is heaven's alabastar jar, broken for us for
  our healing.  He is the Son of God, infinitely precious, yet He took on
  human nature and died on the cross for our sins.  For us then, are we also
  prepared to be broken for Jesus?  I.e, are we ready to give our life to God
  just like how Jesus gave His life for us? }
\end{itemize}
  \section{3rd April 2022: Through the fire and the flames}
\subsection*{Text: Mark 14:27-50}
  \begin{quote}
    [27] And Jesus said to them, “You will all fall away, for it is written,
    ‘I will strike the shepherd, and the sheep will be scattered.’ [28] But
    after I am raised up, I will go before you to Galilee.” [29] Peter said
    to him, “Even though they all fall away, I will not.” [30] And Jesus said
    to him, “Truly, I tell you, this very night, before the rooster crows
    twice, you will deny me three times.” [31] But he said emphatically, “If
    I must die with you, I will not deny you.” And they all said the same.

    [32] And they went to a place called Gethsemane.  And he said to his
    disciples, “Sit here while I pray.” [33] And he took with him Peter and
    James and John, and began to be greatly distressed and troubled.  [34]
    And he said to them, “My soul is very sorrowful, even to death.  Remain
    here and watch.” [35] And going a little farther, he fell on the ground
    and prayed that, if it were possible, the hour might pass from him.  [36]
    And he said, “Abba, Father, all things are possible for you.  Remove this
    cup from me.  Yet not what I will, but what you will.” [37] And he came
    and found them sleeping, and he said to Peter, “Simon, are you asleep?
    Could you not watch one hour?  [38] Watch and pray that you may not enter
    into temptation.  The spirit indeed is willing, but the flesh is weak.”
    [39] And again he went away and prayed, saying the same words.  [40] And
    again he came and found them sleeping, for their eyes were very heavy,
    and they did not know what to answer him.  [41] And he came the third
    time and said to them, “Are you still sleeping and taking your rest?  It
    is enough; the hour has come.  The Son of Man is betrayed into the hands
    of sinners.  [42] Rise, let us be going; see, my betrayer is at hand.”

    [43] And immediately, while he was still speaking, Judas came, one of the
    twelve, and with him a crowd with swords and clubs, from the chief
    priests and the scribes and the elders.  [44] Now the betrayer had given
    them a sign, saying, “The one I will kiss is the man.  Seize him and lead
    him away under guard.” [45] And when he came, he went up to him at once
    and said, “Rabbi!” And he kissed him.  [46] And they laid hands on him
    and seized him.  [47] But one of those who stood by drew his sword and
    struck the servant of the high priest and cut off his ear.  [48] And
    Jesus said to them, “Have you come out as against a robber, with swords
    and clubs to capture me?  [49] Day after day I was with you in the temple
    teaching, and you did not seize me.  But let the Scriptures be
    fulfilled.” [50] And they all left him and fled.
  \end{quote}
\subsection*{Notes}
\begin{itemize}
  \item{Discipleship is not always easy.  Small example: saturday night is a
  prime time for fun and entertainment, which makes waking up in Sunday
  morning difficult.  We can't follow Jesus and be His disciples if we follow
  what our flesh desires.  What our flesh desires is usually contrary to
  following the Lord.}
  \item{True Christian discipleship is a renunciation of our self and a
  conformity to God.  This is expressed in Jesus' prayer in Gethsemane: ``Yet
  not what I will, but what you will''.  As Jesus thought about His upcoming
  betrayal and death, He was distressed.  This was not because Jesus was
  suddenly taken aback by this.  He has known from the start that this was
  His Father's will (e.g Mark 9).  The agony in Jesus' soul wasn't a result
  of an inner wavering, it was a very human reaction to the horrors that was
  to come.  Our Lord is completely human, and completely God (c.f
  Dyophysitism and Dyothelitism).}
  \item{Jesus could have just escaped under the cover of darkness, but He
  didn't.  In the face of this great agony and temptation, He prayed.  He
  knew that His Father could do all things, yet He prayed: ``Yet not what I
  will, but what you will''.  Since the Christian life is a life of following
  the pattern of Jesus, we too must deny ourselves and obey God's will.}
  \item{There are three ways in which we can renounce ourselves and conform
  to God.  The first way is to renounce an overconfidence in ourselves.  We
  must always humbly depend on God to keep us from falling away.  Jesus here
  quotes a prophecy from Zechariah 13 that prophesies the temporary falling
  away of Jesus' disciples.  But when Peter heard that, he over-confidently
  said that he won't deny Jesus.  Yet all of the disciples fled Jesus' side
  at verse 50.  As fallen humans, we are all susceptible to temptation.
  Though we know what we ought to do and what we ought not to do, we might
  still fall into temptation.  ``The Spirit is willing, but the flesh is
  weak''.  If we are overconfident in our ability to follow God, we will
  fall.  The paradox is this: we must realise we are weak, which would push
  us to depend on God, which would then keep us from temptation.  As Jesus
  said: ``watch and pray, that you may not enter into temptation''.  When we
  watch, we are watchful for the external circumstances that we know we are
  weak to.  And when we pray, we are depending on divine assistance against
  temptation (c.f Psalm 121).  Discipleship means a humble dependance on
  God's grace, so we may be conformed to God.  The disciples slept and hence
  they fell; on the other hand, Jesus watched and prayed and depended fully
  on God, hence Jesus overcame temptation.}
  \item{Discipleship also means renouncing false expectations and accept the
  cross.  When it comes to following our Lord, expectations are important,
  because false expectations hinder us from discipleship.  One reason the
  disciples deserted Jesus was because they had a certain expectation of what
  Jesus would do.  When the mob first came, they didn't flee yet; they were
  prepared to defend Jesus, with one of them even drawing his sword.  They
  only fled after Jesus said: ``let the Scriptures be fulfilled''.  The
  disciples did not see that coming.  They were expecting Jesus to establish
  an earthly political kingdom as the messiah.  They were expecting Jesus to
  start a revolution to overthrow the Romans and the Herodian dynasty.  This
  is why when we read the earlier chapters, we see James and John the son of
  Zebedee asking to be on Jesus right hand when the kingdom comes.  When
  Jesus gave Himself over, the disciples all fled because their false
  expectations crumbled.  For us, when we also have false expectations about
  Jesus, and usually these false expectations appeal to our flesh such as the
  prosperity gospel, these false expectations give people a mistaken idea of
  Christianity.  And hence when these false expectations crumble, people will
  also lose their ``faith''.  The real thing about Christianity is to follow
  Jesus to be crucified; i.e while Jesus was literally crucified, we must
  crucify the desires of our flesh which may or may not include matyrdom.  We
  must follow our Lord through the fire and flames of temptation.  Our sinful
  self must die, and our sinful flesh must be crucified.  And our sinful self
  must be replaced by an obedience of faith.  Yet paradoxically, it is only
  though this struggle that we have freedom and peace.  E.g, the struggle to
  quit smoking is tough, yet the final freedom from addiction is truly
  liberating.  Similarly, we must struggle with sin and temptation through
  God's Spirit, and only when we do so, then we are truly free.}
  \item{Discipleship also means renouncing the world.  The disciples were
  ready to defend Jesus to establish a worldly kingdom, yet they all fled
  when Jesus gave himself up to establish a spiritual kingdom.  The way of
  the world is to fight, trample, manipulate and dominate for the sake of
  gain and power.  This is not anything new; this has been around since the
  days of Cain and Abel.  On the other hand, we have Jesus, who was ready to
  give Himself up for the world.  When we were still sinners, Christ died for
  us.  The way of God, as demonstrated by Christ is love and self-sacrifice.
  This is clearly in contradistinction to the way of the world.  If Jesus'
  kingdom were of the world, we would all be taking up arms and manipulating
  people etc for Jesus' kingdom.  Yet since Jesus' kingdom is not of the
  world, the way to bring about Jesus' kingdom is through love, just like
  what Jesus did for the world.  A real life example would be the AWARE (an
  NGO) saga, where a group of Christian women took over the NGO's leadership
  through subterfuge, because they were concered about AWARE's pro-LGBTQ
  stance.  The concerns of the Christians involved were valid, but to resort
  to the ways of the world is to damage our Christian witness.  Jesus'
  kingdom is not of the world, and if we rely on worldly means such as taking
  up the sword, we damage the message of the Kingdom of God, which is a
  message of self-sacrificial love.  }
  \item{Discipleship isn't easy, but our Lord is with us.  Following Him
  means we must renounce ourselves and conform to God, which isn't easy.
  Since we aren't perfectly obedient, we might find ourselves falling away
  from time to time.  But because our Lord is with us, He will pick us up
  when we fall.  In verse 28, after predicting the disciples' abandonment,
  Jesus said that He will be raised up and go ahead of them to meet Him in
  Galilee.  Mark's gospel doesn't really tell us what happens in Galilee, but
  from the other gospels, we see that Jesus commissioned the disciples in
  Galilee.  Right from the start, Jesus was ready to forgive the disciples
  for their failing.  As is written in 1 John 1:9, if we confess our sins,
  God is faithful and just to forgive us our sins and to cleanse us from all
  unrighteousness.  }
\end{itemize}
  \section{10 April 2022: Blessed is He who comes in the name of the Lord}
\subsection*{Text: Mark 11:1-11}
  \begin{quote}
    [1] Now when they drew near to Jerusalem, to Bethphage and Bethany, at
    the Mount of Olives, Jesus sent two of his disciples [2] and said to
    them, “Go into the village in front of you, and immediately as you enter
    it you will find a colt tied, on which no one has ever sat.  Untie it and
    bring it.  [3] If anyone says to you, ‘Why are you doing this?’ say, ‘The
    Lord has need of it and will send it back here immediately.’” [4] And
    they went away and found a colt tied at a door outside in the street, and
    they untied it.  [5] And some of those standing there said to them, “What
    are you doing, untying the colt?” [6] And they told them what Jesus had
    said, and they let them go.  [7] And they brought the colt to Jesus and
    threw their cloaks on it, and he sat on it.  [8] And many spread their
    cloaks on the road, and others spread leafy branches that they had cut
    from the fields.  [9] And those who went before and those who followed
    were shouting, “Hosanna!  Blessed is he who comes in the name of the
    Lord!  [10] Blessed is the coming kingdom of our father David!  Hosanna
    in the highest!”

    [11] And he entered Jerusalem and went into the temple.  And when he had
    looked around at everything, as it was already late, he went out to
    Bethany with the twelve.
  \end{quote}
\subsection*{Notes}
\begin{itemize}
  \item{What Jesus did here was quite different from His previous actions;
  previously He wanted to keep his actions secret, yet in this text He was
  very public.  In the past, Jesus didn't want people to think He was a
  revolutionary who would start a coup, hence His secrecy.  But from this
  portion onwards, Jesus openly confronts the religious leaders in Jerusalem
  for their hypocrisy.  So why would Jesus publicly enter Jerusalem on a
  donkey?  Three points:
  \begin{itemize}
    \item{It is symbolic}
    \item{It is simple(?)}
    \item{It is for suspense}
  \end{itemize}}
  \item{Symbolic: Jesus gives very clear and specific instructions to two of
  His disciples, such as where to find the colt, what to say if they were
  questioned, etc.  In the response `the Lord has need of it', it can either
  mean that the master of the colt needs it, or God needs it.  In verse 2, we
  also see that the colt must not have been sat on before.  This is consonant
  with the OT laws that animals that are used for sacred purposes must not
  have been used before for other stuff.  Here we see that everything Jesus
  does here is intentional.  All his actions are symbols and signs that
  fulfill Old Testament prophecies (e.g Zechariah 9:9).  It shows to the
  people that God remembers and keep His promises about a coming messiah king
  that would come to save His people.  In Zechariah 9:9, we see that the
  coming king will be humble, and this is what exactly Jesus is.  Though
  Jesus is the King of Kings and the Lord of Lords, he entered the city on a
  donkey and on a makeshift saddle.  And looking forward, Jesus will not
  fully display His glory until He goes humbly to the cross.  Also, the
  people's response is from Psalm 118:25-28.  Hosanna used to mean `save us'
  but by that time is also used as a term of worship.  The people's response
  showed that they understood that Jesus was King and savior, though they
  also misunderstoof what Jesus would do as King.}
  \item{Simple: Though His disciples still didn't fully understand what Jesus
  was about to do, they still had faith in Jesus and trusted in Him.  They
  did whatever Jesus told them to do, following His instructions to a T.
  They did not know what the actions would lead to, but their simple faith
  and obedience fulfilled God's purposes.  They recognised that Jesus is the
  messiah king.  Though the disciples are like us, frail and weak, Jesus
  still used them for His purposes.  And from the disciples' POV, through
  their simple faith, they helped to bring about God's plan.}
  \item{Suspense: Will Jesus find His house in order?  From v11, we see that
  the first thing that Jesus did was to enter the temple.  And then
  subsequently, Jesus pronounced judgment on the temple, which similar to OT,
  was an indictment against the religious leaders.}
\end{itemize}

  \section{17th April 2022: He has risen!}
\subsection*{Text: Mark 16:1-8}
  \begin{quote}
    [1] When the Sabbath was past, Mary Magdalene, Mary the mother of James, and Salome bought spices, so that they might go and anoint him. [2] And very early on the first day of the week, when the sun had risen, they went to the tomb. [3] And they were saying to one another, “Who will roll away the stone for us from the entrance of the tomb?” [4] And looking up, they saw that the stone had been rolled back—it was very large. [5] And entering the tomb, they saw a young man sitting on the right side, dressed in a white robe, and they were alarmed. [6] And he said to them, “Do not be alarmed. You seek Jesus of Nazareth, who was crucified. He has risen; he is not here. See the place where they laid him. [7] But go, tell his disciples and Peter that he is going before you to Galilee. There you will see him, just as he told you.” [8] And they went out and fled from the tomb, for trembling and astonishment had seized them, and they said nothing to anyone, for they were afraid.
  \end{quote}
\subsection*{Notes}
\begin{itemize}
  \item{Three points: Certainty of Jesus' death, Certainty of Jesus'
  resurrection, and commisioning of Jesus' disciples.}
  \item{Certainty of Jesus' death: There were three women who knew where
  Jesus were buried.  They brought spices to the tomb because they were
  hoping to annoint Jesus' body.  THe purpose of this was to mask the odor of
  decaying flesh with spices and oil.  They waited until after the Sabbath
  was over to do this.  But as they came, they realised that since Jesus'
  tomb was sealed with a giant stone, someone needed to roll it away for
  them.  Jesus' death was no accident; it was Jesus intentionally giving away
  His life to save everyone.  The timing and manner in which Jesus died was
  in accordance with God's sovereign plan, as it has been attested to
  centuries prior in the Prophets.}
  \item{From 1 John 4:9-10, we see that Jesus died because of Man's sins, but
  also because of God's love.  Death is not natural, death is a result of
  sin.  God is the source of life, and if we turn away from life, the only
  consequence is death.  Jesus didn't come just to be a moral example, He
  came to die, and He came because He loves.  One of the characteristic of
  love is to seek the well-being of others, sometimes self-sacrifically.  God
  gave His only Son to take on human flesh and to die on the cross for our
  sins.  Only Jesus could be the perfect substitute as an atoning sacrifice,
  because of the worthiness of Jesus' person, since Jesus is God's Son.}
  \item{Sin is always costly, instead of downplaying our sins etc, let us
  turn to God in repentance and ask God for forgiveness, so that instead of
  sin, our heart will by gripped by love, a love that will help us walk in
  God's ways.}
  \item{Certainty of Jesus' resurrection: When the women reached the tomb,
  they saw that the rock was removed.  This would have been surprising and
  alarming for them, because if the stone is removed, what about the body
  inside?  Instead of a body, the women saw a white (angelic) figure who told
  them that Jesus is risen, and that they will see Jesus in Galilee.
  Thereafter, the women told the disciples of Jesus, and then the disciples
  also saw the risen Christ, and then the disciples proclaimed the message of
  Jesus' death and resurrection to the whole world.  Now, if this claim was a
  hoax, it would be very easy for the enemies of Jesus to just produce a
  body.  It is also highly unlikely for the disciples to die for a lie.
  Lastly, society in those days was very dismissive of the testimony of
  women; for the writer of Mark's gospel to say that it was the women who
  first saw the risen Christ, this is a mark of authenticity.  If this
  account was made up, Mark would have chosen to use men instead to give
  himself more credibility.}
  \item{Now, if Jesus had not been risen, at most His death could be
  considered heroic.  But His death would still be in vain.  Paul puts it
  nicely in 1 Corinthians 15:17-22.  But Jesus is really risen.  And Jesus'
  resurrection assures us that God has forgiven us of our sins.  Jesus'
  resurrection also assures us that God will free us from sin's curse.  Just
  like Jesus, one day all of us will be given a new body free of pains and
  suffering and death.  Jesus' resurrection shows that God has the power to
  destroy suffering and death.  Apart from God's power in Jesus'
  resurrection, there is nothing else that can end the suffering and death in
  this world. }
  \item{When Jesus rose from the dead, He ushered in a new Kingdom that is
  slowly growing, and as the Kingdom grows, the world will be slowly
  transformed.  As Christians we can all testify to the freedom from the
  bondage of sin.  Though there is still conflict and suffering in the world,
  we know the final end of the world; Jesus will keep His promise and come
  again.  The future renewal of the world does not depend on things like how
  much power/influence us Christians have, but it depends on Jesus' promise;
  and only an alive person can keep promises.}
  \item{Comissioning of Jesus' disciples: After the women met the angelic
  figure, they ran out, for trembling and astonishment has gripped them.
  Their world has been turned upside down; they came to annoint Jesus' dead
  body, but then now they learn that Jesus is alive.  The fact that God chose
  the women to tell the men of Jesus resurrection shows that God uses both
  men and women in His mission.  Despite our failings as men and women, God
  chooses to use both men and women for Gospel proclamation.  The inclusion
  of women in a male-dominated society was really revolutionary.  Now, Mark
  chapter 16 ends at verse 20, though there are a few early manuscripts that
  do not have verses 9-20.  Yet the account in those verses are credible, for
  they are testified to in other gospels.}
  \item{Now, suppose that Mark really ended at verse 8.  What could be Mark's
  point?  The point could be that Mark is trying to say that despite the
  failing of the men (the disciples deserted Jesus) and the failing of the
  women (they were too afraid and didn't tell anyone), God still uses these
  men and women for His own purposes, and that those failures will eventually
  be redeemed as long as the people continue to trust in God.  And in
  history, we do know that those failures are redeemed, for in the other
  gospels, it is testified that the women did tell the disciples.  For us
  today, the possible takeaway for us is that despite our fears, God enables
  both men and women to be faithful in Gospel proclamation.  We can be
  faithful despite our fears because Christ has risen, and that all the
  forces of evil have been decisively defeated.}
  \item{If Jesus really died and rose again, then nothing can stop Him from
  showing forth His glory through us Christians.  Just as Jesus has risen, we
  too will share in that power, which enables us to be faithful, and gives us
  the hope that at the end, all will be made right.}



\end{itemize}
  \section{24th April 2022: Is it all meaningless?}
\subsection*{Text: Ecclesiastes 1:1-11}
  \begin{quote}
    [1] The words of the Preacher, the son of David, king in Jerusalem.

    [2] Vanity of vanities, says the Preacher,
        vanity of vanities! All is vanity.
    [3] What does man gain by all the toil
        at which he toils under the sun?
    [4] A generation goes, and a generation comes,
        but the earth remains forever.
    [5] The sun rises, and the sun goes down,
        and hastens to the place where it rises.
    [6] The wind blows to the south
        and goes around to the north;
    around and around goes the wind,
        and on its circuits the wind returns.
    [7] All streams run to the sea,
        but the sea is not full;
    to the place where the streams flow,
        there they flow again.
    [8] All things are full of weariness;
        a man cannot utter it;
    the eye is not satisfied with seeing,
        nor the ear filled with hearing.
    [9] What has been is what will be,
        and what has been done is what will be done,
        and there is nothing new under the sun.
    [10] Is there a thing of which it is said,
        “See, this is new”?
    It has been already
        in the ages before us.
    [11] There is no remembrance of former things,
        nor will there be any remembrance
    of later things yet to be
        among those who come after.
  \end{quote}
\subsection*{Notes}
\begin{itemize}
  \item{Life sometimes feels like a chore.  E.g endless household chores to
  do; we vacuum and mop the floor, yet tomorrow it becomes dirty again.
  There seems to be a tiresome cycle to life, we just do the same toilsome
  things over and over again.  After a while, we ask: what is life all about?
  Are things all meaningless?}
  \item{Ecclesiastes belongs to the genre of wisdom literature, together with
  Proverbs, Job, and Song of Songs.  Wisdom literature wrestles with the
  question of how to live wisely amid the many challenges of life.
  Ecclesiastes opens with a prologue by the `Narrator' (1:1 to 1:11), and
  then the `Preacher' speaks (1:12 to 12:7), and then there is an epilogue by
  the `Narrator' again (12:8 to 12:14).  Hence, as for the structure of the
  book, we can think of it as the words of the `Preacher' sandwiched in
  between the prologue and the epilogue.  The English name of the book,
  Ecclesiastes, is a transliteration of the Greek (ekklesiastes), but in the
  Hebrew the name is `Qoheleth', which literally means "the one who
  assembles".}
  \item{Who is the `Preacher'?  The `Preacher' seems to be speaking from the
  point of view of Solomon, though some modern bible scholars now think that
  Ecclesiastes is a fictional autobiography written at a later time by
  someone writing in the Solomonic tradition.  Most likely, Ecclesiastes was
  written during the post-exilic period, when Israel was still under foreign
  rule and when there was much social injustice.  While there are many
  indications that the `Preacher' could be Solomon, there are also many
  indications that the `Preacher' is not; for example, Ecclesiastes 1:16a
  would not make sense if the `Preacher' is Solomon, since the only king
  before Solomon was David.}
  \item{OT books like Proverbs and the OT Law describe the action-consequence
  formula in detail.  The idea is: do good things and follow God, and then
  you'll be blessed.  Do bad things and disobey God, and then you'll be
  cursed.  But in the post-exilic period, the action-consequence principle
  doesn't seem to hold anymore.  The wicked prosper, and the righteous
  suffer.  Hence, the Preacher was writing from the POV of someone who was
  frustrated with how the action-consequence principle doesn't seem to apply
  in his day.}
  \item{Four main points for today, all starting with M.  The first M is
  `meaningless'.  `Hebel/Hevel' in Hebrew literally means breath or vapour,
  and in English, that idiom is translated as `vanity' or `meaninglessness'.
  There are things in life that make life seem utterly meaningless.  How did
  the Preacher come to this conclusion?  Firstly, it seems that there was
  nothing to be gained from all the toil under the sun.  Like vapor, the
  things we do in our life has no permanent impact, it makes no lasting
  impression.}
  \item{The next M is monotony.  To support his argument, the Preacher uses
  arguments from the natural world (v4-7), and from human experiences
  (v8-10).  From the arguments from the natural world, we see that one
  generation comes and another generation goes, but the cycles that happen in
  nature just keep repeating itself monotonously, there doesn't seem to be
  any real progress in the things in nature.  From the arguments from human
  experiences, we see that nothing really satisfies us.  Everyday we see an
  endless procession of internet images, from Netflix, Disney+, etc etc, but
  after all our looking and listening, our eyes are not satisfied.  We still
  want to see some more and to hear some more.  There is always one more game
  to watch, one more show to watch, one more song to listen to.  The
  circularity of nature above is paralleled by the repetitiveness of history;
  there is nothing new under the sun.  What has been is what will be, what
  has been done is what will be done.  E.g, war has been a big part of human
  history, and it will continue to be a big part of human history.  While the
  weapons of war are different, the effects are the same; people die, homes
  are destroyed.  While some might say that the cycles in Nature are
  beautiful and even essential, we must empathise with the Preacher as he
  reflects on how small he is and how everything he does seems to be
  inconsequential.}
  \item{The next M is mortality (v11).  If the above two points are not
  depressing enough, the climax of the Preacher's argument is this; one day
  we all have to die, and no one remembers us.  We may be crowned the Olympic
  champion today, but four years later we don't make the cut, and people
  don't remember us.  What more after we die?}
  \item{There are two ways of understanding Ecclesiastes:
  \begin{itemize}
    \item{Approach 1: If God is not in our life, life will be meaningless;
    only when we bring God into the picture will life be meaningful.  Thus,
    the goal of the Preacher is to drown us in pessimism, so that we will
    gasp for air and realise our need for God.}
    \item{Approach 2: Even when God is in our lives, life can still feel
    meaningless.  The Preacher then wants to teach us the following lesson:
    How do we then resolve the tension between what our faith teaches us
    (that life ought to be meaningful), and what we observe and experience in
    our lives that seems to point in the opposite direction?}
  \end{itemize}
  The approach for this sermon series on Ecclesiastes will be to take the
  second approach, as the Pastoral team thinks that the second message is
  more faithful to the text.  The message that is repeated through
  Ecclesiastes (6 times) is the verse in Ecclesiastes 2:24.  Even when life
  seems to be meaningless, when life doesn't make sense, when God seems to be
  absent, we should still fear God.  We should still be joyful and do good.
  We can still eat and drink and be thankful for God's provision even in our
  toil.  As Christians, we can identify with what the Preacher is saying.  In
  Romans 8:20, we see that the creation was subjected to futility.  This is a
  result of sin.  But Christ, in His incarnation, entered into our creation
  which is subjected to futility and took on the full brunt of that futility
  which is a result of us being separated from God through sin.  Hence,
  through Christ's resurrection, we can experience the full meaning of life
  as we are reconciled with God.  But insofar as we are still in the flesh,
  we are `groaning' with the entire creation (also Romans 8) as we await the
  redemption of our bodies which will only happen with Christ comes again.}
  \item{In the end, we have to struggle with that tension between what our
  faith teaches us (that life is meaningful), and what we experience
  sometimes (that life is meaningless).  One way to think about it is that
  when life seems meaningless, we should not conclude that there is no
  meaning to life, but we should conclude that the meaning is currently
  incomprehensible to us.  And while we struggle with the incomprehensibility
  of certain things that happen in life - the apparent meaninglessness,
  monotony and our mortality - we must still acknowledge that God is very
  much present and that we can trust that in all things God works for the
  good of those who love him, and he will make everything beautiful in its
  time.  And as we trust God, we obey Him and fear Him and keep His
  commandments in faith, for that is our duty.  We have the hope that God
  will bring every deed into judgment, with every secret thing, whether good
  or evil.  That is the message of Ecclesiastes, and that is the last M for
  today.}
\end{itemize}
  \section{1st May 2022: Where can we find fulfilment?}
\subsection*{Text: Ecclesiastes 1:12-2:26}
  \begin{quote}
    [12] I the Preacher have been king over Israel in Jerusalem.  [13] And I
    applied my heart to seek and to search out by wisdom all that is done
    under heaven.  It is an unhappy business that God has given to the
    children of man to be busy with.  [14] I have seen everything that is
    done under the sun, and behold, all is vanity and a striving after wind.

    [15] What is crooked cannot be made straight, and what is lacking cannot
    be counted.


    [16] I said in my heart, “I have acquired great wisdom, surpassing all
    who were over Jerusalem before me, and my heart has had great experience
    of wisdom and knowledge.” [17] And I applied my heart to know wisdom and
    to know madness and folly.  I perceived that this also is but a striving
    after wind.

    [18] For in much wisdom is much vexation, and he who increases knowledge
    increases sorrow.


    [1] I said in my heart, “Come now, I will test you with pleasure; enjoy
    yourself.” But behold, this also was vanity.  [2] I said of laughter, “It
    is mad,” and of pleasure, “What use is it?” [3] I searched with my heart
    how to cheer my body with wine—my heart still guiding me with wisdom—and
    how to lay hold on folly, till I might see what was good for the children
    of man to do under heaven during the few days of their life.  [4] I made
    great works.  I built houses and planted vineyards for myself.  [5] I
    made myself gardens and parks, and planted in them all kinds of fruit
    trees.  [6] I made myself pools from which to water the forest of growing
    trees.  [7] I bought male and female slaves, and had slaves who were born
    in my house.  I had also great possessions of herds and flocks, more than
    any who had been before me in Jerusalem.  [8] I also gathered for myself
    silver and gold and the treasure of kings and provinces.  I got singers,
    both men and women, and many concubines, the delight of the sons of man.

    [9] So I became great and surpassed all who were before me in Jerusalem.
    Also my wisdom remained with me.  [10] And whatever my eyes desired I did
    not keep from them.  I kept my heart from no pleasure, for my heart found
    pleasure in all my toil, and this was my reward for all my toil.  [11]
    Then I considered all that my hands had done and the toil I had expended
    in doing it, and behold, all was vanity and a striving after wind, and
    there was nothing to be gained under the sun.

    [12] So I turned to consider wisdom and madness and folly.  For what can
    the man do who comes after the king?  Only what has already been done.
    [13] Then I saw that there is more gain in wisdom than in folly, as there
    is more gain in light than in darkness.  [14] The wise person has his
    eyes in his head, but the fool walks in darkness.  And yet I perceived
    that the same event happens to all of them.  [15] Then I said in my
    heart, “What happens to the fool will happen to me also.  Why then have I
    been so very wise?” And I said in my heart that this also is vanity.
    [16] For of the wise as of the fool there is no enduring remembrance,
    seeing that in the days to come all will have been long forgotten.  How
    the wise dies just like the fool!  [17] So I hated life, because what is
    done under the sun was grievous to me, for all is vanity and a striving
    after wind.

    [18] I hated all my toil in which I toil under the sun, seeing that I
    must leave it to the man who will come after me, [19] and who knows
    whether he will be wise or a fool?  Yet he will be master of all for
    which I toiled and used my wisdom under the sun.  This also is vanity.
    [20] So I turned about and gave my heart up to despair over all the toil
    of my labors under the sun, [21] because sometimes a person who has
    toiled with wisdom and knowledge and skill must leave everything to be
    enjoyed by someone who did not toil for it.  This also is vanity and a
    great evil.  [22] What has a man from all the toil and striving of heart
    with which he toils beneath the sun?  [23] For all his days are full of
    sorrow, and his work is a vexation.  Even in the night his heart does not
    rest.  This also is vanity.

    [24] There is nothing better for a person than that he should eat and
    drink and find enjoyment in his toil.  This also, I saw, is from the hand
    of God, [25] for apart from him who can eat or who can have enjoyment?
    [26] For to the one who pleases him God has given wisdom and knowledge
    and joy, but to the sinner he has given the business of gathering and
    collecting, only to give to one who pleases God.  This also is vanity and
    a striving after wind.
  \end{quote}
\subsection*{Notes}
\begin{itemize}
  \item{Recap: Today's passage seems to point to Solomon as the `preacher'.
  Until the 16th century, both Jewish and Christian interpreters thought that
  Solomon was the author of Ecclesiastes.  The rejection of Solomonic
  authorship is due to linguistic factors etc.  Those who reject Solomonic
  authorship put Ecclesiastes in the time of the return from exile.  In
  Ecclesiastes, there are two voices, that of the narrator and that of the
  Preacher.  The narrator speaks in the prologue and the epilogue, and then
  in the middle we have the voice of the Preacher.  Here, the Preacher is
  speaking from the POV of Solomon.  The goal of the book of Ecclesiastes is:
  what is the meaning of life?}
  \item{Today's sermon point: what will it take for us to find fulfilment?
  What if we were the richest in Singapore, would that make us happy?  Etc.
  If even Solomon, the person with the most resources in the history of the
  world, couldn't find fulfilment through worldly means, what would that mean
  for us?}
  \item{In Eccl 1:12-18, we see the Preacher, in the voice of Solomon,
  telling us about the research he has done to try to find fulfilment.  In
  the Preacher's academic quest to find out what gives fulfilment, he left no
  stone unturned; he even checked if folly can give fulfilment.  In the end,
  he concluded that `it is an unhappy business that God has given to the
  children of men to be busy with', where `unhappy business' refers back to
  the curse of sin that God placed on creation (Gen 3:14-19).  In
  Eccl 1:16-18, we see that no matter how wise we are, because of sin in the
  world, there will always be things that we can't know and things we can't
  fix.  `What is crooked cannot be made straight, and what is lacking cannot
  be counted'.  In the end, much knowledge increases sorrow, since with more
  knowledge comes the growing realisation that knowledge cannot fix
  everything in this world.  Application: we can't find our meaning in life
  through academic means, through signing up for whatever enrichment courses,
  learning from all sorts of gurus, etc.}
  \item{In Eccl 2:1-17, we see Solomon (the Preacher's persona) continuing
  his quest to try to find fulfilment testing if pleasure is the source of
  fulfilment.  Here, the Preacher tries amusment, stimulation with wine,
  finishing many building projects for himself, possessions, collections,
  entertainment, sex, etc.  His conclusion was that all of those means to try
  to find fulfilment were futility.  If we aren't chasing God, we are merely
  chasing the wind.  In fact there is some sort of irony; the harder we try
  to find fulfilment, the emptier we end up.  If we are living in circles, we
  need to turn our lives over to Jesus, the wisdom and power of God.}
  \item{In Eccl 2:18-26, we see that no matter how hard we work to accumulate
  wealth and posessions, we cannot bring them with us to the grave.  The
  lesson here is paralleled with the parable of the rich fool in Luke 12.
  Some people amass great fortune not for their benefit, but for their
  children, but more often than not such great fortunes are very easily
  squandered.  Moreover, such great fortunes for the kids also destorys
  relationships.  Ultimately, hard work is not an end to itself, posessions
  aren't an end to itself, but they are just things that point to God. }
  \item{In the end, in view of eternity, our life on earth is super short in
  comparison.  I.e, if eternity is an infinite line, then our time here is a
  dot (a set of measure zero).  Hence, we should live our lifes with eternity
  in view.  Things on this earth are just meant to point us to eternity.
  True happiness, true joy, is found only in God.  If we treat the things in
  life as an ends in themselves, we will never find fulfilment.  To dethrone
  God is to lose the key to life.  We may pursue many human endeavors, but
  all we will find is vanity, the lack of ultimate fulfilment.  But if we
  treat the things in life as things that point to God, we will find true
  fulfilment in God.  If we enthrone God in our lives, we will enter into the
  fullness of life (John 17:3, Psalm 16:11).  }
  \item{My reflections: in life, we have enough good to point us to God in
  thanksgiving, but also some brokenness due to sin that reminds us that
  these good things depend wholly on God and that these good things aren't an
  ends to themselves.}
\end{itemize}
  \section{8th May 2022: Who really is in control?}
\subsection*{Text: Ecclesiastes 3:1-4:3}
  \begin{quote}
    [1] For everything there is a season, and a time for every matter under
    heaven:

    [2] a time to be born, and a time to die;
    a time to plant, and a time to pluck up what is planted;
    [3] a time to kill, and a time to heal;
    a time to break down, and a time to build up;
    [4] a time to weep, and a time to laugh;
    a time to mourn, and a time to dance;
    [5] a time to cast away stones, and a time to gather stones together;
    a time to embrace, and a time to refrain from embracing;
    [6] a time to seek, and a time to lose;
    a time to keep, and a time to cast away;
    [7] a time to tear, and a time to sew;
    a time to keep silence, and a time to speak;
    [8] a time to love, and a time to hate;
    a time for war, and a time for peace.


    [9] What gain has the worker from his toil?  [10] I have seen the
    business that God has given to the children of man to be busy with.  [11]
    He has made everything beautiful in its time.  Also, he has put eternity
    into man’s heart, yet so that he cannot find out what God has done from
    the beginning to the end.  [12] I perceived that there is nothing better
    for them than to be joyful and to do good as long as they live; [13] also
    that everyone should eat and drink and take pleasure in all his toil—this
    is God’s gift to man.

    [14] I perceived that whatever God does endures forever; nothing can be
    added to it, nor anything taken from it.  God has done it, so that people
    fear before him.  [15] That which is, already has been; that which is to
    be, already has been; and God seeks what has been driven away.

    [16] Moreover, I saw under the sun that in the place of justice, even
    there was wickedness, and in the place of righteousness, even there was
    wickedness.  [17] I said in my heart, God will judge the righteous and
    the wicked, for there is a time for every matter and for every work.
    [18] I said in my heart with regard to the children of man that God is
    testing them that they may see that they themselves are but beasts.  [19]
    For what happens to the children of man and what happens to the beasts is
    the same; as one dies, so dies the other.  They all have the same breath,
    and man has no advantage over the beasts, for all is vanity.  [20] All go
    to one place.  All are from the dust, and to dust all return.  [21] Who
    knows whether the spirit of man goes upward and the spirit of the beast
    goes down into the earth?  [22] So I saw that there is nothing better
    than that a man should rejoice in his work, for that is his lot.  Who can
    bring him to see what will be after him?

    [1] Again I saw all the oppressions that are done under the sun.  And
    behold, the tears of the oppressed, and they had no one to comfort them!
    On the side of their oppressors there was power, and there was no one to
    comfort them.  [2] And I thought the dead who are already dead more
    fortunate than the living who are still alive.  [3] But better than both
    is he who has not yet been and has not seen the evil deeds that are done
    under the sun.
  \end{quote}
\subsection*{Notes}
\begin{itemize}
  \item{Three points that relate to the question: ``Who really is in
  control?'' Modern man likes to think that he is in control, but we need to
  be disabused of that notion.  Three points that show us why we aren't
  really in control:
  \begin{itemize}
    \item{Time: we don't control time, time controls us.}
    \item{Transcience: we may think highly of ourselves, but we are temporary.}
    \item{Theology: God and the teaching of God's word.  What is there in
    theology that we can use to fight the ravages of time and also our
    transcience?}
  \end{itemize}}
  \item{Verse 1-8 is a nice poem, usually it is taken in a positive light to
  to say: ``bad things only last for a season, there will be good that come
  after the bad''.  While the poem is certainly beautiful, the author of
  Ecclesiastes means it here in a negative light; see verse 9 and verse 15.
  Especially verse 15, it seems to imply that God is the one that seeks to
  cycle through what has been done before.  Verse 1-8 can also be taken
  negatively; when we want to dance, sometime happens beyond our control that
  causes us to mourn.  When we want to laugh, something happens beyond our
  control that causes us to cry.  The author seems to say that we are in some
  sort of God ordained cycle of good-and-bad that we have no control over.
  The cycles of life also might cause people to be sick of the tedium.  In
  other religions, this cycle of life is also understood from a negative
  perspective.  In Buddhism and Hinduism, we are stuck in the karmic cycle of
  life-death-rebirth, and the goal is to escape this cycle altogether.}
  \item{We can't control time, and that is made worse by the fact that our
  life is short, or transient.  We might have thought that: ``ok I can't
  control time, but if I can live for longer, I might have some limited
  control''.  God says `nope'.  ``Dust we are, and dust we shall return''.
  This is something that occurs to all, rich/poor, popular/unpopular,
  beautiful/ugly.  And ironically, even though we are all transient, God has
  put eternity in our hearts, so that we all (whether we are religious or
  not) aspire to escape the transcience.  Richard Dawkins, well-known
  materialist/atheist, once told his granddaughter: ``when I'm gone, look up
  at the stars, and remember grandpa''.  For even an atheist like Dawkins,
  even he wants to be remembered past his death; he can't handle the
  ``truth'' that he propounds that we are all cosmic accidents that don't
  matter.  In spite of all our human attempts to overcome transcience, we
  humans can't do that, yet we all naturally want to, since God has put
  eternity in our hearts.  How frustrating is that!  Eternity being put in
  our hearts also means that we have this yearning for the big picture, we
  want to find out how our life fits into this big picture.  Sadly, we can't
  see the big picture as limited human beings, the author of Ecclesiastes
  says: ``yet so that he cannot find out what God has done from the beginning
  to the end''.  How frustrating is that!  We are not in control of how
  people remember us after we are gone, we are not in control of the impact
  that we have on the big picture, etc.  Is God trolling us?  We are all
  proud people, and perhaps this is a way for God to humble us.}
  \item{How do we now face the problems of time and transcience that the
  author of Ecclesiastes has raised?  Note that Ecclesiastes has not resolved
  the answer fully, but has only dropped hints.  The final answer/resolution
  is found only in the gospel.  What are some hints that the author of
  Ecclesiastes has dropped?
  \begin{itemize}
    \item{Time is cyclic, but from God's perspective, time actually has a
    ``telos'', a destination that he wants to direct the flow of time to.
    Time is tedious, but God has made everything beautiful in its time.  The
    solution to the tedium of time is to put God in the equation; once we
    realise that God is at work mysteriously, and that God is with us, then
    every mundane/conventional moment can be a moment of beauty.  E.g for a
    housewife who does nothing but housework daily, it might seem really
    tedious, but with God in her heart, because she knows God is working in
    every moment, she can enjoy every moment.  E.g one person might want to
    celebrate, but God has stricken him with a disease outside of his
    control.  What can he do?  With God with him, he can still celebrate, but
    in a different way; once he knows that all this is part of God's plan for
    His purposes, he can work with God's plan joyfully to proclaim God's
    glory even in his sickness.  These are some of the hints that the author
    of Ecclesiastes has dropped for us to answer the tyranny of time.}
    \item{The question to the transcience of time has an answer.  Richard
    Dawkins and all human beings generally want to be remembered.  The answer
    to this is not in this particular text, but in the later parts of
    Ecclesiastes, we see that there is a final judgement and hence a final
    resurrection.  We are not really transient; while people might forget us,
    God will not, and in fact the people who forget us will get to know us
    when we are all resurrected.  There is an eternity for us after our
    death.  The resurrection and the final judgement is also a hint to the
    problem of us not being able to see the big picture; one day we will be
    able to see the big picture, when we are resurrected.  }
    \item{Now, if we aren't really transient, if we know that the big picture
    exists (though we can't see it), if we know that God ultimately directs
    time to a beautiful end (though we can't see it), we know that our work
    is not meaningless but is ultimately meaningful.  Hence, we can go about
    our work joyfully today, even though we can't see the meaning today,
    because we know that there is an ultimate meaning that we are
    participating in.  We can be joyful in all circumstances, in times of
    rejoicing and also in times of mourning, because we know that things are
    in the hands of God.  Things are in the hands of God, who works all
    things for our good (Romans 8:28). }
  \end{itemize}}
\end{itemize}
  \section{15th May 2022: Right relationships with others and with God}
\subsection*{Text: Ecclesiastes 4:4-5:7}
  \begin{quote}
    [4] Then I saw that all toil and all skill in work come from a man’s envy
    of his neighbor.  This also is vanity and a striving after wind.

    [5] The fool folds his hands and eats his own flesh.

    [6] Better is a handful of quietness than two hands full of toil and a striving after wind.

    [7] Again, I saw vanity under the sun: [8] one person who has no other,
    either son or brother, yet there is no end to all his toil, and his eyes
    are never satisfied with riches, so that he never asks, “For whom am I
    toiling and depriving myself of pleasure?” This also is vanity and an
    unhappy business.

    [9] Two are better than one, because they have a good reward for their
    toil.  [10] For if they fall, one will lift up his fellow.  But woe to
    him who is alone when he falls and has not another to lift him up!  [11]
    Again, if two lie together, they keep warm, but how can one keep warm
    alone?  [12] And though a man might prevail against one who is alone, two
    will withstand him—a threefold cord is not quickly broken.

    [13] Better was a poor and wise youth than an old and foolish king who no
    longer knew how to take advice.  [14] For he went from prison to the
    throne, though in his own kingdom he had been born poor.  [15] I saw all
    the living who move about under the sun, along with that youth who was to
    stand in the king’s place.  [16] There was no end of all the people, all
    of whom he led.  Yet those who come later will not rejoice in him.
    Surely this also is vanity and a striving after wind.

    [1] Guard your steps when you go to the house of God.  To draw near to
    listen is better than to offer the sacrifice of fools, for they do not
    know that they are doing evil.  [2] Be not rash with your mouth, nor let
    your heart be hasty to utter a word before God, for God is in heaven and
    you are on earth.  Therefore let your words be few.  [3] For a dream
    comes with much business, and a fool’s voice with many words.

    [4] When you vow a vow to God, do not delay paying it, for he has no
    pleasure in fools.  Pay what you vow.  [5] It is better that you should
    not vow than that you should vow and not pay.  [6] Let not your mouth
    lead you into sin, and do not say before the messenger that it was a
    mistake.  Why should God be angry at your voice and destroy the work of
    your hands?  [7] For when dreams increase and words grow many, there is
    vanity; but God is the one you must fear.
  \end{quote}
\subsection*{Notes}
\begin{itemize}
  \item{Two points today: relationship with others, and relationship with God.}
  \item{Relationships with others: 
  \begin{enumerate}
    \item{Economic relationships: Some people are caught in the trap of
    working harder and harder, and they cannot rest.  Some of these people
    who are caught in this trap do so because of envy.  Another group of
    people are hardly working at all.  In the end, the Preacher concludes
    that it is better to be content with a little (one handful), rather than
    two handfuls obtained after much toil.  We must find balance between work
    and rest.  As for envy, rather than looking at those who are better than
    us, we should look at those who are worse off than us and help them.  The
    Lord is our giver, and from what He has given us, we should help others.}
    \item{Social relationships: It is better to have companions to share your
    joys and your sorrows.  We are not designed to live for ourselves.  On
    the other hand, the world tells us: "it is all about you, live for your
    own dreams".  We are created by God to be social creatures, when we try
    to live for ourselves, we become empty.  Only in a community is there a
    connection between work and reward, rest and support.}
    \item{Political relationships: leaders and positions will come and go.
    There are people who are old, but they aren't wise anymore, and they will
    eventually give way to other people.  There are those who are young but
    wise, and they rise the ranks to become leaders of the country.  But even
    for these young and wise people, even their wisdom doesn't prevent them
    from one day losing their leadership position, not least due to old age.
    }
  \end{enumerate}}
  \item{Relationship with God: When there is silence, that is when we are
  most ready for God to speak to us.  Moreover, dead formalism is worse than
  a lively faith.  On the other hand, if we are too casual with God, that
  also shows a lack of fear (respect/reverence) of God.  There are two types
  of fear; first is the fear of being judged, fear of facing the
  consequences, second is a great respect that results in awe and reverence,
  which comes from a recognition of the majesty, holiness, and glory of God.
  Also, what is our attitude when we come to church?  Do we get angry with
  people when people cut our lane when we drive to church?  Do we get angry
  with our kids when they aren't cooperative?  Etc.
  
  Everything comes from God, we cannot offer to God anything that doesn't
  already belong to him.  Hence, it makes no sense to try to ``bribe'' God
  with sacrifices etc (Psalm 50).  What God wants is not dead formalism, but
  what God wants is a true and lively faith.  And the first mark of a true
  and lively faith is good works (to listen to God and to obey God; e.g true
  religion is helping the widows and orphans in their affliction).  The
  second mark of a true and lively faith is a reverent awe of God, we must be
  very intentional when we approach God.
  
  In fact, our relationship with God shapes our relationships with others.
  The answer to ``for whom am I toiling'' should be ``for God''.  If we have
  a right view of God, if we know that there is a God in heaven who judges,
  we wouldn't wrong our neighbour.  If we know God's love for us, especially
  in the giving of His Son for us, we will love our neighbour.}
\end{itemize}
  \section{22nd May 2022: Greed is not good}
\subsection*{Text: Ecclesiastes 5:8-6:}
  \begin{quote}
    [8] If you see in a province the oppression of the poor and the violation
    of justice and righteousness, do not be amazed at the matter, for the
    high official is watched by a higher, and there are yet higher ones over
    them.  [9] But this is gain for a land in every way: a king committed to
    cultivated fields.

    [10] He who loves money will not be satisfied with money, nor he who
    loves wealth with his income; this also is vanity.  [11] When goods
    increase, they increase who eat them, and what advantage has their owner
    but to see them with his eyes?  [12] Sweet is the sleep of a laborer,
    whether he eats little or much, but the full stomach of the rich will not
    let him sleep.

    [13] There is a grievous evil that I have seen under the sun: riches were
    kept by their owner to his hurt, [14] and those riches were lost in a bad
    venture.  And he is father of a son, but he has nothing in his hand.
    [15] As he came from his mother’s womb he shall go again, naked as he
    came, and shall take nothing for his toil that he may carry away in his
    hand.  [16] This also is a grievous evil: just as he came, so shall he
    go, and what gain is there to him who toils for the wind?  [17] Moreover,
    all his days he eats in darkness in much vexation and sickness and anger.

    [18] Behold, what I have seen to be good and fitting is to eat and drink
    and find enjoyment in all the toil with which one toils under the sun the
    few days of his life that God has given him, for this is his lot.  [19]
    Everyone also to whom God has given wealth and possessions and power to
    enjoy them, and to accept his lot and rejoice in his toil—this is the
    gift of God.  [20] For he will not much remember the days of his life
    because God keeps him occupied with joy in his heart.

    [1] There is an evil that I have seen under the sun, and it lies heavy on
    mankind: [2] a man to whom God gives wealth, possessions, and honor, so
    that he lacks nothing of all that he desires, yet God does not give him
    power to enjoy them, but a stranger enjoys them.  This is vanity; it is a
    grievous evil.  [3] If a man fathers a hundred children and lives many
    years, so that the days of his years are many, but his soul is not
    satisfied with life’s good things, and he also has no burial, I say that
    a stillborn child is better off than he.  [4] For it comes in vanity and
    goes in darkness, and in darkness its name is covered.  [5] Moreover, it
    has not seen the sun or known anything, yet it finds rest rather than he.
    [6] Even though he should live a thousand years twice over, yet enjoy no
    good—do not all go to the one place?

    [7] All the toil of man is for his mouth, yet his appetite is not
    satisfied.  [8] For what advantage has the wise man over the fool?  And
    what does the poor man have who knows how to conduct himself before the
    living?  [9] Better is the sight of the eyes than the wandering of the
    appetite: this also is vanity and a striving after wind.

    [10] Whatever has come to be has already been named, and it is known what
    man is, and that he is not able to dispute with one stronger than he.
    [11] The more words, the more vanity, and what is the advantage to man?
    [12] For who knows what is good for man while he lives the few days of
    his vain life, which he passes like a shadow?  For who can tell man what
    will be after him under the sun?
  \end{quote}
\subsection*{Notes}
\begin{itemize}
  \item{Greed-the desire for more money than we need-is not good.  Money is
  not itself good too.  Yet it is not hard to see that today, money makes the
  world go round.  Money has no eternal significance, and any desire to
  accumulate more money will lead to sadness.  The only way to satisfaction
  is in Christ.}
  \item{Firstly, we see how greed leads to misery for others.  Even though
  the officials of the land were supposed to serve the common folk, through
  their corruption and greed they exploited the poor peasants.  Chapter 5
  verse 9 is hard to understand in the Hebrew, a better translation would be
  the NLT over the ESV, who says ``even the king milks the land for his own
  profit''.  The greed of the ruling class has brought about a lot of misery
  for the common people.  Greed causes one to care only for his own material
  gain, at the expense of others.  Greed runs contrary to how we should love
  others, which is central to the Christian faith.  As Paul says in
  Philippians 2, we are to look not only to our own needs, but to the needs
  of others.  Love is the antithesis of greed; love shares with others, greed
  hoards for yourself.  We can see this in our world today; just look at some
  billionaires who become rich by exploiting their employees.}
  \item{The reason why people chase and accumulate wealth is because we think
  it makes us happy.  Yet greed, a chase for more money, only make us more
  miserable.  The preacher observes in this text here that greedy and rich
  people are often miserable.  In verse 11, we see that as we have more
  money, more people will come and help us spend it.  Moreover in verse 12,
  we see that the richer we become, the more anxiety we will have over losing
  that money; for example a poor laborer doesn't need to worry about the
  crash of the US stock market lol.  It is a tragedy to chase money so hard
  for happiness.  Beware of greed that leads only to misery by yourself.
  This misery is compounded by the fact that money has no significance at
  all.  We can't take money with us when we die.  As Ecclesiastes 5:15 say,
  we go out of the world as naked and as empty as we came.  Moreover, money
  doesn't help us find favour in God's sight; in fact, to whom more money is
  given, more is expected.  In Luke 16, we see the parable of the rich man
  and Lazarus; the rich man ended up in further misery in the afterlife
  because he didn't use his money properly.  Money ensnares us and draws us
  further from God.  No one can serve two masters, for either he will hate
  the one and love the other, or he will be devoted to the one and despise
  the other.  You cannot serve God and money.  Money is not a neutral entity,
  it is a master that demands to be served.  It has the power to stoke our
  desire, and to entice us to serve it.  The desire for material good is
  vanity, because there is always new stuff to desire.  In our first world
  country, if we want to seek happiness by chasing material goods, we realise
  that there is always more and more material goods to chase.  E.g if we earn
  money to get the iPhone 13, next year there'll be the iPhone 14.}
  \item{But greed, although sinful, is a search for happiness.  People are
  greedy only because they want happiness, but sadly they don't realise that
  money cannot give happiness.  The desire for happiness is actually a good
  thing, its just that using money (which cannot fulfill that desire) as a
  means to happiness is vanity.  So what can give us happiness in this world?
  The answer is Jesus.  Haha.  Through Jesus we can have communion with God,
  and find eternal rest in that communion.  Jesus Christ is our true riches,
  our pearl of great price, our hidden treasure in the field.  When we have
  Jesus, we have everything.  For us that have Jesus as our treasure, we can
  learn to resist the temptations of money.}
  \item{One way we can do that is to enjoy the here and now, in thanksgiving
  to God.  We need to eat and drink everyday to survive, and we need to work
  everyday to survive.  There is nothing especially attractive in these
  mundane activities.  But rather than thinking that we can be happy only in
  the future, e.g only when we have our first million etc, we should realise
  that we can be happy in the here and now, no matter what we are going
  through.  E.g we don't need to go on an expensive vacation with our family
  to enjoy family time; we can enjoy family time at home.  All these mundane
  moments are given to us by God, and we should enjoy them in thanksgiving.
  Who knows when the last mundane moment we will have is...}
  \item{Next, we can resist the temptation to accumulate more money by giving
  money away to the poor.  This is also an obedience to Jesus' command for us
  to love our neighbour.  The more we learn to give money away, the less
  money is able to tempt us into greed.  If we have more than enough for
  ourselves and our family, then we should consider letting the extra money
  go.}
  \item{Ultimately, we are called to seek first the kingdom of God and his
  righteousness.  Whatever we need in our life will be given to us by God.
  Of course we must still work and earn an honest living, but we do so in a
  utilitarian manner (using money to love God and love neighbour and for our
  basic upkeep) rather than seeking money itself as a means to an end.}
\end{itemize}
  \section{29th May 2022: Gospel sunday}
\subsection*{Text: 1 Corinthians 15}
  \begin{quote}
    [1] Now I would remind you, brothers, of the gospel I preached to you, which you received, in which you stand, [2] and by which you are being saved, if you hold fast to the word I preached to you—unless you believed in vain.

    [3] For I delivered to you as of first importance what I also received: that Christ died for our sins in accordance with the Scriptures, [4] that he was buried, that he was raised on the third day in accordance with the Scriptures, [5] and that he appeared to Cephas, then to the twelve. [6] Then he appeared to more than five hundred brothers at one time, most of whom are still alive, though some have fallen asleep. [7] Then he appeared to James, then to all the apostles. [8] Last of all, as to one untimely born, he appeared also to me. [9] For I am the least of the apostles, unworthy to be called an apostle, because I persecuted the church of God. [10] But by the grace of God I am what I am, and his grace toward me was not in vain. On the contrary, I worked harder than any of them, though it was not I, but the grace of God that is with me. [11] Whether then it was I or they, so we preach and so you believed.

    [12] Now if Christ is proclaimed as raised from the dead, how can some of you say that there is no resurrection of the dead? [13] But if there is no resurrection of the dead, then not even Christ has been raised. [14] And if Christ has not been raised, then our preaching is in vain and your faith is in vain. [15] We are even found to be misrepresenting God, because we testified about God that he raised Christ, whom he did not raise if it is true that the dead are not raised. [16] For if the dead are not raised, not even Christ has been raised. [17] And if Christ has not been raised, your faith is futile and you are still in your sins. [18] Then those also who have fallen asleep in Christ have perished. [19] If in Christ we have hope in this life only, we are of all people most to be pitied.

    [20] But in fact Christ has been raised from the dead, the firstfruits of those who have fallen asleep. [21] For as by a man came death, by a man has come also the resurrection of the dead. [22] For as in Adam all die, so also in Christ shall all be made alive. [23] But each in his own order: Christ the firstfruits, then at his coming those who belong to Christ. [24] Then comes the end, when he delivers the kingdom to God the Father after destroying every rule and every authority and power. [25] For he must reign until he has put all his enemies under his feet. [26] The last enemy to be destroyed is death. [27] For “God has put all things in subjection under his feet.” But when it says, “all things are put in subjection,” it is plain that he is excepted who put all things in subjection under him. [28] When all things are subjected to him, then the Son himself will also be subjected to him who put all things in subjection under him, that God may be all in all.

    [29] Otherwise, what do people mean by being baptized on behalf of the dead? If the dead are not raised at all, why are people baptized on their behalf? [30] Why are we in danger every hour? [31] I protest, brothers, by my pride in you, which I have in Christ Jesus our Lord, I die every day! [32] What do I gain if, humanly speaking, I fought with beasts at Ephesus? If the dead are not raised, “Let us eat and drink, for tomorrow we die.” [33] Do not be deceived: “Bad company ruins good morals.” [34] Wake up from your drunken stupor, as is right, and do not go on sinning. For some have no knowledge of God. I say this to your shame.

    [35] But someone will ask, “How are the dead raised? With what kind of body do they come?” [36] You foolish person! What you sow does not come to life unless it dies. [37] And what you sow is not the body that is to be, but a bare kernel, perhaps of wheat or of some other grain. [38] But God gives it a body as he has chosen, and to each kind of seed its own body. [39] For not all flesh is the same, but there is one kind for humans, another for animals, another for birds, and another for fish. [40] There are heavenly bodies and earthly bodies, but the glory of the heavenly is of one kind, and the glory of the earthly is of another. [41] There is one glory of the sun, and another glory of the moon, and another glory of the stars; for star differs from star in glory.

    [42] So is it with the resurrection of the dead. What is sown is perishable; what is raised is imperishable. [43] It is sown in dishonor; it is raised in glory. It is sown in weakness; it is raised in power. [44] It is sown a natural body; it is raised a spiritual body. If there is a natural body, there is also a spiritual body. [45] Thus it is written, “The first man Adam became a living being”; the last Adam became a life-giving spirit. [46] But it is not the spiritual that is first but the natural, and then the spiritual. [47] The first man was from the earth, a man of dust; the second man is from heaven. [48] As was the man of dust, so also are those who are of the dust, and as is the man of heaven, so also are those who are of heaven. [49] Just as we have borne the image of the man of dust, we shall also bear the image of the man of heaven.

    [50] I tell you this, brothers: flesh and blood cannot inherit the kingdom of God, nor does the perishable inherit the imperishable. [51] Behold! I tell you a mystery. We shall not all sleep, but we shall all be changed, [52] in a moment, in the twinkling of an eye, at the last trumpet. For the trumpet will sound, and the dead will be raised imperishable, and we shall be changed. [53] For this perishable body must put on the imperishable, and this mortal body must put on immortality. [54] When the perishable puts on the imperishable, and the mortal puts on immortality, then shall come to pass the saying that is written:

      “Death is swallowed up in victory.”
      [55] “O death, where is your victory?
          O death, where is your sting?”


        [56] The sting of death is sin, and the power of sin is the law. [57] But thanks be to God, who gives us the victory through our Lord Jesus Christ.

    [58] Therefore, my beloved brothers, be steadfast, immovable, always abounding in the work of the Lord, knowing that in the Lord your labor is not in vain.
  \end{quote}
\subsection*{Notes}
\begin{itemize}
  \item{Death is inevitable, whether we choose to think and talk about it, or
  not.  Death sometimes comes in very sudden and surprising ways.  In fact we
  might even know all of these already, yet when death strikes close to home,
  we will still be affected.}
  \item{But from a Christian perspective, we know that death does not have
  the final say.  This is because of Jesus' death on the cross and His
  resurrection.}
  \item{As can be seen in v3 of today's text, ``Christ died for our sins''.
  What does this mean?  When God created everything, everything was good.
  There was no death etc at all.  But when we (Adam and Eve) turned away from
  God, who is the source of life, death came into the world.  So what Jesus
  did on the cross was to take on the penalty of our sin, which is death, so
  that now we no longer need to die.  In fact, Jesus was resurrected from the
  dead, which shows that He was greater than death itself.}
  \item{While Jesus' resurrection might seem hard to believe at first, Paul
  lists a few people who have seen the risen Lord; people like Peter, James,
  etc, and about $500$ more.  So for anyone who wants to verify Paul's
  statement here, they can just go and ask the people that Paul listed here.
  Moreover, anyone who is familiar with the OT would already know that
  whatever that has happened to Jesus has already been prophesied before in
  the past.  One key example is Isaiah 53, who prophesied both Jesus' death
  for our sins and also His resurrection and glorification.}
  \item{Ok but even if Jesus is raised from the dead, what does that have got
  to do with me?  The answer is in v22 of today's text; ``For as in Adam all
  die, so also in Christ shall all be made alive''.  Death gives way to new
  life in Christ.  This is the reason why Christian funeral services have so
  much hope.  When Christians pass on, they are just going on first.  There
  is a certain sadness because we still miss the people who went on ahead,
  but there is also anticipation for the big reunion in the future.}
  \item{But actually we don't have to wait until we die to start this new
  life.  We can have this new life in Christ the moment we believe in Jesus.
  As we walk with Jesus, our mind, character, conduct, etc will all change
  for the better.  I.e, we are a new creation!  Death gives way to new life
  in Christ.  We are also given a new purpose in life, through having a new
  King in our life.  A good life under an evil master won't last very long.
  On the other hand, a good life under Christ, our good king, will last
  forever.  So Christ, in his death, has destroyed ``every rule and every
  authority and power''.  Christ has destroyed all of the evil spiritual
  rulers in the world that are responsible for sin and death, and replaces
  their rule with His own good and perfect rule.  The minute we believe in
  Jesus, we place ourselves under Jesus' good and perfect rule.}
  \item{In the world now there is still evil and suffering, and sin and
  death, but all of these are just machinations of a defeated enemy, who will
  be totally destroyed when Christ comes again.  Right now, God lives in the
  life of Christians through His Holy Spirit that He gives us, and through
  His Holy Spirit, we can overcome all of these darkness that still exists.}
\end{itemize}
  \section{5th June 2022: Sent out to witness}
\subsection*{Text: Acts 1:1-11}
  \begin{quote}
    [1] In the first book, O Theophilus, I have dealt with all that Jesus
    began to do and teach, [2] until the day when he was taken up, after he
    had given commands through the Holy Spirit to the apostles whom he had
    chosen.  [3] He presented himself alive to them after his suffering by
    many proofs, appearing to them during forty days and speaking about the
    kingdom of God.

    [4] And while staying with them he ordered them not to depart from
    Jerusalem, but to wait for the promise of the Father, which, he said,
    “you heard from me; [5] for John baptized with water, but you will be
    baptized with the Holy Spirit not many days from now.”

    [6] So when they had come together, they asked him, “Lord, will you at
    this time restore the kingdom to Israel?” [7] He said to them, “It is not
    for you to know times or seasons that the Father has fixed by his own
    authority.  [8] But you will receive power when the Holy Spirit has come
    upon you, and you will be my witnesses in Jerusalem and in all Judea and
    Samaria, and to the end of the earth.” [9] And when he had said these
    things, as they were looking on, he was lifted up, and a cloud took him
    out of their sight.  [10] And while they were gazing into heaven as he
    went, behold, two men stood by them in white robes, [11] and said, “Men
    of Galilee, why do you stand looking into heaven?  This Jesus, who was
    taken up from you into heaven, will come in the same way as you saw him
    go into heaven.”
  \end{quote}
\subsection*{Notes}
\begin{itemize}
  \item{Christians all share the same mission, regardless of their individual
  backgrounds. They are all supposed to be on the same team.}
  \item{Luke wanted us to understand that Jesus' mission didn't end with
  Jesus going to the Father into heaven.  In fact, Jesus' mission continues
  through His disciples, as God works through them through His Spirit.}
  \item{We are on a mission because Jesus is the risen Lord.  In this text,
  from verse 6, we see that the disciples anticipated Jesus to restore the
  kingdom to Israel.  After all, if Jesus could conquer death, surely He
  could restore the kingdom to Israel.  But Jesus didn't answer the
  disciples' question about the kingdom directly.  Jesus told them that God's
  kingdom will indeed be established (God is sovereign and His purposes
  cannot be thwarted).  But more importantly, Jesus told them not to worry
  about the details about how the kingdom would come, but instead Jesus
  commissioned them (v8) to continue carrying on Kingdom work, and Jesus
  promised them that they would have the power to do so through the power of
  the Holy Spirit.}
  \item{Now, since ``God so loved the world'', and that ``God wills nobody to
  perish, but all to repent'', then God's kingdom work cannot end with the
  disciples' lifetime; it carrys on through the disciples of the disciples,
  and through their disciples, etc, all the way now to us.  }
  \item{We cannot be indifferent to how we conduct our life here, we cannot
  live as if Jesus' death, resurrection and ascension has no effect on us.
  When we confess that ``Jesus is seated at the right hand of the Father'',
  our confession should have an effect on our lives, for example in how we
  relate to other people.}
  \item{Now, also, using the metaphor of the Church being Jesus' body, since
  the Head is on a mission the body must be too!  I.e, just as Jesus who is
  God's Son carried out God's mission, we are to participate in His mission.}
  \item{The purpose of discipleship is to train us and to shape us to perform
  the mission that God called us to.  The process of discipleship trains us
  to understand how things in the world relate to God's mission as revealed
  in His Word, and also produces spiritual fruit in us that is essential for
  God's mission.  It is more than just coming to church, making friends,
  reading the Bible together as a hobby, etc.  We have to be mission-minded.}
  \item{Btw, our mission is to be a witness of the Risen Lord.  A witness is
  someone who testifies to what he has seen or experienced.  So for us, we
  have experienced God's amazing love in our life, and we are to witness to
  it.  In the OT, Israel was said to be God's witness (Isaiah 43), and
  because of their failure, God sent His Son, Jesus to be His perfect
  witness.  And for us today, though we have not seen Jesus' resurrection,
  the content of our witness is also still to be centred on truths about the
  person and work of Christ.  Who is this person Jesus, is Jesus truly Lord
  and truly savior?}
  \item{In our witness, there are the objective truths (that Jesus is a
  historical figure who was crucified under Pontus Pilate, and who was
  resurrected and ascended to God, also that Jesus is the Son of God, etc
  etc), and there also are subjective truths (our conversion experience, how
  Jesus has changed our lives, etc).  Both are important.  Btw our subjective
  truths have to be constantly updated, it is not sufficient to talk about
  how Jesus saved you $10$ years ago, we also need to talk about how Jesus
  saves us now.}
  \item{The scope of our witness also extends to everyone and every aspect of
  our lives.  ``...in Jerusalem and in all Judea and Samaria, and to the end
  of the earth...'' it is said.  Jerusalem is a hostile place to the
  disciples (since Jesus was crucified there), Samaria historically has
  baggage with Jewish people, and ``the ends of the earth'' wasn't even well
  defined back then.  The difficulties are super difficult, but the disciples
  still did it.}
  \item{In our lives today, we might need to witness in ``Jerusalem'', a
  familiar but hostile place, perhaps to our family.  We might also need to
  witness in ``Judea'', still kinda familiar but might be awkward, perhaps to
  our friends.  We might also need to witness in ``Samaria'', to people who
  we don't really know and might hate us.}
  \item{Our witness to the world becomes more convincing if it can also be
  seen in our lives.  Hence, we need to be pure in heart, etc, we need the
  fruits of the Spirit.}
  \item{God empowers us with His Spirit for our mission.  God's Spirit is our
  Helper, and God's Spirit mediates to us the presence of God and His Son.
  See John 14:16-18.  In contemporary Judaism, there is the ``wailing wall'',
  where people go there and cry because under their view, God's presence has
  left them, and they want God to return to them.  But under the Christian
  view, we have God's presence with us through His Spirit!}
  \item{The Spirit is God who convicts hearts and transform lives.  Nothing
  else can make spiritually dead people come back to spiritual life, and that
  power is God's divine power through His Spirit.  As it is said in Ezekiel,
  God's Spirit gives us a heart of flesh and takes away our heart of stone.
  If we are born again, we would also have experienced this power in our
  lives.  God's Spirit will continue to convict the world of sin, and that as
  we try our best through God's Spirit to preach the Word faithfully and as
  to live faithful lives, ultimately it is God's power who changes the hearts
  of those who hear and see us.}
\end{itemize}
  \section{19th June 2022: Being an authentic community}
\subsection*{Text: James 5:13-16}
  \begin{quote}
    [13] Is anyone among you suffering?  Let him pray.  Is anyone cheerful?
    Let him sing praise.  [14] Is anyone among you sick?  Let him call for
    the elders of the church, and let them pray over him, anointing him with
    oil in the name of the Lord.  [15] And the prayer of faith will save the
    one who is sick, and the Lord will raise him up.  And if he has committed
    sins, he will be forgiven.  [16] Therefore, confess your sins to one
    another and pray for one another, that you may be healed.  The prayer of
    a righteous person has great power as it is working.
  \end{quote}
\subsection*{Notes}
\begin{itemize}
  \item{This is a continuation of the church's sermon series on being a
  compelling community (first three sermons were preached earlier in the
  year).  Today's text is on being an authentic community.}
  \item{Three `S' for today: suffering, sickness and sin.  How does a church
  deal with these three things in a way that shows faith at work?}
  \item{Without the context of the letter, our text for today sounds like
  standalone instructions that can easily be misinterpreted.  For example,
  verse 13 does not apply to all forms of suffering, such as if we commit a
  crime and go to jail but we pray that the jail will burn down or something
  so we don't have to go.}
  \item{The content of James' letter suggests that the church was more or
  less established, since there were elders and deacons.  Moreover, it is
  said in Galatians that James was ministering to the Jews.  Hence, it is
  quite probable that the letter of James was written to the Jews who were
  dispersed from Jerusalem after the stoning of Stephen.  The letter of James
  starts with how we should stand fast in the face of tribulation.  Towards
  the end of James' letter, James picks up two groups of people for
  condemnation.  The first group are people who are proud (end of chapter 4),
  and the next group are the rich who oppress the poor (start of chapter 5).
  Right after, James exhorts his hearers to be patient in the face of
  tribulation and not grumble about each other.  These are what James was
  talking about right before our text for today.}
  \item{Hence, our text for today on ``suffering'' (v13) is more about how we
  should have faith in the face of suffering and use the time to pray,
  instead of boasting in arrogance or grumbling.  And we should definitely
  not be the source of suffering like those who oppress the poor.  And
  lastly, we should not rashly make oaths in our suffering; we should pray in
  faith, but we should not say things like ``God if you do $X$ for me, then I
  will do $Y$ for you''.  We should pray for strength to not give up, pray
  for patience in the face of suffering, pray for comfort and help, pray that
  our trials will deepen our relationship with God.}
  \item{That being said, it is not wrong to pray for God's provision for
  ``non-spiritual things'', such as job opportunities etc.  It is not that we
  should only pray for spiritual things.  But we can do seek these
  ``non-spiritual'' things in a way that is wrong, and we must be aware of
  that.}
  \item{As for annointing the sick with oil, this is an ancient custom in
  those days.  These days, we can do things like sending a fruit basket etc.
  But why call the elders?  The answer to that is verse 15.  We note that
  first, James had talked about the steadfastness of Job.  Secondly, as per
  John 9, it is normal back then to automatically tie severe sickness with
  sin.  Hence, James called for the elders (who are spiritually mature)
  because he was afraid of how non-spiritually mature people would be like
  Job's three friends and talk rubbish such as tieing the person's sickness
  with his sin.  The elders were called to assess the situation and to
  discern if the sickness is due to sin, or not.  And whether the ``prayer
  that saves'' refers to saving the sick person from his sin, or his
  sickness, this is disputed\footnote{Ok I'm not sure if I got this part
  correctly...}.  And as for ``the Lord will raise him up'', that might be
  referring to how the sick person's spirit will be raised up, just like how
  Job's spirit was raised up.  What this means practically is that if we are
  sick, the church should share our burdens with us.  If we troubled by
  things that we hear with respect to our sickness, we should call the
  elders.  And as for healing, it is up to God's will and according to God's
  providence.  A note on miraculous healing: when Paul was shipwrecked, he
  healed all who were brought to him.  But it seemed like Paul's healing
  ability was limited to his time on the island; when Paul was in Rome, God
  preferred that Paul preached the gospel with boldness and was eventually
  martyred there.  We note also that healing of diseases do not always lead
  to a changed life, as we have seen in Jesus' ministry, but it is faithfully
  living and proclaiming the Word that does.}
  \item{And as for what James talked about sin, we look at Job again.  In Job
  42, we see also that God asked Job to pray for the three friends.  Job is a
  righteous man here, and therefore his prayer worked greatly.  So what is
  James saying us today?  Sin committed against one another must be dealt
  with.  We must confess our sins to God, but also when we sin against each
  other, we must confess our sins to each other and seek restoration of the
  relationship.  We must forgive each other and be humble with each other
  about our faults.  It is actions like these that are actions of
  righteousness, and these actions show the sincerity of our faith which
  leads to the effectiveness of our prayers.  On the other hand, if we
  confess our sins to God but don't do anything about how our sin against
  someone has hurt him/her, we are being hypocrites and our prayers will be
  ineffectual.  }
\end{itemize}
  \section{26th June 2022: One in ministry to all the world: Learning to live
beyond ourselves}
\subsection*{Text: Romans 15:5-13}
  \begin{quote}
    [5] May the God of endurance and encouragement grant you to live in such
    harmony with one another, in accord with Christ Jesus, [6] that together
    you may with one voice glorify the God and Father of our Lord Jesus
    Christ.  [7] Therefore welcome one another as Christ has welcomed you,
    for the glory of God.

    [8] For I tell you that Christ became a servant to the circumcised to
    show God’s truthfulness, in order to confirm the promises given to the
    patriarchs, [9] and in order that the Gentiles might glorify God for his
    mercy.  As it is written,

      “Therefore I will praise you among the Gentiles,
          and sing to your name.”


      [10] And again it is said,

        “Rejoice, O Gentiles, with his people.”


      [11] And again,

        “Praise the Lord, all you Gentiles,
            and let all the peoples extol him.”


      [12] And again Isaiah says,

        “The root of Jesse will come,
            even he who arises to rule the Gentiles;
        in him will the Gentiles hope.”


      [13] May the God of hope fill you with all joy and peace in believing,
      so that by the power of the Holy Spirit you may abound in hope.
  \end{quote}
\subsection*{Notes}
\begin{itemize}
  \item{This sermon is related to the talks given at the church retreat on 25
  June 2022.  The talks are recorded on the church website.  Yesterday, we
  have looked at ``one with Christ'' and ``one in Christ''.  Today, we have
  ``one for Christ''.  As mentioned yesterday, we need to be one with Christ
  first, which would allow us to be one in Christ.  And as we worship the God
  together as a Church, we are reminded of God's goodness, and after the
  benediction, we go out into the world together for God's mission.  The God
  who gathers His people also sends them out.  This is what it means to be
  ``one for Christ''.}
  \item{One of the things that Jesus prayed for in John 17 was ``that they
  may be one, as we are one''.  The Lord's longing was for oneness to be
  experienced among the people of God.  However, we often fall short of this
  unity.  In Romans 8:34, we read that Jesus is interceding for us, possibly
  still praying for us to be one.}
  \item{The vertical relationship between God and us is essential for us to
  be unity.  As Paul says in verse 5 of our text, we need to have ``the
  Spirit of unity''.  Unity is not a man-made thing, it is not created
  through human efforts of managing organisations, team building exercises,
  etc.  Unity is a supernatural thing.  This Spirit of unity comes ``in
  accord with Christ Jesus'', or as translated in 1984 NIV, the Spirit of
  unity comes ``as we follow Christ Jesus''.  Without Jesus, there will be no
  success in Church work and no unity.  There might be some success in
  worldly eyes, but from a spiritual lens, we can see that there is no
  success. }
  \item{ In Christian organisations, sadly oftentimes we have meetings to
  discuss important stuff without the individual members spending devotional
  time with God first.  Similarly, we should also meet the Lord ourselves
  first before we gather for corporate worship.  It is wrong to say ``we are
  going to worship in Church anyway, I can skip bible reading in the
  morning''.  We must follow Christ individually first, then when we gather
  for corporate worship, we will be united in our worship of God.  The
  essential point for unity is the Lord Jesus Christ.  Take away Jesus, and
  there will be no unity.  It is the name of Jesus, not some institution or
  some brand or project that gives us spiritual unity.}
  \item{Jesus also said that ``He who is not with me is against me'', and
  ``He who does not gather with me scatters''.  Jesus said this when the
  disciples complained about how there was someone else who is not part of
  them driving out demons in Jesus' name.  This means that as long as there
  is someone else doing God's work at Jesus' name, then that person is also a
  brother in Christ doing the same work.  There is no such thing as ``this is
  your mission field'' and ``this is my mission field''.  The vertical
  relationship with God determines our horizontal relationship with others.
  This is also why Paul writes in verse 7 of our text, ``accept one another,
  just as Christ has accepted you''.  How has Christ accepted us?  See
  Romans 5:8; Christ died for us while we were yet sinners!  Jesus also said
  in John 6:37, ``whoever comes to me I will never drive away''.  In light of
  Jesus' love for us, we must accept other brothers and sisters to the same
  extent.  Or as it is said in John 13:34, we love one another as Jesus has
  loved us.  When the vertical is missing, i.e when we don't understand God's
  love for us, then it is hard for us to love one another truly.  Slogans
  telling us to love one another without reminding us about our vertical
  relationship with God are eventually ineffective.  }
  \item{Or put another way, there is a difference between the
  \textbf{kingdom} of God versus a \textbf{republic} of heaven; in a kingdom,
  there is a king that we must all submit to, but in a repulic, there is no
  king and it is all up to us.}
  \item{Paul also said before that the word is God is near us in our mouth
  and in our heart in Romans 10, which is a paraphrase of Deuteronomy 30.
  And Paul also said that if we confess with our mouth and believe in our
  heart, we will be saved.  In the church, there hence must be unity in both
  our hearts (what we believe) and in our mouths (what we confess and do).
  The image to have here is of an orchestra; our hearts must all be in sync
  with God's heartbeat\footnote{speaking figuratively}, and then our hearts
  will be in sync with each other.  Then when all of us plays the same song,
  our message will be effectively proclaimed to the world.  The contrast is
  if all of us are disunited and if we have many different messages; then it
  will be confusion to the world.}
  \item{In verses 9-12, we have four quotations of the Old Testament.  These
  four quotations all mention the Gentiles.  The first quotation comes from
  David, in 2 Samuel 22:50.  The picture is that of a Jew worshipping God
  among the Gentiles, such as Lydia in Philippi worshipping God in a small
  group because there was no synagogue.  The next quotation is from Moses, in
  Deuteronomy 32:43.  The picture is that of Gentiles rejoicing together with
  the Jews.  This is one step further than the previous picture; in the
  previous verse we had Jews worshipping God among the Gentiles, and then now
  the Gentiles join the Jews in worship.  The third picture is from Psalm
  117:1, here we have the Gentiles praising God on their own.  The Gentiles
  who observed the Jews previously and subsequently joined the Jews
  previously have learnt to praise God on their own.  The last picture is
  from Isaiah 11:10, which is that both the Jews who were sent to the
  Gentiles, and the Gentiles who observed them and then joined them and then
  on their own worshipping God, both the Jews and the Gentiles will be ruled
  by Jesus Christ.}
  \item{The above four pictures in succession depicts what sending a
  missionary overseas looks like.  First the missionary worships God alone,
  then the curious townfolk join in to worship God, then the townfolk learn
  to worship God alone, and finally both the missionary and the townfolk are
  ruled by Jesus.  These four pictures here are an encouragement of what
  missionary work can look like, and these four pictures have occured
  throughout history; the fact that we as Gentiles are in church worshipping
  God now is testament to that!  Thanks to the missionaries who came to SG in
  the past.}
  \item{God's plan is not merely for the Jews, it is for the entire world.
  Will we join God in His mission to the world?  We must give up our small
  ambitions (e.g buying a nice house, getting a nice job, etc.), because they
  will fade away and have no lasting impact.  On the other hand, being
  recruited by God to do His work has lasting impact.  God's mission to the
  world is the greatest project in the world, and if we join God in His
  mission, we will surely lead a blessed (but perhaps not easy) life.  When
  we hear God calling us, will we say: ``not my will, but yours be done''. Participating in God's mission can look like going as far to somewhere like Japan, or it can also look like going one street across to let's say the rental flats to reach the poor and giving them help and good news, or it can also look like going one isle across the office to tell our colleague about Jesus. But for this to happen, we need to experience a oneness with God, a oneness with each other, then we can be one in our mission to the world.}
  \end{itemize}
  \section{3rd July 2022: What is spiritual friendship?}
\subsection*{Text: 1 Samuel 20:1-17,35-42}
  \begin{quote}
    [1] Then David fled from Naioth in Ramah and came and said before
    Jonathan, “What have I done?  What is my guilt?  And what is my sin
    before your father, that he seeks my life?” [2] And he said to him, “Far
    from it!  You shall not die.  Behold, my father does nothing either great
    or small without disclosing it to me.  And why should my father hide this
    from me?  It is not so.” [3] But David vowed again, saying, “Your father
    knows well that I have found favor in your eyes, and he thinks, ‘Do not
    let Jonathan know this, lest he be grieved.’ But truly, as the LORD lives
    and as your soul lives, there is but a step between me and death.” [4]
    Then Jonathan said to David, “Whatever you say, I will do for you.” [5]
    David said to Jonathan, “Behold, tomorrow is the new moon, and I should
    not fail to sit at table with the king.  But let me go, that I may hide
    myself in the field till the third day at evening.  [6] If your father
    misses me at all, then say, ‘David earnestly asked leave of me to run to
    Bethlehem his city, for there is a yearly sacrifice there for all the
    clan.’ [7] If he says, ‘Good!’ it will be well with your servant, but if
    he is angry, then know that harm is determined by him.  [8] Therefore
    deal kindly with your servant, for you have brought your servant into a
    covenant of the LORD with you.  But if there is guilt in me, kill me
    yourself, for why should you bring me to your father?” [9] And Jonathan
    said, “Far be it from you!  If I knew that it was determined by my father
    that harm should come to you, would I not tell you?” [10] Then David said
    to Jonathan, “Who will tell me if your father answers you roughly?” [11]
    And Jonathan said to David, “Come, let us go out into the field.” So they
    both went out into the field.

    [12] And Jonathan said to David, “The LORD, the God of Israel, be
    witness!  When I have sounded out my father, about this time tomorrow, or
    the third day, behold, if he is well disposed toward David, shall I not
    then send and disclose it to you?  [13] But should it please my father to
    do you harm, the LORD do so to Jonathan and more also if I do not
    disclose it to you and send you away, that you may go in safety.  May the
    LORD be with you, as he has been with my father.  [14] If I am still
    alive, show me the steadfast love of the LORD, that I may not die; [15]
    and do not cut off your steadfast love from my house forever, when the
    LORD cuts off every one of the enemies of David from the face of the
    earth.” [16] And Jonathan made a covenant with the house of David,
    saying, “May the LORD take vengeance on David’s enemies.” [17] And
    Jonathan made David swear again by his love for him, for he loved him as
    he loved his own soul.

    [35] In the morning Jonathan went out into the field to the appointment
    with David, and with him a little boy.  [36] And he said to his boy, “Run
    and find the arrows that I shoot.” As the boy ran, he shot an arrow
    beyond him.  [37] And when the boy came to the place of the arrow that
    Jonathan had shot, Jonathan called after the boy and said, “Is not the
    arrow beyond you?” [38] And Jonathan called after the boy, “Hurry!  Be
    quick!  Do not stay!” So Jonathan’s boy gathered up the arrows and came
    to his master.  [39] But the boy knew nothing.  Only Jonathan and David
    knew the matter.  [40] And Jonathan gave his weapons to his boy and said
    to him, “Go and carry them to the city.” [41] And as soon as the boy had
    gone, David rose from beside the stone heap and fell on his face to the
    ground and bowed three times.  And they kissed one another and wept with
    one another, David weeping the most.  [42] Then Jonathan said to David,
    “Go in peace, because we have sworn both of us in the name of the LORD,
    saying, ‘The LORD shall be between me and you, and between my offspring
    and your offspring, forever.’” And he rose and departed, and Jonathan
    went into the city.
  \end{quote}
\subsection*{Notes}
\begin{itemize}
  \item{Scripture describes the friendship between Jonathan and David as
  covenantal, and very deep.  In today's age, some of us find this weird.
  This is because our culture has elevated romance to the highest form of
  love, so that the deep love that David and Jonathan share is interpreted in
  that light by some.}
  \item{Our culture's obsession with romance has led us to undervalue other
  human relationships, especially friendships.  Friendships have the
  potential to be deeper, stronger and more enduring that romantic
  relationships.}
  \item{The Christian tradition has never elevated romantic relationships
  above friendship.  CS Lewis has likened friendship to the love between
  angels, for example.}
  \item{A spiritual friendship is a friendship founded on the love of God,
  moved by the pursuit of God, and borrught into and experience of God.
  These are the three points for today.}
  \item{A spiritual friendship is founded on the love of God.  Both Jonathan
  and David knew God's love, and they embodied that love in their
  relationship.  What is happening in our text today is that Saul has been
  trying to assasinate David, and hence he has tried to invite him to a
  banquet.  David knows this, and hence he approached Saul's son, Jonathan,
  for help.  Jonathan expects David to ascend Israel's throne, and this means
  that Jonathan is actually David's direct's competitor to the throne.  But
  Jonathan knew the steadfast love of the Lord, and hence he showed David
  that same steadfast love.  Both Jonathan and David knew the steadfast love
  of the Lord themselves first, so that is how they can show it to each
  other.  This necessitates that we can only have spiritual friendship with
  other Christians.  Spiritual friends have been described as friends that
  agree in both human and divine things.}
  \item{Spiritual friendships are covenantal in nature, spritual friends
  aren't just here today and gone tomorrow.  While it is true that friends
  can come and go in different seasons of life, if we remember that spiritual
  friendship is covenantal, we ought to put in effort to preserve such
  friendships, even if the circumstances of life dictate otherwise (e.g if
  people migrate to another country).  And while spiritual friendships
  between sinners will never be perfect, we should be committed to working
  things like disagreements out.}
  \item{Spiritual friendships are sacrifical in nature, spiritual friends
  care for each other just like how God cares for us.  In the passage today,
  Jonathan risked his life to save David.  Our Lord Jesus Christ affirms this
  in John 15, when He said that ``greater love has no one than this, to lay
  down his life for his friends''.  Most of us today won't be in a situation
  when we would need to die for our friends, but since spiritual friendships
  require commitment and hence sacrifice, we must be prepared to do so.  An
  application of this is to stay close to your friends.}
  \item{Spiritual friendships are confidential in nature.  This doesn't mean
  that we need to hide our friendships from other people, but this means that
  spiritual friends confide in one another.  Here we see David seeking out
  Jonathan to share about his fears about Jonathan's father.  Spiritual
  friends share what is on their hearts with each other, even secret things.
  Our Lord Jesus affirms this too, as in John 15.  If this is the case, then
  this doesn't mean that we can be spiritual friends with everyone in church.
  We might have $100$ friends in church, but we might only confide in a few.
  Most of the time, it boils down to whether we can resonate with the other
  person.  The good thing though is that unlike marriages, spiritual
  friendships are not exclusive.  While we can't be spiritual friends with
  everyone in church, we should be open to be spiritual friends with
  Christians we talk to.}
  \item{A spiritual friendship is also moved by the pursuit of God.
  Spiritual friends have a common love for God, so a spiritual friendship
  entails pursuing God and supporting one another along the way.  Here, we
  see that Jonathan knows that the will of God is for David to become king,
  which means that his family will be destroyed because of his dad's enmity
  with David, yet Jonathan still aids David in his ``destiny'' to become
  king.  Spiritual friends support and help one another to do God's will,
  which for us is our sanctification.}
  \item{Spiritual friends encourage each other, a spiritual friend can be a
  channel of divine grace, when the going gets tough.  We see Jonathan doing
  this for David.  Encouragement can be as simple as sharing how God has
  spoken to us with our Christian friends, etc.  Especially when we live in a
  secular age, we are always bombarded with the philosophies of secular
  humanism.  Unless we encourage each other in our pursuit of God, we will
  regress.}
  \item{Spiritual friends counsel and correct each other.  Especially in we
  get stuck in sin, we need spiritual friends to persuade us of the truth and
  wake us up from our spiritual slumber.  Sometimes this might actually
  require rebuke!  As Proverbs say, ``faithful are the wounds of a friend,
  profuse are the kisses of an enemy''.  It is the responsibility of
  spiritual friends to correct one another and to counsel one another, as per
  what is said in James 5:19-20.  It is better to correct your friend's error
  and to feel abit unpleasant about it, rather than to allow your friend to
  stumble.  In any case, a true spiritual friend will take your correction,
  though it might seem painful for a time.}
  \item{A spiritual friendship is brought into an experience of God.  When
  spiritual friends love each other with the love of God, their friendship
  becomes a place when a deeper experience of God's love can take place.
  When we taste the sweetness of spiritual friendships, it is used as a
  channel for us to experience Jesus' friendship for us.  Spiritual
  friendships that are founded on God and moved by a pursuit of God leads to
  us experiencing God.  Our Lord Jesus has called us friends, and our Lord
  Jesus is the best spiritual friend.}
\end{itemize}
  \section{17th July 2022: Where to find wisdom?}
\subsection*{Text: Ecclesiastes 7}
  \begin{quote}
    [1] A good name is better than precious ointment,
        and the day of death than the day of birth.
    [2] It is better to go to the house of mourning
        than to go to the house of feasting,
    for this is the end of all mankind,
        and the living will lay it to heart.
    [3] Sorrow is better than laughter,
        for by sadness of face the heart is made glad.
    [4] The heart of the wise is in the house of mourning,
        but the heart of fools is in the house of mirth.
    [5] It is better for a man to hear the rebuke of the wise
        than to hear the song of fools.
    [6] For as the crackling of thorns under a pot,
        so is the laughter of the fools;
        this also is vanity.
    [7] Surely oppression drives the wise into madness,
        and a bribe corrupts the heart.
    [8] Better is the end of a thing than its beginning,
        and the patient in spirit is better than the proud in spirit.
    [9] Be not quick in your spirit to become angry,
        for anger lodges in the heart of fools.
    [10] Say not, “Why were the former days better than these?”
        For it is not from wisdom that you ask this.
    [11] Wisdom is good with an inheritance,
        an advantage to those who see the sun.
    [12] For the protection of wisdom is like the protection of money,
        and the advantage of knowledge is that wisdom preserves the life of
        him who has it.
    [13] Consider the work of God:
        who can make straight what he has made crooked?


    [14] In the day of prosperity be joyful, and in the day of adversity
    consider: God has made the one as well as the other, so that man may not
    find out anything that will be after him.

    [15] In my vain life I have seen everything.  There is a righteous man
    who perishes in his righteousness, and there is a wicked man who prolongs
    his life in his evildoing.  [16] Be not overly righteous, and do not make
    yourself too wise.  Why should you destroy yourself?  [17] Be not overly
    wicked, neither be a fool.  Why should you die before your time?  [18] It
    is good that you should take hold of this, and from that withhold not
    your hand, for the one who fears God shall come out from both of them.

    [19] Wisdom gives strength to the wise man more than ten rulers who are
    in a city.

    [20] Surely there is not a righteous man on earth who does good and never
    sins.

    [21] Do not take to heart all the things that people say, lest you hear
    your servant cursing you.  [22] Your heart knows that many times you
    yourself have cursed others.

    [23] All this I have tested by wisdom.  I said, “I will be wise,” but it
    was far from me.  [24] That which has been is far off, and deep, very
    deep; who can find it out?

    [25] I turned my heart to know and to search out and to seek wisdom and the
    scheme of things, and to know the wickedness of folly and the foolishness
    that is madness.  [26] And I find something more bitter than death: the
    woman whose heart is snares and nets, and whose hands are fetters.  He who
    pleases God escapes her, but the sinner is taken by her.  [27] Behold, this
    is what I found, says the Preacher, while adding one thing to another to
    find the scheme of things—[28] which my soul has sought repeatedly, but I
    have not found.  One man among a thousand I found, but a woman among all
    these I have not found.  [29] See, this alone I found, that God made man
    upright, but they have sought out many schemes.
  \end{quote}
\subsection*{Notes}
\begin{itemize}
  \item{Recap: The word ``vanity'' in the ESV is translated from the Hebrew
  word for ``breath'' or ``vapor''.  For example, when we say that wealth is
  ``vanity'', we mean that just like a breath or a vapor, wealth is
  ephemeral, and we cannot really grasp it.  When we say toil is ``vanity'',
  we are saying that like vapor, our toil is futile.  When we say that life
  is ``vanity'', we are saying that life is vanity.  So here, the question
  is, how do we resolve the tension between what our faith teaches us, that
  life ought to be meaningful because we have God, and what we observe and
  experience in the struggles of life ``under the sun'' that tell us
  otherwise?  The key is that even when life doesn't make sense, when life is
  monotonous, when God seems absent, we should still fear God, do good, be
  joyful and thankful to God for food and drink that we receive as food from
  His hand.}
  \item{Our text for today, Ecclesiastes 7, is a lot like Proverbs.  There
  are many sayings about wisdom in this chapter.}
  \item{We can find wisdom in the house of mourning.  In contemporary
  language, it is better to attend funerals than go to parties.  The reason
  is that when we attend funerals, we become more aware of our own mortality.
  The physical body that we have is a tent that is easily collapsible.  In
  this sense, life is ``vanity''.  Some of us might think that we are ``too
  young'' to think about our mortality.  As the Psalmist says, ``teach us to
  number our days, that we may have a heart of wisdom''.  In light of our
  mortality, it is wise to ask ``how then shall we live''.  It is good for us
  to reflect on our lives constantly and see if we need to make any changes,
  so that we won't have any regrets when we live.  The goal is to be such
  that at the end of our days, Jesus can say to us: ``well done, my good and
  faithful servant''.  It is not as if we are saved \textit{by} our works, it
  is more of how we are saved \textit{for} good works.  As Paul said, we are
  to ``work out our own salvation with fear and trembling...''.  When we
  reflect about our lives, we must ask ourselves: are we living any different
  from how non-Christians are living?  Does our faith affect the way we
  relate with people?  Does our faith affect the way we relate with our work?
  Etc.  Or is there something that you feel God is asking you to do?  If so,
  then do not tarry thinking that you still have time; our death is just
  around the corner.}
  \item{We can also find wisdom in situations we have no control over
  (v13-18, v23-24).  Straight and crooked, adversity and prosperity, are all
  the work of God.  Things are the way God wants them to be, we can't
  overrule the work of the Almighty.
  % When the Preacher talks about things that are ``crooked'', we are not
  % saying that God is the source of evil (that cannot be true!)
  Sometimes, despite how much we try to keep fit, we still fall sick.  We
  might try our best to live uprightly, yet we are still victims of injustice
  and slander.  These injustices are things that we wish that we could
  straighten our but we can't.  The fact that sometimes God has made things
  crooked can lead us to fatalism, but it shouldn't.  It shouldn't because
  though we have no control over situations and no answers for certain
  situations, these situations help us realise that sometimes our wisdom is
  limited.  Knowing the limits of our wisdom is itself wisdom.  Instead of
  leading us to fatalism, this understanding of the limits of our own wisdom
  should lead us to trust in God, who alone is wise and good and has all the
  answers.  And in fact, just as how sorrow can bring more wisdom than
  laughter, adversity can bring more wisdom that prosperity, because both
  sorrow and adversity lead us to trust in God's goodness.  ``Better is the
  end of a thing than its beginning, therefore patience is better than
  pride''.  Rather than arrogantly thinking that we know best, we should
  patiently trust God, even when we don't have all the answers.  This is part
  of our discipleship and our sanctification.}
  \item{From verse 15 onwards to verse 18, the Preacher meanders abit.  When
  the Preacher says ``do not be overly righteous'', it is more of ``don't be
  self righteous''.  I.e, we should not think that we are more righteous than
  we really are.  When we think that we are more righteous than we are, we
  become bitter when bad things happen to us, because we think that we don't
  deserve bad things.  When we think that we are more righteous than we are,
  we become bitter and prideful at other people.  When the Preacher says ``do
  not be overly wicked'', it is more of ``don't deliberately commit sins''.
  As sinful people, we do sometimes sin unintentionally.  But we should not
  deliberately sin when we can avoid it, because when we deliberately sin, we
  destroy ourselves.}
  \item{From verse 24, we see that wisdom is always distant and difficult to
  find.  The reason for that is that in verse 20 and verse 29, the reason why
  wisdom is hard to find is because of the sin of humanity.  The troubles of
  life can be traced to the problem of sin!  The Preacher did not in the book
  give answers to this problem of sin, apart from reminding us that God will
  finally judge and that we ought to fear God.  To get more definite answers
  to the problem of sin, we must turn to the New Testament.}
  \item{As in our Scripture reading today (1 Corinthians 1:18-24), Jesus
  Christ is the wisdom of God.  Jesus is the solution to the root problem of
  sin that is the cause of all the vanity in this life.  Jesus is our hope in
  life, however troublesome life is.  Whenever we look at the cross, whenever
  we look at Jesus, however troublesome life seems to be, we can rest assured
  that we will not fall outside God's loving hand (see more in Romans
  8:32-39).  It doesn't make sense for God to save us and then let us be
  destroyed by the trials in this world.  Rather, after we are saved, even
  the trials in this world become means through which God works all things
  for our good.  Even our death, which is inevitable, will not destroy us,
  because Jesus has destroyed the sting of death.  Just as Jesus was
  resurrected, we too can look forward to a future resurrection.}
\end{itemize}
  \section{17th July 2022: Where to find wisdom?}
\subsection*{Text: Ecclesiastes 7}
  \begin{quote}
    [1] A good name is better than precious ointment,
        and the day of death than the day of birth.
    [2] It is better to go to the house of mourning
        than to go to the house of feasting,
    for this is the end of all mankind,
        and the living will lay it to heart.
    [3] Sorrow is better than laughter,
        for by sadness of face the heart is made glad.
    [4] The heart of the wise is in the house of mourning,
        but the heart of fools is in the house of mirth.
    [5] It is better for a man to hear the rebuke of the wise
        than to hear the song of fools.
    [6] For as the crackling of thorns under a pot,
        so is the laughter of the fools;
        this also is vanity.
    [7] Surely oppression drives the wise into madness,
        and a bribe corrupts the heart.
    [8] Better is the end of a thing than its beginning,
        and the patient in spirit is better than the proud in spirit.
    [9] Be not quick in your spirit to become angry,
        for anger lodges in the heart of fools.
    [10] Say not, “Why were the former days better than these?”
        For it is not from wisdom that you ask this.
    [11] Wisdom is good with an inheritance,
        an advantage to those who see the sun.
    [12] For the protection of wisdom is like the protection of money,
        and the advantage of knowledge is that wisdom preserves the life of
        him who has it.
    [13] Consider the work of God:
        who can make straight what he has made crooked?


    [14] In the day of prosperity be joyful, and in the day of adversity
    consider: God has made the one as well as the other, so that man may not
    find out anything that will be after him.

    [15] In my vain life I have seen everything.  There is a righteous man
    who perishes in his righteousness, and there is a wicked man who prolongs
    his life in his evildoing.  [16] Be not overly righteous, and do not make
    yourself too wise.  Why should you destroy yourself?  [17] Be not overly
    wicked, neither be a fool.  Why should you die before your time?  [18] It
    is good that you should take hold of this, and from that withhold not
    your hand, for the one who fears God shall come out from both of them.

    [19] Wisdom gives strength to the wise man more than ten rulers who are
    in a city.

    [20] Surely there is not a righteous man on earth who does good and never
    sins.

    [21] Do not take to heart all the things that people say, lest you hear
    your servant cursing you.  [22] Your heart knows that many times you
    yourself have cursed others.

    [23] All this I have tested by wisdom.  I said, “I will be wise,” but it
    was far from me.  [24] That which has been is far off, and deep, very
    deep; who can find it out?

    [25] I turned my heart to know and to search out and to seek wisdom and the
    scheme of things, and to know the wickedness of folly and the foolishness
    that is madness.  [26] And I find something more bitter than death: the
    woman whose heart is snares and nets, and whose hands are fetters.  He who
    pleases God escapes her, but the sinner is taken by her.  [27] Behold, this
    is what I found, says the Preacher, while adding one thing to another to
    find the scheme of things—[28] which my soul has sought repeatedly, but I
    have not found.  One man among a thousand I found, but a woman among all
    these I have not found.  [29] See, this alone I found, that God made man
    upright, but they have sought out many schemes.
  \end{quote}
\subsection*{Notes}
\begin{itemize}
  \item{Recap: The word ``vanity'' in the ESV is translated from the Hebrew
  word for ``breath'' or ``vapor''.  For example, when we say that wealth is
  ``vanity'', we mean that just like a breath or a vapor, wealth is
  ephemeral, and we cannot really grasp it.  When we say toil is ``vanity'',
  we are saying that like vapor, our toil is futile.  When we say that life
  is ``vanity'', we are saying that life is vanity.  So here, the question
  is, how do we resolve the tension between what our faith teaches us, that
  life ought to be meaningful because we have God, and what we observe and
  experience in the struggles of life ``under the sun'' that tell us
  otherwise?  The key is that even when life doesn't make sense, when life is
  monotonous, when God seems absent, we should still fear God, do good, be
  joyful and thankful to God for food and drink that we receive as food from
  His hand.}
  \item{Our text for today, Ecclesiastes 7, is a lot like Proverbs.  There
  are many sayings about wisdom in this chapter.}
  \item{We can find wisdom in the house of mourning.  In contemporary
  language, it is better to attend funerals than go to parties.  The reason
  is that when we attend funerals, we become more aware of our own mortality.
  The physical body that we have is a tent that is easily collapsible.  In
  this sense, life is ``vanity''.  Some of us might think that we are ``too
  young'' to think about our mortality.  As the Psalmist says, ``teach us to
  number our days, that we may have a heart of wisdom''.  In light of our
  mortality, it is wise to ask ``how then shall we live''.  It is good for us
  to reflect on our lives constantly and see if we need to make any changes,
  so that we won't have any regrets when we live.  The goal is to be such
  that at the end of our days, Jesus can say to us: ``well done, my good and
  faithful servant''.  It is not as if we are saved \textit{by} our works, it
  is more of how we are saved \textit{for} good works.  As Paul said, we are
  to ``work out our own salvation with fear and trembling...''.  When we
  reflect about our lives, we must ask ourselves: are we living any different
  from how non-Christians are living?  Does our faith affect the way we
  relate with people?  Does our faith affect the way we relate with our work?
  Etc.  Or is there something that you feel God is asking you to do?  If so,
  then do not tarry thinking that you still have time; our death is just
  around the corner.}
  \item{We can also find wisdom in situations we have no control over
  (v13-18, v23-24).  Straight and crooked, adversity and prosperity, are all
  the work of God.  Things are the way God wants them to be, we can't
  overrule the work of the Almighty.
  % When the Preacher talks about things that are ``crooked'', we are not
  % saying that God is the source of evil (that cannot be true!)
  Sometimes, despite how much we try to keep fit, we still fall sick.  We
  might try our best to live uprightly, yet we are still victims of injustice
  and slander.  These injustices are things that we wish that we could
  straighten our but we can't.  The fact that sometimes God has made things
  crooked can lead us to fatalism, but it shouldn't.  It shouldn't because
  though we have no control over situations and no answers for certain
  situations, these situations help us realise that sometimes our wisdom is
  limited.  Knowing the limits of our wisdom is itself wisdom.  Instead of
  leading us to fatalism, this understanding of the limits of our own wisdom
  should lead us to trust in God, who alone is wise and good and has all the
  answers.  And in fact, just as how sorrow can bring more wisdom than
  laughter, adversity can bring more wisdom that prosperity, because both
  sorrow and adversity lead us to trust in God's goodness.  ``Better is the
  end of a thing than its beginning, therefore patience is better than
  pride''.  Rather than arrogantly thinking that we know best, we should
  patiently trust God, even when we don't have all the answers.  This is part
  of our discipleship and our sanctification.}
  \item{From verse 15 onwards to verse 18, the Preacher meanders abit.  When
  the Preacher says ``do not be overly righteous'', it is more of ``don't be
  self righteous''.  I.e, we should not think that we are more righteous than
  we really are.  When we think that we are more righteous than we are, we
  become bitter when bad things happen to us, because we think that we don't
  deserve bad things.  When we think that we are more righteous than we are,
  we become bitter and prideful at other people.  When the Preacher says ``do
  not be overly wicked'', it is more of ``don't deliberately commit sins''.
  As sinful people, we do sometimes sin unintentionally.  But we should not
  deliberately sin when we can avoid it, because when we deliberately sin, we
  destroy ourselves.}
  \item{From verse 24, we see that wisdom is always distant and difficult to
  find.  The reason for that is that in verse 20 and verse 29, the reason why
  wisdom is hard to find is because of the sin of humanity.  The troubles of
  life can be traced to the problem of sin!  The Preacher did not in the book
  give answers to this problem of sin, apart from reminding us that God will
  finally judge and that we ought to fear God.  To get more definite answers
  to the problem of sin, we must turn to the New Testament.}
  \item{As in our Scripture reading today (1 Corinthians 1:18-24), Jesus
  Christ is the wisdom of God.  Jesus is the solution to the root problem of
  sin that is the cause of all the vanity in this life.  Jesus is our hope in
  life, however troublesome life is.  Whenever we look at the cross, whenever
  we look at Jesus, however troublesome life seems to be, we can rest assured
  that we will not fall outside God's loving hand (see more in Romans
  8:32-39).  It doesn't make sense for God to save us and then let us be
  destroyed by the trials in this world.  Rather, after we are saved, even
  the trials in this world become means through which God works all things
  for our good.  Even our death, which is inevitable, will not destroy us,
  because Jesus has destroyed the sting of death.  Just as Jesus was
  resurrected, we too can look forward to a future resurrection.}
\end{itemize}
  \section{14th August 2022: Counsel for living when times are uncertain}
\subsection*{Text: Ecclesiastes 11:1-6}
  \begin{quote}
    [1] Cast your bread upon the waters,
        for you will find it after many days.
    [2] Give a portion to seven, or even to eight,
        for you know not what disaster may happen on earth.
    [3] If the clouds are full of rain,
        they empty themselves on the earth,
    and if a tree falls to the south or to the north,
        in the place where the tree falls, there it will lie.
    [4] He who observes the wind will not sow,
        and he who regards the clouds will not reap.

    [5] As you do not know the way the spirit comes to the bones in the womb
    of a woman with child, so you do not know the work of God who makes
    everything.

    [6] In the morning sow your seed, and at evening withhold not your hand,
    for you do not know which will prosper, this or that, or whether both
    alike will be good.
  \end{quote}
\subsection*{Notes}
\begin{itemize}
  \item{Wisdom is important in the life of Christians.  Hence, wisdom
  literature is important for Christians, not just in our thought life, but
  also in our practical life.  Proverbs contains ``conventional wisdom'',
  such as ``a soft answer turns away wrath, but a harsh word stirs up anger''
  (Proverbs 15:1).  Conventional wisdom works like this; when you observe
  swans, if $99$ swans are white, the $100$th would most likely be white too.
  The question is though, what if the $100$th swan is black?  In most cases,
  a gentle answer can de-escalate situations, but what if it doesn't?  It is
  part of wisdom itself to know the limits of conventional wisdom itself, and
  hence we see the limits of conventional wisdom in Job/Ecclesiastes.  This
  is important, because conventional wisdom might not always be applicable in
  our sinful, broken and complicated world.}
  \item{The theme for today is ``wise living in times of uncertainty''.  In
  verses 5-6, the phrase ``you do not know'' appears three times.  In fact,
  chapter 11 is not the first time the theme of ``not knowing'' has appeared
  in Ecclesiastes.  Previously, the theme of how Man ``cannot find out what
  God has done'' (Ecclesiastes 3:11), how Man ``does not know what is to be''
  (Ecclesiastes 8:17) and how Man ``cannot find out the work that is done
  under the sun'' (Ecclesiastes ?:??), and how Man ``does not know his time''
  (Ecclesiastes 9:12).  The uncertainty that the author of Ecclesiastes
  refers to is not uncertainty because of carelessness.  The uncertainty is
  more of how sometimes we can do everything we can, but we still can't find
  out.}
  \item{The key to Ecclesiastes 11:1-6 is found in the latter half of verse
  6; one simply does not know what God is going to do in the future.  One
  cannot be sure of what God is going to do.  So how then should one live?
  We can take all the safeguards humanly possibly, but we still can't assure
  ourselves of the future.}
  \item{Though there is this inherent uncertainty in life, there is still
  some advice that we can heed.  From verse 4, we see a caution against
  idleness/frozenness/inaction.  I.e, it cautions against
  ``analysis-paralysis''.  Doesn't make sense for us to not plant seeds
  because we keep trying to overanalyse the weather.  Though we can't predict
  the weather or whether or crops will receive rain, we should still plant
  seeds.  Though an over-analyser is not a strictly speaking a sluggard (more
  of a paralysed worrywart), the end result is the same (c.f Proverbs 20:4).
  Instead of over-analysing, the answer is in verse 6; in the face of
  uncertainty, we must work, and in fact, we do even more than usual so as to
  extend our safety net.  Not only do we sow our seed in the morning, but we
  also sow our seed in the evening.  Sowing our seed in the morning works
  fine in times of certainty, but in times of uncertainty, we should do more
  than usual.}
  \item{Of course we can go to the extreme and work until we drop dead so as
  to establish the security that we need in our heart.  Yet as was previously
  said, it is impossible to have complete security, because we will never be
  able to fully predict what God does in the future.  We know that clouds
  full of rain will empty themselves (v3), but yet this emptying is not under
  human control.  And as for the latter half of verse three, it is saying
  that we do not know where a tree is going to fall; it is not under human
  control.  ``North and south'' here is abit like ``high and low''.}
  \item{So the balance is this; we do what is responsible in uncertain times,
  which might include working extra, but we also should not try to control
  everything because that is impossible.  Instead of trying to control
  everything, we are to \textbf{trust God}.}
  \item{So a nice term for today is ``the grace of not knowing''.  The grace
  of not knowing what lies ahead, for good or for will, because it frees us
  from the compulsion to control our situation, to secure our own advantage
  in everything and kick ourselves -or curse God- when we guess wrong.  Hence
  while for some of us we might need to do more in times of uncertainty
  (which is responsible), for some of us we actually need to do less and
  trust God more.}
  \item{In v1-2, the idea of ``casting our bread upon the waters'' has two
  main interpretations.  One interpretation is about how we should make wise
  investments.  And when we make these investments, we should diversify our
  portfolio, make ``seven'' or ``eight'' investments, because we don't know
  where disaster will strike.  So here, ``casting your bread'' would be how
  we have ships filled with bread for foreign investment.  Hence, this view
  is more self-focused.}
  \item{In v1-2, the other interpretation is about acts of charity.  Times
  are uncertain not only for you, but also for others.  Hence, this view is
  more other-centred.  When we ``live by the grace of not knowing'', we are
  pushed to also give of ourselves to others, because its like ``since we
  can't predict what will happen to us in the future anyway, we might as well
  go and help others, since they are also facing uncertain times''.  This is
  the favored interpretation of the preacher, and is my favored
  interpretation too.}
\end{itemize}
  \section{28th August 2022: Hear it and take it to heart}
\subsection*{Text: Revelation 1:1-8}
  \begin{quote}
    [1] The revelation of Jesus Christ, which God gave him to show to his
    servants the things that must soon take place.  He made it known by
    sending his angel to his servant John, [2] who bore witness to the word
    of God and to the testimony of Jesus Christ, even to all that he saw.
    [3] Blessed is the one who reads aloud the words of this prophecy, and
    blessed are those who hear, and who keep what is written in it, for the
    time is near.

    [4] John to the seven churches that are in Asia:

    Grace to you and peace from him who is and who was and who is to come,
    and from the seven spirits who are before his throne, [5] and from Jesus
    Christ the faithful witness, the firstborn of the dead, and the ruler of
    kings on earth.

    To him who loves us and has freed us from our sins by his blood [6] and
    made us a kingdom, priests to his God and Father, to him be glory and
    dominion forever and ever.  Amen.  [7] Behold, he is coming with the
    clouds, and every eye will see him, even those who pierced him, and all
    tribes of the earth will wail on account of him.  Even so.  Amen.

    [8] “I am the Alpha and the Omega,” says the Lord God, “who is and who
    was and who is to come, the Almighty.”
  \end{quote}
\subsection*{Notes}
\begin{itemize}
  \item{The book of Revelation is a fascinating book, and is often abused by
  unscrupulous teachers who take advantage of people's curiosity about the
  future.  We should not let current world events affect the way we interpret
  the text, we should let the text speak for itself.}
  \item{These eight verses set the tone for how we should understand the rest
  of the book.  Three points: firstly, the book of Revelation is a gift to
  us.  Secondly, these eight verses give us the foundational theology for
  understanding the book.  Lastly, these eight verses give us the testimony
  of the people of God, about what we should do after hearing the
  Revelation.}
  \item{From verse 1, we see that Revelation is a gift of God to Jesus, and
  then from Jesus to John, and then from John to us.  Since Revelation is
  God's gift to us, we should be able to understand it and take to heart all
  that the Lord is saying in the book of Revelation.  God doesn't give us
  gifts we don't understand.  We might not understand it fully, but we will
  know enough.}
  \item{This revelation is about Jesus Christ, from verse 1.  This means that
  a lot of the book is about Jesus.  We should not concern ourselves too much
  with things like ``rapture'', ``tribulation'', etc at the expense of
  forgetting about Jesus.  }
  \item{This book has a very strong sense of the second coming of Jesus
  Christ.  This book is given us for us to prepare ourselves in light of the
  coming of the last days.  Revelation is given to us a guidance for the last
  days, when Jesus' coming is imminent.}
  \item{From ESV, the book of Revelation also comes with a blessing even for
  the one who read it and those who hear, and most important for those who
  keeps what is written in the book.  We are to hear the words of the book,
  and take the words of the book to heart.  This is similar to Psalm 1;
  blessed is the man who does not walk in the way of sinners, who does not
  sit in the seat of scorners, but who meditates on the Law of the Lord day
  and night.  Similarly, the one who reads, hears and takes to heart the book
  of Revelation will be blessed.}
  \item{So, how are we to face the last days?  Thought experiment time;
  imagine for yourselves a future that you really want to come to pass.  Yet
  our imaginations about the future might be wrong, hence it is helpful for
  Jesus to guide our futures.  We think/imagine things that are good for us
  that aren't actually good for us.  What is good for us is God, since God is
  the ultimate good, yet most of the time our imaginations about the future
  doesn't have God in them.  The book of Revelation stresses this by spending
  five verses on giving titles for God, to expand our idea of God.}
  \item{ From verse 4, we see that God is the One who is, who was, and who is
  to come.  This is a throwback to the story of Moses at the burning bush,
  where God told Moses that ``I am who I am''.  God doesn't need any
  definition, He is the one who is the ultimate reality from which all
  reality is defined, He is who He is.  This also means that God is eternal,
  since God who was.  And lastly, God is the one who is to come; present
  tense with a future looking aspect, which means that God's future is
  defined by how He will come for us.}
  \item{God is also called the ``Alpha'' and the ``Omega'', which means that God is the start and the end of reality (since $\alpha$ and $\Omega$ are the first and last greek alphabets).}
  \item{Jesus Christ here is called the faithful witness. }
  \item{The seven spirits here refers to the Holy Spirit.  Seven is the
  number of perfection in OT, which means that the seven spirits talk about
  how the Spirit is perfect.  We see a clue about how the seven spirits refer
  to the Holy Spirit by referring the Revelation 4 which talk about the seven
  spirits together with the seven lights and the seven eyes, which is a
  throwback to the same analogy in Zechariah 4.  In Zechariah, we see that
  the people of God were very discouraged because they were facing lots of
  troubles, and God encouraged them that they will succeed not by might or by
  numbers, but by His Spirit.  And similarly, for us today, our Church might
  be facing lots of pressures, and we might seem to be struggling, yet we
  know that in the end, we will succeed by the Holy Spirit.}
  \item{With all of the theological foundations in place for us to understand
  the future, what is the purpose for us to understand this future?  As in,
  what should we do after we are able to understand the future?  The idea is
  that after we understand the book, we will be prepared to give testimony of
  God in the last days.  The last days will be challenging, it will be
  urgent, and it means that we also have to be decisive in these last days.
  Tracts and social media are not enough; we need people who are actually
  able to give testimony of God through their words and more importantly
  their lives.} 
  \item{In the modern era, when there is so much competition of
  ideas, how do people know whom to trust?  Sometimes people compare
  different ideas by comparing the logic behind the ideas to see which idea
  is correct, but most of the time people compare different ideas by
  comparing whether the different speakers are trustworthy.  And we can prove
  ourselves to be trustworthy speakers if we live consistently with what we
  say about Jesus.  And eventually, though we might not be able to win all
  the verbal arguments, we might be able to convince people through our
  lives.  And this testimony we give through our lives are eternal, though
  our lives on earth are not, because our lives in heaven are eternal, and at the last days, we will be vindicated, though we might have passed on.  }
\end{itemize}
  \section{18th September 2022: Acceptance in rejection}
\subsection*{Text: Revelation 2:8-11}
  \begin{quote}
    [8] “And to the angel of the church in Smyrna write: ‘The words of the
    first and the last, who died and came to life.

    [9] “‘I know your tribulation and your poverty (but you are rich) and the
    slander of those who say that they are Jews and are not, but are a
    synagogue of Satan.  [10] Do not fear what you are about to suffer.
    Behold, the devil is about to throw some of you into prison, that you may
    be tested, and for ten days you will have tribulation.  Be faithful unto
    death, and I will give you the crown of life.  [11] He who has an ear,
    let him hear what the Spirit says to the churches.  The one who conquers
    will not be hurt by the second death.’
  \end{quote}
\subsection*{Notes}
\begin{itemize}
  \item{God made us social creatures, so we have an innate desire for close
  lasting relationships with others.  A collorary is that we desire social
  acceptance, we want to be accepted by our peers, by our society, etc.
  Hence, a fear of social rejection is common among us humans, as we know
  from our own experience.}
  \item{Today's text is about the church in Smyrna, who are facing religious
  persecution.  The church in Singapore is not facing religious persecution
  of the same kind as the church in Smyrna.  In those days, everyone was
  supposed to sacrifice to the Roman empires and the Roman gods.  Except the
  Jews, who were officially granted an exclusion to that.  At first,
  Christians were worshipping together with Jews in their synagogues, but as
  the number of Christians grew and as the theological differences between
  Jews and Christians grew, the Jews kicked the Christians out of the
  synagogues.  Then, the Jews leveraged the power of the majority of society
  and the power of the state to force the Christians to give up their faith,
  by slandering them and painting a bad picture of them (v9).  As a result,
  the Smyrna society rejected the Christians and excluded them from social
  life.  This led to the poverty of the Christians and etc.}
  \item{In Singapore, we aren't facing such open persecution, but we are
  facing a more subtle form of persecution.  This is the result of society
  being more secularised.  More and more people don't want the Church to have
  a voice in the public square.  The Church is mocked when she talks about
  sin and judgment, and in modern days, the Church is slandered and shunned
  especially when she talks about the Christian view of sexuality.  This
  causes us to subtly shift our values etc, because of a fear of social
  rejeciton.}
  \item{Hence, today's message: when the world rejects you, don't be afraid
  to stand firm for your faith. }
  \item{First reason why we should stand firm: when people reject us, it
  frees us for God.  The more we want to be accepted by others, the more we
  want to care about other's opinion of us, and as a result, the harder it is
  for us to make godly choices because of peer pressure.  But if we embrace
  the rejection of people, we are freed to make godly choices.  The
  Christians in Smyrna had a choice to make; they could have distanced
  themselves from the slander by giving up the public profession of their
  faith, or they could embrace the slander that came with the public
  profession of their faith (it was hard for them to refute the slander,
  since they were in the minority).  But when we accept that being rejected
  by society is an inevitable result of following Jesus, then we can be free
  to ignore public perception and continue to live a faithful life.  Jesus
  promises us that though the world rejects us, He will accept us.  He will
  give us our treasures in heaven, and He will glorify us.}
  \item{Second reason why we should stand firm: when people reject us, it
  refines us for God.  A writer says (and I summarise): ``The church is the
  strongest when it is culturally persecuted, and the church is the weakest
  when it seems identical with the culture''.  It seems from the text that
  society was taking the slander very seriously, and even sending the
  Christians in Smyrna to prison.  The $10$ days in our text suggests that
  the Christians in Smyrna would suffer and be tortured for $10$ days, before
  being executed, as seen from the exhortation to be ``faithful unto
  death''.  God permits us to be tested by the devil, so that it refines our
  faith.  When we are victorious over the devil and his testing, then we will
  be victorious over the second death.  Yet when we fail the devil's test and
  give up our faith in the face of persecution, we show that we are not
  really Christians, and hence we will be hurt by the second death.}
  \item{Third reason why we should stand firm: God Himself is with us in our
  suffering.  We naturally think that God is with us when things are going
  smoothly, for example when Church attendance is booming.  A corollary is
  that we also naturally think that God is not with us or even against us
  when things are going poorly, for example when we have terminal illness.
  The Christians in Smyrna were tempted to think that because of the
  suffering that they are facing, God might not be with them.  Yet that is a
  wrong thought.  The paradox of Christianity is that God is most revealed at
  the cross.  As Paul says, the crucified Christ is a stumbling block to the
  Jews and folly to the Gentiles.  People only imagine God coming in glory,
  but not God coming in suffering and rejection.  For us who understand the
  gospel, Christ crucified is the power of God and the wisdom of God.  Hence,
  when we carry the cross in this life like our Lord Jesus, then God is most
  with us.  Jesus lived a faithful and perfect life, that's why the world
  despised Him and rejected Him.  But in spite of the world's rejection of
  Jesus, Jesus is the Father's beloved Son, and the Father is with Jesus.
  Similarly, when we live faithful lives, the world will naturally reject us,
  but yet like Jesus, the Father will be with us in our suffering through His
  Spirit, the same Spirit that landed on Jesus at His baptism.  }
  \item{And lastly, the Lord's supper is a gracious reminder of all of the above.}
\end{itemize}
  \section{25th September 2022: There is no room for compromise}
\subsection*{Text: Revelation 2:12-17}
  \begin{quote}
    [12] “And to the angel of the church in Pergamum write: ‘The words of him
    who has the sharp two-edged sword.

    [13] “‘I know where you dwell, where Satan’s throne is.  Yet you hold
    fast my name, and you did not deny my faith even in the days of Antipas
    my faithful witness, who was killed among you, where Satan dwells.  [14]
    But I have a few things against you: you have some there who hold the
    teaching of Balaam, who taught Balak to put a stumbling block before the
    sons of Israel, so that they might eat food sacrificed to idols and
    practice sexual immorality.  [15] So also you have some who hold the
    teaching of the Nicolaitans.  [16] Therefore repent.  If not, I will come
    to you soon and war against them with the sword of my mouth.  [17] He who
    has an ear, let him hear what the Spirit says to the churches.  To the
    one who conquers I will give some of the hidden manna, and I will give
    him a white stone, with a new name written on the stone that no one knows
    except the one who receives it.’
  \end{quote}
\subsection*{Notes}
\begin{itemize}
  \item{To each of the seven churches, Christ conveys to them something of
  Him that each church needs to know.  So far, everything has been kinda
  positive (e.g see the letter to Ephesus and Smyrna).  For Pergamum, Christ
  is described as the one who has the sharp double edged sword, which sounds
  terrifying at first.  But more of that later.}
  \item{The city of Pergamum was the capital of Asia minor before Caesar
  Augustus changed the capital to Ephesus.  This was another city where the
  worship of the Roman emperor was popular.  This was the first city in Asia
  minor to build a temple to Caesar.  Thus when Pergamum is described as a
  ``city where Satan dwells'', it might be a reference to this emperor cult.
  Tradition has it that Antipas was a bishop of Pergamum who was roasted
  alive.}
  \item{First, the church at Pergamum was commended for holding fast to the
  name of Jesus.  One's name contains his attributes and his characteristics.
  Holding fast to Jesus' name means believing everything that he has decalred
  about himself as revealed in the Word of God.  This is also what it means
  to fix our eyes on Jesus, which is to believe in everything Jesus said
  about himself and keep faith.}
  \item{Next, the church at Pergamum was condemned for something that the
  Lord Jesus found wanting.  This follows the usual pattern of the letter to
  the seven churches.  For Pergamum, some in the church followed the
  teachings of Baalam and the Nicolaitans.  What Balaam did in the past was
  to influence the people of Israel to follow in the faith of the surrounding
  Canaanites, by teaching them to worship the gods of the Canaanites and by
  teaching them to take the Canaanite women as wives.  So in short, Balaam
  taught the Israelites in the past to be influenced by their surrounding
  culture especially by teaching the Moabite women to cause the Israelite men
  to act treacherously (c.f Numbers 31).  So the reference to Balaam for the
  Pergamum church means that some of the Christians in the Pergamum church
  were doing the same things as their pagan neighbours in doing pagan
  worship.  And as part of ancient pagan worship, it was customary to have
  sex wtih the temple prostitutes, i.e sexual immorality.  An ancient author
  says that Pergamum was given to pagan worship more than all of the cities
  of Asia minor.}
  \item{Are we like the church in Pergamum, i.e are we compromising with the
  world?  Is the church being secularised?  An example is the prosperity
  gospel.  The world treats material wealth as a form of success, and hence
  when the church holds up material wealth as the final goal of Christian
  worship, then that is a compromise.  These churches also do not talk about
  hell and the wrath of God on sin.  Also, some churches and some churchgoes
  have a consumeristic mindset, which is a worldly virtue, not a Christian
  one.  Worship should be us offering our service to God, not about how we feel (because if it is about how we feel, we will hop from church to church to find the church that gives us the best feeling).}
  \item{The Pergamum church is compromising by worship to idols, and this is
  a lesson for us too.  While we do not worship idols of wood and stone, we
  can still set up idols in our hearts (Ezekiel 14:13).  Usually for us,
  these idols are good things (like relationships with people) that become
  ultimate things.  This is what happens when we love God's gifts more that
  God, and as a result, we set up God's gifts as an idol. And this idolatry often hurt us and the people around is.  For parents, they
  might idolise the success of their kid, and love the idea of their kids
  success more than God.  For some people, they might idolise the idea of a
  romantic relationship and place their ultimate happiness in their spouse, a
  burden their spouse cannot bear.  }
  \item{Helpful framework to identify idols: 
  \begin{itemize}
    \item{Imagination - what do you habitually think about to get joy and comfort in the privacy of your heart?}
    \item{Expenditures - where do we spend bulk of our money on to give you pleasure?}
    \item{Prayers - what do you most often pray about and how do you respond to unanswered prayers and frustrated hopes?}
    \item{Emotions - look at your most uncontrollable emotions, especially those what drive you to do things that are wrong.}
  \end{itemize}
  If the answer to any of these is not God, then...  haha.  Oof.}
  \item{After condemning some of the behaviours of the Pergamum church, there
  is an exhortation to repent.  Repentance would lead to eternal life (here
  it is described as manna and a white stone with a new name), and a lack of
  repentance would lead to judgment (Jesus will fight them with the sword of
  his mouth).  The reference to manna makes sense because of the Pergamum
  church's compromise by eating food offered to idols, and the reference to a
  white stone is because in those days, it is what is given to athletes to
  mark their victory.}
\end{itemize}
  \section{9th October 2022: Letter to the church at Sardis}
\subsection*{Text: Revelation 3:1-6}
  \begin{quote}
    [1] “And to the angel of the church in Sardis write: ‘The words of him
    who has the seven spirits of God and the seven stars.

    “‘I know your works.  You have the reputation of being alive, but you are
    dead.  [2] Wake up, and strengthen what remains and is about to die, for
    I have not found your works complete in the sight of my God.  [3]
    Remember, then, what you received and heard.  Keep it, and repent.  If
    you will not wake up, I will come like a thief, and you will not know at
    what hour I will come against you.  [4] Yet you have still a few names in
    Sardis, people who have not soiled their garments, and they will walk
    with me in white, for they are worthy.  [5] The one who conquers will be
    clothed thus in white garments, and I will never blot his name out of the
    book of life.  I will confess his name before my Father and before his
    angels.  [6] He who has an ear, let him hear what the Spirit says to the
    churches.’
  \end{quote}
\subsection*{Notes}
\begin{itemize}
  \item{Sardis, in the Persian period, was a bustling and powerful city
  because its geographical location offered it a lot of natural defenses.
  But by the Roman period, the city has declined a lot cause of war, and
  hence the city can only look back on its glorious past.}
  \item{The letter to the church in Sardis has no commendation, but only
  condemnation.  Compare this with the letter to the church in Smyrna, who
  only had commendation.}
  \item{The church in Sardis had a particular reputation of being alive.
  They received and heard the gospel (v3).  Perhaps in the past they were
  faithful because of the gospel they heard and obeyed, but today they are
  all unfaithful yet their reputation for being faithful and alive persisted.
  Perhaps some of them rested on their laurels about the good old days of
  their faith and thus neglected to work on their faith currently.  This is
  quite similar to the state of Sardis itself, who had an illustrive history
  but is currently rabak.}
  \item{There is nothing wrong for us to look back on the good old days where
  we have been zealous for the Lord and give thanks.  But the important
  question is; what about now?  Are we still passionate for God?  Do we still
  love God as much?  Or do we have the ``been there done that'' mentality
  when it comes to our r/s with God?  This ``been there done that'' mentality
  is dangerous!}
  \item{A quote: ``there is no such thing as stagnating faith.  If your faith
  is not growing, it is likely dying''.  Sometimes, our faith can be dying
  yet we still have a good reputation among people, i.e people who think we
  are very Christian (even to our closest friends!).  We must not deceive
  ourselves by resting on our reputation if we feel like our faith is dying.}
  \item{For us, we must ask ourselves whether we have an abiding r/s with
  Christ and if we are filled with His Spirit (so that we are in the
  invisible church), because if not, we are like whitewashed tombs, outwardly
  looking legit but inside we are totally dead.  Jesus says to the church in
  Sardis: ``\textbf{I know your works}, you have the reputation of being
  alive, but you are dead''.  The phrase: ``I know your works'' is repeated
  to the letters to many of the seven churches here.  The ``works'' here
  include the totality of our Christian conduct, like our acts of service and
  the state of our heart.  For the church at Sardis, their works are
  ``incomplete'', i.e there is something missing in their Christian life,
  perhaps they are acting hypocritically like the Pharisees (got reputation
  but actually dead).  The exhortation to the church at Sardis then is to
  ``remember then what you have heard'', i.e which is to go back to the
  gospel.  The key is to go back to the gospel and be wowed and astounded by
  the magnitude of God's love for us, believe in Jesus who gave us life, and
  then let our gratitude give us motivation to life a Christian life for
  Jesus.}
  \item{The church in Sardis can be described by 2 Time 3:1-5, which says
  especially: ``having the appearance of godliness, but denying its power''.
  }
  \item{The problem with the church in Sardis is that most of them have
  ``soiled their garments''.  From this letter here, the church in Sardis had
  no real problem with persecution, perhaps because most of them are living
  unclean lives like the world anyway.  For us to reflect on: ``have we
  imbibed worldly values into our lives and into the life of our church so
  that we are no different form the world?'' Because if we have, then we have
  soiled our garments. A few examples:
  \begin{itemize}
    \item{In the world, when people are hurt, they just walk away.  But this
    is not what we should do in church.  We are to forgive one another and to
    seek reconciliation with one another.  Practically, what this looks like
    is to talk to the person who has hurt you, and if that is currently too
    hard, then at least talk to the leaders first before leaving.}
    \item{In the world, people are abit too judgey.  We should not be too
    quick to judge in church, especially when people are sharing their
    struggles/sins and their thanksgiving (not everyone who is sharing
    thanksgiving is being boastful).}
    \item{In the increasingly secularised world, the mantra is ``you can
    believe what you want but don't shove your religion down my throat'.
    Would that make us too ashamed of the gospel to share our gospel?  Or do
    we boldly proclaim the gospel and invite people to accept Jesus and
    receive eternal life?} 
  \end{itemize}}
  \item{We must keep watch over our lives and not rest on our reputation.  We
  are saved by faith, but good works and a holy life necessarily flow from
  our faith.  And the good works and a holy life that we produce by our faith
  aren't by our own effort, but by the Holy Spirit who Jesus holds in His
  hand.  Jesus in His hand holds both the churches and the Spirit (v1), and
  He can bring the two together.}
\end{itemize}
  \section{13th November 2022: A missional life}
\subsection*{Text: 1 Thessalonians 1:4-10}
  \begin{quote}
    [4] For we know, brothers loved by God, that he has chosen you, [5]
    because our gospel came to you not only in word, but also in power and in
    the Holy Spirit and with full conviction.  You know what kind of men we
    proved to be among you for your sake.  [6] And you became imitators of us
    and of the Lord, for you received the word in much affliction, with the
    joy of the Holy Spirit, [7] so that you became an example to all the
    believers in Macedonia and in Achaia.  [8] For not only has the word of
    the Lord sounded forth from you in Macedonia and Achaia, but your faith
    in God has gone forth everywhere, so that we need not say anything.  [9]
    For they themselves report concerning us the kind of reception we had
    among you, and how you turned to God from idols to serve the living and
    true God, [10] and to wait for his Son from heaven, whom he raised from
    the dead, Jesus who delivers us from the wrath to come.
  \end{quote}
\subsection*{Notes}
\begin{itemize}
  \item{Today's main point: being a missional people wherever God has placed
  us.  I.e, to be effective salt and light.  Being missional means actually
  doing missions right where we are, which means adopting the posture of a
  missionary, learning and adapting to the culture around us while remaining
  biblically sound.}
  \item{There are lots of non-Christians around us, which means that wherever
  we are, we have opportunities for missions.  But first things first - a
  misisonal life flows out of a life transformed by God.  Until and unless
  God transforms us, we lack the core ingredient to be a truly missional
  people.  A missional life is all about loving, serving, and sharing the
  gospel of Christ to people where God has planted us.}
  \item{A missional people is a people who are called to be \textbf{loved}
  and \textbf{chosen} by God.  Paul was trying to remind the Thessalonians
  the truth of their acceptance from God.  We see that Paul started his
  letter with reminding the Thessalonians of their identity in God.  That is
  important, because we need to know who we are and to whom we belong before
  we can be missional.  Just like the Thessalonians, we too are loved and
  chosen by God.  We know God's love because while we were yet sinners,
  Christ died for us.  Knowing our identity as God's beloved children and
  God's chosen agents to be His ambassadors, we are enabled to love and serve
  those around us.  Christ's love compels us to preach the gospel, as was
  mentioned in 2 Corinthians 5:14-21.}
  \item{Sometimes, we forget our identity in Christ, and that causes us to be
  ineffective missionaries.  We all need to regain a fresh appreciation of
  God's love and calling for us.  That can be done through constant
  thanksgiving for God's love through His provision of our daily bread and
  our spiritual blessings.  We can also regain a fresh appreciation by
  remembering our salvation story.}
  \item{We are also a people empowered by the Holy Spirit.  We see how Paul
  and the Thessalonians were thus empowered by the Holy Spirit in verses 5
  and 6 of our text today.  Regardless of whatever means we use to share the
  gospel, the Holy Spirit is the one who empowers the messenger.  For
  example, the Holy Spirit gives us the power to live a holy life, the Holy
  Spirit gives us the words to say at a particular moment, and the Holy
  Spirit might sometimes give us power to work miracles.  The Holy Spirit
  also gives us supernatural joy in the midst of affliction.  All of these
  help testify to the truth of the gospel message.  By nature, it is
  difficult for us to love others, because we are all sinful.  By nature, it
  is difficult for us to by joyful in the midst of trials.  We need the Holy
  Spirit to help us to do things that are not natural to our sinful nature.
  We need to remember this truth and to always pray for the Holy Spirit to
  work in us.}
  \item{Being empowered by the Holy Spirit involves daily surrendering our
  will to God and asking the Holy Spirit to fill and control us as his
  missional people.  God desires that we live our lives not depending on
  ourselves, but by depending on His Holy Spirit.  We must be constantly
  aware of this fact, and we must be constantly submitting our will to God.}
  \item{Being missional also means that we are called to be a people who
  shares and lives out the gospel of Christ.  We share the gospel with words
  and conviction.  That is what Paul did to the Thessalonians (v5), and that
  is what the Thessalonians did (v8).  We do not wait for people to stumble
  on the gospel message themselves, but we use words to share the gospel.}
  \item{Success in witnessing is simply taking the initiative to share Christ
  in the power of the Holy Spirit and leaving the results to God.}
  \item{Practical methods to share the gospel: learn one method (e.g four
  spiritual laws), and learn how to share your testimony of how God has
  worked in your life.}
  \item{Not only do we use words to share the gospel, we live out the gospel
  through actions in life too.  We see that is what the Thessalonians did
  (v6-8).  Note that sharing the gospel in words must go hand-in-hand with
  living out the gospel through our lives.  We need both!  We can't just live
  a holy life and hope people infer the Christian message without us saying a
  word...  }
\end{itemize}
\end{document}


