\section{13th February 2022: Do you want to be blessed?}
\subsection*{Text: Jeremiah 17:5-10}
  \begin{quote}
    [5] Thus says the LORD:
    “Cursed is the man who trusts in man
        and makes flesh his strength,
        whose heart turns away from the LORD.
    [6] He is like a shrub in the desert,
        and shall not see any good come.
    He shall dwell in the parched places of the wilderness,
        in an uninhabited salt land.


    [7] “Blessed is the man who trusts in the LORD,
        whose trust is the LORD.
    [8] He is like a tree planted by water,
        that sends out its roots by the stream,
    and does not fear when heat comes,
        for its leaves remain green,
    and is not anxious in the year of drought,
        for it does not cease to bear fruit.”


    [9] The heart is deceitful above all things,
        and desperately sick;
        who can understand it?
    [10] “I the LORD search the heart
        and test the mind,
    to give every man according to his ways,
        according to the fruit of his deeds.”
  \end{quote}
\subsection*{Notes}
\begin{itemize}
  \item{The bible has a different take on what it means to be blessed, as
  compared to the secular idea of blessings being in the form of health and
  wealth.  Of course God can bless us with such health and wealth, but having
  health and wealth is not a necessary condition for one consider himself
  blessed by God (see Luke 6:17-26).  Let's understand why the poor etc can
  consider himself blessed.}
  \item{In the Jeremiah text, there are two groups; those who trust in man vs
  those who trust in God.  Those who trusted in man would be those, in
  Jeremiah's time, would be those who trust in military might, in false gods
  and false prophets.  Those who trusted in God, in Jeremiah's time, would be
  those who rely on God for deliverance.  There are also two analogies
  corresponding to these two groups: shrub in the desert vs the tree planted
  by the stream.
  \begin{itemize}
    \item{Those who trust in man are like the shrub who don't bear any fruit.}
    \item{Those who trust in God are like the tree which always bears fruit,
    whose leaves are always green.}
  \end{itemize} Both the shrub and the tree might both experience drought,
  but because the roots of the shrub are not deep, the shrub will wither.
  But for the tree, since the roots are deep, it can still bear fruit in
  times of drought.}
  \item{I.e, those who trust in man are cursed, and those who trust in God
  are blessed.  What is depicted here in Jeremiah is a throwback to the
  Mosaic Law, in Deuteronomy 28.  There will be blessings for obedience and
  curses for disobedience.  Faith and obedience go hand in hand; those who
  trust and have faith in God will obey God naturally.  When we trust God, we
  will obey.  And when we obey, we will be blessed.  Hence we can say that
  those who trust in God will be blessed.}
  \item{In Jeremiah's time, Judah was a nation that didn't trust in God,
  hence they experienced the curses of the Law, experiencing things like
  famine etc and even eventually exile.  But for those in Judah that were
  going through the same difficult times, they are still blessed as long as
  their trust is in the LORD.}
  \item{Hence in Luke, even in adverse circumstances (e.g poverty), we can
  still be blessed when we obey.  Blessing does not necessarily remove
  suffering; in the Jeremiah text, we see that even the tree experiences
  drought (Jeremiah 17:8).  Hence we can say that the poor who trust in God
  are still blessed.}
  \item{What are some of these blessings? They are:
  \begin{itemize}
    \item{Peace that transcends all understanding.}
    \item{Communion and friendship with God.}
    \item{Treasures in heaven.}
  \end{itemize}}
  \item{So how is all of these relevant for us?  The starting point of a
  blessed life is to put our trust in God for our salvation.  We must trust
  in Jesus' finished work on the cross rather than in our own works; our
  hearts are deceitful above all things, and hence our own works are rubbish.
  Jesus has come to seek and save the lost, only Jesus can save us, because
  only He has died on the cross for our sins.  For His suffering on our
  behalf, our sins are cleansed, but if and only if we put our trust in
  Jesus.  If we reject Jesus and put our trust in ourselves (i.e in man)
  instead, we are cursed.  The ``good works'' that we do will never be enough
  before a holy and just God.  Apart from Jesus, we are under the wrath of
  God for our sins; and this will manifest itself especially on judgement
  day.  Hence, the starting point of a blessed life is to trust in Jesus'
  completed work for our salvation.}
  \item{We can also put our trust in man when we trust in the physical
  blessings that God give us rather than recognising that God is behind all
  of those blessings.  When those physical blessings distract us from God,
  then those physical blessings will lead us to be cursed.  E.g, some
  parents' lives revolve around their children so much so that they have no
  time to spend with God, no time to participate in church life, etc.  We
  should work hard and study hard and take care of our kids etc, this is our
  testimony before the world.  But we should not let these things distract us
  from God, because if not we will be like the shrub in the desert.  Hence
  let us repent and seek God and His kingdom first.}
  \item{When we are going through difficult times, are we turning our hearts
  away from God in anger and disappointment or do we cling on to God with
  confidence and hope?  If we turn away from the source of blessings (i.e
  God), how can we be blessed? There are two examples which Pastor Kien Seng talked about:
  \begin{itemize}
    \item{Example 1: Very sian because of physical sickness, don't even want
    to come to church.}
    \item{Example 2: Physical sickness has led to greater trust and
    dependence in God.}
  \end{itemize}
  Here we see two people in the same circumstance but one is evidently more
  blessed than the other, because he is closer to God. The latter person is what it means to be blessed in God's eyes.}
  \item{So do you want to be blessed?  If you do,
  \begin{itemize}
    \item{Believe in Jesus, cultivate your relationship with Jesus.  Cling on
    to Him, put your trust in Him.  Put our trust in who God is and what He
    has done for us, especially in the finished work of the cross.  Putting
    our trust in God also means putting our trust in His Word, because His
    Word reveals who He is and what He has done.  We also put our trust in
    God's promises.  And when we trust in God's promises and His Word, we are
    like the wise man who builds his house on the rock.}
  \end{itemize}}
\end{itemize}

















