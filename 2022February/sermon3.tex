\section{20th February 2022: Turning from anger}
\subsection*{Text: Proverbs 15:1,18;16:32;27:3,4}
  \begin{quote}
    [1] A soft answer turns away wrath,
        but a harsh word stirs up anger.
    [18] A hot-tempered man stirs up strife,
        but he who is slow to anger quiets contention.

    [32] Whoever is slow to anger is better than the mighty,
        and he who rules his spirit than he who takes a city.   

    [3] A stone is heavy, and sand is weighty,
        but a fool's provocation is heavier than both.
    [4] Wrath is cruel, anger is overwhelming,
        but who can stand before jealousy?
  \end{quote}
\subsection*{Notes}
\begin{itemize}
  \item{Anger is a common emotion; even animals can get angry.  It is
    possible to get angry without sinning; see Ephesians (Be angry and do not
    sin).  Anger is not sin it and of itself, but our response to our anger
    could be sin.}
  \item{Anger is part of our body's fight or flight response, more
    specifically our fight response; anger can help us to protect
    ourselves/others, stand up against injustices of the world.}
  \item{To see that anger can be non-sinful, conisder that God can be angry;
    however, God's anger is wholly without sin.  God is also not fundamentally
    angry, he is fundamentally righteous; his anger is directed at the
    sinfulness of Man.}
  \item{But anger can of course be sinful, when it controls us and makes us
    meaningful, full of hatred and bitterness. One simple example is road rage.}
  \item{For Man, anger is usually an emotional response to a threat against
  our \textbf{ego}.  Human experience makes this abundantly clear.}
  \item{To help manage our anger, there are 3 main points:
  \begin{itemize}
    \item{First point: restraint, not react.  Instictively, when we hear
    something we don't like, sometimes we just react based on the emotions we
    are feeling.  These feelings are often based on our first impressions of
    the things we have heard, and contain a lot of our biases. Most of the time, people are triggered when:
    \begin{itemize}
      \item{They feel threatened.}
      \item{They feel frustrated or powerless, hence angry at ourselves.  Or
      angry at the situation.}
      \item{They feel invalitated or unfairly treated.}
      \item{They or their possessions are not respected.}
    \end{itemize}
    For us personally, it is helpful to find out what are our triggers.
    However, all of the above is not an excuse for sin; just because triggers
    exist doesn't mean we have to be triggered.  Self-control is a Christian
    virtue, a fruit of the Spirit (Galatians 5:22-23).  Practically, we can
    restrain when we deliberately interrupt our `reactive' response, by
    walking away, slowly counting to 5, taking a break from the situation,
    etc.  If we must respond, we are to give a `soft answer'.  If we can't
    give a `soft answer', it is better to keep quiet.  If we can't give a
    `soft answer' and shout back because of our anger, then the other person
    will get angry also, and then there'll be a relationship breakdown.  The
    above might seem hard to do, but remember that Jesus faced the biggest
    injustice in the world, yet he was like a lamb led to the slaughter; he
    opened not his mouth.}
    \item{Second point: re-evaluate, not relish.  When we re-evaluate
    something, we do it more objectively as compared to our instinctive
    response, and we widen the possible eplanations or interpreations of the
    actions or what was said.  When we relish, we take delight in going through
    the situation over and over again to stir up our emotions.  We become more
    angry and our burden becomes heavier.  We should also re-evaluate the
    consequences of our possible angry actions, especially in light of James
    1:19-20.  We must remember that our human anger does not produce the
    righteousness of God. One such consequence is that our unrestrained anger is a poor testimony to our Christian faith.}
    \item{Third point: release, not retain.  Keeping our anger within us is
    described as a burden, as in Proverbs 27:3-4.  A stone is a heavy thing,
    and as for sand, a lot of it is a heavy thing.  Sand is a particularly
    good analogy; if everytime we retain our anger we store up a grain of
    sand, then in half a lifetime, all that sand will accumulate to be
    something super heavy.  And also, as per the proverbs verse, one possible
    thing that stirs up our anger is jealousy.  To release our anger to God,
    we should:
    \begin{itemize}
      \item{Be truthful with God about your feelings.  Lament them to God, as
        per in the Psalms.}
      \item{Be ready to forgive those who said those things that made us
        angry.  Just as God forgiven us, we should also forgive.}
      \item{Ask God to turn our problems into solutions.}
      \item{To combat jealousy, we thank God that we are made differently and
        given different gifts.  We affirm the good in our lives and recognise
        the source of goodness which is from God.}
    \end{itemize}}
  \end{itemize}}
  \item{Our angry, sinful response is merely a symptom of a greater problem,
  which is sin.  The Bible is not just an anger management manual; when Jesus
  died on the cross for our sins, He removed the root of our human anger,
  which is sin.  The ultimate remedy for anger is to kill the sin in our
  lives, through the ministry of the Holy Spirit and of the Word.  We are to
  emuluate Jesus' example, as per Philippians 2:3-11, and when we do, anger
  will vanish.  When we are slighted, we must remember that we forgive as God
  forgiven us.}
\end{itemize}