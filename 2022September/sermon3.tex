\section{18th September 2022: Acceptance in rejection}
\subsection*{Text: Revelation 2:8-11}
  \begin{quote}
    [8] “And to the angel of the church in Smyrna write: ‘The words of the
    first and the last, who died and came to life.

    [9] “‘I know your tribulation and your poverty (but you are rich) and the
    slander of those who say that they are Jews and are not, but are a
    synagogue of Satan.  [10] Do not fear what you are about to suffer.
    Behold, the devil is about to throw some of you into prison, that you may
    be tested, and for ten days you will have tribulation.  Be faithful unto
    death, and I will give you the crown of life.  [11] He who has an ear,
    let him hear what the Spirit says to the churches.  The one who conquers
    will not be hurt by the second death.’
  \end{quote}
\subsection*{Notes}
\begin{itemize}
  \item{God made us social creatures, so we have an innate desire for close
  lasting relationships with others.  A collorary is that we desire social
  acceptance, we want to be accepted by our peers, by our society, etc.
  Hence, a fear of social rejection is common among us humans, as we know
  from our own experience.}
  \item{Today's text is about the church in Smyrna, who are facing religious
  persecution.  The church in Singapore is not facing religious persecution
  of the same kind as the church in Smyrna.  In those days, everyone was
  supposed to sacrifice to the Roman empires and the Roman gods.  Except the
  Jews, who were officially granted an exclusion to that.  At first,
  Christians were worshipping together with Jews in their synagogues, but as
  the number of Christians grew and as the theological differences between
  Jews and Christians grew, the Jews kicked the Christians out of the
  synagogues.  Then, the Jews leveraged the power of the majority of society
  and the power of the state to force the Christians to give up their faith,
  by slandering them and painting a bad picture of them (v9).  As a result,
  the Smyrna society rejected the Christians and excluded them from social
  life.  This led to the poverty of the Christians and etc.}
  \item{In Singapore, we aren't facing such open persecution, but we are
  facing a more subtle form of persecution.  This is the result of society
  being more secularised.  More and more people don't want the Church to have
  a voice in the public square.  The Church is mocked when she talks about
  sin and judgment, and in modern days, the Church is slandered and shunned
  especially when she talks about the Christian view of sexuality.  This
  causes us to subtly shift our values etc, because of a fear of social
  rejeciton.}
  \item{Hence, today's message: when the world rejects you, don't be afraid
  to stand firm for your faith. }
  \item{First reason why we should stand firm: when people reject us, it
  frees us for God.  The more we want to be accepted by others, the more we
  want to care about other's opinion of us, and as a result, the harder it is
  for us to make godly choices because of peer pressure.  But if we embrace
  the rejection of people, we are freed to make godly choices.  The
  Christians in Smyrna had a choice to make; they could have distanced
  themselves from the slander by giving up the public profession of their
  faith, or they could embrace the slander that came with the public
  profession of their faith (it was hard for them to refute the slander,
  since they were in the minority).  But when we accept that being rejected
  by society is an inevitable result of following Jesus, then we can be free
  to ignore public perception and continue to live a faithful life.  Jesus
  promises us that though the world rejects us, He will accept us.  He will
  give us our treasures in heaven, and He will glorify us.}
  \item{Second reason why we should stand firm: when people reject us, it
  refines us for God.  A writer says (and I summarise): ``The church is the
  strongest when it is culturally persecuted, and the church is the weakest
  when it seems identical with the culture''.  It seems from the text that
  society was taking the slander very seriously, and even sending the
  Christians in Smyrna to prison.  The $10$ days in our text suggests that
  the Christians in Smyrna would suffer and be tortured for $10$ days, before
  being executed, as seen from the exhortation to be ``faithful unto
  death''.  God permits us to be tested by the devil, so that it refines our
  faith.  When we are victorious over the devil and his testing, then we will
  be victorious over the second death.  Yet when we fail the devil's test and
  give up our faith in the face of persecution, we show that we are not
  really Christians, and hence we will be hurt by the second death.}
  \item{Third reason why we should stand firm: God Himself is with us in our
  suffering.  We naturally think that God is with us when things are going
  smoothly, for example when Church attendance is booming.  A corollary is
  that we also naturally think that God is not with us or even against us
  when things are going poorly, for example when we have terminal illness.
  The Christians in Smyrna were tempted to think that because of the
  suffering that they are facing, God might not be with them.  Yet that is a
  wrong thought.  The paradox of Christianity is that God is most revealed at
  the cross.  As Paul says, the crucified Christ is a stumbling block to the
  Jews and folly to the Gentiles.  People only imagine God coming in glory,
  but not God coming in suffering and rejection.  For us who understand the
  gospel, Christ crucified is the power of God and the wisdom of God.  Hence,
  when we carry the cross in this life like our Lord Jesus, then God is most
  with us.  Jesus lived a faithful and perfect life, that's why the world
  despised Him and rejected Him.  But in spite of the world's rejection of
  Jesus, Jesus is the Father's beloved Son, and the Father is with Jesus.
  Similarly, when we live faithful lives, the world will naturally reject us,
  but yet like Jesus, the Father will be with us in our suffering through His
  Spirit, the same Spirit that landed on Jesus at His baptism.  }
  \item{And lastly, the Lord's supper is a gracious reminder of all of the above.}
\end{itemize}