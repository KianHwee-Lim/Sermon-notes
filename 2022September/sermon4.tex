\section{25th September 2022: There is no room for compromise}
\subsection*{Text: Revelation 2:12-17}
  \begin{quote}
    [12] “And to the angel of the church in Pergamum write: ‘The words of him
    who has the sharp two-edged sword.

    [13] “‘I know where you dwell, where Satan’s throne is.  Yet you hold
    fast my name, and you did not deny my faith even in the days of Antipas
    my faithful witness, who was killed among you, where Satan dwells.  [14]
    But I have a few things against you: you have some there who hold the
    teaching of Balaam, who taught Balak to put a stumbling block before the
    sons of Israel, so that they might eat food sacrificed to idols and
    practice sexual immorality.  [15] So also you have some who hold the
    teaching of the Nicolaitans.  [16] Therefore repent.  If not, I will come
    to you soon and war against them with the sword of my mouth.  [17] He who
    has an ear, let him hear what the Spirit says to the churches.  To the
    one who conquers I will give some of the hidden manna, and I will give
    him a white stone, with a new name written on the stone that no one knows
    except the one who receives it.’
  \end{quote}
\subsection*{Notes}
\begin{itemize}
  \item{To each of the seven churches, Christ conveys to them something of
  Him that each church needs to know.  So far, everything has been kinda
  positive (e.g see the letter to Ephesus and Smyrna).  For Pergamum, Christ
  is described as the one who has the sharp double edged sword, which sounds
  terrifying at first.  But more of that later.}
  \item{The city of Pergamum was the capital of Asia minor before Caesar
  Augustus changed the capital to Ephesus.  This was another city where the
  worship of the Roman emperor was popular.  This was the first city in Asia
  minor to build a temple to Caesar.  Thus when Pergamum is described as a
  ``city where Satan dwells'', it might be a reference to this emperor cult.
  Tradition has it that Antipas was a bishop of Pergamum who was roasted
  alive.}
  \item{First, the church at Pergamum was commended for holding fast to the
  name of Jesus.  One's name contains his attributes and his characteristics.
  Holding fast to Jesus' name means believing everything that he has decalred
  about himself as revealed in the Word of God.  This is also what it means
  to fix our eyes on Jesus, which is to believe in everything Jesus said
  about himself and keep faith.}
  \item{Next, the church at Pergamum was condemned for something that the
  Lord Jesus found wanting.  This follows the usual pattern of the letter to
  the seven churches.  For Pergamum, some in the church followed the
  teachings of Baalam and the Nicolaitans.  What Balaam did in the past was
  to influence the people of Israel to follow in the faith of the surrounding
  Canaanites, by teaching them to worship the gods of the Canaanites and by
  teaching them to take the Canaanite women as wives.  So in short, Balaam
  taught the Israelites in the past to be influenced by their surrounding
  culture especially by teaching the Moabite women to cause the Israelite men
  to act treacherously (c.f Numbers 31).  So the reference to Balaam for the
  Pergamum church means that some of the Christians in the Pergamum church
  were doing the same things as their pagan neighbours in doing pagan
  worship.  And as part of ancient pagan worship, it was customary to have
  sex wtih the temple prostitutes, i.e sexual immorality.  An ancient author
  says that Pergamum was given to pagan worship more than all of the cities
  of Asia minor.}
  \item{Are we like the church in Pergamum, i.e are we compromising with the
  world?  Is the church being secularised?  An example is the prosperity
  gospel.  The world treats material wealth as a form of success, and hence
  when the church holds up material wealth as the final goal of Christian
  worship, then that is a compromise.  These churches also do not talk about
  hell and the wrath of God on sin.  Also, some churches and some churchgoes
  have a consumeristic mindset, which is a worldly virtue, not a Christian
  one.  Worship should be us offering our service to God, not about how we feel (because if it is about how we feel, we will hop from church to church to find the church that gives us the best feeling).}
  \item{The Pergamum church is compromising by worship to idols, and this is
  a lesson for us too.  While we do not worship idols of wood and stone, we
  can still set up idols in our hearts (Ezekiel 14:13).  Usually for us,
  these idols are good things (like relationships with people) that become
  ultimate things.  This is what happens when we love God's gifts more that
  God, and as a result, we set up God's gifts as an idol. And this idolatry often hurt us and the people around is.  For parents, they
  might idolise the success of their kid, and love the idea of their kids
  success more than God.  For some people, they might idolise the idea of a
  romantic relationship and place their ultimate happiness in their spouse, a
  burden their spouse cannot bear.  }
  \item{Helpful framework to identify idols: 
  \begin{itemize}
    \item{Imagination - what do you habitually think about to get joy and comfort in the privacy of your heart?}
    \item{Expenditures - where do we spend bulk of our money on to give you pleasure?}
    \item{Prayers - what do you most often pray about and how do you respond to unanswered prayers and frustrated hopes?}
    \item{Emotions - look at your most uncontrollable emotions, especially those what drive you to do things that are wrong.}
  \end{itemize}
  If the answer to any of these is not God, then...  haha.  Oof.}
  \item{After condemning some of the behaviours of the Pergamum church, there
  is an exhortation to repent.  Repentance would lead to eternal life (here
  it is described as manna and a white stone with a new name), and a lack of
  repentance would lead to judgment (Jesus will fight them with the sword of
  his mouth).  The reference to manna makes sense because of the Pergamum
  church's compromise by eating food offered to idols, and the reference to a
  white stone is because in those days, it is what is given to athletes to
  mark their victory.}
\end{itemize}