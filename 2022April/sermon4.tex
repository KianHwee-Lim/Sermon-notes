\section{24th April 2022: Is it all meaningless?}
\subsection*{Text: Ecclesiastes 1:1-11}
  \begin{quote}
    [1] The words of the Preacher, the son of David, king in Jerusalem.

    [2] Vanity of vanities, says the Preacher,
        vanity of vanities! All is vanity.
    [3] What does man gain by all the toil
        at which he toils under the sun?
    [4] A generation goes, and a generation comes,
        but the earth remains forever.
    [5] The sun rises, and the sun goes down,
        and hastens to the place where it rises.
    [6] The wind blows to the south
        and goes around to the north;
    around and around goes the wind,
        and on its circuits the wind returns.
    [7] All streams run to the sea,
        but the sea is not full;
    to the place where the streams flow,
        there they flow again.
    [8] All things are full of weariness;
        a man cannot utter it;
    the eye is not satisfied with seeing,
        nor the ear filled with hearing.
    [9] What has been is what will be,
        and what has been done is what will be done,
        and there is nothing new under the sun.
    [10] Is there a thing of which it is said,
        “See, this is new”?
    It has been already
        in the ages before us.
    [11] There is no remembrance of former things,
        nor will there be any remembrance
    of later things yet to be
        among those who come after.
  \end{quote}
\subsection*{Notes}
\begin{itemize}
  \item{Life sometimes feels like a chore.  E.g endless household chores to
  do; we vacuum and mop the floor, yet tomorrow it becomes dirty again.
  There seems to be a tiresome cycle to life, we just do the same toilsome
  things over and over again.  After a while, we ask: what is life all about?
  Are things all meaningless?}
  \item{Ecclesiastes belongs to the genre of wisdom literature, together with
  Proverbs, Job, and Song of Songs.  Wisdom literature wrestles with the
  question of how to live wisely amid the many challenges of life.
  Ecclesiastes opens with a prologue by the `Narrator' (1:1 to 1:11), and
  then the `Preacher' speaks (1:12 to 12:7), and then there is an epilogue by
  the `Narrator' again (12:8 to 12:14).  Hence, as for the structure of the
  book, we can think of it as the words of the `Preacher' sandwiched in
  between the prologue and the epilogue.  The English name of the book,
  Ecclesiastes, is a transliteration of the Greek (ekklesiastes), but in the
  Hebrew the name is `Qoheleth', which literally means "the one who
  assembles".}
  \item{Who is the `Preacher'?  The `Preacher' seems to be speaking from the
  point of view of Solomon, though some modern bible scholars now think that
  Ecclesiastes is a fictional autobiography written at a later time by
  someone writing in the Solomonic tradition.  Most likely, Ecclesiastes was
  written during the post-exilic period, when Israel was still under foreign
  rule and when there was much social injustice.  While there are many
  indications that the `Preacher' could be Solomon, there are also many
  indications that the `Preacher' is not; for example, Ecclesiastes 1:16a
  would not make sense if the `Preacher' is Solomon, since the only king
  before Solomon was David.}
  \item{OT books like Proverbs and the OT Law describe the action-consequence
  formula in detail.  The idea is: do good things and follow God, and then
  you'll be blessed.  Do bad things and disobey God, and then you'll be
  cursed.  But in the post-exilic period, the action-consequence principle
  doesn't seem to hold anymore.  The wicked prosper, and the righteous
  suffer.  Hence, the Preacher was writing from the POV of someone who was
  frustrated with how the action-consequence principle doesn't seem to apply
  in his day.}
  \item{Four main points for today, all starting with M.  The first M is
  `meaningless'.  `Hebel/Hevel' in Hebrew literally means breath or vapour,
  and in English, that idiom is translated as `vanity' or `meaninglessness'.
  There are things in life that make life seem utterly meaningless.  How did
  the Preacher come to this conclusion?  Firstly, it seems that there was
  nothing to be gained from all the toil under the sun.  Like vapor, the
  things we do in our life has no permanent impact, it makes no lasting
  impression.}
  \item{The next M is monotony.  To support his argument, the Preacher uses
  arguments from the natural world (v4-7), and from human experiences
  (v8-10).  From the arguments from the natural world, we see that one
  generation comes and another generation goes, but the cycles that happen in
  nature just keep repeating itself monotonously, there doesn't seem to be
  any real progress in the things in nature.  From the arguments from human
  experiences, we see that nothing really satisfies us.  Everyday we see an
  endless procession of internet images, from Netflix, Disney+, etc etc, but
  after all our looking and listening, our eyes are not satisfied.  We still
  want to see some more and to hear some more.  There is always one more game
  to watch, one more show to watch, one more song to listen to.  The
  circularity of nature above is paralleled by the repetitiveness of history;
  there is nothing new under the sun.  What has been is what will be, what
  has been done is what will be done.  E.g, war has been a big part of human
  history, and it will continue to be a big part of human history.  While the
  weapons of war are different, the effects are the same; people die, homes
  are destroyed.  While some might say that the cycles in Nature are
  beautiful and even essential, we must empathise with the Preacher as he
  reflects on how small he is and how everything he does seems to be
  inconsequential.}
  \item{The next M is mortality (v11).  If the above two points are not
  depressing enough, the climax of the Preacher's argument is this; one day
  we all have to die, and no one remembers us.  We may be crowned the Olympic
  champion today, but four years later we don't make the cut, and people
  don't remember us.  What more after we die?}
  \item{There are two ways of understanding Ecclesiastes:
  \begin{itemize}
    \item{Approach 1: If God is not in our life, life will be meaningless;
    only when we bring God into the picture will life be meaningful.  Thus,
    the goal of the Preacher is to drown us in pessimism, so that we will
    gasp for air and realise our need for God.}
    \item{Approach 2: Even when God is in our lives, life can still feel
    meaningless.  The Preacher then wants to teach us the following lesson:
    How do we then resolve the tension between what our faith teaches us
    (that life ought to be meaningful), and what we observe and experience in
    our lives that seems to point in the opposite direction?}
  \end{itemize}
  The approach for this sermon series on Ecclesiastes will be to take the
  second approach, as the Pastoral team thinks that the second message is
  more faithful to the text.  The message that is repeated through
  Ecclesiastes (6 times) is the verse in Ecclesiastes 2:24.  Even when life
  seems to be meaningless, when life doesn't make sense, when God seems to be
  absent, we should still fear God.  We should still be joyful and do good.
  We can still eat and drink and be thankful for God's provision even in our
  toil.  As Christians, we can identify with what the Preacher is saying.  In
  Romans 8:20, we see that the creation was subjected to futility.  This is a
  result of sin.  But Christ, in His incarnation, entered into our creation
  which is subjected to futility and took on the full brunt of that futility
  which is a result of us being separated from God through sin.  Hence,
  through Christ's resurrection, we can experience the full meaning of life
  as we are reconciled with God.  But insofar as we are still in the flesh,
  we are `groaning' with the entire creation (also Romans 8) as we await the
  redemption of our bodies which will only happen with Christ comes again.}
  \item{In the end, we have to struggle with that tension between what our
  faith teaches us (that life is meaningful), and what we experience
  sometimes (that life is meaningless).  One way to think about it is that
  when life seems meaningless, we should not conclude that there is no
  meaning to life, but we should conclude that the meaning is currently
  incomprehensible to us.  And while we struggle with the incomprehensibility
  of certain things that happen in life - the apparent meaninglessness,
  monotony and our mortality - we must still acknowledge that God is very
  much present and that we can trust that in all things God works for the
  good of those who love him, and he will make everything beautiful in its
  time.  And as we trust God, we obey Him and fear Him and keep His
  commandments in faith, for that is our duty.  We have the hope that God
  will bring every deed into judgment, with every secret thing, whether good
  or evil.  That is the message of Ecclesiastes, and that is the last M for
  today.}
\end{itemize}