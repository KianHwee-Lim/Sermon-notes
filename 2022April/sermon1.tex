\section{3rd April 2022: Through the fire and the flames}
\subsection*{Text: Mark 14:27-50}
  \begin{quote}
    [27] And Jesus said to them, “You will all fall away, for it is written,
    ‘I will strike the shepherd, and the sheep will be scattered.’ [28] But
    after I am raised up, I will go before you to Galilee.” [29] Peter said
    to him, “Even though they all fall away, I will not.” [30] And Jesus said
    to him, “Truly, I tell you, this very night, before the rooster crows
    twice, you will deny me three times.” [31] But he said emphatically, “If
    I must die with you, I will not deny you.” And they all said the same.

    [32] And they went to a place called Gethsemane.  And he said to his
    disciples, “Sit here while I pray.” [33] And he took with him Peter and
    James and John, and began to be greatly distressed and troubled.  [34]
    And he said to them, “My soul is very sorrowful, even to death.  Remain
    here and watch.” [35] And going a little farther, he fell on the ground
    and prayed that, if it were possible, the hour might pass from him.  [36]
    And he said, “Abba, Father, all things are possible for you.  Remove this
    cup from me.  Yet not what I will, but what you will.” [37] And he came
    and found them sleeping, and he said to Peter, “Simon, are you asleep?
    Could you not watch one hour?  [38] Watch and pray that you may not enter
    into temptation.  The spirit indeed is willing, but the flesh is weak.”
    [39] And again he went away and prayed, saying the same words.  [40] And
    again he came and found them sleeping, for their eyes were very heavy,
    and they did not know what to answer him.  [41] And he came the third
    time and said to them, “Are you still sleeping and taking your rest?  It
    is enough; the hour has come.  The Son of Man is betrayed into the hands
    of sinners.  [42] Rise, let us be going; see, my betrayer is at hand.”

    [43] And immediately, while he was still speaking, Judas came, one of the
    twelve, and with him a crowd with swords and clubs, from the chief
    priests and the scribes and the elders.  [44] Now the betrayer had given
    them a sign, saying, “The one I will kiss is the man.  Seize him and lead
    him away under guard.” [45] And when he came, he went up to him at once
    and said, “Rabbi!” And he kissed him.  [46] And they laid hands on him
    and seized him.  [47] But one of those who stood by drew his sword and
    struck the servant of the high priest and cut off his ear.  [48] And
    Jesus said to them, “Have you come out as against a robber, with swords
    and clubs to capture me?  [49] Day after day I was with you in the temple
    teaching, and you did not seize me.  But let the Scriptures be
    fulfilled.” [50] And they all left him and fled.
  \end{quote}
\subsection*{Notes}
\begin{itemize}
  \item{Discipleship is not always easy.  Small example: saturday night is a
  prime time for fun and entertainment, which makes waking up in Sunday
  morning difficult.  We can't follow Jesus and be His disciples if we follow
  what our flesh desires.  What our flesh desires is usually contrary to
  following the Lord.}
  \item{True Christian discipleship is a renunciation of our self and a
  conformity to God.  This is expressed in Jesus' prayer in Gethsemane: ``Yet
  not what I will, but what you will''.  As Jesus thought about His upcoming
  betrayal and death, He was distressed.  This was not because Jesus was
  suddenly taken aback by this.  He has known from the start that this was
  His Father's will (e.g Mark 9).  The agony in Jesus' soul wasn't a result
  of an inner wavering, it was a very human reaction to the horrors that was
  to come.  Our Lord is completely human, and completely God (c.f
  Dyophysitism and Dyothelitism).}
  \item{Jesus could have just escaped under the cover of darkness, but He
  didn't.  In the face of this great agony and temptation, He prayed.  He
  knew that His Father could do all things, yet He prayed: ``Yet not what I
  will, but what you will''.  Since the Christian life is a life of following
  the pattern of Jesus, we too must deny ourselves and obey God's will.}
  \item{There are three ways in which we can renounce ourselves and conform
  to God.  The first way is to renounce an overconfidence in ourselves.  We
  must always humbly depend on God to keep us from falling away.  Jesus here
  quotes a prophecy from Zechariah 13 that prophesies the temporary falling
  away of Jesus' disciples.  But when Peter heard that, he over-confidently
  said that he won't deny Jesus.  Yet all of the disciples fled Jesus' side
  at verse 50.  As fallen humans, we are all susceptible to temptation.
  Though we know what we ought to do and what we ought not to do, we might
  still fall into temptation.  ``The Spirit is willing, but the flesh is
  weak''.  If we are overconfident in our ability to follow God, we will
  fall.  The paradox is this: we must realise we are weak, which would push
  us to depend on God, which would then keep us from temptation.  As Jesus
  said: ``watch and pray, that you may not enter into temptation''.  When we
  watch, we are watchful for the external circumstances that we know we are
  weak to.  And when we pray, we are depending on divine assistance against
  temptation (c.f Psalm 121).  Discipleship means a humble dependance on
  God's grace, so we may be conformed to God.  The disciples slept and hence
  they fell; on the other hand, Jesus watched and prayed and depended fully
  on God, hence Jesus overcame temptation.}
  \item{Discipleship also means renouncing false expectations and accept the
  cross.  When it comes to following our Lord, expectations are important,
  because false expectations hinder us from discipleship.  One reason the
  disciples deserted Jesus was because they had a certain expectation of what
  Jesus would do.  When the mob first came, they didn't flee yet; they were
  prepared to defend Jesus, with one of them even drawing his sword.  They
  only fled after Jesus said: ``let the Scriptures be fulfilled''.  The
  disciples did not see that coming.  They were expecting Jesus to establish
  an earthly political kingdom as the messiah.  They were expecting Jesus to
  start a revolution to overthrow the Romans and the Herodian dynasty.  This
  is why when we read the earlier chapters, we see James and John the son of
  Zebedee asking to be on Jesus right hand when the kingdom comes.  When
  Jesus gave Himself over, the disciples all fled because their false
  expectations crumbled.  For us, when we also have false expectations about
  Jesus, and usually these false expectations appeal to our flesh such as the
  prosperity gospel, these false expectations give people a mistaken idea of
  Christianity.  And hence when these false expectations crumble, people will
  also lose their ``faith''.  The real thing about Christianity is to follow
  Jesus to be crucified; i.e while Jesus was literally crucified, we must
  crucify the desires of our flesh which may or may not include matyrdom.  We
  must follow our Lord through the fire and flames of temptation.  Our sinful
  self must die, and our sinful flesh must be crucified.  And our sinful self
  must be replaced by an obedience of faith.  Yet paradoxically, it is only
  though this struggle that we have freedom and peace.  E.g, the struggle to
  quit smoking is tough, yet the final freedom from addiction is truly
  liberating.  Similarly, we must struggle with sin and temptation through
  God's Spirit, and only when we do so, then we are truly free.}
  \item{Discipleship also means renouncing the world.  The disciples were
  ready to defend Jesus to establish a worldly kingdom, yet they all fled
  when Jesus gave himself up to establish a spiritual kingdom.  The way of
  the world is to fight, trample, manipulate and dominate for the sake of
  gain and power.  This is not anything new; this has been around since the
  days of Cain and Abel.  On the other hand, we have Jesus, who was ready to
  give Himself up for the world.  When we were still sinners, Christ died for
  us.  The way of God, as demonstrated by Christ is love and self-sacrifice.
  This is clearly in contradistinction to the way of the world.  If Jesus'
  kingdom were of the world, we would all be taking up arms and manipulating
  people etc for Jesus' kingdom.  Yet since Jesus' kingdom is not of the
  world, the way to bring about Jesus' kingdom is through love, just like
  what Jesus did for the world.  A real life example would be the AWARE (an
  NGO) saga, where a group of Christian women took over the NGO's leadership
  through subterfuge, because they were concered about AWARE's pro-LGBTQ
  stance.  The concerns of the Christians involved were valid, but to resort
  to the ways of the world is to damage our Christian witness.  Jesus'
  kingdom is not of the world, and if we rely on worldly means such as taking
  up the sword, we damage the message of the Kingdom of God, which is a
  message of self-sacrificial love.  }
  \item{Discipleship isn't easy, but our Lord is with us.  Following Him
  means we must renounce ourselves and conform to God, which isn't easy.
  Since we aren't perfectly obedient, we might find ourselves falling away
  from time to time.  But because our Lord is with us, He will pick us up
  when we fall.  In verse 28, after predicting the disciples' abandonment,
  Jesus said that He will be raised up and go ahead of them to meet Him in
  Galilee.  Mark's gospel doesn't really tell us what happens in Galilee, but
  from the other gospels, we see that Jesus commissioned the disciples in
  Galilee.  Right from the start, Jesus was ready to forgive the disciples
  for their failing.  As is written in 1 John 1:9, if we confess our sins,
  God is faithful and just to forgive us our sins and to cleanse us from all
  unrighteousness.  }
\end{itemize}